\section{Lesson 21 Daniel Mendoza}

\begin{paracol}{2}

\nw{Boxing} matches were very popular in England two hundred years ago. 

\switchcolumn

\chinesetext{两百年前,拳击比赛在英国非常盛行。}
\nwe{boxing}{ˈbɑːksɪŋ}{n. 拳击(运动);v. (拳击运动)击打(box的现在分词);}

\switchcolumn*

\ns{In those days}, \nw{boxers} fought with \ns{\nw{bare} fists} for prize money. 

\switchcolumn

\chinesetext{当时,拳击手们不戴手套,为争夺奖金而搏斗。}
\nwe{boxer}{ˈbɑːksər}{n. (尤指职业)拳击手;拳师狗;装箱者,装箱工;}
\nwe{bare}{ber}{adj. 裸露的,光秃秃的;未加装饰的;空的;最低限度的,勉强的;仅有的;v. 使暴露,露出;}
\nse{in those days}{}{当时,那时候;当年;}
\nse{bare fists}{}{赤手空拳}

\switchcolumn*

Because of this, they \ns{were known as} '\nw{prizefighters}'. 

\switchcolumn

\chinesetext{因此,他们被称作“职业拳击手”。}
\nwe{prizefighter}{'praɪzfaɪtə(r)}{n. 职业拳击手;}
\nse{be known as}{bi noʊn æz}{v. 被认为是;号称;叫做;}

\switchcolumn*

However, boxing was very \nw{crude}, for these were no rules and a prizefighter could be seriously injured or even killed during a match.

\switchcolumn

\chinesetext{不过,拳击是十分野蛮的,因为当时没有任何比赛规则,职业拳击手有可能在比赛中受重伤,甚至丧命。}
\nwe{crude}{kruːd}{adj. 粗略的;简陋的;粗俗的;天然的;n. 原油;石油;}

\switchcolumn*

One of the most \ns{colourful figure} in boxing history was Daniel Mendoza, who was born in 1764. 

\switchcolumn

\chinesetext{拳击史上最引人注目的人物之一是丹尼尔·门多萨,他生于1764年。}
\nse{colourful figures}{}{引人注目的人物}

\switchcolumn*

The use of gloves was not introduced until 1860, when the \nw{Marquis} of Queensberry \ns{drew up} the first set of rules. 

\switchcolumn

\chinesetext{1860年昆斯伯里侯爵第一次为拳击比赛制定了规则,拳击比赛这才用上了手套。}
\nwe{marquis}{ˈmɑrkwɪs}{n. 侯爵;世袭贵族;}
\nse{draw up}{drɔː ʌp}{折叠;草拟;(使)停下;排好队伍;}

\switchcolumn*

Though he was \nw{technically} a prizefighter, Mendoza did much to \ns{change crude prizefighting into} a sport, for he brought science to the game. 

\switchcolumn

\chinesetext{虽然门多萨严格来讲不过是个职业拳击手,但在把这种粗野的拳击变成一种体育运动方面,他作出了重大贡献。是他把科学引进了这项运动。}
\nwe{technically}{ˈteknɪkli}{adv. 技术上;学术上;专业上;严格说来;}
\nse{change ... into ...}{}{把...变成...;把...引进...}

\switchcolumn*

\ns{In his day}, Mendoza enjoyed tremendous popularity. 

\switchcolumn

\chinesetext{门多萨在的全盛时期深受大家欢迎。}
\nse{in his day}{}{在他的全盛时期}

\switchcolumn*

He was \nw{adored} by \ns{rich and poor} \nw{alike}.

\switchcolumn

\chinesetext{无论是富人还是穷人都对他祟拜备至。}
\nwe{adore}{əˈdɔːr}{v. 热爱,喜爱;爱慕,崇拜;}
\nwe{alike}{əˈlaɪk}{adj. 相同的;相似的;adv. 相似地;同等地;两者都;}
\nse{rich and poor}{}{富人和穷人}

\switchcolumn*

Mendoza \ns{rose to \nw{fame}} swiftly after a boxing match when he was only fourteen years old. 

\switchcolumn

\chinesetext{门多萨在14岁时参加一场拳击赛后一举成名。}
\nwe{fame}{feɪm}{n. 名声,名誉;}
\nse{rise to fame}{raɪz tu fem}{成名;}

\switchcolumn*

This attracted the attention of Richard Humphries who was then the most \nw{eminent} boxer in England. 

\switchcolumn

\chinesetext{这引起当时英国拳坛名将理查德.汉弗莱斯的注意。}
\nwe{eminent}{ˈemɪnənt}{adj. 卓越的;杰出的;著名的;}

\switchcolumn*

He offered to train Mendoza and his young pupil was quick to learn. 

\switchcolumn

\chinesetext{他主动提出教授门多萨,而年少的门多萨一学就会。}

\switchcolumn*

\ns{In fact}, Mendoza soon became so successful that Humphries \ns{turned against} him. 

\switchcolumn

\chinesetext{事实上,门多萨不久便名声大振,致使汉弗莱斯与他反目为敌。}
\nse{in fact}{}{事实上}
\nse{sb. turned against sb.}{}{反对}

\switchcolumn*

The two men quarrelled \nw{bitterly} and it was clear that the argument could only be settled by a fight. 

\switchcolumn

\chinesetext{两个人争吵不休,显而易见,只有较量一番才能解决问题。}
\nwe{quarrel}{ˈkwɑːrəl}{n. 争吵,分歧;抱怨的理由;v. 争吵;反对;}
\nwe{bitterly}{ˈbɪtərli}{adv. 伤心地;愤怒地;极其,非常;}

\switchcolumn*

A match was held at Stilton, where both men fought for an hour. 

\switchcolumn

\chinesetext{于是两人在斯蒂尔顿设下赛场,厮打了一个小时。}

\switchcolumn*

The public \nw{bet} \ns{a great deal of} money on Mendoza, but he was defeated. 

\switchcolumn

\chinesetext{公众把大笔赌注下到了门多萨身上,但他却输了。}
\nwe{bet}{bet}{n. 打赌,赌注;预计;v. 打赌,下注;希望;敢说;}
\nse{a great deal of}{e ɡret dil ʌv}{大量;}

\switchcolumn*

Mendoza met Humphries in the ring on a later occasion and he lost \nw{for a second time}. 

\switchcolumn

\chinesetext{后来,门多萨与汉弗莱斯再次在拳击场上较量,门多萨又输了一场。}
\nse{for a second time}{}{第二次}

\switchcolumn*

\newsentence{It was not until his third match in 1790 that he finally beat Humphries and became Champion of England. }

\switchcolumn

\chinesetext{直到1790年他们第3次对垒,门多萨才终于击败汉弗莱斯,成了全英拳击冠军。}

\switchcolumn*

Meanwhile, he founded a highly successful \nw{Academy} and even Lord Byron became one of his pupils. 

\switchcolumn

\chinesetext{同时,他建立了一所拳击学校,办得很成功,连拜伦勋爵也成了他的学生。}
\nwe{academy}{əˈkædəmi}{n. 专科院校;研究院,学会;(美国)私立中学;}

\switchcolumn*

He earned \ns{enormous sums of money} and was paid \ns{as much as} £100 \ns{for a single appearance}. 

\switchcolumn

\chinesetext{门多萨挣来大笔大笔的钱,一次出场费就多可达100英镑。}
\nse{enormous sums of money}{}{巨额资金}
\nse{as much as}{æz mʌtʃ æz}{adv. 差不多;足;}
\nse{for a single appearance}{}{单次出场}

\switchcolumn*

Despite this, he was so \nw{extravagant} that he was always \ns{in debt}. 

\switchcolumn

\chinesetext{尽管收入不少,但他挥霍无度,经常债台高筑。}
\nwe{extravagant}{ɪkˈstrævəɡənt}{adj. 奢侈的;过分的;放肆的;}
\nse{in debt}{ɪn det}{欠债;}

\switchcolumn*

After he was defeated by a boxer called Gentleman Jackson, he was quickly forgotten. 

\switchcolumn

\chinesetext{他在被一个叫杰克逊绅士的拳击手击败后很快被遗忘。}

\switchcolumn*

He \ns{was sent to prison} for failing to pay his debts and died in \nw{poverty} in 1836.

\switchcolumn

\chinesetext{他因无力还债而被捕入狱,最后于1836年在贫困中死去。}
\nwe{poverty}{ˈpɑːvərti}{n. 贫穷;短缺;}
\nse{be sent to prison}{}{被送往监狱}

\switchcolumn*

\end{paracol}

\retellingpoints

\begin{multicols}{1}
    \begin{enumerate} 
        \item \textbf{Historical Background:} 
        \begin{itemize} 
            \item Two hundred years ago / bare fists. 
            \item Prizefighters: crude, no rules, serious injuries. 
            \item 1860: Marquis of Queensberry / first set of rules / gloves. 
        \end{itemize}

        \item \textbf{Mendoza’s Significance:}
        \begin{itemize}
            \item Changed crude fighting into a \textbf{sport}.
            \item Brought \textbf{science} to the game.
            \item Adored by rich and poor alike (tremendous popularity).
        \end{itemize}

        \item \textbf{The Rivalry with Humphries:}
        \begin{itemize}
            \item Rose to fame at 14 / Richard Humphries (eminent boxer).
            \item Quarrelled bitterly / settled by a fight.
            \item Defeated twice $\rightarrow$ 1790 (3rd match) $\rightarrow$ Champion of England.
        \end{itemize}

        \item \textbf{Success and Downfall:}
        \begin{itemize}
            \item Founded an Academy / Lord Byron (famous pupil).
            \item Enormous sums of money / extravagant / always in debt.
            \item Defeated by Gentleman Jackson $\rightarrow$ forgotten.
            \item Prison / died in poverty (1836).
        \end{itemize}
    \end{enumerate}
\end{multicols}

\grammarpoints

\wsitem{bitterly}
\begin{multicols}{1}
    bitterly 是修饰 quarrel 最经典、最地道的副词之一。当它们组合在一起时,不仅仅是说“吵架了”,而是强调这场争吵\textbf{“充满了怨恨、痛苦且极难度过”}。

    \begin{enumerate}
        \item \textbf{语义逻辑:为什么用 Bitterly? (Semantic Logic)}
        \begin{itemize}
            \item \textbf{Bitter (苦涩/怨恨) 的引申义}
            
            Bitterly 源自 bitter。在描述争吵时,它不是指味道,而是形容一种\textbf{“像苦胆一样”}的情绪。
            
            它暗示这场争吵伤害了彼此的感情,留下了长久的裂痕,而不仅仅是一时的口角。
            
            \item \textbf{程度的极点}
            
            它比 angrily (愤怒地) 更深刻。愤怒可能很快平息,但 bitterly 形容的冲突通常带有\textbf{敌意和不容妥协}的色彩。
        \end{itemize}
        \item \textbf{语法位置与用法 (Syntax and Usage)}
        \begin{itemize}
            \item \textbf{作状语修饰动词 (Verb + Adverb)}
            
            \es{ They \textbf{quarrelled bitterly over the \textbf{set of rules}.}(他们为了那套规则激烈地/痛苦地吵了一架。)}
            
            \item \textbf{作定语修饰名词 (Adjective + Noun)}
            
            如果把 quarrel 当作名词,则要用形容词形式 bitter:
            
            \es{ They had a \textbf{bitter quarrel that \textbf{settled} nothing.}(他们进行了一场毫无结果的激烈争吵。)}
        \end{itemize}
        \item \textbf{Bitterly 的其他“灵魂搭配” (Common Collocations with Bitterly)}
        
        在英语中,bitterly 专门用来修饰那些带有一种“刺痛感”或“极端性”的词汇:

        \begin{itemize}
            \item \textbf{描写天气:Bitterly cold} (严寒/刺骨的冷)
            
            \es{ It was \textbf{bitterly cold when he crossed the \textbf{Channel}.} (他飞越海峡时,天气冷得刺骨。)}
            
            \item \textbf{描写情绪:Bitterly disappointed} (极其失望)
            
            这种失望带有“心寒”的感觉。
            
            \item \textbf{描写动作:Weep bitterly} (痛哭流涕)
            
            这种哭泣是发自内心深处的悲痛。
        \end{itemize}
    \end{enumerate}
\end{multicols}

\wsitem{Eminent}
\begin{multicols}{1}
    在描述丹尼尔·门多萨(Daniel Mendoza)及其对手时,Eminent 是一个非常关键的词。它不仅仅表示“出名”,更侧重于一种\textbf{“行业内的权威性”和“社会地位的显赫”}。

    \begin{enumerate}
        \item \textbf{词义本质:卓越与显赫 (The Essence of Eminence)}
        \begin{itemize}
            \item \textbf{核心定义 (Core Definition)}
            \begin{itemize}
                \item \textbf{含义:} 卓越的、杰出的、显赫的。
                \item \textbf{侧重点:} 指某人在特定的专业领域(如科学、法律、体育)中处于顶尖地位,并因其成就而受到广泛尊重。
            \end{itemize}
            
            \item \textbf{词根探源 (Etymology)}
            \begin{itemize}
                \item 源自拉丁语 eminentem,原意为“高出、突出”。
                \item \textbf{逻辑:} 想象一群人站在一起,那个“高出一头”的人就是 eminent。这是一种\textbf{“出类拔萃”}的画面感。
            \end{itemize}
        \end{itemize}
        \item \textbf{词汇辨析:Eminent vs. Famous vs. Notorious}
        
        为了精准使用,我们需要区分这几个常被混淆的“名声”词汇:

        \begin{itemize}
            \item \textbf{Famous (最通用)}
            \begin{itemize}
                \item 仅仅表示“名气大”,家喻户晓。不管是好名声还是坏名声,只要知道的人多,就是 famous。
            \end{itemize}
            
            \item \textbf{Eminent (正向且专业)}
            \begin{itemize}
                \item 必须是\textbf{正面的}评价。它是“名声”与“专业造诣”的结合。一个 eminent 的人往往是该领域的领军人物。
                \item \textbf{课文语境:} Humphries 被称为 the most eminent boxer,说明他当时不仅有名,而且拳技被公认为全英第一。
            \end{itemize}
            
            \item \textbf{Notorious (臭名昭著)}
            \begin{itemize}
                \item 侧重于因为负面的、不道德的行为而闻名(如:a notorious kidnapper)。
            \end{itemize}
        \end{itemize}

        \item \textbf{常见搭配与句法位置 (Common Collocations)}
        \begin{itemize}
            \item \textbf{常用搭配领域}
            \begin{itemize}
                \item An \textbf{eminent scholar/scientist} (卓越的学者/科学家)
                \item An \textbf{eminent judge/lawyer} (显赫的法官/律师)
                \item An \textbf{eminent figure in history} (历史上的杰出人物)
            \end{itemize}
            
            \item \textbf{派生词:Eminence (名词)}
            \begin{itemize}
                \item Reach \textbf{eminence in one's field} (在某人的领域达到卓越地位)。
            \end{itemize}
        \end{itemize}
    \end{enumerate}
\end{multicols}

\wsitem{Occasion}
\begin{multicols}{1}
    在描述时间或事件时,Occasion 是一个比 "time" 更具重量感、正式感和背景深度的词。

    \begin{enumerate}
        \item \textbf{语义本质:不只是“时间” (More than just "Time")}
        \begin{itemize}
            \item \textbf{特定的重大事件 (A Specific Event)}
            
            Time 是一个宽泛的时间概念,而 Occasion 指的是\textbf{发生某件重要事情的特定时刻}。
            
            在 Mendoza 的语境中,两人见面不是偶遇,而是一场经过策划、有观众、有规则的正式拳赛。使用 occasion 强调了这次对决的\textbf{正式性}。
            
            \item \textbf{场合与仪式感 (Atmosphere and Setting)}
            
            它暗示了这次见面伴随着特定的环境。它让读者感受到一种“历史性时刻”的张力。
                
            \es{例句: The wedding was a \textbf{grand occasion.} (婚礼是一个盛大的场合。)}
        \end{itemize}
        \item \textbf{搭配与句法逻辑 (Collocations and Syntax)}
        \begin{itemize}
            \item \textbf{On a later occasion (在后来的某个场合)}
            \begin{itemize}
                \item 这是英语中非常地道的\textbf{时间状语表达}。
                \item \textbf{对比:} 
                \begin{itemize}
                    \item Later (后来——太简单,缺乏叙事感)。
                    \item In a later fight (在后来的比赛中——太直白)。
                    \item \textbf{On a later occasion} (在后来的某个场合——优雅、含蓄,且能概括比赛的一切准备工作)。
                \end{itemize}
            \end{itemize}
            
            \item \textbf{介词搭配:On}
            \begin{itemize}
                \item 注意:形容“在某个场合”时,介词通常要用 \textbf{on}。
                \item \textbf{常用短语:} On this occasion (在这个场合), On several occasions (好几次)。
            \end{itemize}
        \end{itemize}
        \item \textbf{深度辨析:Occasion vs. Opportunity}
        
        这两个词在翻译成中文时都有“机会”的意思,但用法迥异:

        \begin{itemize}
            \item \textbf{Occasion (契机/时刻)}
            \begin{itemize}
                \item 侧重于“事件发生的时间或理由”。
                \item 例句: I have \textbf{no occasion to see him.} (我没有理由/机会见他。)
            \end{itemize}
            
            \item \textbf{Opportunity (机遇/好机会)}
            \begin{itemize}
                \item 侧重于“有利于成功的时机”。
                
                \es{ 例句: This is a \textbf{great opportunity for your career.} (这是你职业生涯的大好机遇。)}
            \end{itemize}
        \end{itemize}
        \item \textbf{实战模仿:如何使用 "Occasion"}
        
        你可以用它来替换掉平淡的 "time",提升文章的格调:

        \begin{itemize}
            \item \textbf{描述重要会面:}
            
            \es{I have met the \textbf{eminent professor \textbf{on several occasions}.}(我曾在好几个场合见过那位卓越的教授。)}
            
            \item \textbf{描述人生转折:}
            
            \es{The completion of his \textbf{life work was a \textbf{great occasion} \textbf{regarded as} a milestone.}(他毕生心血的完成是一个重大的时刻,被视为一个里程碑。)}
        \end{itemize}
    \end{enumerate}
\end{multicols}

\wsitem{Quarrel vs. Argue}
\begin{multicols}{1}
    这两个词都表示“争吵、争论”,但在激烈程度、目的性以及表现形式上有很明显的区别。简单来说,Argue 往往是为了“道理”,而 Quarrel 往往是为了“发泄情绪”。

    \begin{enumerate}
        \item \textbf{核心动机对比 (Core Motivation)}
        \begin{itemize}
            \item \textbf{Argue (争论/辩论)}
            \begin{itemize}
                \item \textbf{侧重点:} 强调通过引用事实、理由或论点来\textbf{说服别人},或者证明某事是正确/错误的。
                \item \textbf{性质:} 可以是理性的、有逻辑的,甚至可以是建设性的(如学术争论)。
            \end{itemize}
            
            \item \textbf{Quarrel (争吵/吵架)}
            \begin{itemize}
                \item \textbf{侧重点:} 强调由于\textbf{愤怒或不满}而产生的言语冲突。
                \item \textbf{性质:} 通常是情绪化的、私人的。它往往涉及到个人感情的伤害,而非逻辑的博弈。
            \end{itemize}
        \end{itemize}
        \item \textbf{表现形式与结果 (Form and Result)}
        \begin{itemize}
            \item \textbf{表现形式}
            \begin{itemize}
                \item Argue 可能只是在讨论一个问题(如:argue about politics)。
                \item Quarrel 通常包含大声叫嚷、指责或冷战(如:quarrel with a spouse)。
            \end{itemize}
            
            \item \textbf{结果}
            \begin{itemize}
                \item Argue 的结果可能是达成共识或决定。
                \item Quarrel 的结果往往是\textbf{关系破裂}或暂时不说话。
            \end{itemize}
        \end{itemize}
        \item \textbf{常见搭配与介词 (Collocations and Prepositions)}
        \begin{itemize}
            \item \textbf{Argue}
            \begin{itemize}
                \item Argue \textbf{with sb. \textbf{about} sth.} (与某人争论某事)
                \item Argue \textbf{for/against sth.} (据理力争赞成/反对某事)
            \end{itemize}
            
            \item \textbf{Quarrel}
            \begin{itemize}
                \item Quarrel \textbf{with sb. \textbf{over} sth.} (因某事与某人吵架)
                \item Have a quarrel \textbf{with sb.} (与某人吵了一架)
            \end{itemize}
        \end{itemize}
    \end{enumerate}
\end{multicols}

\wsitem{Injure vs Damage}
\begin{multicols}{1}
    这两个词都表示“损害”,但它们最大的区别在于受损的对象:Injure 针对的是“有生命的人或动物”,而 Damage 针对的是“无生命的物品或抽象事物”。

    \begin{enumerate}
        \item \textbf{核心受众对比 (Target Audience)}
        \begin{itemize}
            \item \textbf{Injure (伤害/使受伤)}
            \begin{itemize}
                \item \textbf{受体:} 人或动物(Living beings)。
                \item \textbf{含义:} 指身体上的伤害(如骨折、流血)或名誉、自尊心上的损害。
                \item \textbf{常用场景:} 事故、比赛、战争。
            \end{itemize}
            
            \item \textbf{Damage (损坏/破坏)}
            \begin{itemize}
                \item \textbf{受体:} 物品(Objects)或抽象事物(Abstract things)。
                \item \textbf{含义:} 指价值、功能或外表的降低或毁坏。
                \item \textbf{常用场景:} 自然灾害、火灾、名誉受损。
            \end{itemize}
        \end{itemize}
        \item \textbf{典型例句与语境 (Contextual Examples)}
        \begin{itemize}
            \item \textbf{物理层面的区分}
            \begin{itemize}
                \item "Two people were \textbf{injured in the boxing match."} (两名拳击手在比赛中受伤了。) —— \textbf{不能用 damaged}。
                \item "The storm \textbf{damaged several houses."} (风暴损坏了几栋房子。) —— \textbf{不能用 injured}。
            \end{itemize}
            
            \item \textbf{抽象层面的细微差别}
            \begin{itemize}
                \item "The scandal \textbf{injured his pride."} (丑闻伤害了他的自尊。) —— 侧重于内心的痛苦。
                \item "The scandal \textbf{damaged his reputation."} (丑闻破坏了他的名声。) —— 侧重于名声作为一种“资产”的贬值。
            \end{itemize}
        \end{itemize}
        \item \textbf{词形变化与名词用法 (Word Forms)}
        \begin{itemize}
            \item \textbf{Injury (名词:伤口/伤害)}
            
            \es{ "He had to retire due to a serious leg \textbf{injury."} (他因严重的腿伤不得不退役。)}
            
            \item \textbf{Damage (名词:损坏/赔偿金)}
            
            \es{"The fire caused a lot of \textbf{damage."} (大火造成了巨大的损失。)}
            
            \textbf{注意:} Damages (复数形式) 在法律上特指“赔偿金”。
        \end{itemize}
    \end{enumerate}
\end{multicols}

\wsitem{Injure, Hurt, Wound, Harm}
\begin{multicols}{1}
    在描述“伤害”或“疼痛”时,英语中有几个非常相近的词:Injure, Hurt, Wound, Harm。它们在严重程度、起因以及是“身体痛”还是“心理痛”上各有侧重。

    \begin{enumerate}
        \item \textbf{核心差异对比 (Core Differences)}
        \begin{itemize}
            \item \textbf{Hurt (最通用、侧重“疼痛感”)}
            
            \textbf{特点:} 既可以指身体上的疼,也可以指内心的伤心。它往往强调的是“感觉”,而不是伤口本身。
            
            \es{ \textbf{用法:} My leg hurts. (我腿疼。) / You hurt my feelings. (你伤了我的心。)}
            
            \item \textbf{Injure (较正式、侧重“功能受损”)}
            
            \textbf{特点:} 常用于事故或运动中,指身体部位受到损害(如骨折、拉伤)。
            
            \es{ \textbf{用法:} He was \textbf{injured in a car accident.} (他在车祸中受伤了。)}
            
            \item \textbf{Wound (侧重“外部创伤”)}
            
            \textbf{特点:} 特指由武器(刀、枪)造成的出血、开放性伤口。常用于战场或暴力事件。
            
            \es{The soldier was \textbf{wounded in the arm.} (士兵手臂受了枪伤。)}
            
            \item \textbf{Harm (侧重“负面影响”)}
            
            \textbf{特点:} 强调造成的“坏处”或“危害”。通常不指具体的流血伤口,而指对健康、环境或名誉的损害。
            
            \es{用法: Smoking will \textbf{harm your health.} (吸烟会危害健康。)}
        \end{itemize}
        \item \textbf{场景化应用 (Scenario Mapping)}
        
        我们来看看在同一个拳击场上,这些词怎么用:
        \begin{itemize}
            
            \item Mendoza's eye \textbf{hurts.} (门多萨眼睛疼。——强调感觉)
            \item He \textbf{injured his shoulder during the \textbf{fight}.} (他在比赛中肩膀脱臼/受伤了。——强调功能受损)
            \item The opponent was \textbf{wounded by a sharp object.} (对手被利器割伤出血了。——强调器械创伤)
            \item Losing the match did \textbf{harm to his reputation.} (输掉比赛损害了他的名誉。——强调抽象的坏处)
        \end{itemize}
        \item \textbf{语法小贴士 (Grammar Tips)}
        \begin{itemize}
            \item \textbf{Hurt 的特殊性:} Hurt 的过去式和过去分词还是 hurt。而且它可以是不及物动词(My head hurts)。
            \item \textbf{被动语态:} Injure 和 Wound 最常用于被动语态(be injured / be wounded)。
        \end{itemize}
    \end{enumerate}
\end{multicols}

\wsitem{Beat vs. Defeat}
\begin{multicols}{1}
    这两个词都翻译为“打败”,但在句法结构上有一个绝对不能错的硬性规定。
    \begin{enumerate}
        \item \textbf{句法结构的“唯一真理” (Syntactic Rule)}
        
        这是中国学生最容易出错的地方。虽然它们意思相近,但在接宾语时逻辑完全一致:后面接的都是“被击败的那个人/那支队”,而不是“比赛”。
        
        \begin{itemize}
            \item \textbf{共同点:[Subject] + beat/defeat + [Opponent]}
            
            \es{\textbf{正确:} Mendoza \textbf{beat/defeated Humphries.} (门多萨打败了汉弗莱斯。)}
            
            \es{\textbf{错误:} He beat the game. (在正式英语中,我们不用 beat 来“赢得”比赛。)}
            
            \item \textbf{区分 Win}
            
            \textbf{Win} 后面接的是\textbf{奖项或比赛}(Win a race/game/prize)。
        \end{itemize}
        \item \textbf{语感与程度的细微差别 (Subtle Nuances)}
        \begin{itemize}
            \item \textbf{Beat (口语化、动作感强)}
            
            \textbf{语感:} 比较非正式,听起来更具活力,有时暗示一种轻松的胜利。
            
            \textbf{词形:} Beat (原形) - Beat (过去式) - Beaten (过去分词)。注意过去式不加 -ed。
            
            \item \textbf{Defeat (正式、彻底性强)}
            
            \textbf{语感:} 更加书面化,常用于历史记录、政治选举或军事战争。
            
            \textbf{含义:} 往往暗示一种\textbf{彻底的、决定性}的胜利。
        \end{itemize}
        \item \textbf{词性扩展:名词用法 (Noun Forms)}
        \begin{itemize}
            \item \textbf{Defeat 可作为名词}
            
            \es{It was a \textbf{crushing defeat for the team.} (对这支队伍来说,这是一次惨重的失败。)}
            
            \textbf{注意:} Beat 作名词时通常指“心跳”或“节奏”,不指“失败”。
            
            \item \textbf{常见表达}
            \begin{itemize}
                \item \textbf{Admit defeat:} 认输。
                \item \textbf{Suffer a defeat:} 遭受失败。
            \end{itemize}
        \end{itemize}
    \end{enumerate}
\end{multicols}

\wsitem{Be known as vs. Be regarded as}
\begin{multicols}{1}
    这两个短语都和“看作、称为”有关,但在语义重心、主客观程度以及侧重点上有着明显的区别。

    \begin{enumerate}
        \item \textbf{语义重心对比 (Semantic Focus)}
        \begin{itemize}
            \item \textbf{Be known as (以……身份/名称而闻名)}
            \begin{itemize}
                \item \textbf{侧重点:} 强调“名称”、“头衔”或“公众公认的身份”。
                \item \textbf{性质:} 比较客观,通常指一个事实上的标签或大家熟知的绰号。
                \item \textbf{例句:} He is \textbf{known as the "Father of Aviation".} (他被誉为“航空之父”。——这是一个广为人知的头衔。)
            \end{itemize}
            
            \item \textbf{Be regarded as (被看作/被视为……)}
            \begin{itemize}
                \item \textbf{侧重点:} 强调“评价”、“观点”或“心理上的认同”。
                \item \textbf{性质:} 带有一定的主观性。它暗示经过人们的思考、衡量后得出的看法。
                \item \textbf{例句:} She is \textbf{regarded as the most talented artist of her time.} (她被认为是那个时代最有才华的艺术家。——这是一种评价,可能有人不这么认为。)
            \end{itemize}
        \end{itemize}
        \item \textbf{逻辑属性差异 (Logical Attributes)}
        \begin{itemize}
            \item \textbf{身份 vs. 地位}
            \begin{itemize}
                \item Be known as 往往连接的是一个特定的**身份名片**。
                \item Be regarded as 往往连接的是一个**社会地位或品质评价**。
            \end{itemize}
            
            \item \textbf{固定性 vs. 变化性}
            \begin{itemize}
                \item 一个人的 known as 身份通常很稳定(如:纽约被 known as "The Big Apple")。
                \item 一个人的 regarded as 评价可能会随着时间或人们的标准而改变。
            \end{itemize}
        \end{itemize}
        \item \textbf{常见搭配与近义词 (Collocations and Synonyms)}
        \begin{itemize}
            \item \textbf{与 Be known as 相似:}
            \begin{itemize}
                \item Be called... (被称为……)
                \item Be titled... (被冠以……的头衔)
            \end{itemize}
            
            \item \textbf{与 Be regarded as 相似:}
            \begin{itemize}
                \item Be considered as/to be... (被认为……)
                \item Be viewed as... (被看作……)
                \item Be thought of as... (被当作……)
            \end{itemize}
        \end{itemize}
    \end{enumerate}
\end{multicols}

\wsitem{Die in poverty}
\begin{multicols}{1}
    "Die in poverty" 是一个极具叙事冲击力的短语,常用来描述历史人物、艺术家或运动员悲剧性的晚年结局。它描绘了一个曾经辉煌的人在极度贫困中离世的强烈反差。

    \begin{enumerate}
        \item \textbf{核心含义与情感色彩 (Core Meaning and Emotional Tone)}
        \begin{itemize}
            \item \textbf{极度的贫困 (Extreme Destitution)}
            
            Poverty 是一个抽象名词。在这个短语中,它不只是指“没钱”,通常暗示一种\textbf{赤贫、无依无靠}的状态。

            \textbf{逻辑:} 与 die in peace (安详去世) 或 die in wealth (在富有中去世) 形成对比。
            
            \item \textbf{叙事中的讽刺与同情 (Irony and Pathos)}
            
            当这个短语修饰一个曾 enjoyed tremendous popularity (享有极大名望) 的人时,会产生巨大的情感张力。
        \end{itemize}
        \item \textbf{语法结构解析 (Syntactic Analysis)}
        \begin{itemize}
            \item \textbf{介词短语作状语 (Prepositional Phrase as Adverbial)}
            
            In poverty 作为状语,修饰不及物动词 die,说明去世时的\textbf{环境或状态}。
            
            \item \textbf{零冠词用法 (Zero Article)}
            
            就像 in debt 一样,这里的 poverty 是抽象概念,不需要冠词。
        \end{itemize}
        \item \textbf{词汇关联与对比 (Lexical Associations)}
        \begin{itemize}
            \item \textbf{近义表达 (Similar Expressions)}
            \begin{itemize}
                \item Die penniless: 死时分文不剩(更加强调具体的金钱丧失)。
                \item Die a pauper: 作为赤贫者死去(pauper 指极度贫困的人)。
            \end{itemize}
            
            \item \textbf{常见的前置条件}
            
            一个人之所以 die in poverty,往往是因为他早年 fell heavily into debt (陷入重债) 或 lost his popularity (失去了名气)。
        \end{itemize}
    \end{enumerate}
\end{multicols}

\wsitem{The first set of rules}
\begin{multicols}{1}
    \begin{enumerate}
        \item \textbf{Set (套/组)}
        
        在这里表示一系列相互关联、构成完整体系的事物。
        
        \textbf{逻辑:} 拳击规则不是孤立的一条,而是包含回合时间、手套使用、击倒判定等一系列规定的“合集”。
        
        \item \textbf{Rules (规则/章程)}
        
        特指《昆斯伯里侯爵规则》 (The Marquess of Queensberry Rules)。
        
        \textbf{历史背景:} 在这套规则出现之前,拳击主要是裸拳格斗 (bare-knuckle fighting),非常野蛮且缺乏统一标准。
        
        \item \textbf{The first (第一)}
        
        强调这是\textbf{现代拳击运动的开端}。虽然以前也有非正式的约定,但这是第一部被广泛公认并引入“必须佩戴手套”规定的官方典章。
    \end{enumerate}
\end{multicols}

\wsitem{Quick to learn}
\begin{multicols}{1}
    \textbf{Quick to do sth. (敏于做某事):} 
    \begin{itemize}
        \item 这是一个非常地道的表达,用来形容某人在某方面有天赋、反应快。
        \item \textbf{对比:} 
        \begin{itemize}
            \item He learned quickly. (他学得很快。——侧重速度。)
            \item He was quick to learn. (他很聪明,一学就会。——侧重人的素质和天赋。)
        \end{itemize}
    \end{itemize}
\end{multicols}

\wsitem{Rise to fame}
\begin{multicols}{1}
    "Rise to fame" 是一个非常生动且富有画面感的短语,它将“成名”的过程比作像太阳升起或星辰运行一样的向上运动。

    \begin{enumerate}
        \item \textbf{动态的成功 (The Dynamics of Success)}
        
        \begin{itemize}
            \item \textbf{向上运动的隐喻 (The Upward Metaphor)}
            \begin{itemize}
                \item \textbf{Rise:} 暗示了一个从“无名小卒”(底层)到“大众偶像”(顶层)的过程。它比单纯的 become famous 更有叙事张力,强调了地位的攀升。
                \item \textbf{Fame:} 指广泛的公众认可和名声。
            \end{itemize}
            
            \item \textbf{过程的性质}
            \begin{itemize}
                \item 这个短语通常用来描述一个**持续的、有迹可循**的成功过程。
                \item \textbf{常见副词搭配:} Rise \textbf{swiftly to fame} (迅速成名), Rise \textbf{gradually to fame} (逐渐成名)。
            \end{itemize}
        \end{itemize}
        \item \textbf{词汇对比:Rise vs. Gain vs. Achieve}
        \begin{itemize}
            \item \textbf{Rise to fame (侧重“地位”的改变)}
            
            强调你所处的**阶层或能见度**发生了变化。

            \es{Example: Mendoza \textbf{rose to fame} when he was only 14.}
            
            \item \textbf{Gain fame (侧重“获得”的过程)}
            
            像赚钱或获得知识一样,强调通过某种行为“得到了”名声。
            
            \item \textbf{3. Achieve fame (侧重“目标”的达成)}
            
            强调经过长期的努力,最终“实现了”成名的目标,带有成就感。
        \end{itemize}
        \item \textbf{常见搭配与变化 (Collocations and Variations)}
        \begin{itemize}
            \item \textbf{Rise to prominence / Rise to power}
            
            分别表示“崭露头角”或“走上权力巅峰”。逻辑与 rise to fame 一致。
            
            \item \textbf{2. Find fame}
            
            侧重于“偶然发现”或“终于寻得”名声。
            
            \item \textbf{3. Sudden rise to fame}
            
            一夜成名。

        \end{itemize}
    \end{enumerate}
\end{multicols}

\wsitem{Settle an argument}
\begin{multicols}{1}
    是的,settle an argument 是最标准、最地道的搭配。
    
    在英语中,Settle 的核心含义是“终结一种不稳定的状态”,所以当它和 Argument(争论/争端)连用时,意味着双方通过某种方式(如谈判、证据、甚至是决斗)给出了一个最终的定论,使争吵不再继续。

    \begin{enumerate}
        \item \textbf{语义逻辑:为什么用 Settle? (Semantic Logic)}
        
        \begin{itemize}
            \item \textbf{“尘埃落定”的终结感}
            \begin{itemize}
                \item Settle 暗示争论已经到了一个必须解决的关头。
                \item \textbf{课文语境:} The argument could \textbf{only be \textbf{settled} by a fight.} (这场争端只能通过一场决斗来解决。) —— 这里强调“拳头”是唯一的最终裁决方式。
            \end{itemize}
            
            \item \textbf{解决冲突的不同动词对比}
            \begin{itemize}
                \item \textbf{Solve a problem:} 解决一个数学题或技术困难(找答案)。
                \item \textbf{Resolve a conflict:} 解决冲突(更正式,侧重消除分歧)。
                \item \textbf{Settle an argument:} 平息争论(侧重于达成最后的结果,让大家都闭嘴)。
            \end{itemize}
        \end{itemize}
        \item \textbf{常见的 Settle 搭配 (Common Phrases with Settle)}
        
        除了平息争论,Settle 经常用于以下需要“清算”或“定居”的场景:

        \begin{itemize}
            \item \textbf{Settle a debt (清偿债务)}
            
            \es{ He worked hard to \textbf{settle his \textbf{debts} before he \textbf{died in poverty}.}(他在贫困去世前,努力还清了他的债务。)}
            
            \item \textbf{Settle a score (报仇/算账)}
            
            指报复以前伤害过你的人。
            
            \item \textbf{Settle for sth. (委曲求全/满足于)}
            
            指在拿不到最好的东西时,勉强接受次好的。
        \end{itemize}

        \item \textbf{在描述“解决争端”时,经常使用被动语态来强调“结果”:}
        \begin{itemize}
            \item \textbf{Structure: [The argument] + be settled + by + [Method]}
            
            \es{The \textbf{bitter quarrel was finally \textbf{settled} by the \textbf{eminent} judge.} (那场激烈的争吵最终由那位卓越的法官平息了。)}
        \end{itemize}
    \end{enumerate}
\end{multicols}

\wsitem{Attract attention}
\begin{multicols}{1}
    "Attract attention" 是一个非常实用的搭配,用来描述某人或某事因为独特、出色或反常而\textbf{“吸引了大众的目光”}。在 Mendoza 的故事中,他因为卓越的拳击技巧和成功而引起了Richard Humphries的关注。

    \begin{enumerate}
        \item \textbf{语义本质:磁铁般的吸引 (The Magnetic Pull)}
        
        \begin{itemize}
            \item \textbf{核心定义 (Core Definition)}
            \begin{itemize}
                \item \textbf{Attract:} 意为“吸引”,像磁铁吸住铁屑一样,强调一种\textbf{拉力}。
                \item \textbf{Attention:} 意为“注意力”或“关注”。
                \item \textbf{综合:} 使人们开始注意到某人的存在或某事的发生。
            \end{itemize}
            
            \item \textbf{为什么用 Attract?}
            
            相比于 get attention(得到关注),attract 更有\textbf{主动性和魅力感},暗示是由于主体的某些特质(如美貌、才华、奇特)自发产生的吸引力。
        \end{itemize}
        \item \textbf{程度修饰与常见搭配 (Degrees and Collocations)}
        
        你可以通过添加形容词来精确描述这种“吸引力”的强度:

        \begin{itemize}
            \item \textbf{程度修饰 (Intensity)}
            \begin{itemize}
                \item Attract \textbf{considerable attention:} 吸引了相当大的关注(稳重、正式)。
                \item Attract \textbf{tremendous attention:} 吸引了巨大的关注(强调轰动性)。
                \item Attract \textbf{wide attention:} 引起了广泛关注。
            \end{itemize}
            
            \item \textbf{结果导向 (Outcome)}
            
            Attract attention \textbf{to sth.:} 将注意力吸引到某事上。
        \end{itemize}
        \item \textbf{词汇辨析:Attract vs. Catch vs. Draw}
        \begin{itemize}
            \item \textbf{Catch one's attention (瞬间捕捉)}
            
            侧重于\textbf{瞬间、偶然}的行为。比如路过橱窗时,某个亮闪闪的东西“抓住了”你的眼球。
            
            \item \textbf{Draw attention to (刻意引导)}
            
            往往带有特定的\textbf{目的性}。比如老师敲黑板是为了“引导学生注意”某个知识点。
            
            \item \textbf{Attract attention (持续吸引)}
            
            侧重于\textbf{特质带来的持续关注}。Mendoza 的成功是持续的,所以他 attracted tremendous attention。
        \end{itemize}
    \end{enumerate}
\end{multicols}

\wsitem{Pupil vs. Student}
\begin{multicols}{1}
    这是一个非常细致的问题。虽然两者在中文里都译为“学生”,但在英语的语境中,它们在年龄、关系亲疏以及专业程度上有着明显的界限。
    
    在 Mendoza 的故事里,作者提到 Lord Byron(拜伦勋爵)是他的 pupil,这绝非随手乱用,而是精准地捕捉了那种“师徒式”的紧密关系。

    \begin{enumerate}
        \item \textbf{核心定义与年龄差异 (Age & Definition)}
        \begin{itemize}
            \item \textbf{Pupil (小学生/受监护者)}
            \begin{itemize}
                \item \textbf{对象:} 通常指\textbf{小学生}或 12 岁以下的孩子。
                \item \textbf{内涵:} 强调在老师的直接监督和保护下学习。
            \end{itemize}
            
            \item \textbf{Student (学生/研究者)}
            \begin{itemize}
                \item \textbf{对象:} 通常指\textbf{中学生或大学生}。
                \item \textbf{内涵:} 强调自主学习和对某个学科(subject)进行研究。
            \end{itemize}
        \end{itemize}
        \item \textbf{师生关系的性质 (Nature of Relationship)}
        
        这是理解 Mendoza 文章的关键。即使拜伦勋爵当时已经成年,作者依然称他为 pupil:

        \begin{itemize}
            \item \textbf{Pupil (师徒/私塾关系)}
            \begin{itemize}
                \item Pupil 往往暗示一种\textbf{一对一}或\textbf{私人的}指导关系。
                \item 在艺术、武术或技艺传授(如 Mendoza 的 Academy)中,老师不仅教知识,还塑造学生的行为。
                \item \textbf{语感:} “门生”或“徒弟”。
            \end{itemize}
            
            \item \textbf{Student (机构/职业关系)}
            \begin{itemize}
                \item Student 往往指在一个大的教育机构(School/University)中的一员。
                \item 师生之间通常只是单纯的教学关系,没有那么深的人格依附。
            \end{itemize}
        \end{itemize}
        \item \textbf{词汇关联:为什么拜伦是 Pupil?}
        \begin{itemize}
            \item \textbf{The Narrative Logic:}
            \begin{itemize}
                \item Mendoza 创办的是 \textbf{Academy}(专业学院),他教授的是 \textbf{scientific boxing}(科学拳击)。
                \item 拜伦勋爵并不是去“上大课”,而是接受 Mendoza 的\textbf{亲自传授}。
                \item 因此,用 \textbf{pupil} 凸显了 Mendoza 作为“一代宗师”的地位,同时也体现了拜伦对他技艺的传承。
            \end{itemize}
        \end{itemize}
    \end{enumerate}
\end{multicols}

\wsitem{It is/was not until ... that ...}
\begin{multicols}{1}
    \begin{enumerate}
        \item \textbf{这是英语中最常用的强调句型之一,用来强调时间。}
        \begin{itemize}
            \item \textbf{句型结构拆解 (Sentence Structure)}
            \begin{itemize}
                \item \textbf{强调成分:} not until his third match in 1790 (直到1790年的第三场比赛)。
                \item \textbf{被强调部分:} 原本的逻辑是 "He did not beat Humphries until his third match"。
                \item \textbf{that 引导的部分:} 后面接句子的其余部分,注意这里要用**正常语序**(不需要倒装)。
            \end{itemize}
            
            \item \textbf{语义重心 (Semantic Focus)}
            \begin{itemize}
                \item 该句型强调的是“**时间的迟到性**”或“**过程的艰难**”。
                \item 翻译时通常处理为:“直到……才……”。它暗示了在第三场比赛之前,他经历了失败或漫长的等待。
            \end{itemize}
        \end{itemize}

        \item \textbf{句法还原对比 (Transformation)}
        
        为了更好地理解这个强调句,我们可以看看它的不同变形:

        \begin{itemize}
            \item \textbf{普通陈述句 (Neutral):}
            \begin{itemize}
                \es{ He \textbf{did not} beat Humphries \textbf{until} his third match in 1790.}
                \item (语气平淡,只是叙述事实。)
            \end{itemize}
            
            \item \textbf{倒装句 (Inversion - 更加正式):}
            
            \es{\textbf{Not until} his third match in 1790 \textbf{did he} beat Humphries.}

            (这种写法通过将 Not until 提前,并对主句进行部分倒装,来增加戏剧性。)
            
            \item \textbf{本文强调句 (Emphasis - 最具叙事张力):}
            
            \es{\textbf{It was not until} ... \textbf{that} ...}

            (这是《新概念英语》中常见的文学化表达,能突出结果的来之不易。)
        \end{itemize}

        \item \textbf{其他值得注意的表达 (Other Noteworthy Expressions)}
        
        \begin{itemize}
            \item \textbf{Finally (终于):}
            
            与 not until 呼应,进一步强化了获胜的艰难。
            
            \item \textbf{Become Champion of England (成为英格兰冠军):}
            
            这里的 Champion 前没有冠词,在表示“独一无二的头衔或职位”时,冠词常被省略。
        \end{itemize}
    \end{enumerate}

\end{multicols}

\wsitem{... is not ... until ... , when ...}
\begin{multicols}{1}
    这是一个非常精妙且高级的复合时间强调句型。它将“直到……才”的否定结构与“就在那时”的定语从句结合在一起,常用于描述历史转折点或关键性的时间节点。

    \begin{enumerate}
        \item \textbf{句型结构拆解 (Structural Breakdown)}
        \begin{itemize}
            \item \textbf{第一层:Not... until... (否定直到...才)}
            \begin{itemize}
                \item \textbf{逻辑:} 某事(A)在某个时间点(B)之前一直没有发生。它强调了漫长的等待或延迟。
                \item \textbf{成分:} Subject + is not + [done] + until + [Time/Year].
            \end{itemize}
            
            \item \textbf{第二层:..., when... (非限制性定语从句)}
            \begin{itemize}
                \item \textbf{逻辑:} 对前文提到的具体时间点进行补充说明,交代在那一刻发生了什么重大的事情。
                \item \textbf{成分:} ..., when + [Subject] + [Verb].
            \end{itemize}
        \end{itemize}
        \item \textbf{课文实例解析 (Textual Analysis)}
        
        以你读到的句子为例:

        \es{The use of gloves was not introduced until 1860, when the Marquis of Queensberry drew up the first set of rules.}

        \begin{itemize}
            \item \textbf{前半部分:} 在 1860 年之前,拳击手们不用手套(强调了很长一段时间的裸拳格斗)。
            \item \textbf{后半部分:} 就在 1860 年那一年,昆斯伯里侯爵制定了规则(解释了为什么在那一年发生了改变)。
            \item \textbf{综合语感:} 这种写法比单纯用 because 更具**历史厚重感**,它把“时间的推移”和“事件的发生”紧密扣合在了一起。
        \end{itemize}
        \item \textbf{为什么不直接用 "It was not until... that..."?}
        
        虽然意思相近,但侧重点不同:

        \begin{itemize}
            \item \textbf{It was not until 1860 that gloves were used.} 
            
            这仅仅是强调“时间”。
            
            \item \textbf{...not introduced until 1860, when...}
            
            这更像是在讲故事。它不仅给出了时间,还提供了一个\textbf{背景舞台},让读者明白这个时间点之所以重要,是因为有另一个关键动作同时发生。
        \end{itemize}
    \end{enumerate}
\end{multicols}

\grammarquestions

\wsitem{boxing match为什么是match,而不是boxing game?}
\begin{multicols}{1}
    在中文里,我们统一用“比赛”来概括,但在英语中,Match 和 Game 代表了两种截然不同的竞技逻辑。

    拳击之所以被称为 Match 而不是 Game,核心原因在于“对抗的本质”和“规则的结构”。

    \begin{enumerate}
        \item \textbf{逻辑区分:Match vs. Game (The Core Difference)}
        \begin{itemize}
            \item \textbf{Match (对等/匹配的对抗)}
            \begin{itemize}
                \item \textbf{本质:} Match 原意是“匹配”。它强调的是\textbf{双方实力的直接对抗},通常是 1 对 1 或 2 对 2。
                \item \textbf{特点:} 它往往意味着一场“较量”,胜负取决于谁能压倒对手。
                \item \textbf{典型:} Boxing match, Tennis match, Wrestling match (都是直接的身体或技巧对抗)。
            \end{itemize}
            
            \item \textbf{Game (娱乐/规则驱动的博弈)}
            \begin{itemize}
                \item \textbf{本质:} Game 强调的是\textbf{玩耍、策略和复杂的计分规则}。
                \item \textbf{特点:} 通常包含球类、团队协作或通过操纵某种物体(如球、棋子)来得分。
                \item \textbf{典型:} Football game, Basketball game, Chess game (更多是关于如何“玩”好这套规则)。
            \end{itemize}
        \end{itemize}
        \item \textbf{为什么拳击绝对不能叫 Game?}
        如果你对一个职业拳击手说 "Nice game",他可能会觉得你在羞辱他。
        \begin{itemize}
            \item \textbf{严肃性 (Seriousness)}
            
            Game 带有“游戏、消遣”的味道。但拳击是一种极其严肃、甚至可能导致受伤的生死较量。
            
            这种关乎荣誉和生存的战斗,必须用 Match 这种更稳重、更具对抗色彩的词。
            
            \item \textbf{匹配性 (The element of 'Matching')}
            
            在拳击中,双方必须在体重、等级上进行“匹配” (match) 才能开赛。所以这场比赛本身就是一次 、\textbf{Matching} (配对) 后的结果。
        \end{itemize}
        \item \textbf{词汇关联:与 Mendoza 故事的契机}
        在文中,Mendoza 和 Humphries 的对决被称为 Fight,而在正式场合则称为 Match:
        \begin{itemize}
            \item \textbf{The Progression:}
            \begin{itemize}
                \item \textbf{Fight:} 强调打斗的动作(比较原始)。
                \item \textbf{Match:} 强调在 set of rules (规则) 下进行的正式比赛。
                \item \textbf{Result:} Mendoza 在一场 \textbf{match} 中 \textbf{defeated} (击败) 了对手。
            \end{itemize}
        \end{itemize}

        \item \textbf{辨析总结表 (Quick Reference)}
        \begin{itemize}
            \item 格斗/对抗类使用match,因为强调 1 对 1 的较量
            \item 大型球类/团队使用game,因为强调强调趣味、策略与球
            \item 田径/竞速类使用race,因为强调速度的竞争
            \item 锦标赛/系列赛使用Tournament,因为强调多轮筛选的过程
        \end{itemize}
    \end{enumerate}
\end{multicols}

\wsitem{Mendoza enjoyed tremendous popularity 为什么popularity要用enjoy?}
\begin{multicols}{1}
    这是一个非常地道的动词搭配。在英语中,enjoy 的含义远不止“享受某种乐趣”,它在正式语境中常用来表示\textbf{“享有、拥有(某种积极的地位、优势或状态)”}。

    \begin{enumerate}
        \item \textbf{"Enjoy" 的核心语义扩展 (Semantic Extension)}
        \begin{itemize}
            \item \textbf{状态的占有 (Possession of a Positive State)}
            \begin{itemize}
                \item Enjoy 在这里不一定指心理上的愉悦,而是指\textbf{客观上处于一种令人羡慕的地位}。
                \item 当主语是一个公众人物(如 Mendoza)时,说他 enjoyed popularity 是指他“\textbf{享有}”很高的名望,这种名望是他身份的一部分。
            \end{itemize}
            
            \item \textbf{翻译的艺术 (Translation Nuance)}
            
            在这种语境下,enjoy 不翻译为“喜欢”,而翻译为:\textbf{享有、拥有、得到}。
        \end{itemize}
        \item \textbf{为什么不用 "Have" 或 "Had"? (Enjoy vs. Have)}
        \begin{itemize}
            \item \textbf{语感的强度 (Tone and Intensity)}
            \begin{itemize}
                \item Had popularity 听起来非常平淡,只是叙述一个事实。
                \item Enjoyed popularity 则带有一种\textbf{动态感和光环感},暗示这种名望给他的生活带来了积极的影响。
            \end{itemize}
            
            \item \textbf{强调“持续性”与“广泛性”}
            
            Enjoy 常用于形容那些需要社会认同、长期积累的状态。
        \end{itemize}
        \item \textbf{常见的“Enjoy + 抽象名词”搭配 (Common Collocations)}
        \begin{itemize}
            \item \textbf{Enjoy a reputation (享有声誉)}
            
            \es{ The school \textbf{enjoys a high reputation for its \textbf{life work} in education.}(这所学校在教育事业上享有极高的声誉。)}
            
            \item \textbf{Enjoy a right/privilege (享有权利/特权)}
            
            \es{ Citizens \textbf{enjoy the right to vote.} (公民享有投票权。)}
            
            \item \textbf{Enjoy success (享有成功)}
            
            \es{ Bleriot \textbf{enjoyed great success after his flight \textbf{across the Channel}.(布莱里奥在飞越海峡后获得了巨大的成功。)}}
            
            \item \textbf{Enjoy good health (享有健康的体魄)}
            
            指一个人长期保持健康状态。
        \end{itemize}
        \item \textbf{语境解析:Tremendous Popularity}
        
        \textbf{Tremendous (巨大的/极大的):} 这是一个语气很强的形容词,用来修饰 popularity,说明 Mendoza 在当时不仅仅是出名,而是全民偶像级别的红人。
    \end{enumerate}
\end{multicols}

\wsitem{He was adored by rich and poor alike.为什么不是he was adored by both rich and poor?}
\begin{multicols}{1}
    这也是一个非常地道的表达差异。虽然 both...and... 在语法上完全正确,但作者选择使用 "rich and poor alike" 具有更强的修辞效果和叙事美感。

    \begin{enumerate}
        \item \textbf{词义重心与强调 (Focus and Emphasis)}
        \begin{itemize}
            \item \textbf{"Alike" 的包容性 (Inclusivity of 'Alike')}
            \begin{itemize}
                \item Alike 在句末作为副词,意为“**同样地、不分彼此地**”。
                \item \textbf{语义重心:} 它强调的是不同群体在“崇拜他”这件事上表现出的**一致性**。它暗示了 Mendoza 的魅力跨越了阶级鸿沟,让贫富双方在情感上达到了统一。
            \end{itemize}
            
            \item \textbf{"Both...and..." 的分割性 (Division of 'Both...and...')}
            \begin{itemize}
                \item Both...and... 倾向于将两者看作两个**独立的个体或类别**。
                \item \textbf{语义重心:} 它仅仅是在陈述一个事实:A 喜欢他,B 也喜欢他。它缺乏那种“所有人合而为一”的感染力。
            \end{itemize}
        \end{itemize}
        \item \textbf{文学色彩与韵律 (Literary Tone and Rhythm)}
        \begin{itemize}
            \item \textbf{叙事的节奏感 (Narrative Rhythm)}
            \begin{itemize}
                \item Rich and poor alike 读起来更加朗朗上口,有一种古典英语的优雅。在《新概念英语》这种文学性较强的教材中,这种表达能提升文章的格调。
            \end{itemize}
            
            \item \textbf{强化“众口一词”的效果}
            \begin{itemize}
                \item 在描述像 Mendoza 这样的传奇英雄(colourful figure)时,这种表达能更好地衬托他极高的人气(tremendous popularity)。
            \end{itemize}
        \end{itemize}
        \item \textbf{常见用法与固定搭配 (Idiomatic Usage)}
        
        这种 "A and B alike" 的结构在描述“普遍性”时非常高效:

        \begin{itemize}
            \item \textbf{描述广泛的兴趣:}
            
            \es{ His \textbf{life work inspired \textbf{young and old alike}.}(他的毕生心血激励了年轻人和老人,不分长幼。)}
            
            \item \textbf{描述普遍的影响:}
            
            \es{ The engine failure was a blow to \textbf{experts and amateurs alike.}(引擎故障对专家和业余爱好者来说同样是个打击。)}
            
            \item \textbf{描述自然或规则:}
            
            \es{Death comes to \textbf{kings and peasants alike}.(死亡对国王和平民一视同仁。)}
        \end{itemize}
    \end{enumerate}
\end{multicols}

\wsitem{Mendoza rose to fame swiftly after a boxing match when he was only fourteen years old. 为什么用swiftly不用quickly?}
\begin{multicols}{1}
    这是一个非常敏锐的观察。在描述 Mendoza 的成名之路时,作者选择 "rose to fame swiftly" 而不是 quickly,主要是为了捕捉那种“势如破竹、平稳上升且极具力量感”的过程。

    \begin{enumerate}
        \item \textbf{成名轨迹的质感 (The Quality of Rising to Fame)}
        \begin{itemize}
            \item \textbf{动态的平滑感 (Smoothness of Action)}
            
            Rise(上升)本身是一个平滑的连续动作。Swiftly 完美地契合了这种轨迹,形容他从无名小卒到大众偶像的过程非常**流畅且毫无阻碍**。
            
            相比之下,quickly 往往带有“急促、突发”的味道,听起来可能像是一夜成名后的偶然,而 swiftly 更像是因为实力出众而产生的必然结果。
            
            \item \textbf{优雅与力量的结合 (Grace and Power)}
            
            常用在文学中描写像风、流水或雄鹰这种自然且强大的力量。

            曼多萨(Mendoza)是“科学拳击”的创始人,他的风格以敏捷著称。使用 swiftly 在潜意识里就呼应了他这种**灵动、高效**的拳击风格。
        \end{itemize}
        \item \textbf{搭配习惯与语境 (Collocation and Context)}
        \begin{itemize}
            \item \textbf{固定搭配:Rise swiftly}
            
            在英语书面语中,形容职位的晋升或地位的提高,rise swiftly 是一个高频的地道搭配。
            
            \item \textbf{强调“时机”与“效率”}
            
            句子提到他当时“只有 14 岁”。Swiftly 在这里强调他不仅快,而且在极短的时间内就达到了顶峰,展现出一种少年天才的**高效率**。
        \end{itemize}
        \item \textbf{句式回顾:Rise to fame (成名)}
        \begin{itemize}
            \item Rise to fame 是一个非常经典的短语。
            \item \textbf{等价表达:} Gained fame, became famous, achieved celebrity status.
            \item \textbf{文学提升:} Rise to fame 赋予了成名一种“冉冉升起”的画面感。
        \end{itemize}
    \end{enumerate}
\end{multicols}

\wsitem{This attracted the attention of Richard Humphries who was then the most eminent boxer in England. 为什么是the attention of Richard Humphries不是Richard Humphries‘s attention?}
\begin{multicols}{1}
    这是一个关于英语句子平衡(Sentence Focus)和修饰成分处理的精彩问题。
    
    虽然 Richard Humphries's attention 在语法上是正确的,但在这个特定的句子里,使用 the attention of Richard Humphries 是为了后续的\textbf{定语从句(who was then...)}能顺畅连接。

    \begin{enumerate}
        \item \textbf{句式重心与“就近原则” (Proximity Principle)}
        \begin{itemize}
            \item \textbf{方便引导定语从句}
            \begin{itemize}
                \item 在英语中,who 引导的定语从句通常要紧跟在它所修饰的人名后面。
                \item 如果用 Richard Humphries's attention,紧跟在从句前面的词就变成了 attention。这会导致逻辑混乱,因为 who 不能用来修饰“注意力”。
                \item 使用 the attention of \textbf{Richard Humphries},确保了人名位于句子的末端,从而让 who 能够完美地衔接。
            \end{itemize}
            
            \item \textbf{句子重心的后移 (End-weight Principle)}
            \begin{itemize}
                \item 英语倾向于把较长的、复杂的修饰成分放在句末。
                \item of Richard Humphries 加上后面的长从句非常重。通过这种结构,句子保持了头轻脚重的平稳感,读起来更加顺畅。
            \end{itemize}
        \end{itemize}
        \item \textbf{语义上的细微差别 (Nuance in Meaning)}
        \begin{itemize}
            \item \textbf{客观性与正式感}
            \begin{itemize}
                \item 's 所有格通常用于非正式、简短的表达(如 my friend's house)。
                \item the ... of ... 结构则更具\textbf{客观性}和\textbf{正式感},常用于历史叙事或正式的人物传记中,体现了 Mendoza 引起关注这一事件的历史重量。
            \end{itemize}
            
            \item \textbf{强调对象不同}
            \begin{itemize}
                \item Richard Humphries's attention 强调的是“注意力”。
                \item The attention of Richard Humphries 将重心更多地引向了“Richard Humphries”这个人,为后文介绍这位“全英最杰出的拳击手”做好了铺垫。
            \end{itemize}
        \end{itemize}
    \end{enumerate}
\end{multicols}

\wsitem{He offered to train Mendoza ... 为什么用offer to ,不是he trained Mendoza ...}
\begin{multicols}{1}
    这是一个非常微妙的叙事细节。作者选择使用 "offered to train" 而不是直接说 "trained",是为了刻画两人之间互动性质以及身份地位的差异。

    \begin{enumerate}
        \item \textbf{动作的“主动性”与“善意” (Initiative and Goodwill)}
        \begin{itemize}
            \item \textbf{表达主观意愿 (Subjective Willingness)}
            
            Offer 强调的是 Richard Humphries \textbf{主动提出帮助}。
            
            作为一个当时“全英最杰出的拳击手” (most eminent boxer),他主动提出教导一个 14 岁的孩子,这表现出他对手下才华的\textbf{赏识}和一种\textbf{提携后辈}的慷慨。
            
            \item \textbf{叙事的层次感 (Layers of Narrative)}
            
            如果只说 "He trained Mendoza",这只是在陈述一个教学的事实,听起来像是一份普通的工作。
            
            说 "He offered to train",则交代了这段师徒关系的起点——是由于 Mendoza 的天赋引起了大师的注意(attracted the attention),才有了后来的提议。
        \end{itemize}
        \item \textbf{逻辑上的“因果链条” (Logical Causality)}
        \begin{itemize}
            \item \textbf{提议与反馈的呼应}
            
            前半句是 He \textbf{offered to train}(他提议教);
            
            后半句是 Mendoza was \textbf{quick to learn}(曼多萨学得快)。
            
            这种“\textbf{提议 $\rightarrow$ 响应}”的结构,比单纯说“他教了,他学了”要生动得多,体现了师徒之间极高的默契。
            
            \item \textbf{强调“机会”的珍贵}
            
            在那个时代,得到顶级拳击手的亲自指导是巨大的荣誉。用 offer 能体现出这个机会的来之不易。
        \end{itemize}
    \end{enumerate}
\end{multicols}

\wsitem{Mendoza met Humphries in the ring中met的作用,感觉和中文差异很大}
\begin{multicols}{1}
    你敏锐地察觉到了中英表达在动态感上的差异。在中文里,我们常说“门多萨和汉弗莱斯在拳台上交手了”或“对决了”。但英语却用了看起来很平和的 met (meet 的过去式)。

    这里 met 的作用非常巧妙,它不仅仅是“遇见”,而是一种\textbf{“狭路相逢”或“正式对垒”}的表达方式。以下是深度解析:

    \begin{enumerate}
        \item \textbf{语义的“错位”:从“遇见”到“对阵” (The Semantic Shift)}
        
        \begin{itemize}
            \item \textbf{中文思维:强调动作 (Action-oriented)}
            
            中文倾向于直接描述动作:打斗、比武、较量。
            
            \item \textbf{英文思维:强调“遭遇”与“交汇” (Encounter-oriented)}
            
            Meet 的本质是两个移动的物体在同一个点交汇。
            
            在竞技体育(拳击、足球、棋类)语境下,meet 自动转化为“\textbf{交锋/对垒}”的意思。它强调的是双方在约定的时间、约定的地点(in the ring)\textbf{正式碰头并开始较量}。
        \end{itemize}

        \item \textbf{语境的力量:地点决定动作 (Location Defines the Action)}
        在英语中,动词的含义往往被其后的地点状语重新定义:

        \begin{itemize}
            \item \textbf{Meet + in a cafe:} 喝咖啡聊天(普通遇见)。
            \item \textbf{Meet + in the ring:} 拳台见(那就是打架)。
            \item \textbf{Meet + on the battlefield:} 战场见(那就是生死搏斗)。
            \item \textbf{Meet + in court:} 法庭见(那就是打官司)。
        \end{itemize}

        这种用法让语言显得含蓄但充满力量。 它不需要说 "they fought",因为在拳台(ring)这个背景下,除了打斗,没有别的“碰面”方式。
        \item \textbf{叙事功能:仪式感与宿命感 (Narrative Function)}
        \begin{itemize}
            \item \textbf{强调“宿命的重逢”}
            
            结合前文提到的 for a second time(第二次),使用 met 营造了一种\textbf{“冤家路窄”}的仪式感。它把这场比赛看作是两个历史人物轨迹的再次交汇。
            
            \item \textbf{职业化的表述}
            
            在体育新闻或历史记录中,用 met 显得非常客观、专业,类似于“双方运动员入场并展开对决”。
        \end{itemize}
        
        \item \textbf{相似用法拓展}
        
        你可以用这个词来描述任何形式的对抗:

        \begin{itemize}
            \item \textbf{描述决赛:}
            
            \es{ The two \textbf{eminent teams will \textbf{meet} in the final next Sunday.} (两支杰出的球队将在下周日的决赛中交锋。)}
            
            \item \textbf{描述挑战:}

            \es{ He was ready to \textbf{meet the \textbf{challenge} head-on.}(他准备好正面迎接这项挑战了。)}
        \end{itemize}
    \end{enumerate}
\end{multicols}

\wsitem{meet有地点决定动作的意义,但是meet in the police station是什么动作?}
\begin{multicols}{1}
    这是一个非常敏锐且深刻的反问!你抓住了英语中“语境暗示”的核心。
    
    "Meet in the police station" 的具体动作含义,取决于你在那个场景中的身份(Identity)。警察局是一个特殊的功能性场所,它不像拳台(只有对决)那样单一,但它依然符合“地点决定动作”的逻辑。

    \begin{enumerate}
        \item \textbf{身份决定动作逻辑 (Role-Based Action)}
        \begin{itemize}
            \item \textbf{如果你是罪犯/嫌疑人:}
            
            \textbf{隐含动作:} \textbf{接受审讯 (Be interrogated)}。
                
            \es{ Example: "The detective and the suspect \textbf{met in the police station."}(侦探和嫌疑人在警察局碰面了。) —— 这里的 met 暗示了压力重重的审讯或对质。}
            
            \item \textbf{如果你是纠纷双方:}
            
            \textbf{隐含动作:} \textbf{调解/报案 (Mediate/Report)}。
                
            \es{ Example: "The two neighbors \textbf{quarrelled bitterly and finally \textbf{met in the police station}."} (邻居两人激烈争吵,最后在警察局见了。) —— 这里的 met 暗示了法律干预下的调解。}
            
            \item \textbf{如果你是警察同事:}
            
            \textbf{隐含动作:} \textbf{交接/办公 (Hand over/Work)}。
                
            \es{ Example: "The two officers \textbf{met in the police station to \textbf{draw up} the final report."(两名警官在局里碰头起草最终报告。) —— 这里的 met 就是纯粹的职业交汇。}}
        \end{itemize}

        \item \textbf{为什么这种现象在英语中很普遍?}
        
        英语是一种非常\textbf{简约(Concise)}的语言,它喜欢用最基础的动词(如 meet, get, go)配合具体的名词。中英文的思维模式有相当不同。

        \begin{itemize}
             \item 中文倾向于使用具体的动词来定义行为(审讯、对质、报警)。
            \item 英文倾向于使用通用动词(meet)来搭建舞台,让读者根据舞台背景(police station, ring, court)自己去脑补具体的戏份。
        \end{itemize}
        \item \textbf{更多“地点决定动作”的例子}
        \begin{itemize}
            \item In the ring: 隐含的动作 (Implicit Action)为搏斗、比赛 (Fight/Compete)
            \item In court:诉讼、对质 (Litigate/Confront)
            \item In the boardroom:谈判、决策 (Negotiate/Decide)
            \item In the operating room:手术、救治 (Operate/Save)
            \item In a back alley:密谋、非法交易 (Plot/Deal)
        \end{itemize}

        \item \textbf{总结}
        \begin{itemize}
            \item Met 本身是中性的。
            \item 地点 (Police Station) 给它穿上了“制服”。
            \item 身份 给了它具体的脚本。
        \end{itemize}

    \end{enumerate}
\end{multicols}

\wsitem{能用obtain attention 或者acquire attention吗?}
\begin{multicols}{1}
    在绝大多数情况下,不能这么用。
    
    虽然 obtain 和 acquire 都有“获得”的意思,但它们与 attention(注意力)搭配时会显得非常不自然。这涉及到英语中的\textbf{词汇搭配(Collocation)}习惯。

    \begin{enumerate}
        \item \textbf{语义冲突:为什么不能用 Obtain/Acquire?}
        
        \begin{itemize}
            \item \textbf{Obtain (获得/得到 —— 侧重“努力后的实体占有”)}
            \begin{itemize}
                \item Obtain 通常指通过努力、申请或正式途径得到某种\textbf{具体的、可以持有的}东西。
                \item \textbf{正确搭配:} Obtain a visa (获得签证), Obtain information (获取信息)。
                \item \textbf{冲突点:} “注意力”是一种虚幻的、流动的能量,你无法像拿证书一样把它“占有” (obtain)。
            \end{itemize}
            
            \item \textbf{Acquire (习得/收购 —— 侧重“逐渐积累”或“所有权变更”)}
            \begin{itemize}
                \item Acquire 强调通过长期的过程“学到”知识,或者“买下”资产。
                \item \textbf{正确搭配:} Acquire a language (习得语言), Acquire a company (收购公司)。
                \item \textbf{冲突点:} 你不能“买下”或“长期习得”别人的注意力,注意力往往是瞬间或自发产生的。
            \end{itemize}
        \end{itemize}
        \item \textbf{应该用什么词? (Correct Collocations)}
        
        如果你想表达“得到关注”,英语中有几套固定的“黄金搭配”:

        \begin{itemize}
            \item \textbf{吸引关注 (The Most Common)}
            \begin{itemize}
                \item \textbf{Attract attention:} 像磁铁一样把目光吸过来。
                \item \textbf{Draw attention:} 类似于“拉动”人们的视线(常用于 Draw attention to sth.)。
            \end{itemize}
            
            \item \textbf{赢得/获得关注 (Focus on Success)}
            \begin{itemize}
                \item \textbf{Gain attention:} 侧重于经过努力后,名气或关注度“增加了”。
                \item \textbf{Get attention:} 最口语化的表达,“得到关注”。
            \end{itemize}
            
            \item \textbf{寻求关注 (Focus on Intent)}
            \begin{itemize}
                \item \textbf{Seek attention:} 刻意做某些事来让别人看自己(有时带贬义,如 attention-seeking behavior)。
            \end{itemize}
        \end{itemize}
    \end{enumerate}
\end{multicols}

\wsitem{One of the most colourful figure in boxing history was Daniel Mendoza, who was born in 1764.这里为什么是figure,而不是person,或者是man,或者是boxer, prizefighter?}
\begin{multicols}{1}
    在英语写作中,词汇的选择往往决定了叙事的“高度”和“质感”。作者之所以避开 man 或 boxer 而选用 figure,是为了精准地传达出 Mendoza 在历史长河中的地位感和传奇色彩。

    \begin{enumerate}
        \item \textbf{Figure 的“重量感” (The Weight of the Word)}
        \begin{itemize}
            \item \textbf{从“人”到“人物” (From 'Person' to 'Figure')}
            \begin{itemize}
                \item \textbf{Man/Person:} 仅仅描述他的生物属性或个体身份,显得平淡无奇。
                \item \textbf{Figure:} 特指历史、艺术或某个领域中**举足轻重、具有代表性的人**。中文常翻译为“人物”。
                \item \textbf{逻辑:} 既然 Mendoza 是“最丰富多彩的”(most colourful),他就不再是一个普通的 man,而是一个刻在历史背景板上的“身影”。
            \end{itemize}
            
            \item \textbf{形象的饱满度 (Multi-dimensional Image)}
            
            Figure 包含了一个人的成就、性格、影响力以及他所代表的时代。
        \end{itemize}
        \item \textbf{为什么不用 Boxer 或 Prizefighter?}
        
        虽然 Mendoza 确实是拳击手(boxer),但作者在文章开头需要一个更宏大的视角:

        \begin{itemize}
            \item \textbf{避免语义重复 (Avoiding Redundancy)}
            
            句子后半部分提到了 boxing history。如果说 the most colourful \textbf{boxer in boxing history},在修辞上略显累赘。
            
            \item \textbf{强调“跨界”影响力}
            
            Mendoza 不仅仅是会打拳,他还是第一个将拳击**科学化**的人(set of rules)。Figure 暗示了他对整个社会的文化影响,而不仅仅是他在拳台(ring)上的动作。
        \end{itemize}
        \item \textbf{Colourful 与 Figure 的“化学反应”}
        \begin{itemize}
            \item \textbf{Colourful (丰富多彩/传奇的)}
            
            这个词常用来形容那些经历坎坷、个性鲜明、结局戏剧性的人。
            \item \textbf{固定搭配的张力}
            
            \textbf{colourful figure} 是英语中的固定高级表达,专门用来形容那些“有故事的人”。
            
            如果用 colourful man,听起来像是在形容他的衣服颜色很鲜艳。
        \end{itemize}
    \end{enumerate}
\end{multicols}

\wsitem{... he founded a highly successful Academy ...为什么不是highly successful school}
\begin{multicols}{1}
    作者在这里使用 Academy 而不是 School,不仅是为了避免重复,更是为了精准地传达出 Mendoza 所创办机构的专业级别、社会地位以及它对拳击这项运动的贡献。
    \begin{enumerate}
        \item \textbf{词义的“阶层感” (Hierarchy of Meaning)}
        \begin{itemize}
            \item \textbf{School (大众化/基础教育)}
            
            \textbf{语感:} 通常指提供基础知识的普通学校,或者是学生们被动接受教育的地方。
            
            \textbf{局限性:} 如果说 "boxing school",听起来像是一个小孩子学业余爱好或强身健体的地方。
            
            \item \textbf{Academy (专门化/高等或专业机构)}
            
            \textbf{语感:} 侧重于**专门技能、艺术或高等科学**的培训。它带有“研究院”或“专业学院”的色彩。
            
            \textbf{地位:} Academy 暗示这里教授的是一套**科学的体系**。这完美契合了 Mendoza 的贡献——他将拳击带入了“科学化”时代(scientific boxing)。
        \end{itemize}

        \item \textbf{身份的匹配:为什么能吸引勋爵 (Social Status)}
        
        文中提到 Lord Byron(拜伦勋爵)成为了他的学生。这种高层社会人物的出现,决定了该机构的称呼:

        \begin{itemize}
            \item \textbf{贵族的选择}
            
            在 18 世纪,贵族们不会去简陋的 "school" 和普通人混在一起。他们去的是 **Academy**,因为那里代表了优雅、先进和精英阶层。
            
            \item \textbf{品牌效应 (Branding)}
            
            Academy 让这种体育活动听起来更像是一门“艺术”或“绅士的技艺”(The Noble Science),而不仅仅是互殴。
        \end{itemize}
    \end{enumerate}
\end{multicols}

\newpage