\section{Lesson 31 A lovable eccentric}

\begin{paracol}{2}

True \nw{eccentrics} never \nw{deliberately} \ns{set out} to \ns{draw attention to} themselves.

\switchcolumn

\chinesetext{真正古怪的人从不有意引人注意。}
\nwe{lovable}{ˈlʌvəbəl}{adj. 可爱的;惹人爱的;}
\nwe{eccentrics}{ɪkˈsentrɪk}{adj. 古怪的;n. 怪人;}
\nwe{deliberately}{dɪˈlɪbərətli}{adv. 有意地;从容地;不慌不忙地;}
\nse{set out}{sɛt aʊt}{动身; 出发;着手;安排;}
\nse{draw attention to}{drɔ əˈtɛnʃən tu}{吸引…的注意力;促使…注意;}

\switchcolumn*

They \ns{disregard} \ns{social \nw{conventions}} without being \nw{conscious} that they are doing anything extraordinary.

\switchcolumn

\chinesetext{他们不顾社会习俗,意识不到自己所作所为有什么特殊之处。}
\nwe{disregard}{ˌdɪsrɪˈɡɑːrd}{v. 不理会;漠视;n. 漠视,忽视;}
\nwe{convention}{kənˈvenʃn}{n. 习俗,惯例;公约;大会; (艺术等)传统风格;}
\nwe{conscious}{ˈkɑːnʃəs}{adj. 意识到的;神志清醒的;慎重的;关注的;}
\nse{social conventions}{}{社会习俗}

\switchcolumn*

This \nw{invariably} wins them the love and respect of others, for they add colour to the \ns{dull} \ns{routine} of \ns{everyday life}.

\switchcolumn

\chinesetext{他们总能赢得别人的喜爱与尊敬,因为他们给平淡单一的日常生活增添了色彩。}
\nwe{invariably}{ɪnˈveriəbli}{adv. 总是;不变地;一贯地;}
\nwe{dull}{dʌl}{adj. 枯燥无味的;晦暗的;多云的;(声音)不清晰的;(感觉)不明显的;(迟)钝的;不景气的;v. 变迟钝;变暗淡;变麻木;}
\nwe{routine}{ruːˈtiːn}{n. 惯例,常规;例程;生活乏味;一套动作;adj. 常规的;日常的;平常的;乏味的;}
\nse{everyday life}{ˈɛvriˈde laɪf}{日常生活;}

\switchcolumn*

\ns{Up to the time of} his death, Richard Colson was one of the most \nw{notable} figures in our town.

\switchcolumn

\chinesetext{理查德.科尔森生前是我们镇上最有名望的人之一。}
\nwe{notable}{ˈnoʊtəbl}{adj. 值得注意的;显著的;著名的;n. 名人;显要人物;}

\switchcolumn*

He was a \nw{shrewd} and wealthy businessman, but most people in the town hardly knew anything about this side of his life.

\switchcolumn

\chinesetext{他是个精明能干、有钱的商人,但镇上大部分人对他生活中的这一个方面几乎一无所知。}
\nwe{shrewd}{ʃruːd}{adj. 精明的,敏锐的;奸诈的,狡猾的;有眼光的;精于盘算的;}

\switchcolumn*

He \ns{was known to us all as} Dickie and his \nw{eccentricity} had become \nw{legendary} \ns{long before} he died.

\switchcolumn

\chinesetext{大家都管他叫迪基。早在他去世前很久,他的古怪行为就成了传奇故事了。}
\nwe{eccentricity}{ˌeksenˈtrɪsəti}{n. 古怪;反常,怪癖;[物]偏心距;[数]离心率;}
\nwe{legendary}{ˈledʒənderi}{adj. 传说的;传奇的;极其著名的;}
\nse{be known to sb. as ...}{bi non tu}{为…所熟知;名字被记录在案;}
\nse{long before}{lɔŋ bɪˈfɔr}{很久以前;}

\switchcolumn*

Dickie disliked \nw{snobs} \nw{intensely}.

\switchcolumn

\chinesetext{迪基痛恨势利小人。}
\nwe{snob}{snɑːb}{n. 势利小人,势利眼;附庸风雅之徒,假内行;}
\nwe{intensely}{ɪnˈtɛnsli}{adv. 强烈地;极度;剧烈地;}

\switchcolumn*

Though he owned a large car, he hardly ever used it, \ns{preferring always to} go \ns{on foot}.

\switchcolumn

\chinesetext{尽管他有一辆豪华小轿车,但却很少使用,常常喜欢以步代车。}
\nse{prefer to}{prɪˈfɜːr tu}{较喜欢,宁愿;提升某人为;}
\nse{on foot}{ɑn fʊt}{步行,进行起来,在筹划中;徒步;徒;}

\switchcolumn*

Even when it was raining heavily, he refused to carry an umbrella.

\switchcolumn

\chinesetext{即使大雨倾盆,他也总是拒绝带伞。}

\switchcolumn*

One day, he walked into an expensive shop after having \ns{\ns{been caught in} a particularly \ns{heavy shower}}.

\switchcolumn

\chinesetext{一天,他遇上一场瓢泼大雨,淋得透湿,他走进一家高级商店。}
\nse{heavy shower}{ˈhɛvi ˈʃaʊɚ}{暴雨,大阵雨;骤雨;}
\nse{be caught in ...}{bi kɔːt ɪn}{淋雨;罹;陷于;}
\nse{be caught in a heavy shower}{}{遇上一场大雨;被大雨淋到;突遇暴雨}

\switchcolumn*

He wanted to buy a £300 watch for his wife, but he was in such a \nw{bedraggled} condition that an assistant refused to serve him.

\switchcolumn

\chinesetext{他要为妻子买一块价值300英镑的手表,但店员见他浑身泥水的样子,竟不肯接待他。}
\nwe{bedraggle}{bɪ'drægəl}{vi. 弄脏,弄湿,弄皱;}

\switchcolumn*

Dickie left the shop \ns{without a word} and returned carrying a large \ns{cloth bag}.

\switchcolumn

\chinesetext{迪基二话没说就走了。一会儿,他带着一个大布口袋回到店里。}
\nse{without a word}{}{一句话也没说;一声不吭;}
\nse{cloth bag}{klɔθ bæɡ}{n. 布袋;}

\switchcolumn*

As it was extremely heavy, he \nw{dumped} it on the counter.

\switchcolumn

\chinesetext{布袋很沉,他重重地把布袋扔在柜台上。}
\nwe{dump}{dʌmp}{v. 倾倒;丢弃,乱放;推卸;倾销;转存;与…断绝关系;n. 垃圾场;脏地方;转存;临时堆积处;}

\switchcolumn*

The assistant asked him to leave, but Dickie \ns{paid no attention to} him and requested to see the manager.

\switchcolumn

\chinesetext{店员让迪基走开,他置之不理,并要求见经理。}
\nse{pay no attention to sb.}{pei nəu əˈtenʃən tu:}{置之不理…;毫不在乎…;不问;不在意;}

\switchcolumn*

Recognizing who the customer was, the manager was most \nw{apologetic} and \nw{reprimanded} the assistant severely.

\switchcolumn

\chinesetext{经理认出了这位顾客,表示了深深的歉意,还严厉地训斥了店员。}
\nwe{apologetic}{əˌpɑləˈdʒetɪk}{adj. 道歉的,谢罪的;愧疚的;}
\nwe{reprimand}{ˈreprɪmænd}{vt. 谴责;惩戒;责难;}

\switchcolumn*

When Dickie was given the watch, he \ns{presented the assistant with} the cloth bag.

\switchcolumn

\chinesetext{店员为迪基拿出了那块手表,迪基把布口袋递给他。}
\nse{presented sb. with sth.}{}{}

\switchcolumn*

It contained £300 \ns{in \nw{pennies}}.

\switchcolumn

\chinesetext{口袋里面装着300镑的便士。}
\nwe{pennies}{'peniz}{n. of penny;便士( penny的名词复数 );(美国、加拿大的)一分钱;少量的钱;}
\nse{in pennies}{}{以便士计}

\switchcolumn*

He \ns{insisted on} the assistant's counting the money before he left--30, 000 pennies \ns{in all}!

\switchcolumn

\chinesetext{他坚持要店员点清那些硬币后他才离去。这些硬币加在一起共有30,000枚!}
\nse{insist on}{}{坚决要求;}
\nse{in all}{}{总共,合计;通共;拢共;成总儿;}

\switchcolumn*

\ns{On another occasion}, he invited \ns{a number of} important critics to see his \ns{private collection} of \ns{modern paintings}.

\switchcolumn

\chinesetext{还有一次,他邀请一些著名评论家来参观他私人收藏的现代画。}
\nse{on another occasion}{}{另一次}
\nse{a number of}{}{许多的;一些;}
\nse{private collection}{}{私人收藏;}
\nse{modern painting}{}{现代绘画;}

\switchcolumn*

This exhibition \ns{received \ns{\ns{a great deal of} attention}} in the press, for though the pictures \ns{were supposed to} be \ns{the work of} famous artists, they had \ns{in fact} been painted by Dickie.

\switchcolumn

\chinesetext{这次展览引起报界广泛注意,因为这些画名义上是名家的作品,事实上是迪基自己画的。}
\nse{receive attention }{}{受到关注;}
\nse{a great deal of}{}{大量;}
\nse{a great deal of attention}{}{广泛关注}
\nse{be supposed to}{}{v. 应该,被期望;}
\nse{the work of ...}{}{...的作品}
\nse{in fact}{}{实际上,其实;实则;事实上;说起来;}

\switchcolumn*

It took him four years to \nw{stage} this \ns{\nw{elaborate} joke} simply to prove that critics do not always know what they are talking about.

\switchcolumn

\chinesetext{他花了4年时间策划这出精心设计的闹剧,只是想证明评论家们有时并不解他们所谈论的事情。}
\nwe{stage}{steɪdʒ}{n. 阶段;舞台;v. 上演;举办;组织;使发生;}
\nwe{elaborate}{ɪ'læbərət}{vi. 详尽说明;变得复杂;vt. 详细制定;详尽阐述;[生理学]加工;尽心竭力地做;adj. 复杂的;详尽的;精心制作的;}
\nse{elaborate joke}{}{精心设计的闹剧}

\switchcolumn*


\end{paracol}


%True eccentrics never deliberately set out to draw attention to themselves. They disregard social conventions without being conscious that they are doing anything extraordinary. This invariably wins them the love and respect of others, for they add colour to the dull routine of everyday life.

%Up to the time of his death, Richard Colson was one of the most notable figures in our town. He was a shrewd and wealthy businessman, but most people in the town hardly knew anything about this side of his life. He was known to us all as Dickie and his eccentricity had become legendary long before he died.

%Dickie disliked snobs intensely. Though he owned a large car, he hardly ever used it, preferring always to go on foot. Even when it was raining heavily, he refused to carry an umbrella. One day, he walked into an expensive shop after having been caught in a particularly heavy shower. He wanted to buy a \$300 watch for his wife, but he was in such a bedraggled condition than an assistant refused to serve him. Dickie left the shop without a word and returned carrying a large cloth bag. As it was extremely heavy, he dumped it on the counter. The assistant asked him to leave, but Dickie paid no attention to him and requested to see the manager. Recognizing who the customer was, the manager was most apologetic and reprimanded the assistant severely. When Dickie was given the watch, the presented the assistant with the cloth bag. It contained \$300 in pennies. He insisted on the assistant's counting the money before he left -- 30,000 pennies in all! On another occasion, he invited a number of important critics to see his private collection of modern paintings. This exhibition received a great deal of attention in the press, for though the pictures were supposed to be the work of famous artists, they had in fact been painted by Dickie. It took him four years to stage this elaborate joke simply to prove that critics do not always know what they are talking about.


\grammarpoints

\wordex{Eccentric}
\begin{multicols}{2}

    它涵盖了从物理轨迹、人格特质到社会行为的多个维度,强调的是\textbf{“偏离中心”与“独特个性”}。

    \begin{enumerate}
        \item \textbf{核心内涵:偏离轨道的独特性}

        $Eccentric$ 的核心在于其拉丁语词源 *ex- (出)* + *centrum (中心)*。它代表了一种不符合常规、偏离标准圆心或社会准则的状态。其逻辑可以概括为: 

        \begin{itemize}
            \item \textbf{Deviation from Center (偏离中心):} 最初用于天文学,指轨道不是完美圆形的。 
            \item \textbf{Non-conformity (非习俗性):} 核心在于其行为、思想或外表不符合大众公认的“正常”标准。 
            \item \textbf{Harmless Peculiarity (无害的奇特):} 与“疯狂”或“危险”不同,这种特质通常带有某种智力上的趣味或迷人的古怪。 
        \end{itemize}


        \item \textbf{多维语境下的语义表达}

        \begin{itemize} 
            \item \textbf{人格特质(社会心理):} 
            
            描述那些拥有独特生活方式或不寻常习惯的人。 
            
            \es{The \textbf{eccentric} billionaire lives in a house made entirely of glass.} (那位古怪的亿万富翁住在一间完全由玻璃制成的房子里。) 
            
            \item \textbf{机械与科学(技术语境):} 
            
            在工程学中,形容圆盘或轮子的轴心不在几何中心。 
            
            \es{An \textbf{eccentric} cam is used to convert rotary motion into linear motion.} (偏心凸轮被用来将旋转运动转化为直线运动。) 
            
            \item \textbf{艺术与审美(风格表达):} 
            
            探讨打破常规的美学选择。 
            
            \es{Her \textbf{eccentric} fashion sense made her a favorite of street-style photographers.} (她那古怪的时尚品味使她成为街拍摄影师的宠儿。) 
        \end{itemize}

        \item \textbf{语法进阶:程度强弱与修饰变体}

        为了精准描述这种“偏离”的状态,常用以下变体:

        \begin{itemize}
            \item \textbf{程度修饰:} 
            \begin{itemize}
                \item \textit{Wildly/Highly eccentric} (极度古怪的。) 
                \item \textit{Charmingly eccentric} (古怪得可爱的。) 
                \item \textit{Somewhat eccentric} (有些许古怪。) 
                \item \textit{Distinctly eccentric} (显而易见的古怪。) 
            \end{itemize}

            \item \textbf{动词与句式搭配:} 
            \begin{itemize}
                \item \textit{Regarded as eccentric} (被视为古怪):侧重于外界对某人的评价。 
                \item \textit{Border on the eccentric} (近乎古怪):形容行为已经接近某种异常的边缘。 
                \item \textit{Known for being eccentric} (以古怪闻名):侧重于这种特质已成为其个人标志。 
            \end{itemize}
        
            \item \textbf{合成词与派生扩展:} 
            \begin{itemize}
                \item \textit{Eccentricity} (名词:古怪的行为、特性,或物理上的偏心率)。 
                \item \textit{Eccentrically} (副词:古怪地,不寻常地)。 
            \end{itemize}
        \end{itemize}
    \end{enumerate}
\end{multicols}

\wordex{Deliberately}
\begin{multicols}{2}

    它涵盖了从主观意图、行为节奏到法律责任的多个维度,强调的是\textbf{“经过权衡的意志”与“不慌不忙的执行”}

    \begin{enumerate}
        \item \textbf{核心内涵:理性控制下的行动}

        $Deliberately$ 源于拉丁语 *librare*(衡量/称重),其核心逻辑在于行动前经过了大脑的“称重”与预判。其语义逻辑可以概括为:

        \begin{itemize}
            \item \textbf{Intentionality (意图性):} 行为并非偶然或意外,而是主观意识明确追求的结果。 
            \item \textbf{Measured Pace (衡量节奏):} 动作本身缓慢且谨慎,展现出一种不被外界干扰的掌控感。 
            \item \textbf{Calculated Consequence (计算后果):} 核心在于对行为影响有清晰认知,常用于描述具有负面动机或高度专业性的动作。 
        \end{itemize}

        \item \textbf{多维语境下的语义表达}

        \begin{itemize} 
            \item \textbf{主观动机(社会行为):} 
            
            强调违背偶然性的故意行为。 
            
            \es{I believe the fire was \textbf{deliberately} set to destroy the evidence.} (我相信火灾是故意纵的,目的是销毁证据。) 
            
            \item \textbf{行为风格(节奏感知):} 
            
            描述为了确保准确或沉稳而特意放慢速度。 
            
            \es{He spoke \textbf{deliberately}, choosing each word with great care.} (他说话慢条斯理,谨慎地挑选每一个词。) 
            
            \item \textbf{法律与伦理(责任判定):} 
            
            探讨行为人在完全清醒状态下的选择。 
            
            \es{The company was accused of \textbf{deliberately} misleading its investors.} (该公司被指控蓄意误导投资者。) 
        \end{itemize}

        \item \textbf{语法进阶:程度强弱与修饰变体}

        为了精准描述这种“刻意”的状态,常用以下变体:

        \begin{itemize}
            \item \textbf{程度修饰与同义辨析:}
            \begin{itemize}
                \item \textit{Quite deliberately} (非常从容地/完全故意地。) 
                \item \textit{Slowly and deliberately} (缓慢且谨慎地,常用于动作描写。) 
                \item \textit{Deliberate intent} (蓄意图谋,法律常用。) 
            \end{itemize}

            \item \textbf{动词搭配:}
            \begin{itemize}
                \item \textit{Deliberately target...} (蓄意针对...):侧重于攻击或计划的精确性。 
                \item \textit{Deliberately ignore} (故意无视):侧重于主观上的回避。 
                \item \textit{Move deliberately} (动作沉稳):侧重于体态的自信与从容。 
            \end{itemize}
        
            \item \textbf{合成词扩展:}
            \begin{itemize}
                \item \textit{Deliberation} (名词:审议、深思熟虑)。 
                \item \textit{Deliberative} (形容词:审议的,如 a deliberative body 审议机构)。 
            \end{itemize}
        \end{itemize}
        \end{enumerate}
\end{multicols}

\wsitem{Disregard}
\begin{multicols}{2}
    它涵盖了从心理忽略、规则漠视到专业判断的多个维度,强调的是\textbf{“主观上的不予理会”与“信息的过滤”}。
    \begin{enumerate}
        \item \textbf{核心内涵:认知的选择性屏蔽}

        $Disregard$ 由前缀 *dis- (剥离/相反)* 与 *regard (注视/关心)* 组成。它代表了一种在感知到某事后,主动决定不给予其注意力或重要性的行为。其逻辑可以概括为:

        \begin{itemize}
            \item \textbf{Selective Inattention (选择性忽视):} 核心在于“已经看到或听到”,但主观上认为其不重要或不相关。 
            \item \textbf{Active Rejection (主动拒绝):} 不同于无意的疏忽(overlook),这通常是一种有意识的、有时甚至是傲慢的拒绝。 
            \item \textbf{Neutral Filtering (中性过滤):} 在专业指令中,它也代表清理冗余信息,以确保核心任务的执行。 
        \end{itemize}

        \item \textbf{多维语境下的语义表达}

        \begin{itemize} 
            \item \textbf{规则与权威(社会约束):} 
            
            描述对他人的建议、警告或法律条文的公然漠视。 
            
            \es{He drove with a total \textbf{disregard} for the safety of others.} (他开车时完全不顾及他人的安全。) 
            
            \item \textbf{修正与纠偏(信息处理):} 
            
            在正式通讯中,用于撤回前言或要求忽略错误信息。 
            
            \es{Please \textbf{disregard} my previous email; it was sent by mistake.} (请忽略我之前发的邮件,那是误发。) 
            
            \item \textbf{心理防御(情感状态):} 
            
            探讨通过无视外界评价来保持自我。 
            
            \es{To succeed, you must learn to \textbf{disregard} the critics and focus on your goal.} (为了成功,你必须学会无视批评者,专心于你的目标。) 
        \end{itemize}

        \item \textbf{语法进阶:用法强度与固定搭配}

        为了精准描述这种“忽视”的性质,常用以下变体:

        \begin{itemize}
            \item \textbf{修饰语与程度:} 
            \begin{itemize}
                \item \textit{Utter/Total disregard} (完全的漠视,语气最强。) 
                \item \textit{Blatant disregard} (公然的无视,常带有谴责色彩。) 
                \item \textit{Flagrant disregard} (悍然不顾,通常指对法律或规则的挑战。) 
            \end{itemize}

            \item \textbf{动词与介词搭配:} 
            \begin{itemize}
                \item \textit{Disregard for/of...} (对...的无视):注意其后常接抽象名词,如 safety, feelings, rules。 
                \item \textit{Show a disregard} (表现出漠视):侧重于通过行为流露出的态度。 
                \item \textit{Carefully disregard} (刻意忽略):带有目的性的筛选过程。 
            \end{itemize}
        
            \item \textbf{近义词辨析(Synonym Comparison):} 
            \begin{itemize}
                \item \textit{Ignore} (更通俗,可能包含没看见或不理睬)。 
                \item \textit{Neglect} (侧重于职责上的疏忽,带有负面后果)。 
                \item \textit{Overlook} (通常指无意的漏看,或宽容地略过细节)。 
            \end{itemize}
        \end{itemize}
    \end{enumerate}
\end{multicols}

\wordex{Convention}
\begin{multicols}{2}
    \begin{enumerate}
        \item \textbf{核心内涵:集体共识的交汇点}

        $Convention$ 源于拉丁语 *convenire*(聚集/开会/一致)。其逻辑可以概括为:

        \begin{itemize}
            \item \textbf{Collective Agreement (集体认同):} 不是由法律强制,而是基于社会成员长期达成的默契。
            \item \textbf{Formal Assembly (正式集会):} 核心在于人们为了特定目的“聚在一起”,衍生为大型会议。
            \item \textbf{Established Practice (既定惯例):} 在艺术或专业领域,指代那些被广泛接受的创作模式或操作标准。
        \end{itemize}

        \item \textbf{多维语境下的语义表达}

        \begin{itemize} 
            \item \textbf{社会习俗(文化感知):} 
            
            描述社会公认的行为准则。 
            
            \es{By \textbf{convention}, the bride wears white at her wedding.} (按照习俗,新娘在婚礼上穿白色婚纱。) 
            
            \item \textbf{政治与国际关系(正式契约):} 
            
            指国家间达成的正式条约或协定。 
            
            \es{The Geneva \textbf{Convention} establishes the standards of international law for humanitarian treatment.} (《日内瓦公约》确立了人道主义待遇的国际法标准。) 
            
            \item \textbf{行业与集会(专业领域):} 
            
            描述大规模的行业展览或政治代表大会。 
            
            \es{The city is hosting a comic book \textbf{convention} this weekend.} (这周末该市将举办一场动漫展。) 
        \end{itemize}

        \item \textbf{语法进阶:用法强度与修饰变体}

        为了精准描述这种“惯例”的状态,常用以下变体:

        \begin{itemize}
            \item \textbf{修饰语与程度:}
            \begin{itemize}
                \item \textit{Social/Cultural convention} (社会/文化习俗。)
                \item \textit{Literary/Artistic convention} (文学/艺术惯例。)
                \item \textit{Rigid convention} (僵化的习俗。)
                \item \textit{Defy/Break with convention} (打破常规。)
            \end{itemize}

            \item \textbf{动词搭配:}
            \begin{itemize}
                \item \textit{Follow/Adhere to convention} (遵循惯例):侧重于对传统的尊重或保守。
                \item \textit{Call a convention} (召开大会):侧重于组织行为。
                \item \textit{Govern by convention} (受惯例约束):描述一种非强制性但有效的统治方式。
            \end{itemize}
        
            \item \textbf{合成词与派生扩展:}
            \begin{itemize}
                \item \textit{Conventional} (形容词:传统的、常规的,如 conventional weapons 常规武器)。
                \item \textit{Unconventional} (形容词:不因循守旧的、标新立异的)。
            \end{itemize}
        \end{itemize}

        \item \textbf{Conventional vs. Traditional}
        
        虽然两者都翻译为“传统的”,但在语境和情感色彩上有着显著的区别:

        \begin{itemize}
            \item \textbf{Traditional}侧重于历史的传承,是由一代代人传下来的信仰、习俗或方法。强调\textbf{时间的深度}。如果一件事被称为 traditional,它通常关联着某种身份认同。通常是中性或积极的,带有文化尊重和怀旧感。常见语境包括节日、手工艺、家庭价值观、中医 (Traditional Chinese Medicine)。
            \item \textbf{Conventional}侧重于社会的共识,是当前大众普遍接受、认为“正常”的标准或做法。强调\textbf{群体的广度}。如果一件事被称为 conventional,是因为大家都这么做,它是“不出错”的选择。有时带有消极色彩,暗示平庸、缺乏创意或过于守旧。常见语境包括商业模式、思维方式、常规武器 (Conventional weapons)。
        \end{itemize}
    \end{enumerate}
\end{multicols}

\wordex{Notable}
\begin{multicols}{2}
    它涵盖了从感官察觉、社会地位到学术评价的多个维度,强调的是**“值得关注的卓越性”与“明显的差异”**

    \begin{enumerate}
        \item \textbf{核心内涵:跳出平庸的关注点}

        $Notable$ 源于动词 *note*(注意/记录)。其核心逻辑在于:某事物具备了足够的强度、质量或独特性,以至于让观察者觉得“有必要记录下来”。其逻辑可以概括为:

        \begin{itemize}
            \item \textbf{Worthy of Notice (值得注意):} 并非所有的存在都值得被记录,只有那些偏离了背景基准线的事物才被称为 $notable$。
            \item \textbf{Distinction in Quality (品质卓越):} 核心在于其优越性,通常与成功、才华或重要的历史节点挂钩。
            \item \textbf{Visual or Statistical Significance (显著性):} 在描述数据或外观时,强调那种一眼就能看出的、不可忽视的变化。
        \end{itemize}

        \item \textbf{多维语境下的语义表达}

        \begin{itemize} 
            \item \textbf{社会成就(人物评价):} 
            
            描述在特定领域有影响力的杰出人物。
            
            \es{The event was attended by many \textbf{notable} figures from the local community.} (许多当地社区的知名人士出席了这次活动。)
            
            \item \textbf{特征与属性(客观描述):} 
            
            强调某个事物最突出的优点或特点。
            
            \es{The most \textbf{notable} feature of the new building is its sustainable design.} (这座新建筑最引人注目的特点是其可持续设计。)
            
            \item \textbf{程度与变化(数据感知):} 
            
            描述明显的增长、减少或差异。
            
            \es{There has been a \textbf{notable} improvement in his performance this semester.} (这学期他的表现有了显著的进步。)
        \end{itemize}

        \item \textbf{语法进阶:用法强度与修饰变体}

        为了精准描述这种“值得注意”的程度,常用以下变体:

        \begin{itemize}
            \item \textbf{常用修饰语:}
            \begin{itemize}
                \item \textit{Particularly/Especially notable} (格外引人注目的。)
                \item \textit{A notable exception} (一个显著的例外,常用于打破规律的讨论。)
                \item \textit{Notable for something} (因某事而闻名/显著。)
            \end{itemize}

            \item \textbf{辨析对比(Synonym Nuances):}
            \begin{itemize}
                \item \textit{Notable} vs. \textit{Famous}: $Famous$ 侧重于被大众所知(名气),而 $notable$ 侧重于客观上值得被注意(质量或重要性)。
                \item \textit{Notable} vs. \textit{Noticeable}: $Noticeable$ 仅指“能被察觉到的”(比如墙上一个小划痕),而 $notable$ 则蕴含了“重要、卓越”的评价色彩。
            \end{itemize}
        
            \item \textbf{合成词与派生扩展:}
            \begin{itemize}
                \item \textit{Notability} (名词:显著、知名、值得关注的事物)。
                \item \textit{Notably} (副词:显著地;尤其/特别是,如 ...notably in the field of science)。
            \end{itemize}
        \end{itemize}
    \end{enumerate}
\end{multicols}

\wordex{Shrewd}
\begin{multicols}{2}
    它涵盖了从直觉洞察、商业手腕到性格评价的多个维度,强调的是\textbf{“敏锐的判断力”与“实际的利益导向”}

    \begin{enumerate}
        \item \textbf{核心内涵:剥离迷雾的判断力}

        $Shrewd$ 描述的是一种能够迅速看穿复杂局面、并识别出对自己最有利路径的能力。其逻辑可以概括为:

        \begin{itemize}
            \item \textbf{Astute Observation (精明的观察):} 核心在于能注意到他人忽视的微小细节,并预判其走向。
            \item \textbf{Practical Intelligence (务实智能):} 不同于纯粹的学术智慧(academic intelligence),这是一种更接地气的、“街头智慧”般的生存本能。
            \item \textbf{Calculation of Interest (利益权衡):} 隐含了一定程度的冷静甚至冷酷,行为动机往往是为了获取成功或避免损失。
        \end{itemize}

        \item \textbf{多维语境下的语义表达}

        \begin{itemize} 
            \item \textbf{商业与投资(决策属性):} 
            
            描述那些能精准捕捉市场时机、做出高回报决策的行为。
            
            \es{She made a \textbf{shrewd} investment in the tech sector just before the boom.} (在繁荣到来之前,她在科技领域进行了一项精明的投资。)
            
            \item \textbf{政治与社交(博弈特质):} 
            
            形容一个人善于审时度势,在复杂的人际或政治斗争中占据主动。
            
            \es{As a \textbf{shrewd} politician, he knows exactly when to compromise and when to stand firm.} (作为一名精明的政治家,他清楚地知道何时该妥协,何时该坚持。)
            
            \item \textbf{直觉与洞察(认知表现):} 
            
            指对事物本质的快速且准确的把握。
            
            \es{He is a \textbf{shrewd} judge of character; he saw through the salesman's lies immediately.} (他看人很准,一眼就看穿了推销员的谎言。)
        \end{itemize}

        \item \textbf{语法进阶:近义辨析与评价维度}

        为了精准区分“聪明”的不同侧面,常用以下对比:

        \begin{itemize}
            \item \textbf{近义词梯队 (Synonym Spectrum):}
            \begin{itemize}
                \item \textit{Wise}: 侧重于阅历深厚、有长远眼光且道德高尚。
                \item \textit{Clever}: 侧重于脑筋灵活、学得快(有时带有一点“小聪明”)。
                \item \textit{Shrewd}: 侧重于\textbf{务实、敏锐、尤其在商业和利益方面}。
                \item \textit{Cunning}: 侧重于\textbf{狡猾、欺骗性}(通常是负面评价)。
            \end{itemize}

            \item \textbf{常用搭配:}
            \begin{itemize}
                \item \textit{A shrewd move} (精明之举):强调行动的策略性。
                \item \textit{Shrewd business sense} (精明的商业头脑)。
                \item \textit{A shrewd suspicion} (一种敏锐的怀疑/预感):指基于迹象推断出的接近事实的猜测。
            \end{itemize}
        
            \item \textbf{派生词:}
            \begin{itemize}
                \item \textit{Shrewdness} (名词:精明,敏锐)。
                \item \textit{Shrewdly} (副词:精明地,敏锐地)。
            \end{itemize}
        \end{itemize}
    \end{enumerate}
\end{multicols}

\wordex{Eccentricity}
\begin{multicols}{2}
    \begin{enumerate}
        \item \textbf{核心内涵:偏离常轨的量化与表现}

        $Eccentricity$ 探讨的是一个系统或个体偏离“圆心”或“正轨”的距离。其逻辑可以概括为:

        \begin{itemize}
            \item \textbf{Geometric Deviation (几何偏心率):} 在数学中,它是一个数值,描述圆锥曲线(如椭圆)偏离圆形的程度。
            \item \textbf{Behavioral Singularity (行为奇特性):} 核心在于一种不落俗套、不随大流的独特性,通常伴随着极高的辨识度。
            \item \textbf{Social Tolerance (社会包容度):} 这个词通常带有中性偏褒义的色彩,暗示这种“古怪”源于天赋或独特的视野,而非病态。
        \end{itemize}



        \item \textbf{多维语境下的语义表达}

        \begin{itemize} 
            \item \textbf{天文学与数学(精确测量):} 
            
            描述行星轨道或几何图形的扁平化程度。 
            
            \es{The \textbf{eccentricity} of Earth's orbit is very small, meaning it is nearly a perfect circle.} (地球轨道的偏心率非常小,这意味着它近乎一个完美的圆。) 
            
            \item \textbf{人格特质(人文描述):} 
            
            形容那些生活方式独特、不被传统束缚的人的习惯。 
            
            \es{The artist was famous for his \textbf{eccentricity}, such as taking his pet lobster for walks.} (这位艺术家以他的古怪行为闻名,比如带着他的宠物龙虾去散步。) 
            
            \item \textbf{工程学(机械结构):} 
            
            指转动轴与几何中心的不重合度。 
            
            \es{Engineers must account for the \textbf{eccentricity} of the loading to prevent structural failure.} (工程师必须考虑荷载的偏心距,以防止结构失效。) 
        \end{itemize}

        \item \textbf{语法进阶:程度与修饰变体}

        为了精准描述这种“偏离”的性质,常用以下变体:

        \begin{itemize}
            \item \textbf{程度修饰:}
            \begin{itemize}
                \item \textit{A degree of eccentricity} (一定程度的古怪。)
                \item \textit{Endearing eccentricity} (讨人喜欢的古怪/萌点。)
                \item \textit{Zero eccentricity} (零偏心率/完美对称。)
            \end{itemize}

            \item \textbf{动词搭配:}
            \begin{itemize}
                \item \textit{Display/Exhibit eccentricity} (表现出古怪):侧重于外在观察到的行为。
                \item \textit{Tolerate eccentricity} (包容古怪):侧重于社会或环境对他人的接纳。
                \item \textit{Calculate the eccentricity} (计算偏心率):侧重于科学领域的定量分析。
            \end{itemize}
        
            \item \textbf{语感对比:}
            
            \textbf{Eccentricity} vs. \textbf{Oddity}: \textit{Eccentricity} 常指有思想、有风格的独特;而 \textit{Oddity} 更侧重于事物本身的奇怪或不协调。
        \end{itemize}
    \end{enumerate}
\end{multicols}

\wordex{Legendary}
\begin{multicols}{2}
    它涵盖了从神话传说、历史声望到现代通俗赞誉的多个维度,强调的是\textbf{“超越时代的叙事”与“极高的知名度”}。

    \begin{enumerate}
        \item \textbf{核心内涵:被传颂的卓越}

        $Legendary$ 源于 \textbf{legend}(传说/传奇)。其核心逻辑在于:一个人物或事物的表现如此出色,以至于它不再仅仅是一个事实,而成为了人们口耳相传的故事。其逻辑可以概括为:

        \begin{itemize}
            \item \textbf{Mythical Origins (神话起源):} 植根于古代民间传说,往往带有超自然或英雄主义色彩。
            \item \textbf{Enduring Fame (持久的声望):} 经受住了时间的考验,其名声在不同的世代间流传。
            \item \textbf{Hyperbolic Excellence (夸张的卓越):} 在现代语境下,常用来形容某种达到了顶峰、令人难以置信的技能或特质。
        \end{itemize}

        \item \textbf{多维语境下的语义表达}

        \begin{itemize} 
            \item \textbf{文学与民俗(虚构维度):} 
            
            描述那些存在于古老故事中、真假难辨的人物或地方。
            
            \es{King Arthur is a \textbf{legendary} figure in British history.} (亚瑟王是英国历史上的一位传奇人物。)
            
            \item \textbf{行业巅峰(现实声望):} 
            
            形容在某个领域拥有极高地位、被后人视为标杆的人物。
            
            \es{The \textbf{legendary} basketball player announced his retirement today.} (那位传奇篮球运动员今天宣布退役。)
            
            \item \textbf{现代口语(程度夸张):} 
            
            在非正式场合,用来形容非常出名或令人印象极深的事物。
            
            \es{The traffic jams in this city are \textbf{legendary}.} (这个城市的堵车是出了名的。)
        \end{itemize}

        \item \textbf{语法进阶:近义辨析与评价维度}

        为了精准区分“出名”的不同侧面,常用以下对比:

        \begin{itemize}
            \item \textbf{近义词梯队 (Synonym Spectrum):}
            \begin{itemize}
                \item \textit{Famous}: 最通用,仅表示“被很多人知道”。
                \item \textit{Iconic}: 侧重于具有“代表性”,成为了某种文化或符号的象征。
                \item \textit{Legendary}: 侧重于**“被传颂”**,带有某种神圣感或史诗感。
                \item \textit{Notorious}: 侧重于**“臭名昭著”**,因不好的事情而出名。
            \end{itemize}

            \item \textbf{常用搭配:}
            \begin{itemize}
                \item \textit{A legendary hero} (传奇英雄):强调其超越常人的功绩。
                \item \textit{Of legendary proportions} (极大的/传奇般的规模):形容程度非常惊人。
                \item \textit{Legendary hospitality} (出名的好客):形容某种品质名声在外。
            \end{itemize}
        
            \item \textbf{派生词:}
            \begin{itemize}
                \item \textit{Legend} (名词:传奇人物/传说)。
                \item \textit{Legendarily} (副词:传奇地/极其有名地)。
            \end{itemize}
        \end{itemize}
    \end{enumerate}
\end{multicols}

\wordex{Intensely}
\begin{multicols}{2}
    它涵盖了从物理能量、情感深度到感官知觉的多个维度,强调的是**“极高的浓度”与“强大的穿透力”**

    \begin{enumerate}
        \item \textbf{核心内涵:能量的极度聚焦}

        $Intensely$ 描述的是某种性质、情感或力量达到了极高的等级或密度的状态。其逻辑可以概括为:

        \begin{itemize}
            \item \textbf{High Concentration (高浓度):} 核心不在于广度(如 $extensively$),而在于深度。它像一束激光,将所有能量集中在一点。
            \item \textbf{Emotional Depth (情感深度):} 形容主观感受极其强烈,往往伴随着生理上的紧绷感或心理上的震颤。
            \item \textbf{Sensory Sharpness (感官敏锐):} 用于描述光线、气味或声音等物理刺激对感官产生的巨大冲击。
        \end{itemize}

        \item \textbf{多维语境下的语义表达}

        \begin{itemize} 
            \item \textbf{心理与情感(主观状态):} 
            
            描述内心活动的剧烈程度。 
            
            \es{She was \textbf{intensely} jealous of her sister's success.} (她对姐姐的成功感到极度嫉妒。) 
            
            \item \textbf{感官体验(客观刺激):} 
            
            描述外界环境带来的强烈冲击。 
            
            \es{The desert sun shone \textbf{intensely} throughout the day.} (沙漠里的太阳整天都在剧烈地照射。) 
            
            \item \textbf{专注与投入(行为表现):} 
            
            描述在进行某项任务时的高度集中。 
            
            \es{The two scientists collaborated \textbf{intensely} to solve the problem before the deadline.} (两位科学家为了在截止日期前解决问题,进行了紧张而密切的合作。) 
        \end{itemize}

        \item \textbf{语法进阶:程度强弱与辨析变体}

        为了精准描述这种“强烈”的状态,常用以下变体:

        \begin{itemize}
            \item \textbf{常用修饰搭配:}
            \begin{itemize}
                \item \textit{Intensely private} (极其隐秘的/注重隐私的。)
                \item \textit{Intensely hot/cold} (极度炎热或寒冷。)
                \item \textit{Intensely competitive} (竞争极其激烈的。)
                \item \textit{Stare intensely} (死死地盯着/目不转睛地看。)
            \end{itemize}

            \item \textbf{近义辨析 (Synonym Nuances):}
            \begin{itemize}
                \item \textit{Intensely} vs. \textit{Strongly}: $Strongly$ 侧重于力量或影响(如 *strongly recommend*),而 $intensely$ 侧重于内心的“热度”或“纯度”。
                \item \textit{Intensely} vs. \textit{Extremely}: $Extremely$ 只是简单的“非常”,而 $intensely$ 往往带有一种**紧张、严肃或情感波动**的氛围感。
            \end{itemize}
        
            \item \textbf{派生扩展:}
            \begin{itemize}
                \item \textit{Intensity} (名词:强度、剧烈、热烈)。
                \item \textit{Intensify} (动词:加强、加剧)。
                \item \textit{Intensive} (形容词:加强的、集中的,如 an intensive course 强化课程)。
            \end{itemize}
        \end{itemize}
    \end{enumerate}
\end{multicols}

\wordex{Snob}
\begin{multicols}{2}
    它涵盖了从社会阶层、审美优越感到现代职场偏见的多个维度,强调的是**“基于虚荣的排他性”与“自上而下的审视”**

    \begin{enumerate}
        \item \textbf{核心内涵:优越感的错位与投射}

        $Snob$ 描述的是一种通过贬低他人或模仿更高阶层来确立自我价值的人。其逻辑可以概括为:

        \begin{itemize}
            \item \textbf{Social Hierarchy (社会等级观):} 极度看重社会地位、财富或头衔,并以此作为衡量人的唯一标准。
            \item \textbf{Exclusivity (排他性):} 核心在于“区分”——通过展示某种昂贵或小众的品味,将自己与“普通人”区别开来。
            \item \textbf{Insecurity (内在不安全感):} 心理学认为,大多数势利行为源于对被边缘化的恐惧,通过俯视他人来获得暂时的安全感。
        \end{itemize}

        \item \textbf{多维语境下的语义表达}

        \begin{itemize} 
            \item \textbf{阶层势利(传统语境):} 
            
            描述那些巴结权贵、看不起社会地位较低者的人。
            
            \es{He is such a \textbf{snob} that he only associates with people who have Ivy League degrees.} (他是个十足的势利小人,只跟那些拥有常春藤学位的人交往。)
            
            \item \textbf{品味/知识势利(现代语境):} 
            
            描述在特定领域(如红酒、艺术、咖啡)因拥有专业知识而产生的傲慢。
            
            \es{I’m a bit of a wine \textbf{snob}, so I can’t stand drinking the cheap stuff at parties.} (我对红酒有点挑剔,受不了在派对上喝那些廉价货。)
            
            \item \textbf{职场偏见(社会心理):} 
            
            探讨根据职位高低来决定礼貌程度的行为。
            
            \es{Her \textbf{snobbish} attitude toward the interns made her very unpopular in the office.} (她对实习生那副势利的神态使她在办公室里非常不受欢迎。)
        \end{itemize}

        \item \textbf{语法进阶:评价色彩与辨析变体}

        为了精准描述这种“势利”的性质,常用以下变体:

        \begin{itemize}
            \item \textbf{常见合成词:}
            \begin{itemize}
                \item \textit{Intellectual snob} (知识分子式的势利,看不起学历低的人。)
                \item \textit{Inverse snobbery} (反向势利:指故意表现得草根,反而看不起有钱人或受过高深教育的人。)
                \item \textit{A coffee/movie snob} (咖啡/电影鉴赏方面的挑剔者,带有某种自嘲色彩。)
            \end{itemize}

            \item \textbf{近义词辨析 (Synonym Nuances):}
            \begin{itemize}
                \item \textit{Snob} vs. \textit{Elitist}: $Elitist$ (精英主义者) 认为社会应由少数优秀的人领导,偏向政治/社会理念;而 $snob$ 更侧重于个人态度上的傲慢。
                \item \textit{Snobbish} vs. \textit{Arrogant}: $Arrogant$ (傲慢) 是由于自大,而 $snobbish$ (势利) 必然涉及对他人的**社会性比较**。
            \end{itemize}
        
            \item \textbf{派生扩展:}
            \begin{itemize}
                \item \textit{Snobbery} (名词:势利的行为或态度)。
                \item \textit{Snobbish} (形容词:势利的)。
                \item \textit{Snobbishly} (副词:势利地)。
            \end{itemize}
        \end{itemize}
    \end{enumerate}
\end{multicols}

\wordex{Bedraggle}
\begin{multicols}{2}
    它捕捉的是一种\textbf{“从有序到狼狈”}的视觉转化,强调的是\textbf{“由于水或泥造成的湿乱”}

    \begin{enumerate}
        \item \textbf{核心内涵:被环境摧残的凌乱感}

        $Bedraggle$ 的核心在于“被动性”与“潮湿的混乱”。它描述的事物原本是整洁、干燥或体面的,但因为暴露在雨水、泥泞或长途跋涉中而变得不堪入目。其逻辑可以概括为:

        \begin{itemize}
            \item \textbf{Saturation & Weight (浸透与重感):} 强调纤维(头发、衣服、羽毛)吸饱了水分,失去了原有的蓬松或形状。
            \item \textbf{Loss of Dignity (尊严的流失):} 视觉上带有一种颓丧、狼狈甚至令人同情的色彩。
            \item \textbf{External Cause (外部致因):} 这种混乱并非天生,而是由于环境(通常是恶劣天气)强加的结果。
        \end{itemize}

        \item \textbf{多维语境下的语义表达}

        \begin{itemize} 
            \item \textbf{视觉形象(外观描述):} 
            
            描述在大雨中行走后,衣服或头发贴在身上的样子。
            
            \es{After the storm, his expensive suit was completely \textbf{bedraggled}.} (风暴过后,他那昂贵的西装变得破烂不堪且湿漉漉的。)
            
            \item \textbf{生物观察(自然状态):} 
            
            形容淋雨后羽毛不再顺滑的鸟类或毛发打结的流浪动物。
            
            \es{A \textbf{bedraggled} pigeon sat on the windowsill, shivering from the cold.} (一只浑身湿透、羽毛凌乱的鸽子守在窗台上,冻得瑟瑟发抖。)
            
            \item \textbf{情绪比喻(心理延伸):} 
            
            形容一个人像被淋透的落汤鸡一样,不仅外表乱,精神也极度疲惫。
            
            \es{She returned from the protest looking tired and \textbf{bedraggled}.} (她从抗议现场回来,看起来疲惫不堪且狼狈不已。)
        \end{itemize}

        \item \textbf{语法进阶:用法变体与近义辨析}

        为了精准区分“弄湿”的不同程度,常用以下对比:

        \begin{itemize}
            \item \textbf{常用词性(最常用其形容词形式 bedraggled):}
            \begin{itemize}
                \item \textit{A bedraggled appearance} (狼狈的外表。)
                \item \textit{To look bedraggled} (看起来像落汤鸡。)
            \end{itemize}

            \item \textbf{近义辨析 (Synonym Nuances):}
            \begin{itemize}
                \item \textit{Bedraggled} vs. \textit{Soaked/Drenched}: $Soaked$ 只强调“水多”,而 $bedraggled$ 强调“\textbf{水多且看起来很乱、很脏}”。
                \item \textit{Bedraggled} vs. \textit{Untidy}: $Untidy$ 只是单纯的不整洁,不一定和水、泥有关。
            \end{itemize}
        
            \item \textbf{搭配艺术:}
            \begin{itemize}
                \item \textit{Bedraggled by the rain} (被雨淋得狼狈不堪。)
                \item \textit{A bedraggled group of refugees} (一群衣衫褴褛、神情憔悴的难民。)
            \end{itemize}
        \end{itemize}
        \end{enumerate}
\end{multicols}

\wsitem{Presented sb. with sth.}
\begin{multicols}{2}
    这个表达是英语中非常经典的\textbf{“授予/呈现”句型。它不仅仅是“给”,通常带有一定的正式感或仪式感}。

    $Present\ sb.\ with\ sth.$ 的核心逻辑在于\textbf{“交互的方向性”}。它强调的是\textbf{“将某物置于某人面前供其接受或处理”}。

    \begin{enumerate}
        \item \textbf{核心语法结构}

        \begin{itemize}
            \item \textbf{及物动词:} $Present$ (注意重音在后 [prɪˈzent])。
            \item \textbf{介词固定搭配:} 必须使用 \textit{with}。
            \item \textbf{同义转换:} $Present\ sth.\ to\ sb.$ (注意:如果物在前,介词要换成 \textit{to})。
        \end{itemize}

        \item \textbf{三种主要语义语境}

        \begin{itemize} 
            \item \textbf{颁奖与赠送 (Gifting/Awarding):} 
            
            带有仪式感的给予。
            
            \es{The committee \textbf{presented her with} a gold medal.} (委员会授予她一枚金牌。)
            
            \item \textbf{面临挑战或问题 (Facing/Tasking):} 
            
            指环境或人使某人不得不应对某种状况。
            
            \es{The new job \textbf{presented him with} many unexpected challenges.} (这份新工作让他面临许多意想不到的挑战。)
            
            \item \textbf{出示证据或文件 (Submitting):} 
            
            正式递交。
            
            \es{The lawyer \textbf{presented the witness with} the leaked documents.} (律师向证人出示了泄露的文件。)
        \end{itemize}

        \item \textbf{词义辨析:Present vs. Give}
        \begin{itemize}
            \item Give,语气通俗、日常,典型对象为任何物品 (钱、书、建议)
            \item Present,语气正式、庄重,典型对象为奖项、机会、难题、证据
        \end{itemize}
    \end{enumerate}
\end{multicols}

\newpage