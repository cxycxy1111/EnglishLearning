\section{Lesson 23 One man's meat is another man's poison}

\begin{paracol}{2}

People become quite \nw{illogical} when they try to decide what can be eaten and what cannot be eaten. 

\switchcolumn

\chinesetext{在决定什么能吃而什么不能吃的时候,人们往往变得不合情理。}
\nwe{illogical}{ɪˈlɑdʒɪkl}{adj. 不合逻辑的;无意义的;}

\switchcolumn*

If you lived in the Mediterranean, \ns{for instance}, you would consider \nw{octopus} a great \nw{delicacy}. 

\switchcolumn

\chinesetext{比如,如果你住在地中海地区,你会把章鱼视作是美味佳肴。}
\nwe{octopus}{ˈɑktəpəs}{n. 章鱼;}
\nwe{delicacy}{ˈdelɪkəsi}{n. 精美,娇嫩,雅致;微妙;棘手;周到;佳肴;}
\nse{for instance}{fɔr ˈɪnstəns}{例如;比如;}

\switchcolumn*

You would not be able to understand why some people find it \nw{repulsive}. 

\switchcolumn

\chinesetext{同时不能理解为什么有人一见章鱼就恶心。}
\nwe{repulsive}{rɪˈpʌlsɪv}{adj. 令人厌恶的;相斥的;可憎的;}

\switchcolumn*

\ns{On the other hand}, your \nw{stomach} would turn \ns{at the idea of} \nw{frying} potatoes in animal fat -- the normally accepted practice in many northern countries. 

\switchcolumn

\chinesetext{另一方面,你一想到动物油炸土豆就会反胃,但这在北方许多国家却是一种普通的烹任方法。}
\nwe{stomach}{ˈstʌmək}{n. 腹部;胃;v. 欣赏;吃得下;}
\nwe{fry}{fraɪ}{n. 鱼苗,鱼秧;弗赖伊(姓氏);油炸食物;(口语)烦恼,愤激;vi. 用油煎;用油炸烤炒;油炸;vt. 油炸;油煎;(美俚)使被处电刑;瓦解;}
\nse{on the other hand}{ɑːn ðə ˈʌðər hænd}{在另一方面;}
\nse{at the idea of}{æt ði aɪˈdiə ʌv}{一想起…就;}

\switchcolumn*

The sad truth is that most of us \ns{have been brought up} to eat certain foods and we \ns{stick to} them all our lives.

\switchcolumn

\chinesetext{不无遗憾的是,我们中的大部分人,生来就只吃某几种食品,而且一辈子都这样。}
\nse{be brought up}{}{养育,抚养}
\nse{stick to}{}{坚持}

\switchcolumn*

No creature has received more \nw{praise} and \nw{abuse} than the common garden \nw{snail}. 

\switchcolumn

\chinesetext{没有一种生物所受到的赞美和厌恶会超过花园里常见的蜗牛了。}
\nwe{praise}{preɪz}{v. 称赞;(对上帝)称颂;n. 称赞;(对上帝)敬拜;}
\nwe{abuse}{əˈbjuːs , əˈbjuːz}{n. 虐待,辱骂;滥用;v. 滥用;虐待,辱骂;}
\nwe{snail}{sneɪl}{n. 蜗牛;慢性子;}

\switchcolumn*

Cooked in wine, snails are a great \nw{luxury} in various parts of the world. 

\switchcolumn

\chinesetext{蜗牛加酒烧煮后,便成了世界上许多地方的一道珍奇的名菜。}
\nwe{luxury}{ˈlʌkʃəri}{n. 奢侈,豪华;奢侈品,美食,美衣;乐趣,享受;不常有的乐趣(或享受、优势);}

\switchcolumn*

There are \nw{countless} people who, \ns{ever since} their early years, have learned to \nw{associate} snails with food. 

\switchcolumn

\chinesetext{有不计其数的人们从小就知道蜗牛可做菜。}
\nwe{countless}{ˈkaʊntləs}{adj. 无数的,多得数不清的;恒河沙数;指不胜屈;数不胜数;}
\nwe{associate}{v.美[əˈsoʊsieɪt , əˈsəʊʃieɪt]n. adj.美[əˈsoʊsiət]}{v. 联想,联系;使与(某个组织、事业或观点)有关系;公开支持;与…交往;n. 同事,伙伴;合伙人;adj. (表示头衔)副的,准的;有关联的;}
\nse{ever since}{ˈevər sɪns}{adv. 从那时到现在; 自从;自…以后;从…起;}

\switchcolumn*

My friend, Robert, lives in a country where snails are \nw{despised}. 

\switchcolumn

\chinesetext{但我的朋友罗伯特却住在一个厌恶蜗牛的国家中。}
\nwe{despise}{dɪˈspaɪz}{v. 鄙视,看不起;}

\switchcolumn*

As his flat is in a large town, he has no garden of his own. 

\switchcolumn

\chinesetext{他住在大城市里的一所公寓里,没有自己的花园。}

\switchcolumn*

\ns{For years} he has been asking me to collect snails from my garden and take them to him. 

\switchcolumn

\chinesetext{多年来,他一直让我把我园子里的蜗牛收集起来给他捎去。}
\nse{for years}{}{多年以来,好多年;累月经年;}

\switchcolumn*

The idea never \ns{\nw{appealed} to} me very much, but one day, after \ns{heavy \nw{shower}}, I \ns{happened to} be walking in my garden when I noticed \ns{a huge number of} snails \ns{taking a \nw{stroll} on} some of my prize plants. 

\switchcolumn

\chinesetext{一开始,他的这一想法没有引起我多大兴趣。后来有一天,一场大雨后,我在花园里漫无目的散步,突然注意到许许多多蜗牛在我的一些心爱的花木上慢悠悠的蠕动着。}
\nwe{appeal}{əˈpiːl}{v. 呼吁,恳请;上诉,申诉;对…有吸引力;劝说;n. 呼吁,恳求;募捐;上诉,申诉;吸引力,感染力;打动;}
\nwe{shower}{ˈʃaʊər}{n. 淋浴间;沐浴;一阵;送礼聚会;v. 洒落;大量地给;(洗)淋浴;}
\nwe{stroll}{stroʊl}{n. 闲逛;漫步;vi. 散步;奔波;vt. 溜达;}
\nse{appeal to}{əˈpiːl tu}{有吸引力;有感染力;呼吁;上诉;打动;}
\nse{happen to}{ˈhæpən tu}{发生在…身上;碰巧;赶巧;值;}
\nse{take a stroll on ...}{}{在...散步}
\nse{a huge number of}{}{大量}
\nse{heavy shower}{}{阵雨}

\switchcolumn*

\ns{Acting on} a sudden \nw{impulse}, I collected several dozen, put them in a paper bag, and took them to Robert. 

\switchcolumn

\chinesetext{我一时冲动,逮了几十只,装进一只纸袋里,带着去找罗伯特。}
\nwe{impulse}{ˈɪmpʌls}{n. 凭冲动行事;突如其来的念头;[电子]脉冲;[医]冲动,搏动;}
\nse{act on}{}{根据……行事;遵照……起作用}

\switchcolumn*

Robert was delighted to see me and equally pleased with my little gift. 

\switchcolumn

\chinesetext{罗伯特见到我很高兴,对我的薄礼也感到满意。}

\switchcolumn*

I left the bag in the hall and Robert and I went into the living room where we talked for \ns{a couple of hours}. 

\switchcolumn

\chinesetext{我把纸袋放在门厅里,与罗伯特一起进了起居室,在那里聊了好几个钟头。}
\nse{a couple of hours}{}{好几个钟头}

\switchcolumn*

I had forgotten all about the snails when Robert suddenly said that I must stay to dinner. 

\switchcolumn

\chinesetext{我把蜗牛的事已忘得一干二净,罗伯特突然提出一定要我留下来吃晚饭,这才提醒了我。}

\switchcolumn*

Snails would, of course, be the main dish. 

\switchcolumn

\chinesetext{蜗牛当然是道主菜。}

\switchcolumn*

I did not \nw{fancy} the idea and I reluctantly followed Robert out of the room. 

\switchcolumn

\chinesetext{我并不喜欢这个主意,所以我勉强跟着罗伯特走进了起居室。}
\nwe{fancy}{ˈfænsi}{v. 想要;爱慕;自命不凡;竟然;认为…会赢;n. 想象;喜好;幻想;adj. 太花哨的;精致的;复杂的;昂贵的;优质的;}

\switchcolumn*

To our dismay, we saw that there were snails everywhere: they had escaped from the paper bag and had \ns{taken complete possession of} the hall! 

\switchcolumn

\chinesetext{使我们惊愕的是门厅里到处爬满了蜗牛:它们从纸袋里逃了出来,爬得满厅都是!}
\nse{take possession of}{}{占有,拥有}

\switchcolumn*

I have never been able to look at a snail \ns{since then}.

\switchcolumn

\chinesetext{从那以后,我再也不能看一眼蜗牛了。}
\nse{since then}{}{从哪以后}

\switchcolumn*

\end{paracol}

%People become quite illogical when they try to decide what can be eaten and what cannot be eaten. If you lived in the Mediterranean, for instance, you would consider octopus a great delicacy. You would not be able to understand why some people find it repulsive. On the other hand, your stomach would turn at the idea of frying potatoes in animal fat -- the normally accepted practice in many northern countries. The sad truth is that most of us have been brought up to eat certain foods and we stick to them all our lives.
%No creature has received more praise and abuse than the common garden snail. Cooked in wine, snails are a great luxury in various parts of the world. There are countless people who, ever since their early years, have learned to associate snails with food. My friend, Robert, lives in a country where snails are despised. As his flat is in a large town, he has no garden of his own. For years he has been asking me to collect snails from my garden and take them to him. The idea never appealed to me very much, but one day, after heavy shower, I happened to be walking in my garden when I noticed a huge number of snails taking a stroll on some of my prize plants. Acting on a sudden impulse, I collected several dozen, put them in a paper bag, and took them to Robert. Robert was delighted to see me and equally pleased with my little gift. I left the bag in the hall and Robert and I went into the living room where we talked for a couple of hours. I had forgotten all about the snails when Robert suddenly said that I must stay to dinner. Snails would, of course, be the main dish. I did not fancy the idea and I reluctantly followed Robert out of the room. To our dismay, we saw that there were snails everywhere: they had escaped from the paper bag and had taken complete possession of the hall! I have never been able to look at a snail since then.
%在决定什么能吃而什么不能吃的时候,人们往往变得不合情理。比如,如果你住在地中海地区,你会把章鱼视作是美味佳肴,同时不能理解为什么有人一见章鱼就恶心。另一方面,你一想到动物油炸土豆就会反胃,但这在北方许多国家却是一种普通的烹任方法。不无遗憾的是, 我们中的大部分人,生来就只吃某几种食品,而且一辈子都这样。
%没有一种生物所受到的赞美和厌恶会超过花园里常见的蜗牛了。蜗牛加酒烧煮后,便成了世界上许多地方的一道珍奇的名菜。有不计其数的人们从小就知道蜗牛可做菜。但我的朋友罗伯特却住在一个厌恶蜗牛的国家中。他住在大城市里的一所公寓里,没有自己的花园。多年来,他一直让我把我园子里的蜗牛收集起来给他捎去。一开始,他的这一想法没有引起我多大兴趣。后来有一天,一场大雨后,我在花园里漫无目的散步,突然注意到许许多多蜗牛在我的一些心爱的花木上慢悠悠的蠕动着。我一时冲动,逮了几十只,装进一只纸袋里,带着去找罗伯特。罗伯特见到我很高兴,对我的薄礼也感到满意。我把纸袋放在门厅里,与罗伯特一起进了起居室,在那里聊了好几个钟头。我把蜗牛的事已忘得一干二净,罗伯特突然提出一定要我留下来吃晚饭,这才提醒了我。蜗牛当然是道主菜。我并不喜欢这个主意,所以我勉强跟着罗伯特走进了起居室。使我们惊愕的是门厅里到处爬满了蜗牛:它们从纸袋里逃了出来,爬得满厅都是!从那以后,我再也不能看一眼蜗牛了。


\retellingpoints

\begin{multicols}{1}
    \begin{enumerate} 
        \item \textbf{Cultural Perspectives on Food} 
        \begin{itemize} 
            \item \textbf{Illogical:} Not sensible or reasonable (describing food choices). 
            \item \textbf{Great delicacy:} A choice or expensive food. 
            \item \textbf{Repulsive:} Causing intense distaste or disgust. 
            \item \textbf{Stomach would turn:} To feel sick because of something unpleasant. 
            \item \textbf{Stick to them:} To continue doing or eating something without changing. 
        \end{itemize}

        \item \textbf{The Controversial Snail}
        \begin{itemize}
            \item \textbf{Praise and abuse:} Positive and negative comments/treatment.
            \item \textbf{The common garden snail:} The type of snail found in ordinary gardens.
            \item \textbf{Associate snails with food:} To connect the idea of snails with eating.
            \item \textbf{Despised:} Deeply hated or looked down upon.
        \end{itemize}

        \item \textbf{The Collection}
        \begin{itemize}
            \item \textbf{After a heavy shower:} After it has rained hard (when snails come out).
            \item \textbf{Taking a stroll:} Walking in a relaxed way (humorous description of snails).
            \item \textbf{Prize plants:} Highly valued or best plants in a garden.
            \item \textbf{Acting on a sudden impulse:} Doing something without thinking it through first.
            \item \textbf{Several dozen:} A large number (1 dozen = 12).
        \end{itemize}

        \item \textbf{The Incident}
        \begin{itemize}
            \item \textbf{Equally pleased:} Just as happy with the gift as he was to see the friend.
            \item \textbf{In the hall:} In the entrance area of the flat.
            \item \textbf{Stay to dinner:} An invitation to eat the evening meal.
            \item \textbf{Did not fancy the idea:} Did not like or want to do something.
            \item \textbf{Reluctantly:} Unwillingly; hesitantly.
        \end{itemize}

        \item \textbf{The "Takeover"}
        \begin{itemize}
            \item \textbf{To our dismay:} A feeling of unhappiness and disappointment.
            \item \textbf{Escaped from the paper bag:} Got out of the container.
            \item \textbf{Taken complete possession of:} To take control of or occupy an entire space.
            \item \textbf{Snails everywhere:} Highlighting the chaos in the hall.
        \end{itemize}
    \end{enumerate}
\end{multicols}

\grammarpoints

\wsitem{Attract vs. Appeal}
\begin{multicols}{1}
    \begin{enumerate} 
        \item \textbf{核心逻辑的差异 (Mechanism)} 
        \begin{itemize} 
            \item \textbf{Attract}: 侧重于“拉力” (Pull)。它通常描述一种物理上的、直接的、或是外部特征引发的吸引力。 
            
            \es{Magnets \textbf{attract} iron. (磁铁吸引铁。) }
            
            \es{Her bright dress \textbf{attracted} everyone's attention. (她的亮色裙子吸引了所有人的注意。) }
            
            \item \textbf{Appeal}: 侧重于“感染力” (Catering)。它通常描述一种内在的、心理上的、或是审美上的契合。它意味着某物“合某人的心意”。 
            
            \es{The idea of living in the countryside \textbf{appeals} to me. (乡村生活的想法很吸引我/合我心意。) }
        \end{itemize}

        \item \textbf{语法结构的限制}
        \begin{itemize}
            \item \textbf{Attract}: 是及物动词,直接加宾语。
            \begin{itemize}
                \item \textbf{A attracts B}. (A 吸引了 B。)
            \end{itemize}
            \item \textbf{Appeal}: 是不及物动词,必须接介词 \textbf{to} 才能加人。
            \begin{itemize}
                \item \textbf{A appeals to B}. (A 对 B 有吸引力。)
            \end{itemize}
        \end{itemize}

        \item \textbf{时间跨度的不同}
        \begin{itemize}
            \item \textbf{Attract}: 往往是\textbf{瞬间的、表面性的}。比如被广告吸引、被噪音吸引。
            \item \textbf{Appeal}: 往往是\textbf{长期的、深层次的}。它涉及个人的价值观、品味和长期的兴趣。
        \end{itemize}

        \item \textbf{那为什么是attract one's attention,不是appeal to one's attention呢?}
        
        这是一个非常敏锐的语义逻辑问题。要回答这个问题,我们需要从\textbf{“注意力(Attention)的性质”以及这两个动词的“动作方向”}来剖析。

        \begin{itemize} 
            \item \textbf{动作的方向性 (The Direction of Action)} \begin{itemize} 
                \item \textbf{Attract}: 是一个“向心力”。当一个物体 \textbf{attracts} 你的注意力时,它像磁铁一样把你的目光从别处“拽”了过去。这通常是一个被动的过程——你本来在看书,突然的一声巨响 \textbf{attracted} 了你的注意。 
                \item \textbf{Appeal}: 是一个“契合度”。它描述的是某物符合你的审美、价值观或兴趣。 
            \end{itemize}

            \item \textbf{“Attention” 是什么?}
            \begin{itemize}
                \item \textbf{Attention} 在英语语境中被视为一种“资源”或“能量”。它可以被\textbf{捕获 (catch)}、\textbf{吸引 (attract)}、\textbf{转移 (distract)} 或\textbf{占据 (occupy)}。
                \item 因为注意力是向外投射的,所以我们需要一个能表达“强行拉动”的词。而 \textbf{Appeal to} 侧重于内心的某种情感,我们通常不会说“某物符合我的注意力的品味”。
            \end{itemize}

            \item \textbf{“Appeal to” 后面接的是什么?}
            \begin{itemize}
                \item \textbf{Appeal to} 后面通常接的是具备“判断力”或“感受力”的对象:
                \begin{itemize}
                    \item \textit{Appeal to one's \textbf{emotions}} (诉诸情感)
                    \item \textit{Appeal to one's \textbf{sense of humor}} (符合某人的幽默感)
                    \item \textit{Appeal to one's \textbf{reason}} (诉诸理智)
                \end{itemize}
                \item \textbf{Attention} 本身不具备判断力,它只是一个“聚光灯”。你可以吸引聚光灯,但你不能“符合”聚光灯的品味。
            \end{itemize}
            \end{itemize}

            形象化对比 (Visual Analogy)
            我们可以把 Attention 想象成一个人的目光:

            \begin{itemize} \item \textbf{Attract one's attention}: 就像你放了一个烟花,强行让那个人的头转过来盯着你看。

            \item \textbf{Appeal to someone}: 就像你展示了一幅美丽的画,那个人看了一会儿,心里觉得“嗯,这幅画真不错,我挺喜欢的”。
        \end{itemize}
    \end{enumerate}
\end{multicols}

\wsitem{Act on a sudden impulse}
\begin{multicols}{1}
    这是一个非常地道的短语,属于\textbf{“动词 + 介词 + 名词”}构成的固定习语。它生动地描述了那种“不假思索、随性而为”的行为方式。

    \begin{enumerate} 
        \item \textbf{词组结构拆解} 
        \begin{itemize} 
            \item \textbf{Act on}: 意为“根据……行事”或“遵照……起作用”。 
            \item \textbf{Impulse}: 名词,指“冲动”或“一时的念头”。 
            \item \textbf{合体含义}:指受到某种突然念头的驱使而立即采取行动。 
        \end{itemize}

        \item \textbf{语法功能:分词短语作状语}
        \begin{itemize}
            \item 在你给出的句子中,它以 \textit{Acting on...} 的现在分词形式开头。
            \item \textbf{功能}:作为全句的**背景/原因状语**,解释了“我”为什么要收集那几十个东西——是因为那一瞬间的冲动。
        \end{itemize}

        \item \textbf{为什么用 “on”?}
        \begin{itemize}
            \item 介词 \textit{on} 在这里表示“依据”或“基础”。
            \item \textbf{类比}:正如 \textit{act on advice}(根据建议行事)、\textit{act on information}(根据情报行动)。这里是把“冲动”作为行动的依据。
        \end{itemize}

        \item \textbf{常见搭配变体}
        \begin{itemize}
            \item \textbf{On (an) impulse}: 意为“冲动地”。例如:\textit{I bought the dress \textbf{on impulse}.}(我一时冲动买了那条裙子。)
            \item \textbf{A sudden impulse}: 强调这种念头来的非常快。
            \item \textbf{Resist the impulse}: 抑制住冲动(反义表达)。
        \end{itemize}
    \end{enumerate}
\end{multicols}

\wsitem{The sad truth is that ...}
\begin{multicols}{1}
    这又是一个非常地道的引导性骨架句型,通常被称为评价性前置短语 (Evaluative Fronting) 或 主语从句的先行结构。

    它不仅仅是在陈述事实,更是在陈述事实之前先定下了情感基调。

    \begin{enumerate} 
        \item \textbf{句式结构拆解} 
        \begin{itemize} 
            \item \textbf{公式}:The + 形容词 (sad/plain/honest) + truth + is + that + [事实陈述]。 
            \item \textbf{语法功能}:The sad truth 是主语,is 是系动词,that 引导的是一个表语从句,解释“真相”的具体内容。 
        \end{itemize}
        \item \textbf{为什么这个句式好用?}
        \begin{itemize}
            \item \textbf{预设情感}:在说出事实之前,先告诉读者/听众你对这个事实的态度(比如觉得很无奈、很讽刺、很遗憾)。
            \item \textbf{吸引注意力}:比起直接说 "A is B",使用这种铺垫能起到“话语转折”的作用,引导听众关注接下来的核心信息。
        \end{itemize}

        \item \textbf{常见的同类变体}
        \begin{itemize}
            \item \textbf{The plain truth is that...} (事实很明显……/开诚布公地说……)
            \item \textbf{The frightening truth is that...} (可怕的事实是……)
            \item \textbf{The simple truth is that...} (事实很简单……)
        \end{itemize}

        \item \textbf{在复述与写作中的应用}
        \es{\textbf{The sad truth is that} despite his great success, Mendoza died in poverty.}
        \es{\textbf{The simple truth is that} most people are afraid of what they do not understand.}
    \end{enumerate}
\end{multicols}

\wsitem{No sth do sth. more sth than sth}
\begin{multicols}{1}
    这是一个非常经典的最高级替代句型。

    \begin{enumerate}
        \item \textbf{句式结构拆解}
        \begin{itemize}
            \item \textbf{公式}:No + 名词(A) + 动词(do sth.) + more + 副词/形容词 + than + 目标对象(B)。
            \item \textbf{核心逻辑}:通过全面否定其他对象(No A),来无限抬高目标对象(B)。
            \item \textbf{语义等值}:No one runs faster than him. $\rightarrow$ He runs the fastest.
        \end{itemize}
        \item \textbf{常见的变形与用法}
        \begin{itemize}
            \item \textbf{修饰动作的程度}:No one \textbf{inspired} me more \textbf{than} my teacher.(没有人比我的老师更激励我。)
            \item \textbf{修饰情感或认知}:No animal \textbf{frightens} people more \textbf{than} the great white shark.(没有什么动物比大白鲨更让人恐惧。)
        \end{itemize}

        \item \textbf{为什么它是“高级”句式?}
        
        \textbf{强调对比}:直接说最高级(The best/The most)只是陈述事实,而这个句式在陈述事实的同时,完成了一次“全场扫射”般的对比,语气更加果断、自信。
        
        \textbf{修辞力量}:在文学和演说中,这种句式能产生更强的节奏感和感染力。

        \item \textbf{在复述与写作中的应用}
        
        \es{场景:Mendoza 的影响:No boxer \textbf{changed} the nature of the sport more \textbf{than} Daniel Mendoza.}
        
        \es{场景:对章鱼的偏见:No creature \textbf{suffers} from more misunderstanding \textbf{than} the octopus.}

        \item \textbf{进阶:同类“否定+比较”句式扩展}
        
        在英语写作中,这种“曲径通幽”表达最高级的方法还有几种变体,建议一并掌握:
        
        \begin{itemize} 
            \item \textbf{Nothing is more ... than ...} 
            
            \es{Nothing is more important than health. (没有什么比健康更重要。) }

            \item \textbf{I have never seen a more ... than ...} 
            
            \es{I have never seen a more exciting match than this one. (我从未见过比这场更精彩的比赛了。) }
        \end{itemize}
    \end{enumerate}
\end{multicols}

\wsitem{The idea never appealed to sb...}
\begin{multicols}{1}
    这是一个非常地道的否定陈述句型,主要用来表达“某人对某事不感兴趣”或“某事不合某人心意”。

    \begin{enumerate} 
        \item \textbf{句式结构拆解} 
        \begin{itemize} 
            \item \textbf{公式}:The idea + never + appealed to + somebody. 
            \item \textbf{核心词汇}:\textit{appeal to} 是一个固定短语,意为“吸引”或“引起……的兴趣”。 
            \item \textbf{逻辑方向}:注意它的逻辑是**“物吸引人”**(The thing appeals to me),而不是“人喜欢物”。 
        \end{itemize}

        \item \textbf{为什么这个句式比 "I don't like" 更高级?}
        \begin{itemize}
            \item \textbf{客观性}:它将重点放在“主意/想法”本身(The idea),暗示这个想法本身缺乏吸引力,而不仅仅是主观的不喜欢。
            \item \textbf{语气坚定}:使用 \textit{never} 比 \textit{didn't} 语气更强烈,表达出一种“压根没考虑过”或“从始至终没兴趣”的决绝感。
        \end{itemize}

        \item \textbf{常见的同类变体}
        \begin{itemize}
            \item \textbf{The prospect of ... appealed to him.}(……的前景很吸引他。)
            \item \textbf{Does the suggestion appeal to you?}(这个建议对你有吸引力吗?)
            \item \textbf{The idea of working abroad never appealed to her.}(出国工作的想法从未吸引过她。)
        \end{itemize}

        \item \textbf{在复述与写作中的应用}
        
        \es{Mendoza was a fighter, but the idea of staying in debt \textbf{never appealed to} him, yet he couldn't help his extravagance.}
            
        \es{Many people thought flying was dangerous, and the idea of crossing the Channel \textbf{never appealed to} them.}
    \end{enumerate}
\end{multicols}

\grammarquestions
\wsitem{You would not be able to understand why some people find it repulsive. 中为什么用be able to,不用can}
\begin{multicols}{1}
    这是一个非常深入的语法细节。虽然 can 和 be able to 经常可以互换,但在你给出的这个句子 "You would not be able to understand..." 中,使用 be able to 是为了满足语法和语气上的双重需求。

    \begin{enumerate} 
        \item \textbf{与情态动词 (would) 的兼容性} 
        \begin{itemize} 
            \item \textbf{语法禁忌}:在英语中,两个情态动词(Modal Verbs)不能直接连用。我们不能说 \textit{"would can"}。 
            \item \textbf{功能替换}:当你需要表达“将来能够”或“假设能够”时,必须使用 be able to 来替代 can。 
            \item \textbf{本句逻辑}:这里的 \textit{would} 表示一种强烈的假设或推测。因为已经有了 \textit{would},所以只能接 \textit{be able to}。 
        \end{itemize}
        \item \textbf{侧重“能力”而非“许可”}
        \begin{itemize}
            \item \textbf{be able to}:更侧重于通过智力、体力或环境具备的“某种具体的能力”。
            \item \textbf{本句语境}:不理解某事是因为缺乏某种感知或逻辑基础(即“不具备理解的能力”),用 \textit{be able to} 显得更加严谨。
        \end{itemize}

        \item \textbf{语气的委婉与距离感}
        \begin{itemize}
            \item \textbf{would not be able to}:听起来比 \textit{cannot} 更加委婉、礼貌,常用于辩论或阐述观点,给人一种“基于某种前提下的推论”的感觉。
            \item \textbf{对比}:\textit{You cannot understand} 听起来像是在指责对方“你没这本事理解”,带有攻击性;而 \textit{You would not be able to} 则是在探讨一种客观的可能性。
        \end{itemize}

        \item \textbf{固定句型:假设语气中的骨架}
        \begin{itemize}
            \item 这是一个典型的虚拟语气或推测结构:\textit{If you were in my shoes, you would not be able to...}
            \item 在复述这类句子时,记住这个组合:\textbf{would (not) + be able to}。
        \end{itemize}
    \end{enumerate}
\end{multicols}

\wsitem{If you lived in the Mediterranean, for instance, you would consider octopus a great delicacy.这句话中为什么用consider,不用think? }
\begin{multicols}{1}
    这是一个非常经典的用词习惯问题。虽然 think 和 consider 都有“认为”的意思,但在结构、正式程度和逻辑深度上,它们有着显著的区别。

    \begin{enumerate} 
        \item \textbf{语法结构的简洁性} 
        \begin{itemize} 
            \item \textbf{Consider 的结构}:\textit{consider A (to be) B}。在这个句子里,它是 \textit{consider [octopus] [a great delicacy]}。你可以直接把 A 和 B 连在一起,中间不需要任何多余的词。 
            \item \textbf{Think 的结构}:如果你用 \textit{think},通常需要接 \textit{that} 从句,或者接 \textit{of... as...}。 
            
            \es{You would think \textbf{that} octopus is a delicacy. (较繁琐) }
            
            \es{You would think \textbf{of} octopus \textbf{as} a delicacy. (这是 think 的常用固定搭配) }
        \end{itemize}
        \item \textbf{语义深度的差异}
        
        \textbf{Think}:侧重于“直觉”或“简单的想法”。它通常描述一种瞬间的、主观的观点。
        
        \textbf{Consider}:侧重于“经过权衡、评估后的看法”。它带有一种“定性”的色彩。在你的例句中,地中海人将章鱼视为美味,这是一种饮食文化的“定论”和“共识”,而非拍脑袋的想法,所以用 \textit{consider} 更有深度。

        \item \textbf{正式程度与语境}
        
        \textbf{Consider}:属于中高级词汇(Academic/Formal),更符合《新概念英语》这类文章追求的雅致感。
        
        \textbf{Think}:最基础的口语词汇,在正式写作中往往显得略显单薄。

        \item \textbf{饮食文化中的特定用法}
        
        \item 在讨论“视某物为……(珍馐/垃圾/挑战)”时,英语母语者几乎本能地会选择 \textit{consider} 或 \textit{regard...as}。
            
        \es{例句:Many people \textbf{consider} bird's nest soup a luxury. (很多人认为燕窝是奢侈品。)}
    \end{enumerate}
\end{multicols}

\wsitem{and took them to Robert. 为什么时took them to Robert不是took them to Robert‘s home?}
\begin{multicols}{1}
    这是一个非常地道的英语习惯。在英语中,人名本身就可以代表一个“目的地”,尤其是当你和那个人有约、或者去他的住处/工作地点见他时。

    \begin{enumerate}
        \item \textbf{“目的地”的省约原则 (Implicit Destination)}
        
        在英语口语和叙事文学中,当你朝向某人移动时,人名 = 该人所在的场所。

        \begin{itemize}
            \item \textbf{Take sth. to Robert:} 侧重于将东西交给罗伯特这个\textbf{人}。
            \item \textbf{Take sth. to Robert's (home):} 侧重于将东西送到罗伯特那个\textbf{地点}。
        \end{itemize}

        在你的句子里,作者不仅是去那个房子,更是要去见 Robert 并把东西交给他。因此,直接用人名作为终点,显得动作更加连贯且具有指向性。


        \item \textbf{常见的类似用法 (Common Usage)}
        
        这种用法在日常生活中极其普遍,你会发现“人”和“地”经常是合二为一的:

        \es{ \textbf{I'm going to the doctor.},意思是“去诊所看医生。”}
            
        \es{ \textbf{I'll take you to Robert.},意思是“我带你去罗伯特那儿(见他)。”}
        
        \es{ \textbf{I'm at Robert's.},意思是“我在罗伯特家。”}

        \item \textbf{叙事节奏:动作的简练 (Narrative Pace)}
        
        \es{ took them to \textbf{Robert's home} —— 听起来像是一个物流快递员在送货,强调物理距离。}
            
        \es{ took them to \textbf{Robert} —— 听起来像是一个朋友间的互动,强调人与人的会面。}

        由于作者的目的是把那几打(dozen)东西给罗伯特看,所以人才是这个动作的真正终点。
    \end{enumerate}

\end{multicols}

\wsitem{I had forgotten all about the snails...中的all about snails是什么语法?}
\begin{multicols}{1}
    这句话的语法结构非常地道,它是一个典型的\textbf{“宾语 + 范围状语”}组合。


    我们可以把这个结构拆解为:forget (谓语) + all (宾语) + about the snails (后置修饰/范围状语)。

    \begin{enumerate}
        \item \textbf{语法成分拆解 (Grammatical Breakdown)}
        \begin{itemize}
            \item \textbf{All 作为代词 (Pronoun):} 
            
            在这里 all 是 forget 的直接宾语,代表“全部事情”。
            
            \item \textbf{About the snails 作为介词短语:} 
            
            这个短语紧跟在 all 后面,限定了“全部”的具体范畴——不是忘了所有事,而是仅仅忘了“关于蜗牛的所有事”。

        \end{itemize}
        \item \textbf{语感辨析:Forget the snails vs. Forget all about the snails}
        虽然意思相近,但加入 all about 后,语感发生了质变

        \textbf{I forgot the snails.},予以重点在于忘了那个具体的物体,听起来像是:我忘了带蜗牛,或者忘了喂它们。

        \textbf{I forgot all about the snails.},语义重点在于完全抛诸脑后,强调“关于蜗牛的这件事、整个念头”都被彻底忘记了。

        \item \textbf{常见的类似用法 (Common Patterns)}
        
        这种结构在英语中非常万能,常用来表示对某个话题的“彻底性”:
        
        \es{\textbf{Tell me all about it!} (把那件事的始末全部告诉我!)}
        
        \es{\textbf{I know all about him.} (他的底细我全清楚。)}
        
        \es{\textbf{Don't worry all about the details.} (别担心所有的细节。)}
    \end{enumerate}
\end{multicols}

\wsitem{On the other hand, your stomach would turn at the idea of ....,解析一下at the idea of ...}
\begin{multicols}{1}
    
    这句话用得非常绝,它捕捉到了\textbf{“生理反应”与“心理念头”}之间那种极其迅速的联动。

    \begin{enumerate}
        \item \textbf{语法结构解析 (Syntactic Analysis)}
        
        \begin{itemize}
            \item \textbf{核心公式:} [Reaction] + at the idea of + [Doing sth./Sth.]
            \item \textbf{介词 "at" 的妙用:} 
            \begin{itemize}
                \item 这里的 at 表示\textbf{“原因”}或\textbf{“诱因”},且通常暗示这种反应是\textbf{即时的、条件反射式的}。
                \item 就像 jump at the sound(听见声音吓一跳),at the idea 意味着你甚至还没看到实物,仅仅是“一想到”那个念头,生理反应就产生了。
            \end{itemize}
        \end{itemize}


        \item \textbf{语感深度辨析:为什么不直接用 because?}
        
        对比以下两种说法:

        \begin{itemize}
            \item \textit{Your stomach turns \textbf{because} you think about it.} (因为你思考了它,所以胃难受。—— 听起来像是一个缓慢的逻辑推导过程。)
            \item \textit{Your stomach turns \textbf{at the idea of} it.} (一想到这事儿你就反胃。—— 强调瞬间的刺激。)
        \end{itemize}

        at the idea of 传递出一种“大脑刚接收到这个信号,胃部立刻就收缩了”的画面感。   

        \item \textbf{词汇关联:Stomach turn (反胃)}
        
        在这一句中,stomach would turn(胃部翻江倒海/感到恶心)与 at the idea of 是完美搭档:

        \begin{itemize}
            \item \textbf{Stomach turn:} 这是一个很生动的地道表达,用来形容极度的反感或厌恶。
            
            \textbf{搭配扩展:}
            
            \es{ I \textbf{shudder at the idea of...} (一想到……我就不寒而栗)}
            
            \es{ She \textbf{smiled at the idea of...} (一想到……她就露出了微笑)}
        \end{itemize}
    \end{enumerate}
\end{multicols}

\wsitem{... at the idea of frying potatoes in animal fat ....可以改为by animal fat吗?还是必须用in animal fat?}
\begin{multicols}{1}
    这是一个非常棒的介词辨析问题。简单直接的回答是:通常必须用 "in",不能改为 "by"。
    
    虽然在中文里我们都翻译成“用动物油炸”,但在英语的底层逻辑中,炸东西的“油”被视为一种物理介词环境。

    \begin{enumerate}
        \item \textbf{逻辑辨析:介词的环境属性 (Environment vs. Means)}
        \begin{itemize}
            \item \textbf{In animal fat (在油脂中/用油脂):}
            \begin{itemize}
                \item \textbf{逻辑:} 当我们“炸” (fry) 东西时,食物是被\textbf{浸没}或\textbf{包裹}在油脂里的。
                \item \textbf{空间感:} 油是一个容器或介质,食物在其中受热。因此,英语使用表示空间包围感的 in。
                \item \textbf{同类表达:} cook in oil, fry in butter, boil in water。
            \end{itemize}

            \item \textbf{By animal fat (通过油脂):}
            \begin{itemize}
                \item \textbf{逻辑:} By 强调的是\textbf{手段、工具或动力来源}。
                \item \textbf{语感:} 如果你说 fry by animal fat,听起来像是你把动物油当成了一种“燃料”或是某种抽象的“中介方案”,这不符合烹饪的物理描述。
            \end{itemize}
        \end{itemize}
        \item \textbf{什么时候能用 With?}
        
        除了 in,有时候你会看到 with。

        \textbf{Fry with animal fat:} 侧重于配料的选择(我选择用这种油而不是那种油)。

        \textbf{但在描述“炸”这个动作本身时,} in 依然是绝对的主流。 因为“炸”的本质就是让食物处于液体环境之中。

        \item \textbf{深度拓展:介词与烹饪动作的固定搭配}
        \begin{itemize}
            \item \textbf{Fry / Sauté}: 使用介词\textbf{in} oil / butter,逻辑是食物被包裹在油脂环境中。
            \item \textbf{Bake / Roast}:使用介词\textbf{in} the oven,逻辑是食物处于烤箱的空间内。
            \item \textbf{Cook},使用介词\textbf{on} a stove / over an open fire,强调热源的位置(在火炉上/在火堆上方)。
            \item \textbf{Season},使用介词\textbf{with} salt / pepper,强调添加的工具或调料。
        \end{itemize}

    \end{enumerate}
\end{multicols}

\wsitem{... he has no garden of his own.  可以改为he hasn't got his own garden吗?}
\begin{multicols}{1}
    简单直接的回答:可以!
    
    但在语感和使用场景上,这两者有着非常有趣的差别。“He has no garden of his own” 听起来更像文学叙述,而 “He hasn't got his own garden” 则充满了地道的英国生活气息。

    \begin{enumerate}
        \item \textbf{语法结构的合法性 (Grammatical Validity)}
        \begin{itemize}
            \item \textbf{A. Has no garden of his own}
            \begin{itemize}
                \item \textbf{结构:} Verb (has) + Negative Determiner (no) + Noun.
                \item \textbf{特点:} no 直接修饰名词,语气非常坚决,且 of his own 作为后置定语起到强调作用。
            \end{itemize}

            \item \textbf{B. Hasn't got his own garden}
            \begin{itemize}
                \item \textbf{结构:} Auxiliary (has) + not + got (Past Participle) + Object.
                \item \textbf{特点:} 这是典型的 have got 结构。在这里 his own 变成了前置形容词直接修饰 garden。
            \end{itemize}
        \end{itemize}
        \item \textbf{语感与语体辨析 (Nuance and Register)}
        
        虽然意思一样,但它们给读者的感觉截然不同:

        He has no garden...语体属性为书面语、文学性,适合写小说、正式报告或《新概念英语》这种典雅的叙事风格。。

        He hasn't got a garden...,语体属性为口语化、非正式,是典型的英式口语。如果你在伦敦街头和人聊天,用这个会显得非常自然。

        \item \textbf{关于 "Own" 的位置变幻}
        
        \es{ \textbf{Garden of his own:} 这种“名词 + of + 所有格”的结构更具强调性,仿佛在叹息:哪怕一个属于他自己的小小花园都没有。}
        
        \es{ \textbf{His own garden:} 这种“所有格 + own + 名词”的结构更直接、平铺直叙。}
    \end{enumerate}
\end{multicols}
\newpage