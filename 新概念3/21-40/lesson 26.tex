\section{Lesson 26 Wanted: a large biscuit tin}

\begin{paracol}{2}

No one can \ns{avoid being} \ns{influenced} by advertisements. 

\switchcolumn

\chinesetext{没有人能避免受广告的影响。}
\nse{avoid doing sth.}{}{}
\nse{influence}{ˈɪnfluəns}{n. 势力;影响;有影响的人/事物;v. 影响;支配;}

\switchcolumn*

Much as we may \ns{\nw{pride} ourselves on} our good \nw{taste}, we are \ns{no longer} free to choose the things we want, for advertising \ns{\nw{exerts} a \nw{subtle} influence on} us. 

\switchcolumn

\chinesetext{尽管我们可以自夸自己的鉴赏力如何敏锐,但我们已经无法独立自主地选购自己所需的东西了。这是因为广告在我们身上施加着一种潜移默化的影响。}
\nwe{pride}{praɪd}{v. 为…感到自豪;}
\nwe{taste}{teɪst}{n. 味道;味觉;一小口;体验;品味;爱好;}
\nwe{exert}{ɪɡˈzɜːrt}{v. 运用,施加;努力;}
\nwe{subtle}{ˈsʌtl}{adj. 微妙的;敏感的;狡猾的;巧妙的;}
\nse{pride oneself on ...}{}{为……感到自豪}
\nse{exerts a subtle influence on ...}{}{对……施加微妙的影响}
\nse{no longer}{}{不再}

\switchcolumn*

In their efforts to persuade us to buy this or that product, \nw{advertisers} have \ns{made a close study of} \ns{human nature} and have \ns{\nw{classified} all our little weaknesses}.

\switchcolumn

\chinesetext{做广告的人在力图劝说我们买下这种产品或那种产品之前,已经仔细地研究了人的本性,并把人的弱点进行了分类。}
\nwe{advertiser}{ˈædvərtaɪzər}{n. 登广告的人;}
\nwe{classify}{ˈklæsɪfaɪ}{v. 将…分类;}
\nse{persuade sb. to do sth.}{}{说服...做...}
\nse{make a close study of}{}{对……进行严密研究}
\nse{human nature}{ˌhjuːmən ˈneɪtʃər}{n. 人性,人情;人之常情;}
\nse{classify our little weaknesses}{}{对我们的小弱点进行分类}

\switchcolumn*

Advertisers discovered \ns{years ago} that all of us love to \ns{get something for nothing}. 

\switchcolumn

\chinesetext{做广告的人们多年前就发现我们大家都喜欢免费得到东西。}
\nse{years ago}{}{多年前}
\nse{get something for nothing}{}{白吃白拿,不劳而获}

\switchcolumn*

An advertisement which begins with the \nw{magic} word FREE can rarely \ns{go wrong}. 

\switchcolumn

\chinesetext{凡是用"免费"这个神奇的词开头的广告很少会失败的。}
\nwe{magic}{ˈmædʒɪk}{adj. 有魔力的;神奇的;极好的;}
\nse{go wrong}{ɡo rɔŋ}{v. 走错路,误入歧途,(机器等)发生故障;出岔子;出乱子;}

\switchcolumn*

These days, advertisers not only offer free \nw{samples}, but free cars, free houses, and free trips \ns{round the world} as well. 

\switchcolumn

\chinesetext{目前,做广告的人不仅提供免费样品,而且还提供免费汽车,免费住房,免费周游世界。}
\nwe{sample}{ˈsæmpl}{n. 样品;样本;}

\switchcolumn*

They \nw{devise} hundreds of competitions which will enable us to win \ns{huge sums of money}. 

\switchcolumn

\chinesetext{他们设计数以百计的竞赛,竞赛中有人可赢得巨额奖金。}
\nwe{devise}{dɪˈvaɪz}{v. 设计,发明;想出;}
\nse{huge sums of money}{}{巨额资金}

\switchcolumn*

Radio and television have made it possible for advertisers to \ns{\nw{capture} the attention of} millions of people in this way.

\switchcolumn

\chinesetext{电台、电视使做广告的人可以用这种手段吸引成百万人的注意力。}
\nwe{capture}{ˈkæptʃər}{v. 俘虏,占领;引起(注意、想象、兴趣);拍摄;表现,体现(感情、气氛);}
\nse{capture the attention of ...}{}{吸引……的注意}

\switchcolumn*

During a radio programme, a company of biscuit manufacturers once asked listeners to bake biscuits and send them to their factory. 

\switchcolumn

\chinesetext{有一次,在电台播放的节目里,一个生产饼干的公司请听众烘制饼干送到他们的工厂去。}
\nwe{manufacturer}{ˌmænjuˈfæktʃərər}{n. 制造商,制造厂;厂主;[经]厂商;}

\switchcolumn*

They offered to pay \$10 a pound for the biggest biscuit baked by a listener. 

\switchcolumn

\chinesetext{他们愿意以每磅10美元的价钱买下由听众烘制的最大的饼干。}

\switchcolumn*

\nse{The response to} this competition was tremendous. 

\switchcolumn

\chinesetext{这次竞赛在听众中引起极其热烈的反响。}
\nse{tremendous response}{}{巨大的反响/响应}
\nse{the response to}{}{对...的反应}

\switchcolumn*

\ns{Before long}, \ns{biscuits of all shapes and sizes} began arriving at the factory. 

\switchcolumn

\chinesetext{不久,形状各异,大小不一的饼干陆续送到工厂。}
\nse{before long}{}{不久以后}
\nse{biscuits of all shapes and sizes}{}{各种形状和大小的饼干}

\switchcolumn*

One lady brought in a biscuit on a \ns{wheelbarrow}. 

\switchcolumn

\chinesetext{一位女士用手推车运来一个饼干。}
\nwe{wheelbarrow}{ˈwiːlbæroʊ}{n. 独轮手推车;}

\switchcolumn*

It weighed nearly 500 pounds. 

\switchcolumn

\chinesetext{重达500磅左右。}

\switchcolumn*

\ns{A little later}, a man \ns{came along with} a biscuit which occupied the whole boot of his car. 

\switchcolumn

\chinesetext{相隔不一会儿,一个男子也带来一个大饼干,那个饼干把汽车的行李箱挤得满满的。}
\nwe{boot}{buːt}{n. 靴子;猛踢;(汽车)行李箱;v. 猛踢;装入操作系统;}
\nse{a little later}{}{晚一点;}
\nse{come along with}{}{随同,和……一起来;}

\switchcolumn*

All the biscuits that were sent were carefully weighed. 

\switchcolumn

\chinesetext{凡送来的饼干都仔细地称量。}

\switchcolumn*

The largest was 713 pounds. 

\switchcolumn

\chinesetext{最重的一个达713磅。}

\switchcolumn*

It seemed certain that this would win the prize. 

\switchcolumn

\chinesetext{看来这个饼干获奖无疑了。}

\switchcolumn*

But just before the competition closed, a \nw{lorry} arrived at the factory with a truly \ns{colossal} biscuit which weighed 2,400 pounds. 

\switchcolumn

\chinesetext{但就在竞赛截止时间将到之际,一辆卡车驶进了工厂,运来了一个特大无比、重达2,400磅的饼干。}
\nwe{lorry}{ˈlɔːri}{n. 运货汽车,机动卡车;}
\nwe{colossal}{kəˈlɑːsl}{adj. 巨大的;<口语>异常的;}

\switchcolumn*

It had been baked by a college student who had used over 1,000 pounds of flour, 800 pounds of sugar, 200 pounds of fat, and 400 pounds of various other ingredients. 

\switchcolumn

\chinesetext{它是由一个大学生烘制的,用去1,000多磅的面粉、800磅食糖、200磅动物脂肪及400磅其他各种原料。}
\nwe{ingredient}{ɪnˈɡriːdiənt}{n. (混合物的)组成部分;(烹调的)原料;(构成)要素;因素;}

\switchcolumn*

It was so heavy that a crane had to be used to remove it from the lorry. 

\switchcolumn

\chinesetext{饼干份量太重了,用了一台起重机才把它从卡车上卸下。}
\nwe{crane}{[kreɪn]}{n. 起重机;鹤;v. 伸长(脖子),探着身子(看);}


\switchcolumn*

The manufacturers had to pay more money than they had anticipated, for they bought the biscuit from the student for \$24,000.

\switchcolumn

\chinesetext{饼干公司不得不付出比他们预计多得多的钱,因为为买下那学生烘制的饼干他们支付了24,000美元。}
\nwe{anticipate}{ænˈtɪsɪpeɪt}{v. 预期;预计;期盼;先于…做;}

\switchcolumn*

\end{paracol}

\grammarpoints

\wsitem{Subtle vs. Tiny vs. Little}
\begin{multicols}{1}
    虽然它们都表示“小”,但 subtle 属于大脑和感官的范畴,而 tiny little 则属于眼睛和空间的范畴。

    \begin{enumerate}
        \item \textbf{本质属性的区别}
        \begin{itemize} 
            \item \textbf{Subtle (微妙的/隐约的):} 指程度极其细微,以至于很难察觉或描述。它不一定是物理体积小,而是指那种“不显眼”或“需要敏锐观察力”才能发现的特性。 
            \item \textbf{Tiny little (极小的/微小的):} 这是一个非常形象且带有口语色彩的叠加词,专门描述物理体积或尺寸极其之小。 
        \end{itemize}
        \item \textbf{核心特征解析}
        \begin{itemize} 
            \item \textbf{Subtle — 侧重“难以捉摸”:} 
            \begin{itemize} 
                \item \textit{A \textbf{subtle} difference:} 两者之间只有极小的差别(可能肉眼看不出,需要专业分析)。 
                \item \textit{A \textbf{subtle} hint:} 一个委婉的暗示(不是直说)。 
                \item \textit{A \textbf{subtle} change in a patient's pulse:} 病人脉搏极其细微的变化(普通人感觉不到,但老练的医生能察觉)。 
            \end{itemize} 
            \item \textbf{Tiny little — 侧重“体积极小”:} 
            \begin{itemize} 
                \item \textit{A \textbf{tiny little} pill:} 一颗极小的药片。 \item \textit{A \textbf{tiny little} crack:} 一个非常细小的裂缝(看得见,只是很小)。 
                \item \textit{A \textbf{tiny little} spider:} 一只小不点的蜘蛛。 
            \end{itemize} 
        \end{itemize}
        \item \textbf{语体色彩对比}
        \begin{itemize} 
            \item \textbf{Subtle:} 书面语,正式且优雅。如果你在描述某种高深的艺术或复杂的病理,它是首选。 
            \item \textbf{Tiny little:} 非正式口语。将 \textit{tiny} 和 \textit{little} 放在一起是一种重叠强调,听起来甚至带有一点可爱的感情色彩(或者极度的厌烦感)。 
        \end{itemize}
        \item \textbf{实操}
        
        想象你在实验室观察切片:
        
        \begin{itemize} 
            \item 你在显微镜下看到一个 \textbf{tiny little} cell (极小的细胞)。这是描述它的物理尺寸。 
            \item 你观察到这个细胞的结构有 \textbf{subtle} abnormalities (微妙的异常)。这是描述某种不明显的病理特征。 
        \end{itemize}
    \end{enumerate}
\end{multicols}

\wsitem{Round the world vs Around the world}
\begin{multicols}{1}
    \begin{itemize} 
        \item \textbf{Round:} 更多见于英式英语,暗示一个完整的圆圈,一种周而复始的旅程。 
        \item \textbf{Around:} 更多见于美式英语,侧重于在各处漫游。 
    \end{itemize}
\end{multicols}

\wsitem{Occupy}
\begin{multicols}{1}
    \begin{enumerate}
        \item \textbf{核心义项与常见搭配}
        \begin{itemize} 
            \item \textbf{占据空间 (Physical Space)} 
            \begin{itemize} 
                \item \textbf{含义:} 指物理上的占用、居住或军事上的占领。 
                \item \textbf{经典搭配:} \textit{occupy a seat/room} (占用座位/房间);\textit{occupied territory} (占领区)。 
                \item \textbf{医学延伸:} \textit{Space-occupying lesion} (占位性病变)。这是临床描述肿瘤、囊肿等压迫周围组织时的标准术语。 
            \end{itemize}

            \item \textbf{占据时间 (Time)}
            \begin{itemize}
                \item \textbf{含义:} 指某事消耗了大量的精力或时间跨度。
                \item \textbf{经典搭配:} \textit{occupy much of one's time} (占据某人大部分时间)。
                \item \textbf{例句:} \textit{Preparing for the medical exam occupied all my weekends.}
            \end{itemize}
            \item \textbf{占据思想/注意力 (Mind/Attention)}
            \begin{itemize}
                \item \textbf{含义:} 指使某人忙碌或某念头萦绕在脑海。
                \item \textbf{经典搭配:} \textit{occupy oneself with/in...} (使自己忙于...);\textit{be fully occupied} (完全没空)。
            \end{itemize}
            \item \textbf{担任职位 (Position/Status)}
            \begin{itemize}
                \item \textbf{含义:} 占据某种社会地位或担任特定公职。
                \item \textbf{经典搭配:} \textit{occupy a leading position} (占据领先地位);\textit{occupy the throne} (占据王位)。
            \end{itemize}
        \end{itemize}
        \item \textbf{近义词辨析:Occupy vs. Take up vs. Reside}
        \begin{itemize}
            \item \textbf{Occupy (最通用/功能性)}
            \begin{itemize}
                \item \textbf{侧重:} 强调“使用权”或“状态”。当一个地方被 $occupy$ 时,重点是它现在“不空闲”。
            \end{itemize}
            \item \textbf{Take up (物理消耗性)}
            \begin{itemize}
                \item \textbf{侧重:} 强调“比例”或“填充”。比如家具太宽、事情太多。
                \item \textbf{对比:} \textit{The skeleton occupies the cupboard} (它住在里面) vs. \textit{The skeleton takes up the whole cupboard} (它把柜子塞满了)。
            \end{itemize}
            \item \textbf{Reside (法律/长期居住)}
            \begin{itemize}
                \item \textbf{侧重:} 极其正式,仅用于指人(或法律实体)居住在某地。
            \end{itemize}
        \end{itemize}
        \item \textbf{词汇辨析:Occupied vs. Busy}
        \begin{itemize} 
            \item \textbf{Busy (动作状态)} 
            \begin{itemize} 
                \item \textbf{语感:} 日常、通用。强调你正在不停地做动作。 
            \end{itemize} 
            \item \textbf{Occupied (被占用的状态)} 
            \begin{itemize} 
                \item \textbf{语感:} 正式。暗示你被某件事“困住了”或者某个位置被“占住了”。 
                \item \textbf{典型语境:} 厕所门上的“有人”通常写着 \textit{Occupied},而不会写 \textit{Busy}。 
            \end{itemize} 
        \end{itemize}
        \item \textbf{派生词及其逻辑}
        \begin{itemize} 
            \item \textbf{Occupation (职业/占用)} 
            \begin{itemize} 
                \item \textbf{逻辑:} 你的职业就是那个“每天占据你最多时间的事”。 
            \end{itemize} 
            \item \textbf{Occupant (居住者/占用者)} 
            \begin{itemize} 
                \item \textbf{例句:} \textit{The previous occupant of this desk left a stethoscope.} (这桌子的前任主人留下了一个听诊器。) 
            \end{itemize} 
        \end{itemize}
    \end{enumerate}
\end{multicols}

\wsitem{Anticipate vs. Expect}
\begin{multicols}{1}
    这两个词在中文里都可以翻译为“预料”,但在英语逻辑中,它们之间存在着\textbf{“被动观察”与“主动应对”}的根本区别。

    \begin{enumerate}
        \item \textbf{心理逻辑:思维 vs 行动}
        \begin{itemize} 
            \item \textbf{Expect (预期/期望):} 侧重于心理上的认知。你认为某事“很可能会发生”,但你只是坐在那里等它发生。它不包含任何行动。 
            \item \textbf{Anticipate (预见/预知):} 侧重于提前的准备。你不仅预料到它会发生,而且已经在脑海中预演了场景,或者已经在现实中采取了应对措施。 
        \end{itemize}
        
        \begin{itemize} 
            \item \textit{The surgeon \textbf{expects} the surgery to take three hours.} (医生估计手术要三个小时——这是一个时间预测。) 
            \item \textit{The surgeon \textbf{anticipates} potential complications and prepares extra blood units.} (医生预见到了并发症并准备了备用血袋——这包含了预判 + 准备。)
        \end{itemize}
        \item \textbf{语法搭配的硬要求}
        
        这是你在改写句子时最需要注意的地方,Anticipate 的用法比 Expect 稍微挑剔一些:
        
        \begin{itemize} 
            \item \textbf{Expect 可接不定式:} \textit{I expect \textbf{to see} you.} (正确) 
            \item \textbf{Anticipate 不接不定式:} \textit{I anticipate \textbf{seeing} you.} (正确,习惯接 \textit{-ing} 形式或名词)。 
        \end{itemize}
        \item \textbf{情感色彩:确定性 vs 期待感}
        \begin{itemize} 
            \item \textbf{Expect:} 往往带有“理所应当”的确定感,有时甚至是一种“要求”。 
            \begin{itemize} 
                \item \textit{I expect you to be on time.} (我要求/准许你准时,带有一种权力感。) 
            \end{itemize} 
            \item \textbf{Anticipate:} 往往带有一种“屏息以待”的张力,可能是期待,也可能是担忧。 \begin{itemize} 
                \item \textit{We \textbf{anticipate} a breakthrough in the cancer research.} (我们预见并期待着癌症研究的突破。) 
            \end{itemize} 
        \end{itemize}
    \end{enumerate}
\end{multicols}

\wsitem{Devise}
\begin{multicols}{1}
    Devise 是一个充满了“智慧感”的词,它的核心在于创造性的策划。
    \begin{enumerate}
        \item \textbf{核心义项与常见搭配}
        \begin{itemize} 
            \item \textbf{策划与发明 (Plan or Invent)} 
            \begin{itemize} 
                \item \textbf{含义:} 指通过巧妙的思考,想出(计划、系统或方法)。它强调“从无到有”且“构思精巧”。 
                \item \textbf{经典搭配:} \textit{devise a scheme/plan} (策划方案);\textit{devise a method} (发明方法)。 
                \item \textbf{例句:} \textit{Scientists are working to \textbf{devise} a new test for the virus.} (科学家正致力于研发一种新的病毒检测方法。) 
            \end{itemize} 
            \item \textbf{法律语境 (Legal Legacy)} \begin{itemize} 
                \item \textbf{含义:} (在遗嘱中)遗赠(不动产)。 
                \item \textbf{补充:} 虽然这个义项较偏,但在阅读关于“遗产”或“法律纠纷”的文学作品时可能会遇到。 
            \end{itemize} 
        \end{itemize}
        \item \textbf{近义词辨析:Devise vs. Design vs. Invent}
        \begin{itemize} 
            \item \textbf{Devise (侧重:脑力构思)} 
            \begin{itemize} 
                \item \textbf{特征:} 重点在于“想出办法”来解决复杂问题。它通常不涉及具体的绘图,而是逻辑上的完善。 
                \item \textbf{场景:} 策划一场复杂的恶作剧,或者制定一套康复计划。 
            \end{itemize} 
            \item \textbf{Design (侧重:结构与蓝图)} 
            \begin{itemize} 
                \item \textbf{特征:} 强调外观、结构或功能布局。通常需要图纸或模型。 
                \item \textbf{场景:} 设计一栋医院大楼 (\textit{design a hospital})。 
            \end{itemize} 
            \item \textbf{Invent (侧重:科技突破)} 
            \begin{itemize} 
                \item \textbf{特征:} 指创造出以前不存在的全新机器或工具。 
                \item \textbf{场景:} 爱迪生发明灯泡 (\textit{invent the light bulb})。 
            \end{itemize} 
        \end{itemize}
        \item \textbf{词汇辨析:Devise vs. Device (易混淆点)}
        
        这是一对经典的“动名混淆”,你可以通过后缀和发音来区分:

        \begin{itemize} 
            \item \textbf{Devise (动词 /dɪˈvaɪz/)} 
            \begin{itemize} 
                \item \textbf{结尾:} \textit{-se}。 
                \item \textbf{动作:} 指“策划”的过程。 
            \end{itemize} 
            \item \textbf{Device (名词 /dɪˈvaɪs/)} 
            \begin{itemize} 
                \item \textbf{结尾:} \textit{-ce}。 
                \item \textbf{物体:} 指具体的“装置、设备”。 
                \item \textbf{医学联系:} \textit{Medical device} (医疗器械)。 
            \end{itemize} 
        \end{itemize}
        \item \textbf{派生词与进阶表达}
        \begin{itemize} 
            \item \textbf{Devisable (可设计的/可策划的)} 
            \begin{itemize} 
                \item \textbf{用法:} 描述某个问题是否有可行的解决方案。 
            \end{itemize} 
            \item \textbf{Well-devised (精心策划的)} 
            \begin{itemize} 
                \item \textbf{例句:} \textit{The experiment was \textbf{well-devised}, leaving no room for error.} (实验设计严密,没有给错误留空间。) 
            \end{itemize} 
        \end{itemize}
    \end{enumerate}
\end{multicols}

\wsitem{Capture the attention vs. Attract the attention}
\begin{multicols}{1}
    这两个短语在日常使用中经常互换,但如果你追求表达的精确性和文学张力,它们在“力度”和“结果”上有着显著的区别。

    \begin{enumerate}
        \item \textbf{核心义项与力度对比}
        \begin{itemize}
            \item \textbf{Attract (诱发性的吸引)}
            \begin{itemize}
                \item \textbf{特征:} 通常是某种特质(如美貌、响声、亮色)自然而然产生的结果。
                \item \textbf{场景:} 广告商通过免费样品来 \textit{attract attention}。
            \end{itemize}
            \item \textbf{Capture (征服性的占据)}
            \begin{itemize}
                \item \textbf{特征:} 带有某种“掌控感”。一旦注意力被 $capture$,对方就处于一种被吸引住的状态(fascinated/spellbound)。
                \item \textbf{场景:} 一场精彩的手术演示 \textit{captures the attention} of all medical students.
            \end{itemize}
        \end{itemize}
        \item \textbf{常见搭配与语境应用}
        \begin{itemize} 
            \item \textbf{常用动词替换} 
            \begin{itemize} 
                \item \textbf{Grab/Catch attention:} 比 \textit{attract} 更随意、更迅速(常用于标题)。 
                \item \textbf{Hold/Engage attention:} 强调在捕获之后如何“维持”住注意力。 
            \end{itemize} 
            \item \textbf{医学/学术语境} 
            \begin{itemize} 
                \item \textbf{场景 A:} 论文的标题要足够 \textit{catchy} 以便 \textbf{attract} 读者的注意。 
                \item \textbf{场景 B:} 论文的内容必须足够扎实,才能 \textbf{capture} 审稿人的兴趣。 
            \end{itemize} 
        \end{itemize}
    \end{enumerate}
\end{multicols}

\wsitem{Colossal}
\begin{multicols}{1}
    Colossal 是一个重量级的形容词。它的核心逻辑不仅是“大”,而是\textbf{“像巨像一样宏伟、巨大”}。这个词源于希腊语的 kolossos(巨像,特指罗德岛太阳神巨像)。

    \begin{enumerate}
        \item \textbf{核心义项与常见搭配}
        \begin{itemize} 
            \item \textbf{体积/规模巨大 (Physical Size or Scale)} 
            \begin{itemize} 
                \item \textbf{含义:} 极其巨大的,庞大的。通常用于建筑物、雕像或自然景观。 
                \item \textbf{经典搭配:} \textit{a colossal statue} (巨型雕像);\textit{a colossal building} (宏伟的建筑)。 
            \end{itemize} 
            \item \textbf{程度/数量极大 (Degree or Amount)} 
            \begin{itemize} 
                \item \textbf{含义:} 极度的,巨大的(常带有一种令人震撼或惊愕的语气)。 
                \item \textbf{经典搭配:} \textit{a colossal waste of time} (极大的时间浪费);\textit{a colossal failure} (惨败);\textit{a colossal sum of money} (巨款)。 
            \end{itemize} 
        \end{itemize}
        \item \textbf{近义词辨析:Colossal vs. Huge vs. Enormous}
        
        这些词虽然都表示“大”,但它们的“重量感”和“情感色彩”不同:
        
        \begin{itemize} 
            \item \textbf{Huge (最通用)} 
            \begin{itemize} 
                \item \textbf{特征:} 最日常的词,描述物理体积大或数量多。 
                \item \textbf{场景:} \textit{a huge dog} (一只大狗)。 
            \end{itemize} 
            \item \textbf{Enormous (超越常规)} 
            \begin{itemize} 
                \item \textbf{特征:} 侧重于“超出正常限度”的巨大,常带有某种程度上的不可思议。 
                \item \textbf{场景:} \textit{an enormous amount of work} (繁重得惊人的工作量)。 
            \end{itemize} 
            \item \textbf{Colossal (宏伟/震撼)} 
            \begin{itemize} 
                \item \textbf{特征:} 等级最高。它不仅是大,而且带有一种\textbf{“压迫感”或“纪念碑式”}的宏大。 
                \item \textbf{场景:} 如果一个广告商送了一辆车,那是 \textit{huge};但如果他送了一座摩天大楼,那是 \textit{colossal}。 
            \end{itemize} 
        \end{itemize}
        \item \textbf{词源小知识}
        \begin{itemize} 
            \item \textbf{The Colosseum (罗马斗兽场):} 
            
            它的名字就源于它旁边曾经矗立着的一尊尼禄皇帝的 colossal statue。 
            
            所以当看到 \textbf{Coloss-} 开头的词,脑子里就要浮现出“罗马斗兽场”那种体量。 
        \end{itemize}
    \end{enumerate}
\end{multicols}

\wsitem{Capture}
\begin{multicols}{1}
    Capture 是一个极具动态感的词,其核心逻辑在于\textbf{“克服阻力并获得控制权”}。

    \begin{enumerate}
        \item \textbf{核心义项与常见搭配}
        \begin{itemize} 
            \item \textbf{武力或强制抓捕 (Physical Seizure)} 
            \begin{itemize} 
                \item \textbf{含义:} 指通过武力或努力抓住某人、动物,或在战争中占领某地。 
                \item \textbf{经典搭配:} \textit{capture a prisoner} (抓获囚犯);\textit{capture the city} (攻占城市)。 
            \end{itemize} 
            \item \textbf{记录与采集 (Recording or Representation)} 
            \begin{itemize} 
                \item \textbf{含义:} 用文字、照片、视频或数据准确地“记录”下瞬间或特征。 
                \item \textbf{经典搭配:} \textit{capture the moment} (定格瞬间);\textit{capture the essence of...} (捕捉到...的精髓)。 
                \item \textbf{医学/技术应用:} \textit{Image capture} (影像采集);\textit{Capture the patient's heartbeat data} (采集患者心跳数据)。 
            \end{itemize} 
            \item \textbf{赢得与争取 (Winning or Gaining)} 
            \begin{itemize} 
                \item \textbf{含义:} 成功赢得某些难以获得的东西,如选票、市场份额或注意力。 
                \item \textbf{经典搭配:} \textit{capture the market share} (占领市场份额);\textit{capture the headlines} (登上头条)。 
            \end{itemize} 
        \end{itemize}
        \item \textbf{近义词辨析:Capture vs. Catch vs. Seize}
        
        虽然它们都有“抓”的意思,但在动作的“力度”和“意图”上区别显著:
        
        \begin{itemize} 
            \item \textbf{Catch (最通用/瞬时性)} 
            \begin{itemize} 
                \item \textbf{特征:} 通常指抓住移动中的物体,或者无意中碰到。 
                \item \textbf{场景:} \textit{Catch a ball} (接球);\textit{Catch a cold} (感冒)。 
            \end{itemize} 
            \item \textbf{Capture (持续性/掌控感)} 
            \begin{itemize} 
                \item \textbf{特征:} 强调通过某种“计划”或“技巧”将其置于自己的控制之下。 
                \item \textbf{场景:} \textit{Capture a butterfly} (捕捉蝴蝶——强调不仅抓到了,还把它关起来了)。 
            \end{itemize} 
            \item \textbf{Seize (爆发性/强制性)} 
            \begin{itemize} 
                \item \textbf{特征:} 强调突然、有力、甚至带有侵略性的夺取。 
                \item \textbf{医学场景:} \textit{Seizure} (癫痫发作) —— 就像大脑被某种力量突然“攫住”了。 
            \end{itemize} 
        \end{itemize}
        \item \textbf{派生词与进阶词组}
        \begin{itemize} 
            \item \textbf{Captivating (迷人的/有魅力的)} 
            \begin{itemize} 
                \item \textbf{逻辑:} 就像你的心被“捕获”了一样,形容极具吸引力。 
            \end{itemize} 
            \item \textbf{Captive (被监禁的/俘虏)} 
            \begin{itemize} 
                \item \textbf{例句:} \textit{The skeleton was kept captive in the dark cupboard.} (那具骷髅被囚禁在黑暗的橱柜里。) 
            \end{itemize} 
        \end{itemize}
    \end{enumerate}
\end{multicols}

\wsitem{Study}
\begin{multicols}{1}
    在英语的高级用法中,它远不止“学习”或“读书”这么简单。它的核心逻辑在于\textbf{“专注且系统地观察与分析”}。

    \begin{enumerate}
        \item \textbf{核心义项与常见搭配}
        \begin{itemize} 
            \item \textbf{学术研究与调查 (Research and Investigation)} 
            \begin{itemize} 
                \item \textbf{含义:} 指对某一特定课题进行系统性的、科学的调查。 
                \item \textbf{经典搭配:} \textit{conduct a study} (进行研究);\textit{a clinical study} (临床研究);\textit{a pilot study} (初步/试点研究)。 
                \item \textbf{医学延伸:} \textit{Case study} (个案研究/病例分析)。 
            \end{itemize}
            \item \textbf{细致观察 (Close Observation)}
            \begin{itemize}
                \item \textbf{含义:} 凝神细看,仔细审视。
                \item \textbf{经典搭配:} \textit{study someone's face} (审视某人的脸);\textit{study the map} (研读地图)。
                \item \textbf{例句:} \textit{The surgeon studied the X-ray for several minutes before making the first incision.} (手术医生在动第一刀前,仔细研究了几分钟 X 光片。)
            \end{itemize}
            \item \textbf{学科或领域 (Field of Learning)}
            \begin{itemize}
                \item \textbf{含义:} 指某门特定的学问。
                \item \textbf{经典搭配:} \textit{the study of anatomy} (解剖学研究);\textit{social studies} (社会研究)。
            \end{itemize}
            \item \textbf{特定空间 (Physical Room)}
            \begin{itemize}
                \item \textbf{含义:} 指家里专门用来阅读、写字的“书房”。
            \end{itemize}
        \end{itemize}
        \item \textbf{近义词辨析:Study vs. Learn vs. Research}
        
        这三个词在中文里都可能译为“学”或“研究”,但侧重点完全不同:
        
        \begin{itemize}
            \item \textbf{Learn (侧重:获得结果)}
            \begin{itemize}
                \item \textbf{特征:} 强调从“不懂”到“懂”,或者掌握某种技能。
                \item \textbf{场景:} \textit{Learn how to suture} (学习如何缝合)。
            \end{itemize}
            \item \textbf{Study (侧重:过程与行为)}
            \begin{itemize}
                \item \textbf{特征:} 强调付出努力去阅读、练习或观察的过程。你不一定能 $learn$ 会,但你可以一直在 $study$。
                \item \textbf{场景:} \textit{I studied all night, but I learned nothing.} (我学了一宿,但啥也没学会。)
            \end{itemize}
            \item \textbf{Research (侧重:发现新知)}
            \begin{itemize}
                \item \textbf{特征:} 极其正式,通常指为了发现新事实、新理论而进行的系统性工作。
                \item \textbf{场景:} \textit{Research into a cure for cancer.} (研究癌症的疗法。)
            \end{itemize}
        \end{itemize}
        \item \textbf{习惯用法与搭配(你的“逻辑性”偏好)}
        \begin{itemize} 
            \item \textbf{In a study of... (在关于...的研究中)} 
            \begin{itemize} 
                \item 这是医学论文中最常见的开头。 
            \end{itemize} 
            \item \textbf{Under study (正在被研究中)} 
            \begin{itemize} 
                \item \textbf{例句:} \textit{The long-term effects of the new drug are still under study.} (这种新药的长期副作用仍在研究中。) 
            \end{itemize} 
            \item \textbf{A quick study (学得快的人)} 
            \begin{itemize} 
                \item 形容某人很有天赋,能迅速掌握新东西。 
            \end{itemize} 
        \end{itemize}
        \item \textbf{派生词与进阶表达}
        \begin{itemize} 
            \item \textbf{Studious (好学的/勤奋的)} 
            \begin{itemize} 
                \item \textbf{语感:} 形容一个学生非常用功,经常埋头苦读。 
            \end{itemize} 
            \item \textbf{Studied (故意的/刻意的)} 
            \begin{itemize} 
                \item \textbf{注意:} 这是一个高级用法!指某种行为是经过“仔细考虑和练习”的,而不是自然的。 
                \item \textbf{例句:} \textit{He spoke with \textbf{studied} politeness.} (他说话带着一种刻意的礼貌。) 
            \end{itemize} 
        \end{itemize}
    \end{enumerate}
\end{multicols}

\wsitem{Persuade}
\begin{multicols}{1}
    Persuade 是一个在沟通、市场营销以及日常生活中极具分量的词。它的核心逻辑不仅是“说”,而是\textbf{“通过理据或情感,成功使某人改变主意或采取行动”}。

    \begin{enumerate}
        \item \textbf{核心义项与语法结构}
        \begin{itemize} 
            \item \textbf{说服某人做某事 (Action)} 
            \begin{itemize} 
                \item \textbf{结构:} \textit{persuade somebody \textbf{to do} something}。 
                \item \textbf{例句:} \textit{I managed to \textbf{persuade} my patient to quit smoking.} (我成功说服了我的病人戒烟。) 
            \end{itemize} 
            \item \textbf{使某人信服 (Belief)} 
            \begin{itemize} 
                \item \textbf{结构:} \textit{persuade somebody \textbf{that...}} 或 \textit{be persuaded \textbf{of...}}。 
                \item \textbf{例句:} \textit{The test results \textbf{persuaded} him that the treatment was working.} (化验结果让他信服治疗是有疗效的。) 
            \end{itemize} 
            \item \textbf{劝阻某人 (Deterrence)} 
            \begin{itemize} 
                \item \textbf{结构:} \textit{persuade somebody \textbf{out of} doing something}。 
                \item \textbf{例句:} \textit{She \textbf{persuaded} him \textbf{out of} resigning.} (她劝服他不要辞职。) 
            \end{itemize} 
        \end{itemize}
        \item \textbf{近义词辨析:Persuade vs. Convince}
        
        这两个词常被混用,但在严谨的语境下,它们代表了影响他人的不同阶段:
        
        \begin{itemize}
            \item \textbf{Convince (侧重:思想转变)}
            \begin{itemize}
                \item \textbf{特征:} 重点在于改变一个人的“想法”或“信念”,让他觉得某事是真的。
                \item \textbf{结果:} 他的大脑“信了”。
            \end{itemize}
            \item \textbf{Persuade (侧重:行为转变)}
            \begin{itemize}
                \item \textbf{特征:} 重点在于促成一个“动作”。你可能已经 $convince$ 他运动有益健康,但只有当你 $persuade$ 他,他才会真正穿上跑鞋。
                \item \textbf{经典公式:} \textbf{Arguments} $\rightarrow$ \textbf{Convince} (Mind) $\rightarrow$ \textbf{Persuade} (Action).
            \end{itemize}

            \item \textbf{“证据”与“信念”的关联}
            \begin{itemize}
                \item \textbf{Convince (改变认知):} 当主语是客观事物(如 \textit{test results}, \textit{evidence}, \textit{facts})时,它们的作用是让人“相信”某个事实。
                \item \textbf{逻辑链:} 化验结果(事实) $\rightarrow$ 证明了疗效(逻辑) $\rightarrow$ 大脑接受了这一事实(信念转变)。这个过程用 \textbf{convinced} 堪称完美。
            \end{itemize}
        \end{itemize}
        \item \textbf{派生词与进阶词组}
        \begin{itemize} 
            \item \textbf{Persuasive (有说服力的)} 
            \begin{itemize} 
                \item \textbf{例句:} \textit{He gave a \textbf{persuasive} argument for the new policy.} (他为新政策提出了一个有说服力的论据。) 
            \end{itemize} 
            \item \textbf{Persuasion (说服/派别)} \begin{itemize} 
                \item \textbf{含义 1:} 说服的行为。 
                \item \textbf{含义 2:} (较正式) 宗教或政治信仰。例如:\textit{People of all religious \textbf{persuasions}.} 
            \end{itemize} 
        \end{itemize}
    \end{enumerate}
\end{multicols}

\wsitem{Trip}
\begin{multicols}{1}
    你对 Trip 的理解可能最初停留在“旅行”,但在英语中,它不仅是一个物理上的移动,更是一个关于\textbf{“状态切换”和“失衡”}的词。

    \begin{enumerate}
        \item \textbf{核心义项与常见搭配}
        \begin{itemize} 
            \item \textbf{短途旅程 (Journey)} 
            \begin{itemize} 
                \item \textbf{含义:} 指为了特定目的(如度假、出差)而进行的往返。通常比 \textit{journey} 或 \textit{voyage} 持续时间短。 
                \item \textbf{经典搭配:} \textit{a business trip} (出差);\textit{a round trip} (往返旅行);\textit{a field trip} (实地考察)。 
                \item \textbf{辨析:} 你之前看到的广告商提供的 \textit{free trips round the world},这里用 \textit{trip} 强调的是一种轻松、享受的性质。 
            \end{itemize} 
            \item \textbf{绊倒/失足 (Stumble)} 
            \begin{itemize} 
                \item \textbf{含义:} 脚被碰到而失去平衡,或者使某人跌倒。 
                \item \textbf{经典搭配:} \textit{trip over something} (被某物绊倒);\textit{trip up} (犯错/使某人露出破绽)。 
                \item \textbf{医学应用:} \textit{Gait and balance disorders} (步态与平衡障碍) 的评估中,\textit{tripping} 是跌倒风险的重要指标。 
            \end{itemize}
            \item \textbf{触发开关 (Mechanics)}
            \begin{itemize}
                \item \textbf{含义:} 触动开关或制动装置,使机械或电路停止工作。
                \item \textbf{经典搭配:} \textit{trip a circuit breaker} (使断路器跳闸);\textit{trip an alarm} (触动警报)。
            \end{itemize}
            \item \textbf{幻觉体验 (Slang/Psychology)}
            \begin{itemize}
                \item \textbf{含义:} 药物(尤其是致幻剂)诱发的感觉体验;或者一种极其怪异的经历。
                \item \textbf{经典搭配:} \textit{a bad trip} (糟糕的幻觉体验/恐怖的经历);\textit{power trip} (权力欲膨胀/滥用职权)。
            \end{itemize}
        \end{itemize}
        \item \textbf{近义词辨析:Trip vs. Journey vs. Travel}
        \begin{itemize} 
            \item \textbf{Trip (侧重:往返的目的性)} 
            \begin{itemize} 
                \item \textbf{特征:} 名词居多,强调“去哪儿干点啥然后回来”。 
            \end{itemize} 
            \item \textbf{Journey (侧重:过程的漫长)} 
            \begin{itemize} 
                \item \textbf{特征:} 强调从 A 到 B 的漫长旅途,往往带有艰辛或心灵成长的意味。 
            \end{itemize} 
            \item \textbf{Travel (侧重:动作本身)} 
            \begin{itemize} 
                \item \textbf{特征:} 泛指“旅行”这一行为,常作为动词使用。 
            \end{itemize} 
        \end{itemize}
        \item \textbf{派生词与进阶表达}
        \begin{itemize} 
            \item \textbf{Tripper (旅游者/服用幻觉剂的人)} 
            \begin{itemize} 
                \item \textbf{语感:} 在英国,\textit{day trippers} 指那些当天往返的游客。 
            \end{itemize} 
            \item \textbf{Trip-wire (陷阱线/触发线)} 
            \begin{itemize} 
                \item \textbf{逻辑:} 引申为某种“隐形陷阱”或“预设的底线”。 
            \end{itemize} 
        \end{itemize}
    \end{enumerate}
\end{multicols}

\wsitem{Much as...}
\begin{multicols}{1}
    这个结构是《新概念英语》第三册中非常出彩的高级语法点。很多同学初看会觉得它和 $As$ 引导的普通状语从句混淆,但其实它是一个强烈的让步状语从句。

    \begin{enumerate}
        \item \textbf{基本含义与功能}
        
        Much as 在句首时,含义等同于 Although 或 Even though(尽管/虽然),但它的语气比 although 更强,且带有明显的强调意味,强调“程度之深”。

        \begin{itemize}
            \item \textbf{原文:} Much as we may pride ourselves on our good taste...
            \item \textbf{翻译:} 尽管我们可能对自己的高雅品位感到非常自豪……
            \item \textbf{潜台词:} 虽然我们确实非常自豪,但事实(被广告影响)却无情地摆在面前。
        \end{itemize}
        \item \textbf{语法结构拆解}
        
        这个句型实际上是一种部分倒装(Fronting)。

        \begin{itemize}
            \item \textbf{正常语序:} Though we may pride ourselves on our good taste \textbf{much}...
            \item \textbf{倒装语序:} Much + as + 主语 + 谓语...
        \end{itemize}
        
        注意: 这里的 as 虽然引导让步从句,但它不能和 but 连用(英语语法通病:有了“虽然”就不能有“但是”)。

        \item \textbf{同类句式拓展(形容词/副词 + as)}
        
        Much as 是这个家族中最常用的成员,但你也可以替换成其他形容词或副词来表达“尽管”:

        \begin{itemize}
            \item \textbf{形容词开头:}

            \es{Young as he is, he is very experienced. (尽管他很年轻,他却很有经验。)}

            \item \textbf{副词开头:}

            \es{Hard as he worked, he failed the exam. (尽管他工作很努力,他还是考试挂科了。)}

            \item \textbf{名词开头(注意名词前通常不加冠词):}

            \es{Child as he is, he knows a lot. (尽管他是个孩子,他懂得却很多。)}
        \end{itemize}

        \item \textbf{Much as 的特殊性}
        
        为什么课文里用 Much as 而不用 Although?

        \begin{itemize}
            \item \textbf{强调程度:} Much 直接摆在句首,一眼就让读者感受到那股“自豪劲儿”有多大。
            \item \textbf{固定搭配:} Much as 后面经常接 \textbf{feel, like, hate, want, pride oneself on} 等表示情感或心理状态的词。
        \end{itemize}
        \item \textbf{举一反三(对比练习)}
        \begin{itemize}
            \item \textbf{普通说法:} Although I like the car, I can't afford it.
            
            \es{ \textbf{高级说法:} Much as I like the car, I can't afford it. (尽管我非常喜欢这辆车,但我买不起。)}
            
            \item \textbf{普通说法:} Although I hate to say it, he is right.
            
            \es{ \textbf{高级说法:} Much as I hate to say it, he is right. (尽管我很不情愿这么说,但他是对的。)}
        \end{itemize}
    \end{enumerate}
\end{multicols}

\wsitem{Make it possible for sb. to do sth}
\begin{multicols}{1}
    这是一个非常高效的\textbf{“功能性”句型}。在《新概念英语》第三册中,它经常被用来描述某种技术、条件或行为如何“打破了原有的限制”,让某事变得可行。
    
    其核心在于使用 it 作为形式宾语,以保持句子的平衡感。

    \begin{enumerate}
        \item \textbf{语法结构拆解}
        
        标准的公式如下:
        
        主语 + make + it + 形容词 (possible/easy/difficult) + for somebody + to do something

        \begin{itemize}
            \item \textbf{it:} 形式宾语(代指后面真正的内容)。
            \item \textbf{possible:} 宾语补足语(形容词,描述状态)。
            \item \textbf{to do something:} 真正的宾语(如果不把这个长块放到后面,句子会显得“头重脚轻”)。
        \end{itemize}
        \item \textbf{为什么不用 "Make sb. possible"? (常见错误)}
        
        这是很多学习者最容易掉进去的陷阱:

        \begin{itemize}
            \item \textbf{Wrong:} Advertisers make us possible to buy things.
            \item \textbf{Right:} Advertisers make it possible for us to buy things.
        \end{itemize}

        逻辑辨析: 在英语中,possible 的主语通常是一件事,而不是一个人。你不能说“一个人是可能的”,只能说“做某事是可能的”。所以必须借用 it 来指代“做某事”这件事。

        \item \textbf{灵活变体}
        
        你可以通过更换中间的形容词,来表达不同的“难易程度”:

        \begin{itemize}
            \item Easy:使……变得容易
            
            \es{Smartphones make it easy for us to stay in touch.}
            
            \item Difficult:使……变得困难
            
            \es{The heavy rain made it difficult for the ship to stay on course.}
            
            \item Necessary:使……变得必要
            
            \es{The new law makes it necessary for companies to protect user data.}
        \end{itemize}
    \end{enumerate}
\end{multicols}

\wsitem{Not only... but... as well}
\begin{multicols}{1}
    这是一个非常经典的变体结构。通常我们最熟悉的是 "Not only... but also...",但在地道的书面语和口语中,为了避免重复或调整节奏,经常会将 also 替换为放在句末的 as well。
    
    这个结构有三个最核心的知识点:连词位置、倒装要求以及语感平衡。

    \begin{enumerate}
        \item \textbf{语法结构与位置}
        当你使用 as well 替代 also 时,它的位置必须挪到句子的末尾。
        \begin{itemize}
            \item \textbf{标准型:} \text{Not only A \textbf{but also} B.}
            \item \textbf{变体型:} \text{Not only A \textbf{but} B \textbf{as well}.}
        \end{itemize}
        \item \textbf{核心考点:部分倒装 (Partial Inversion)}
        
        这是该句型最容易出错的地方:当 Not only 位于句首引导句子时,第一个分句必须进行部分倒装(即将助动词/情态动词提到主语前),而 but 后的分句顺序保持不变。

        \es{\textbf{错误写法:} Not only he is a doctor, but he is an author as well.}

        \es{\textbf{正确写法:} Not only is he a doctor, but he is an author as well.}
        
        \textbf{Inversion Formula:} Not only + [Auxiliary/Do/Be] + Subject + Verb...

        \item \textbf{为什么选择用 "As well"?}
        在《新概念英语》或文学作品中,选择这个变体通常有两个目的:

        \begin{itemize}
            \item \textbf{避免单调:} 如果一篇文章里多次出现 also,换成 as well 或 too 会让文字更有灵性。
            \item \textbf{节奏感:} as well 放在句末可以起到一种“压轴”的效果,读起来更有停顿感和力度。
        \end{itemize}
    \end{enumerate}
\end{multicols}

\wsitem{So... that...}
\begin{multicols}{1}
    这是一个在英语中极高频且非常实用的程度状语从句结构,核心含义是**“如此……以至于……”**。
    
    它通过连接“因”与“果”,让句子的逻辑感瞬间拉满。

    \begin{enumerate}
        \item \textbf{语法结构剖析 (Structure)}
        
        So + 形容词/副词 + that + 结果从句
        

        \begin{itemize}
            \item \textbf{So:} 后面紧跟的是\textbf{程度}(因)。
            \item \textbf{That:} 引导的是\textbf{结果}(果)。在非正式口语中,that 常常可以省略。
        \end{itemize}
        
        
        
        
        \item \textbf{四大核心用法 (Key Usage)}
        \begin{itemize}
            \item \textbf{形容词:} So + adj + that
            
            \es{The book was so interesting that I read it in one night.}

            \item \textbf{副词:} So + adv + that
            
            \es{He ran so fast that I couldn't catch him.}
            
            \item \textbf{数量词:} So + many/few + n. (pl)
            
            \es{There were so many people that we couldn't find a seat.}
            
            \item \textbf{不可数:}So + much/little + n. (u)
            
            \es{He has so much money that he can buy anything.}
        \end{itemize}
        \item \textbf{易混淆辨析:So... that vs. Such... that}
        
        这是很多学习者最头疼的地方。区分它们的秘诀在于:So 后面跟“词” (adj/adv),Such 后面跟“名” (noun)。

        \begin{itemize}
            \item \textbf{So:} The weather was \textbf{so} cold that... (cold 是形容词)
            \item \textbf{Such:} It was \textbf{such} a cold day that... (day 是名词)
        \end{itemize}
        \item \textbf{句式进阶:倒装 (Inversion)}
        
        为了强调某种程度,你可以把 So 提前到句首,此时主句需要进行\textbf{部分倒装}:

        \begin{itemize}
            \item \textbf{正常语序:} The wind was \textbf{so} strong that the trees fell.
            \item \textbf{倒装语序:} \textbf{So} strong \textbf{was the wind} that the trees fell. (这种写法在书面语中非常高级,很有张力。)
        \end{itemize}
    \end{enumerate}
\end{multicols}

\grammarquestions
\wsitem{Much as we may pride ourselves on our good taste...中的.much as we may pride...是什么写法?}
\begin{multicols}{1}
    这是一个非常地道的让步状语从句 (Concessive Clause)。这种写法在《新概念三册》中非常常见,主要用于增强语气的对比感。
    
    这种结构通常被称为 “as / though 引导的倒装让步从句”。

    \begin{enumerate}
        \item \textbf{结构拆解:公式化理解}
        
        其标准公式为:

        [形容词/副词/名词/动词] + as + 主语 + 动词

        在这一句中:
        \begin{itemize}
            \item Much: 是副词,原句正常语序应该是 Although we may pride ourselves on our good taste much...
            \item As: 相当于 Although,但它强制要求把被强调的词(Much)提前。
        \end{itemize}

        \item \textbf{语气差异 (Tone)} \begin{itemize} \item \textbf{Although / Even if}: 平铺直叙,表示“虽然”。 \item \textbf{Much as...}: 带有明显的\textbf{转折力度}。它先肯定一个事实(我们非常引以为傲),紧接着给出一个截然相反的结论(其实我们并不自由)。这种写法在议论文中显得非常有张力。 \end{itemize}

        \item \textbf{变换练习 (Transformations)}
        \begin{itemize}
            \item \textbf{形容词提前}: \textit{Rich as he is, he is unhappy.} (虽然他很有钱,但他并不快乐。)
            \item \textbf{名词提前}: \textit{Child as he is, he knows a lot.} (虽然他是个孩子,但他懂很多。注意:名词提前通常不带冠词。)
            \item \textbf{动词提前}: \textit{Try as he might, he couldn't open the door.} (尽管他努力尝试了,但还是打不开门。)
        \end{itemize}

        \item \textbf{语义重心:广告的“软暴力”}
    
        作者使用这种写法,是为了形成强烈的讽刺感:

        \begin{itemize}
            \item \textbf{前半句:}我们自以为审美高级、品味卓越(虚假的自由感)。
            \item \textbf{后半句:}我们实际上被广告牵着鼻子走(残酷的真相)。
        \end{itemize}
    \end{enumerate}

\end{multicols}

\wsitem{These days, advertisers not only offer free samples, but free cars, free houses, and free trips round the world as well. 为什么不是These days, advertisers offer not only free samples,...}
\begin{multicols}{1}
    从传统语法(甚至是很多英语考试的满分标准)来看,你的直觉是对的:not only 和 but (also) 后面连接的成分应该是性质相同、结构对称的。

    但在实际的文学写作和地道表达中,这种“不称”反而有一种特殊的节奏感。

    \begin{enumerate}
        \item \textbf{严格对称 vs. 实际用法}
        
        按照你追求的对称美,句子应该是这样的:
        
        \begin{itemize}
            \item \textbf{对称版:} \textit{Advertisers offer \textbf{not only [free samples]}, but \textbf{also [free cars]}.}
            \item (分析:not \ only 后接名词短语,$but$ 后也接名词短语。这是完美的平衡。)
        \end{itemize}而原句的结构是:
        \begin{itemize}
            \item \textbf{原句版:} \textit{Advertisers \textbf{not only [offer free samples]}, but \textbf{[free cars...]} as well.}
            \item (分析:not \ only 后接的是动词短语,而 but 后接的是一串名词短语。看起来确实“歪”了。)
        \end{itemize}
        \item \textbf{为什么作者敢这么写?(语法的“包容性”)}
        这种“不对称”其实在英语中被广泛接受,原因有二:
        \begin{itemize}
            \item \textbf{动词共享逻辑:} 在读者的潜意识里,but 后面其实省略了重复的动词。原句逻辑是:\textit{...not only [offer samples], but [offer free cars... as well].} 既然第二个 offer 是多余的,作者就把它省掉了。
            \item \textbf{强调重心的偏移:} 当作者把 not \ only 放在 offer 前面时,他其实是在提醒你:“注意看,这帮广告商的行为要开始升级了!”这是一种语气上的预警,而不是单纯的逻辑归类。
        \end{itemize}
        \item \textbf{As well 的补偿作用}
        
        注意原句末尾那个 as well。这也是为什么作者敢在前面搞“不对称”的原因。在英语中,如果你没有用标准配对的 but \ also,而是用了 but \ ... \ as \ well,句子的结构重心就会变得非常灵活。
        
        \textbf{语义天平:} Not \ only 负责拉开序幕,as \ well 在句末负责收尾。这种“头尾呼应”抵消了中间成分不对称带来的突兀感。

        \item \textbf{这种写法在医学/科学写作中可行吗?}
        
        作为前医学生,这是一个非常关键的区别:
        
        \begin{itemize} 
            \item \textbf{在论文/报告中:} 请务必坚持对称! 
            
            \es{ The drug \textbf{not only} reduced the fever \textbf{but also} eliminated the infection. (对称,专业,严谨)。 }

            \item \textbf{在文学/散文中(如你读的这段):} 可以追求节奏。 
            
            这种略带随意的结构听起来更像是一个人在你耳边侃侃而谈,而不是在读说明书。 
        \end{itemize}
    \end{enumerate}
\end{multicols}

\newpage