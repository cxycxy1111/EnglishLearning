\section{Lesson 17 The longest suspension bridge in the world}

\begin{paracol}{2}

Verrazano, \newsentence{an Italian about whom little is known}, sailed into New York Harbour in 1524 and named it Angouleme. 

\switchcolumn

\chinesetext{1524年,一位鲜为人知的意大利人维拉萨诺驾船驶进纽约港,并将该港名为安古拉姆。}

\switchcolumn*

He described it as 'a very \nw{agreeable} situation located within two small hills in the \nw{midst} of which flowed a great river.' 

\switchcolumn

\chinesetext{他对该港作了这样的描述:“地理位置十分适宜,位于两座小山的中间,一条大河从中间流过”。}
\nwe{agreeable}{əˈɡriːəbl}{adj. 令人愉快的;友善的,讨人喜欢的;欣然同意的;适合的;}
\nwe{situation}{ˌsɪtʃuˈeɪʃn}{n. 情况;地理位置;职业;}
\nwe{locate}{ˈloʊkeɪt}{v. 找到,探明,确定…的位置;放置,设立,建立;创办于;}
\nwe{midst}{mɪdst}{n. 中部,中间,当中;}

\switchcolumn*

Though Verrazano is by no means considered to be a great explorer, his name will probably remain \nw{immortal}, for on November 21st, 1964, the longest \nw{suspension} bridge in the world was named after him.

\switchcolumn

\chinesetext{虽然维拉萨诺绝对算不上一个伟大的探险家,但他的名字将流芳百世,因为1964年11月21日建成的一座世界上最长的吊桥是以他的名字命名。}
\nwe{immortal}{ɪˈmɔːrtl}{adj. 不死的;永恒的,不朽的;神的;流芳百世的;n. 神仙;流芳百世的人;不朽的作家;}
\nwe{suspension}{səˈspenʃn}{n. 悬浮;暂停,暂缓;悬架;悬浮液;}

\switchcolumn*

The Verrazano Bridge, which was designed by Othmar Ammann, joins \nw{Brooklyn} to \nw{Staten} Island. 

\switchcolumn

\chinesetext{维拉萨诺大桥由奥斯马.阿曼设计,连结着布鲁克林与斯塔顿岛。}
\nwe{Brooklyn}{ˈbrʊklɪn}{n. 布鲁克林(美国纽约市西南部的一区);}
\nse{Staten}{}{n. 斯塔藤(岛),位于美国纽约;}

\switchcolumn*

\newsentence{It has a span of 4,260 feet. }

\switchcolumn

\chinesetext{桥长4,260英尺。}
\nwe{span}{spæn}{n. 持续时间,间隔;\textbf{跨度},墩距;范围;v. 持续;横跨,跨越;包括,涵盖;}

\switchcolumn*

The bridge is so long that the shape of the earth had to be \ns{taken into account} by its designer. 

\switchcolumn

\chinesetext{由于桥身太长,设计者不得不考虑了地表的形状。}
\nse{take into account}{}{把……计算在内;考虑到;}

\switchcolumn*

Two great towers support four huge \nw{cables}. 

\switchcolumn

\chinesetext{两座巨塔支撑着4根粗大的钢缆。}
\nwe{cable}{ˈkeɪbl}{n. 电缆;缆绳,钢索;有线电视;电报;v. 给…发电报;给…安装有线电视;}

\switchcolumn*

The towers are built on immense underwater platforms make of steel and \nw{concrete}. 

\switchcolumn

\chinesetext{塔身建在巨大的水下钢盘混凝土平台上。}
\nwe{concrete}{ˈkɑːŋkriːt}{n. 混凝土;v. 用混凝土浇筑;adj. 混凝土制的;具体的;实在的;}

\switchcolumn*

\newsentence{The platforms extend to a depth of over 100 feet under the sea. }

\switchcolumn

\chinesetext{平台深入海底100英尺。}

\switchcolumn*

These alone took sixteen months to build. 

\switchcolumn

\chinesetext{仅这两座塔就花了16个月才建成。}

\switchcolumn*

Above the surface of the water, \newsentence{the towers rise to a height of nearly 700 feet}. 

\switchcolumn

\chinesetext{塔身高出水面将近700英尺。}

\switchcolumn*

\newsentence{They support the cables from which the bridge has been \nw{suspended}. }

\switchcolumn

\chinesetext{高塔支撑着钢缆,而钢缆又悬吊着大桥。}
\nwe{suspend}{səˈspend}{v. 暂停;延缓;挂;悬;}

\switchcolumn*

\newsentence{Each of the four cables contains 26,108 \nw{lengths} of wire. }

\switchcolumn

\chinesetext{4根钢缆中的每根由26,108股钢绳组成。}
\nwe{length}{leŋθ}{n. 长度;篇幅;时间的长短;有…长度的;一节;池长;艇位;马位;\textbf{根,段;}}

\switchcolumn*

\newsentence{It has been \nw{estimated} that} if the bridge were packed with cars, it would still only be carrying a third of its total \nw{capacity}. 

\switchcolumn

\chinesetext{据估计,若桥上摆满了汽车,也只不过是桥的总承载力的1/3。}
\nwe{estimate}{ˈestɪmət , ˈestɪmeɪt}{n. 估计,预测;报价,预算书;评价,判断;vt. 估计,估算;评价,评论;估量,估价;}
\nwe{capacity}{kəˈpæsəti}{n. 容量;才能;职责;生产能力;载客量;adj. 座无虚席的;}

\switchcolumn*

However, size and strength \newsentence{are not the only important things about} this bridge. 

\switchcolumn

\chinesetext{然而,这座桥重要特点不仅是它的规模与强度。}

\switchcolumn*

Despite its \nw{immensity}, it is both simple and \nw{elegant}, fulfilling its designer's dream to create 'an enormous object drawn as \nw{faintly} as possible'.


\switchcolumn

\chinesetext{尽管此桥很大,但它的结构简单,造型优美,实现了设计者企图创造一个“尽量用细线条勾画出一个庞然大物”的梦想。}
\nwe{immensity}{ɪˈmɛnsɪti}{n. 无限,广大,巨大;}
\nwe{elegant}{ˈelɪɡənt}{adj. 优美的;典雅的;简洁的;}
\nwe{faintly}{feɪntlɪ}{adv. 微弱地;隐约地;虚弱地;有点;}

\switchcolumn*

\end{paracol}

\grammarpoints
\wsitem{...an Italian about whom little is known,..}
\begin{multicols}{1}
    这是一个非常典型的\textbf{“介词 + 关系代词”}引导的定语从句结构。在英语中,这种结构通常用于正式文体,用来对前面的名词(先行词)进行补充说明。
    \begin{enumerate}
        \item \textbf{拆解结构 (Deconstruction)}
        \begin{itemize}
            \item \textbf{先行词:} \text{Verrazano} (维拉扎诺,一位意大利航海家)。
            \item \textbf{关系代词:} \text{whom} (指代 \text{Verrazano},在从句中作宾语)。
            \item \textbf{介词:} \text{about} (与后面的动词或形容词搭配,表示“关于”)。
            \item \textbf{原句还原:} \text{Little is known \textbf{about} Verrazano.} (关于维拉扎诺,世人知之甚少。)
        \end{itemize}

        \item \textbf{为什么用 Whom 而不是 Who?}
        \begin{itemize}
            \item \textbf{宾格要求:} 因为它跟在介词 \text{about} 后面,必须使用宾格形式 \text{whom}。
            \item \textbf{正式度:} 在正式写作(如《新概念英语》或历史文献)中,介词提前到关系代词之前的做法非常普遍,这能让句子结构更加严谨、庄重。
        \end{itemize}

        \item \textbf{逻辑功能:插入式补充 (Parenthetical Information)}
        \begin{itemize}
            \item 这个短语被放在两个逗号之间,作为一个**非限制性定语从句**。
            \item 它不影响主句的完整性(主句是:\textit{Verrazano sailed into New York Harbour...}),但它提供了一个重要的背景:这位航海家在历史上其实挺神秘的。
        \end{itemize}

        \item 实战替换练习(掌握“介词 + Whom”的逻辑)
        \begin{itemize}
            \item \textbf{With whom (与某人一起)}
            \begin{itemize}
                \item \textbf{基础逻辑:} I work \textit{with} him.
                \item \textbf{高级表达:} He is a colleague \textbf{with whom} I work.
                \item \textit{(他是我的一位同事,我与他一起工作。)}
            \end{itemize}
            
            \item \textbf{To whom (对/向某人)}
            \begin{itemize}
                \item \textbf{基础逻辑:} I spoke \textit{to} the manager.
                \item \textbf{高级表达:} This is the manager \textbf{to whom} I spoke.
                \item \textit{(这就是那位我曾与之谈话的经理。)}
            \end{itemize}
            
            \item \textbf{For whom (为某人)}
            \begin{itemize}
                \item \textbf{基础逻辑:} She bought the gift \textit{for} the child.
                \item \textbf{高级表达:} The child \textbf{for whom} she bought the gift was happy.
                \item \textit{(那个收到她买的礼物的孩子很快乐。)}
            \end{itemize}
            
            \item \textbf{By whom (被某人/由某人)}
            \begin{itemize}
                \item \textbf{基础逻辑:} The book was written \textit{by} an author.
                \item \textbf{高级表达:} The author \textbf{by whom} the book was written is famous.
                \item \textit{(写这本书的那位作家很有名。)}
            \end{itemize}
        \end{itemize}
    \end{enumerate}

\end{multicols}

\wsitem{It has been estimated that ...}
\begin{multicols}{1}
    "It has been estimated that..." 是一个非常地道的英语句式,常用于引用数据、统计结果或专家推测。它属于\textbf{“无人称被动结构” (Impersonal Passive Structure)},能让你的陈述听起来既客观又权威。

    \begin{enumerate}
        \item \textbf{"It has been estimated that..." 深度解析}
    
    
    \begin{enumerate}
        \item \textbf{语法功能:形式主语与现完被动 (Present Perfect Passive)}
        \begin{itemize}
            \item \textbf{It (形式主语):} 替代后面沉重的 \textit{that} 从句。
            \item \textbf{Has been estimated:} 使用现在完成时的被动语态,暗示这是一个\textbf{直到目前为止已经得出的估算结果}。
            \item \textbf{逻辑意义:} 相当于 "People estimate that...",但使用被动语态隐去了具体是谁估算的,使重点集中在\textbf{数据}本身。
        \end{itemize}
        
        \item \textbf{语用功能:增加说服力}
        \begin{itemize}
            \item 在学术写作或新闻报道中,当你没有精准数字,但有一个大概范围时,这个句式是最佳选择。
            \item 它避免了主观臆断(如 \textit{I think}),让结论显得经过了多方核实。
        \end{itemize}
    \end{enumerate}

    \item \textbf{高级表达进阶:引述事实的同类句式}

    除了 estimated,你可以根据事实的性质更换动词,这种结构在雅思写作中非常加分:

    \begin{enumerate}
        \item \textbf{It has been reported that...}
        \begin{itemize}
            \item \textbf{语感:} \textit{据报道……} (常用于新闻或公认的事实)。
            
            \es{\textbf{It has been reported that} more than 1,500 people lost their lives when the Titanic sank.}
        \end{itemize}
        
        \item \textbf{It has been suggested that...}
        \begin{itemize}
            \item \textbf{语感:} \textit{有人提议/暗示……} (用于非定论、待讨论的观点)。
            
            \es{\textbf{It has been suggested that} the lookouts did not have binoculars at the time.}
        \end{itemize}
        
        \item \textbf{It is generally believed that...}
        \begin{itemize}
            \item \textbf{语感:} \textit{普遍认为……} (描述大众共识)。
            
            \es{\textbf{It is generally believed that} the ship was unsinkable.}
        \end{itemize}
        
        \item \textbf{It has been calculated that...}
        \begin{itemize}
            \item \textbf{语感:} \textit{据计算……} (侧重于数学或科学运算结果)。
            
            \es{\textbf{It has been calculated that} the iceberg was twice the size of the ship.}
        \end{itemize}
    \end{enumerate}

    \end{enumerate}
\end{multicols}

\wsitem{... be not the only important things about ...}
\begin{multicols}{1}
    这个句式通常用于拓宽读者的视野。它的逻辑核心是:虽然你现在关注的这些点确实重要,但绝不是全部,还有其他同样(甚至更重要)的因素需要被考虑。

    这是一个非常好的“引出新观点”的过渡句。

    \begin{enumerate}
        \item \textbf{逻辑功能:打破局限 (Breaking Limitations)}
        \begin{itemize}
            \item \textbf{核心逻辑:} 这是一个“不仅……而且……”的隐含表达。它承认当前关注点(A)的重要性,但强调(B, C, D)的存在。
            \item \textbf{语用:} 常用于文章的中间段落,作为从“表面事实”转向“深层分析”的桥梁。
        \end{itemize}
        
        \item \textbf{句式结构解析}
        \begin{itemize}
            \item \textbf{Subject + are/were not the only important things about + [Noun/Topic]}
            
            \es{Speed and size were \textbf{not the only important things about} the Titanic. (速度和尺寸并不是泰坦尼克号唯一重要的事情。)}
        \end{itemize}

        \item \textbf{高级表达进阶:同类逻辑句式}

        如果你想在写作中表达“这还不是全部”,可以使用以下更具张力的句式:

        \begin{itemize}
            \item \textbf{...is but one aspect of...}
            \begin{itemize}
                \item \textbf{语感:} \textit{……仅仅是……的一个方面} (极其正式,暗示还有很多其他层面)。
                
                \es{The size of the iceberg \textbf{is but one aspect of} the tragedy.}
            \end{itemize}
            
            \item \textbf{...cannot be defined solely by...}
            \begin{itemize}
                \item \textbf{语感:} \textit{……不能仅仅由……来定义} (强调事物的复杂性)。
                
                \es{Verrazano's voyage \textbf{cannot be defined solely by} the discovery of New York Harbour.}
            \end{itemize}
            
            \item \textbf{There is more to ... than ...}
            \begin{itemize}
                \item \textbf{语感:} \textit{关于……,除了……之外,还有更多值得关注的内容} (非常地道的表达)。
                
                \es{\textbf{There is more to} the story of Dimitri \textbf{than} just a missing lamb.}
            \end{itemize}
        \end{itemize}
    \end{enumerate}
\end{multicols}

\wsitem{Lengths of ...}
\begin{multicols}{1}
    在英语中,lengths of 是一种非常精准的“单位化”用法。它将不可数或难以计数的“线条状物质”(如电线、绳索、布料)切分成可以计数的特定长度的单位。

    \begin{enumerate}
        \item \textbf{"Lengths of" 的逻辑与语法解析}
        \begin{itemize}
            \item \textbf{物质的“量词化” (Quantification of Continuous Matter)}
            \begin{itemize}
                \item \textbf{核心逻辑:} 像 \textit{wire} (电线) 这种东西在物理属性上通常被视为不可数的“连续体”。如果我们直接说 \textit{26,108 wires},听起来像是 2 万多根各不相同的独立电线。
                \item \textbf{功能:} 使用 \textbf{lengths of} 明确了这些是“\textbf{一段段等长/特定长度的}”材料。它把“物质”变成了“可数的单体”。
            \end{itemize}
            
            \item \textbf{强调“跨度”与“规模”}
            \begin{itemize}
                \item \textbf{语境解析:} 在描述大桥缆索时,使用这个词能体现出工程的严谨。这 26,108 段钢丝被紧密缠绕在一起,每一段都贡献了特定的长度。
                \item \textbf{画面感:} 它让读者联想到工厂里切割出来的一卷卷、一段段标准化的建筑材料。
            \end{itemize}
            
            \item \textbf{语法结构:数量词 + lengths of + 不可数名词}
            \begin{itemize}
                \item 这里的 \textbf{lengths} 是复数,因为前面的数字大于 1。
                \item 类似的结构包括:\textit{pieces of} (件/块), \textit{sheets of} (张), \textit{strips of} (条)。
            \end{itemize}
        \end{itemize}

        \item \textbf{高级表达进阶:类似“形状/长度”量词}
        
        在描述不同形状的物体时,选用精准的量词能让你的英语瞬间变得专业:

        \begin{itemize}
            \item \textbf{Strips of (条/带状)}
            \begin{itemize}
                \item \textbf{适用:} 纸条、布条、金属带。
                
                \es{The technician used \textbf{strips of} copper to reinforce the cable.}
            \end{itemize}
            
            \item \textbf{Strands of (股/缕)}
            \begin{itemize}
                \item \textbf{适用:} 头发、细丝、纤维 (多指拧在一起的细线)。
                
                \es{The bridge cable is made of thousands of \textbf{strands of} steel wire.}
            \end{itemize}
            
            \item \textbf{Sections of (节/段)}
            \begin{itemize}
                \item \textbf{适用:} 管道、铁轨、隧道 (强调可以拆分或组合的各个部分)。
                
                \es{Engineers are replacing broken \textbf{sections of} the pipe.}
            \end{itemize}
        \end{itemize}
    \end{enumerate}
\end{multicols}

\wsitem{... rise to a height of ...}
\begin{multicols}{1}
    这是一个非常地道的英语书面语表达,通常出现在描述建筑、地理特征或工程成就的文本中。它的核心功能是将“上升”这一动作与“高度”这一具体数值优雅地连接起来。

    \begin{enumerate}
        \item \textbf{"Rise to a height of" 深度解析}
        \begin{enumerate}
            \item \textbf{逻辑构成:动作 + 终点状态}
            \begin{itemize}
                \item \textbf{Rise (动词):} 描述物体向上延伸的状态。注意这里用一般现在时,表示该建筑或地理特征的“恒定属性”。
                \item \textbf{To a height of (介词短语):} 这里的 \textit{to} 表示程度或终点,引出具体的高度数据。
                \item \textbf{对比:}
                \begin{itemize}
                    \item \textit{The towers are 700 feet high.} (简单陈述事实,语气平淡。)
                    \item \textit{The towers \textbf{rise to a height of} 700 feet.} (富有动感,仿佛能看到塔楼拔地而起的雄伟姿态。)
                \end{itemize}
            \end{itemize}
            
            \item \textbf{语境功能:强调“规模”与“高度”}
            \begin{itemize}
                \item 这种表达方式常见于导游手册、百科全书或工程报告。它不仅告诉读者数字,还在心理上构建了一种从地面向上仰望的\textbf{视觉透视感}。
            \end{itemize}
            
            \item \textbf{介词的精准使用}
            \begin{itemize}
                \item \textbf{Above the surface of the water:} 这句话先设定了一个参照基准(水面)。
                \item \textbf{Height of [Number]:} \textit{of} 后面直接接具体的度量衡。
            \end{itemize}
        \end{itemize}
        \item \textbf{高级表达进阶:描述物理维度的同类句式 }
        
        在描述一个物体的长、宽、高或深度时,可以套用以下高级公式:

        \begin{itemize}
            \item \textbf{Extend to a length of... (延伸至……的长度)}
            
            \es{The bridge \textbf{extends to a length of} over 2 miles.(这座桥延伸出的长度超过了 2 英里。)}
            
            \item \textbf{Reach a depth of... (达到……的深度)}
            
            \es{In the North Atlantic, the \textbf{icy waters} \textbf{reach a depth of} thousands of feet.(在北大西洋,冰冷的海域深度可达数千英尺。)}
            
            \item \textbf{Span a distance of... (横跨……的距离)}
            
            \es{The main cables \textbf{span a distance of} 4,200 feet between the towers.(主缆索在两塔之间横跨了 4,200 英尺的距离。)}
        \end{itemize}
    \end{enumerate}
\end{multicols}

\wsitem{Extend to a depth of...}
\begin{multicols}{1}
    这个表达与 "Rise to a height of" 在结构上是镜像对称的。如果说 "rise" 是向上的雄伟,那么 "extend to a depth of" 则是向下的深邃。它常用于描述地基、海底平台、油井或潜水艇的作业范围。
    \begin{enumerate}
        \item \textbf{"Extend to a depth of" 深度解析}
        \item \textbf{高级表达进阶:物理维度的深度描述}
        
        在描述具有“深度”或“厚度”的事物时,可以参考以下表达:

        \begin{enumerate}
            \item \textbf{Penetrate to a depth of... (穿透至……的深度)}
            \begin{itemize}
                \item \textbf{语感:} 强调“刺破”或“钻入”某个坚硬的物体(如土壤或岩层)。
                \es{The drill \textbf{penetrated to a depth of} 500 meters into the seabed.}
            \end{itemize}
            
            \item \textbf{Be submerged to a depth of... (淹没/潜入至……的深度)}
            \begin{itemize}
                \item \textbf{语感:} 描述物体整体位于水下的状态。
                \es{The submarine was \textbf{submerged to a depth of} 200 feet in the \textbf{icy waters}.}
            \end{itemize}
            
            \item \textbf{Anchored at a depth of... (锚定在……的深度)}
            \begin{itemize}
                \item \textbf{语感:} 强调固定、稳固。
                \es{The massive pillars are \textbf{anchored at a depth of} 150 feet below the mud.}
            \end{itemize}
        \end{enumerate}
    \end{enumerate}
\end{multicols}

\wsitem{They support the cables from which the bridge has been suspended}
\begin{multicols}{1}
    这也是一个\textbf{“介词 + 关系代词”}引导的定语从句结构,与我们about whom 逻辑一致,但它描述的是物理上的支撑与悬挂关系。
    \begin{enumerate}
        \item \textbf{"From which" 的逻辑与语法解析}
        \begin{itemize}
            \item \textbf{拆解与还原 (Deconstruction)}
            \begin{itemize}
                \item \textbf{先行词:} The cables (缆索)。
                \item \textbf{关系代词:} which (指代 the cables)。
                \item \textbf{介词:} from (表示来源或支撑点)。
                \item \textbf{逻辑还原:} The bridge has been suspended \textbf{from} the cables. (大桥被悬挂在这些缆索上。)
            \end{itemize}
            
            \item \textbf{为什么用 "From"? (物理逻辑)}
            \begin{itemize}
                \item 在英语中,当你表达“某物吊在/挂在另一物上”时,标准的动词搭配是 \textbf{suspend from} 或 \textbf{hang from}。
                \item \textbf{想象画面:} 缆索在上面,桥面在下面。力是从缆索向下传递的,所以桥是“从” (from) 缆索上垂下来的。
            \end{itemize}
            
            \item \textbf{语法身份:正式定语从句}
            \begin{itemize}
                \item \textbf{正式度:} 把介词 \textit{from} 放在 \textit{which} 前面是极其正式的书面语。
                \item \textbf{非正式改写:} \textit{The cables which the bridge has been suspended \textbf{from}.} (虽然意思一样,但在描述伟大的工程如维拉扎诺大桥时,这种写法显得不够庄重。)
            \end{itemize}
        \end{itemize}
        \item \textbf{实战替换练习:掌握“物理关系”中的介词}
        
        在描述建筑、机械或空间关系时,介词的选择决定了描述的精准度:
        \begin{itemize}
            \item \textbf{Through which (通过/穿过)}
            \begin{itemize}
                \item \textbf{逻辑:} 描述通过某个通道或媒介。
                
                \es{例句: This is the tunnel \textbf{through which} the train passes.(这就是火车穿过的隧道。)}
            \end{itemize}
            
            \item \textbf{On which (在……之上)}
            \begin{itemize}
                \item \textbf{逻辑:} 描述支撑面或基础。
                
                \es{The towers stand on platforms \textbf{on which} the entire weight rests.(塔楼立在平台上,整个重量都压在这些平台之上。)}
            \end{itemize}
            
            \item \textbf{Under which (在……之下)}
            \begin{itemize}
                \item \textbf{逻辑:} 描述位于下方的状态。
                
                \es{There are icy waters \textbf{under which} many secrets are hidden.(在那冰冷的海面之下,隐藏着许多秘密。)}
            \end{itemize}
        \end{itemize}
    \end{enumerate}
\end{multicols}

\wsitem{A span of ...}
\begin{multicols}{1}
    在描述桥梁、翅膀、或者任何具有“跨度”的物体时,"a span of..." 是一个非常专业且精准的量词表达。它不仅指长度,更强调两个支撑点之间的距离。
    \begin{enumerate}
        \item \textbf{深度解析}
        \begin{itemize}
            \item \textbf{物理逻辑:跨越的空间 (The Space Bridged)}
            \begin{itemize}
                \item \textbf{Span (名词):} 原意是指大拇指和小指张开后的距离。在建筑学中,它特指桥梁、拱门在两个支撑塔楼(towers)或桥墩(pillars)之间的那段“悬空长度”。
                \item \textbf{量词化:} 使用 \textit{"a span of"} 将一段距离转化为一个整体的度量单位。
            \end{itemize}
            
            \item \textbf{动词与名词的联动}
            \begin{itemize}
                \item \textbf{作为名词:} \textit{It has \textbf{a span of} 4,260 feet.} (它有 4,260 英尺的跨度。)
                \item \textbf{作为动词:} \textit{The bridge \textbf{spans} the harbour.} (大桥横跨海港。)
            \end{itemize}
            
            \item \textbf{语境功能:强调“无支撑”的壮举}
            
            当你说一座桥长 5 英里时,大家可能觉得没什么;但当你强调它拥有 \textit{"a span of 4,260 feet"} 时,专业人士会立刻意识到这意味着在那 4,000 多英尺的空间里是没有桥墩支撑的,这完全依赖于悬索的力量。
        \end{itemize}

        \item \textbf{高级表达进阶:描述“跨度与范围”的同类句式}
        \begin{itemize}
            \item \textbf{A wing span of... (翼展)}
            \begin{itemize}
                \item \textbf{适用:} 飞机、鸟类。
                \es{The eagle has \textbf{a wing span of} over two meters.}
            \end{itemize}
            
            \item \textbf{A time span of... (时间跨度)}
            \begin{itemize}
                \item \textbf{适用:} 历史时期、项目周期。
                \es{The project was completed over \textbf{a time span of} five years.}
            \end{itemize}
            
            \item \textbf{Over a range of... (在……范围内)}
            \begin{itemize}
                \item \textbf{适用:} 价格、声音、频率。
                \es{The bridge is designed to function \textbf{over a range of} extreme temperatures.}
            \end{itemize}
        \end{itemize}
    \end{enumerate}

\end{multicols}

\newpage