
\section{Lesson 15 Fifty pence worth of trouble}

\begin{paracol}{2}
    Children always \nw{appreciate} small gifts of money. 

    \switchcolumn

    \chinesetext{孩子们总是喜欢得到一些零花钱。}

    \nwe{appreciate}{əˈpriːʃieɪt}{v. 欣赏,重视;感激,欢迎;理解,领会;升值;}

    \switchcolumn*

    Mum or dad, of course, provide a regular supply of \nw{pocket money}, but uncles and ants are always a source of extra income. 

    \switchcolumn

    \chinesetext{爸爸妈妈当然经常给孩子零花钱,但是,叔舅婶姨也是孩子们额外收入来源。}

    \nwe{pocket money}{ˈpɑkɪt ˈmʌni}{n. 零用钱;}

    \switchcolumn*

    \newsentence{With some children}, small sums \ns{go a long way}. 

    \switchcolumn

    \chinesetext{对于有些孩子来说,少量的钱可以花很长一段时间。}

    \nse{go a long way}{}{v. 大有帮助,走了一大段路,采取主动;可以维持很久;}

    \switchcolumn*

    If fifty pence pieces are not exchanged for sweets, they \nw{rattle} for months inside \ns{money boxes}. 

    \switchcolumn

    \chinesetext{如果50便士不拿来换糖吃,则可以放在储蓄罐里叮当响上好几月。}

    \nwe{rattle}{ˈrætl}{vt.\& vi. (使)发出格格的响声,(使)作嘎嘎声;喋喋不休地说话;vi. 迅速而嘎嘎作响地移动,堕下或走动;vt. 使紧张,使恐惧;给(桅索)扎梯绳;n. 嘎嘎声;发出嘎嘎声的儿童玩具;呼噜声;音响构造;}
    \nse{money box}{}{n. [经] 银(钱)箱,金柜;}

    \switchcolumn*

    Only very \nw{thrifty} children manage to \ns{fill up} a money box. 

    \switchcolumn

    \chinesetext{但是能把储蓄罐装满的只有屈指可数的几个特别节俭的孩子。}

    \nwe{thrifty}{ˈθrɪfti}{adj. 节俭的;节约的;}
    \nse{fill up}{}{(使)充满;补;垫;填充;}

    \switchcolumn*

    For most of them, fifty pence is a small price to pay for \nw{a nice big bar of} chocolate.

    \switchcolumn

    \chinesetext{对大部分孩子来说,用50便士来买一大块好的巧克力,是算不了什么的。}
    \nse{a bar of}{}{一块...}

    \switchcolumn*

    My \nw{nephew}, George, has a money box but it is always empty. 

    \switchcolumn

    \chinesetext{我的外甥乔治有一个储蓄罐,但总是空空的。}

    \nwe{nephew}{ˈnefjuː}{n. 侄子,外甥;}

    \switchcolumn*

    \newsentence{Very few of the fifty pence pieces and pound coins I have given him \ns{have found their way there}}. 

    \switchcolumn

    \chinesetext{我给了不少50便士的硬币,但没有几个存到储蓄罐里。}

    \nse{have found one's way there}{}{其中,there指的是储蓄罐。find one's way在这里作“进入”讲,即“放到(储蓄罐)里”。}

    \switchcolumn*

    I gave him fifty pence yesterday and advised him to save it. 

    \switchcolumn

    \chinesetext{昨天,我给了他50便士让存起来。}

    \switchcolumn*

    Instead he bought himself fifty pence worth of trouble. 

    \switchcolumn

    \chinesetext{他却拿这钱给自己买了50便士的麻烦。}

    \switchcolumn*

    \ns{On his way to} the \ns{sweet shop}, he dropped his fifty pence and it \nw{bounced} along the \nw{pavement} and then disappeared down a \nw{drain}. 

    \switchcolumn

    \chinesetext{在他去糖果店的路上,50便士掉在地上,在人行道上跳了几下,掉进了阴沟里。}

    \nwe{bounce}{baʊns}{v. (使)弹起;反射;蹦跳,上下晃动;被拒付,退回;试探;开除;与…探讨;n. 弹跳;弹性;活力;反弹;}
    \nwe{drain}{dreɪn}{v. 排出;(使)流干;喝光;使…耗尽;n. 下水道,排水管;}
    \nse{on one's way to ...}{}{在去往...的路上}
    \nse{sweet shop}{}{糖果店}

    \switchcolumn*

    George \ns{took off} his jacket, rolled up his \nw{sleeves} and pushed his right arm through the drain cover. 

    \switchcolumn

    \chinesetext{乔治脱掉外套,卷起袖子,将右胳膊伸进了阴沟盖。}
    \nse{take off}{}{脱掉;起飞;(使)离开;突然成功;}
    \nwe{sleeve}{sliːv}{n. [机]套筒,套管;袖子,袖套;唱片套;}

    \switchcolumn*

    He could not find his fifty pence piece anywhere, and \ns{what is more}, he could no get his arm out. 

    \switchcolumn

    \chinesetext{但他摸了半天也没找到那50便士硬币,他的胳膊反倒退不出来了。}

    \nse{what is more}{}{更有甚者}

    \switchcolumn*

    A crowd of people gathered round him and a lady \nw{rubbed} his arm with soap and butter, but George was firmly \nw{stuck}. 

    \switchcolumn

    \chinesetext{这时在他周围上了许多人,一位女士在乔治胳膊上抹了肥皂,黄油,但乔治的胳膊仍然卡得紧紧的。}

    \nwe{rub}{rʌb}{v. 摩擦,涂抹;(使)加深痛苦;惹恼;n. 抹;困难;}
    \nwe{stick}{stɪk}{n. 枯枝;棍,棒(状物); 批评,指责;v. 刺,戳;粘贴,附着在;容忍;被证明成立;放置;附着于;固定不动;}

    \switchcolumn*

    The \ns{fire brigade} was called and two \ns{fire \nw{fighter}} freed George using a special type of \nw{grease}. 

    \switchcolumn

    \chinesetext{有人打电话叫来消防队,两位消防队员使用了一种特殊的润滑剂才使乔治得以解脱。}

    \nwe{fighter}{ˈfaɪtɚ}{n. 战士(尤指士兵或拳击者);[航]战斗机,歼击机;斗争者,奋斗者;好斗的人;}
    \nwe{grease}{ɡriːs}{n. 动物油脂;油膏,润滑油;〈俚〉贿赂;vt. 涂油脂于,用油脂润滑;贿赂;}
    \nse{fire brigade}{}{n. 消防队;[军]〈美俚〉特速紧急分遣队;}
    \nse{fire fighter}{}{n. 救火队员;}

    \switchcolumn*

    George was not too upset by his experience because the lady who owns the sweet shop heard about his troubles and \ns{rewarded him with} large box of chocolates.

    \switchcolumn

    \chinesetext{不过,此事并没使乔治过于伤心,因为糖果店老板娘听说了他遇到的麻烦后,赏给他一大盒巧克力。}

    \nse{rewarded sb. with sth.}{}{奖励某人某物}

    \switchcolumn*

\end{paracol}

\grammarpoints
\wsitem{With some children ...}
\begin{multicols}{1}
    \begin{enumerate}
        \item \textbf{视角不同:内在属性 (With) vs. 外在对象 (For)}
        \begin{itemize}
            \item \textbf{With 的逻辑:} 强调“在……的身上”或“对……这种类型的人来说”。它侧重于被修饰对象的性格、特质或心理预期。
            \item \textbf{For 的逻辑:} 侧重于“目的”或“获利者”。如果说 \textit{for children},语感上更像是这笔钱是留给孩子用的,或者对孩子有好处。
            \item \textbf{深度解析:} 这里用 \textbf{With} 是为了表达:在“有些孩子”这类特定的心理账本里,一丁点钱的“价值密度”是很高的。
        \end{itemize}

        \item \textbf{逻辑关联:伴随状态 (Accompaniment)}
        \begin{itemize}
            \item \textbf{核心语义:} \textbf{With} 引导一个前提条件——“一旦涉及到了某些孩子”。
            \item \textbf{搭配:} \textit{Small sums go a long way}(小钱起大作用)。这个结果是伴随着“这些孩子”这个前提而发生的。
        \end{itemize}

        \item \textbf{语感延伸:固定用法的逻辑}
        \begin{itemize}
            \item 在英语中,描述某种规律适用于某人时,常用 \textbf{With}。
            \item \textbf{例句:} \text{It is a different story \textbf{with} John.} (在约翰身上,情况就不同了。)
            \item \textbf{本句逻辑:} 在这些孩子身上,少量的钱就能发挥极大的效用。
        \end{itemize}
    \end{itemize}
\end{multicols}

\wsitem{he dropped his fifty pence and it bounced along the pavement and then disappeared down a drain}
\begin{multicols}{1}
    这句话精准地描绘了物体的运动轨迹。在英语中,介词不仅是连接词,更是“导航仪”。使用 along 和 down 是为了构建一个三维的、连续的动态画面。

    \begin{enumerate}
        \item \textbf{Along (沿着……向前)}
        \begin{itemize}
            \item \textbf{核心逻辑:} 描述物体在水平面上,顺着某个长条状物体的边缘或表面移动。
            \item \textbf{语境解析:} 硬币掉下后,由于惯性和重力,它不会原地停止,而是顺着平整的“人行道” (\textit{pavement}) 的延伸方向水平向前滚动或弹跳。
            \item \textbf{画面感:} 就像在一条直线上滑行。
        \end{itemize}

        \item \textbf{Down (向……下/进入……中)}
        \begin{itemize}
            \item \textbf{核心逻辑:} 描述物体从高处往低处运动,或者从地表进入地下空间。
            \item \textbf{语境解析:} 硬币在水平运动(\textit{along})之后,遇到了一个垂直的缺口——排水沟 (\textit{drain})。此时运动方向发生了 90 度的转变,从地表掉进了深处。
            \item \textbf{画面感:} 一个突然消失的重力坠落。
        \end{itemize}
    \end{itemize}
\end{multicols}

\wsitem{pushed his right arm through the drain cover.}
\begin{multicols}{1}
    在英语的逻辑中,through 强调的是“从三维空间的内部穿过”或“穿过一个有边界的孔隙/障碍物”。

    \begin{enumerate}
        \item \textbf{物理空间的“通道”逻辑 (The "Tunnel" Logic)}
        \begin{itemize}
            \item \textbf{核心定义:} \textbf{Through} 描述的是一个物体从一个开口进入,穿过中间的限制空间,并从另一端(或至少进入其内部)出来的过程。
            \item \textbf{语境解析:} 下水道盖子 (\textit{drain cover}) 通常是由许多金属栅格(\textit{bars})组成的。他的手臂不是简单地碰到了盖子,而是\textit{穿过了栅格之间的空隙},进入了盖子下方的空间。
        \end{itemize}

        \item \textbf{介词对比:Through vs. Across vs. Into}
        \begin{itemize}
            \item \textbf{Across:} 强调在“二维平面”上横跨(如 \textit{walk across the street})。
            \item \textbf{Into:} 侧重于进入动作的结果(如 \textit{put his arm into the drain}),但没有表现出挤过窄缝的“穿透感”。
            \item \textbf{Through:} 侧重于“挤、钻、穿”的过程,精准体现了手臂在栅格缝隙中移动的动态。
        \end{itemize}

        \item \textbf{Through 的“穿透感”公式}
        
        当你想表达“穿过一个窄缝、孔洞或复杂的中间地带”时,请使用 through:

        \begin{itemize}
            \item \textbf{Push a thread through a needle:} 穿针。
            \item \textbf{Walk through a forest:} 穿过森林(树木形成了三维的障碍)。
            \item \textbf{Look through a window:} 透过窗户看(视线穿过介质)。
        \end{itemize}
    \end{itemize}
\end{multicols}

\wsitem{Reward sb. with sth}
\begin{multicols}{1}
    \begin{itemize}
        \item \textbf{1. 动词的“逻辑配方” (Verb Valency)}
        \begin{itemize}
            \item \textbf{Reward} 是一个及物动词,它的\textbf{直接宾语}(Direct Object)通常是“被奖励的人”。
            \item 如果你想说明奖励的具体内容(奖品),英语逻辑要求使用介词 \textbf{with} 来引导这个“工具”或“手段”。
            \item \textbf{公式:} \text{Reward + \textbf{someone} + \textbf{with} + \textbf{something}}。
        \end{itemize}

        \item \textbf{2. 逻辑对比:Reward vs. Give}
        \begin{itemize}
            \item \textbf{Give:} 可以接双宾语。你可以说 \textit{Give him a box of chocolates.}(这里不需要介词)。
            \item \textbf{Reward:} 不支持双宾语。如果你说 \textit{Reward him a box...},听起来就像是把“一盒巧克力”当成了被奖励的对象,逻辑不通。
        \end{itemize}

        \item \textbf{3. 介词 "with" 的功能}
        \begin{itemize}
            \item 这里的 \textbf{with} 表示“伴随”或“通过……方式”。它告诉读者:这个奖励的行为是通过“一盒巧克力”这个载体来实现的。
        \end{itemize}
    \end{itemize}
\end{multicols}

\wsitem{Be upset by...}
\begin{multicols}{1}
    "Be upset by..." 是一个非常实用的情感表达结构,逻辑上表示“因……而感到心烦/难受/沮丧”。它强调的是一种由于外部因素导致的心理平衡失调。

    \begin{enumerate}
        \item \textbf{核心语义 (Core Meaning)}
        \begin{itemize}
            \item \textbf{Upset 的逻辑:} 字面意思是“翻转、颠倒”。在情感上,它指原本平静的心情被搅乱了。
            \item \textbf{情感广度:} 这是一个“多功能”的情感词。它既可以包含 angry (生气),也可以包含 sad (伤心) 或 worried (担心)。
            \item \textbf{By 的作用:} \textbf{by} 引导的是产生这种负面情绪的“触发点”或“原因”。
        \end{itemize}

        \item \textbf{被动 vs. 主动 (Passive vs. Active)}
        \begin{itemize}
            \item \textbf{Passive (常用):} I was \textbf{upset by} the news. (我被这消息搞得心烦意乱。) —— 强调受到的影响。
            \item \textbf{Active (使动):} The news \textbf{upset} me. (这消息让我很难受。) —— 强调消息本身的破坏力。
        \end{itemize}

        \item \textbf{逻辑辨析:Upset by vs. Upset about}
        \begin{itemize}
            \item \textbf{Upset by:} 侧重于具体的动作或事件直接冲击了你。 (e.g., \textit{upset by his rude words})
            \item \textbf{Upset about:} 侧重于对某个话题或状况感到忧虑。 (e.g., \textit{upset about the economy})
        \end{itemize}
    \end{enumerate}
\end{multicols}

\wsitem{Very few of the fifty pence pieces and pound coins I have given him have found their way there}
\begin{multicols}{1}
    作者从讲述历史转为讲述自己儿子的存钱罐。这里使用现在完成时 (Present Perfect) —— "have found their way",是由其特定的逻辑语境决定的。
    \begin{enumerate}
        \item \textbf{延续性逻辑:从过去持续到现在 (Continuity)}
        \begin{itemize}
            \item \textbf{逻辑意义:} 存钱是一个\textbf{持续的行为}。从孩子拥有存钱罐那天起,直到作者说话的这一刻,这个“给钱”和“钱进入存钱罐”的过程一直在断断续续地发生。
            \item \textbf{对比:} 
            \begin{itemize}
                \item 如果用过去时 (\textit{found}),听起来像是只给过一次钱,动作已经彻底结束。
                \item 用现在完成时 (\textbf{have found}),强调的是截止到\textbf{目前为止}的累积结果。
            \end{itemize}
        \end{itemize}

        \item \textbf{强调“现状”的影响 (Resulting State)}
        \begin{itemize}
            \item \textbf{核心逻辑:} 现在完成时的精髓在于“\textbf{过去发生的动作对现在造成了影响}”。
            \item \textbf{语境解析:} 作者想表达的是现在的状态——“存钱罐现在还是空的”。那些钱在过去的某个时间点消失了,导致了现在的匮乏。
        \end{itemize}
    \end{enumerate}
\end{multicols}

\newpage