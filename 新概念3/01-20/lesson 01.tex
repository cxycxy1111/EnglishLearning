\section{Lesson 1 A puma at large}

\begin{paracol}{2}

\englishtext{\nw{Pumas} are large, cat-like animals which are found in America.}

\switchcolumn

\nwe{puma}{ˈpuːmə}{n. 美洲狮;彪马;于1948年成立于德国荷索金劳勒(Herzogenaurach)的国际运动品牌;创始人:鲁道夫及达斯勒。;}

\switchcolumn*

\englishtext{When reports came into London Zoo that a wild puma had been \nw{spotted} forty-five miles south of London, they were not taken seriously.}

\switchcolumn

\nwe{spot}{spɑːt}{n. 地点;斑点;丘疹;少量;v. 注意到;让步;adj. 立即支付的;}

\switchcolumn*

\englishtext{However, as the evidence began to \nw{accumulate}, experts from the Zoo felt \nw{obliged} to investigate, for the descriptions given by people who \nw{claimed} to have seen the puma were extraordinarily similar.}

\switchcolumn

\nwe{accumulate}{əˈkjuːmjəleɪt}{v. 积累,积聚;堆积;}
\nwe{oblige}{əˈblaɪdʒ}{vt. 强制,强迫;使负债务;使感激;施惠于;vi. 施恩惠;帮忙,效劳;}
\nwe{claim}{kleɪm}{v. 声称;索取;赢得;需要;夺去(生命);n. 声称,主张;权利;索赔;
}

\switchcolumn*

\englishtext{The hunt for the puma began in a small village where a woman picking \nw{blackberries} saw ‘a large cat’ only five yards away from her. }

\switchcolumn

\nwe{blackberry}{ˈblækberi}{n. 黑莓;
}

\switchcolumn*

\englishtext{It immediately \ns{ran away} when she saw it, and experts confirmed that a puma will not attack a \ns{human being} unless \ns{it is cornered}. }

\switchcolumn

\nse{run away}{}{逃跑,走掉;逃脱;(使)流走[掉];出奔;}
\nse{human being}{}{n. 人类;}
\nse{be cornered}{}{被逼到墙角}

\switchcolumn*

\englishtext{\newsentence{The search proved difficult}, for the puma was often observed at one place in the morning and at another place twenty miles away in the evening. }

\switchcolumn

\switchcolumn*

\englishtext{Wherever it went, it \ns{left behind} it \ns{a trail of} dead deer and small animals like rabbits.}

\switchcolumn

\nse{leave behind}{}{忘带; 留下; 丢弃;使落后;}
\nse{a trail of}{}{一串的;}

\switchcolumn*

\englishtext{\ns{Paw prints} were seen in \ns{a number of} places and puma fur was found \ns{\nw{clinging} to bushes}. }

\switchcolumn

\nwe{cling}{klɪŋ}{v. 抓紧;黏附;缠住;坚持(观念或行为方式) ;}
\nse{paw print}{}{爪印}
\nse{a number of}{}{许多的;一些;}
\nse{clinging to bushes}{}{紧贴灌木}

\switchcolumn*

\englishtext{Several people \ns{complained of} ‘\ns{cat-like noises}’ at night and a businessman on a fishing trip \newsentence{saw the puma up a tree}. }

\switchcolumn

\nse{complain of}{}{诉说(病痛、不适);投诉(令人烦恼的事物)}
\nse{cat-like noises}{}{一个非常高效的构词法,表示“像……一样的”,Child-like (孩子般的), God-like (神一般的)}

\switchcolumn*

\englishtext{The experts were now fully \nw{convinced} that the animal was a puma, but where had it come from? }

\switchcolumn

\nwe{convince}{kənˈvɪns}{v. 使相信;说服;}

\switchcolumn*

\englishtext{As no pumas had been reported missing from any zoo in the country, this one must have been \ns{in the possession of} a \ns{private collector} and \nw{somehow} \ns{managed to} escape. }

\switchcolumn

\nwe{somehow}{ˈsʌmhaʊ}{adv. 以某种方式,用某种方法;不知什么缘故,不知怎的;}
\nse{in the possession of}{}{n. 在…控制/支配下,被…拥有/占有;}
\nse{private collector}{}{私人收藏家}

\nse{manage to}{}{达成,设法;}

\switchcolumn*

\englishtext{The hunt \ns{went on} for several weeks, but the puma was not caught.}

\switchcolumn

\nse{go on}{}{发生;进行;过去;向前走;继续;}

\switchcolumn*

\englishtext{\newsentence{It is \nw{disturbing} to think that a dangerous wild animal is still \ns{at large} in the quiet countryside.}}

\switchcolumn

\nwe{disturb}{dɪˈstɜːrb}{v. 打扰,打断;使焦虑;搅乱;}
\nse{at large}{}{(囚犯)在逃;逍遥法外;一般说来;详细地;}

\switchcolumn*

\end{paracol}

%Pumas are large, cat-like animals which are found in America. When reports came into London Zoo that a wild puma had been spotted forty-five miles south of London, they were not taken seriously. However, as the evidence began to accumulate, experts from the Zoo felt obliged to investigate, for the descriptions given by people who claimed to have seen the puma were extraordinarily similar.
%The hunt for the puma began in a small village where a woman picking blackberries saw 'a large cat' only five yards away from her. It immediately ran away when she saw it, and experts confirmed that a puma will not attack a human being unless it is cornered. The search proved difficult, for the puma was often observed at one place in the morning and at another place twenty miles away in the evening. Wherever it went, it left behind it a trail of dead deer and small animals like rabbits. Paw prints were seen in a number of places and puma fur was found clinging to bushes. Several people complained of "cat-like noises' at night and a businessman on a fishing trip saw the puma up a tree. The experts were now fully convinced that the animal was a puma, but where had it come from? As no pumas had been reported missing from any zoo in the country, this one must have been in the possession of a private collector and somehow managed to escape. The hunt went on for several weeks, but the puma was not caught. It is disturbing to think that a dangerous wild animal is still at large in the quiet countryside.

%美洲狮是一种体形似猫的大动物,产于美洲。当伦敦动物园接到报告说,在伦敦以南45英里处发现一只美洲狮时,这些报告并没有受到重视。可是,随着证据越来越多,动物园的专家们感到有必要进行一番调查,因为凡是声称见到过美洲狮的人们所描述的情况竟是出奇地相似。
%搜寻美洲狮的工作是从一座小村庄开始的。那里的一位妇女在采摘黑莓时的看见"一只大猫",离她仅5码远,她刚看见它,它就立刻逃走了。专家证实,美洲狮非被逼得走投无路,是决不会伤人的。事实上搜寻工作很困难,因为常常是早晨在甲地发现那只美洲狮,晚上却在20英里外的乙地发现它的踪迹。无论它走哪儿,一路上总会留下一串死鹿及死兔子之类的小动物,在许多地方看见爪印,灌木丛中发现了粘在上面的美洲狮毛。有人抱怨说夜里听见"像猫一样的叫声";一位商人去钓鱼,看见那只美洲狮在树上。专家们如今已经完全肯定那只动物就是美洲狮,但它是从哪儿来的呢?由于全国动物园没有一家报告丢了美洲狮,因此那只美洲狮一定是某位私人收藏豢养的,不知怎么设法逃出来了。搜寻工作进行了好几个星期,但始终未能逮住那只美洲狮。想到在宁静的乡村里有一头危险的野兽继续逍遥流窜,真令人担心。

\retellingpoints
\begin{multicols}{1}
    \begin{itemize} 
        \item \textbf{Initial Reports (消息传闻)} 
        \begin{itemize} 
            \item \textbf{Cat-like animals} 
            \item \textbf{Spotted forty-five miles south of London} 
            \item \textbf{Not taken seriously} 
            \item \textbf{Evidence began to accumulate} 
            \item \textbf{Felt obliged to investigate} 
            \item \textbf{Extraordinarily similar descriptions} 
        \end{itemize}

        \item \textbf{The First Encounter (首次目击)}
        \begin{itemize}
            \item \textbf{Picking blackberries}
            \item \textbf{Only five yards away}
            \item \textbf{Immediately ran away}
            \item \textbf{Unless it is cornered}
        \end{itemize}

        \item \textbf{Signs of the Puma (追踪痕迹)}
        \begin{itemize}
            \item \textbf{The search proved difficult}
            \item \textbf{Left behind a trail of dead deer}
            \item \textbf{Paw prints}
            \item \textbf{Puma fur clinging to bushes}
            \item \textbf{Cat-like noises at night}
            \item \textbf{Up a tree}
        \end{itemize}

        \item \textbf{Expert Conclusions (专家定论)}
        \begin{itemize}
            \item \textbf{Fully convinced}
            \item \textbf{Reported missing}
            \item \textbf{In the possession of a private collector}
            \item \textbf{Managed to escape}
        \end{itemize}

        \item \textbf{The Aftermath (结局状态)}
        \begin{itemize}
            \item \textbf{The hunt went on for several weeks}
            \item \textbf{Was not caught}
            \item \textbf{Disturbing to think}
            \item \textbf{At large}
            \item \textbf{Quiet countryside}
        \end{itemize}
    \end{itemize}
\end{multicols}

\grammarpoints

\wsitem{Accumulate}
\begin{multicols}{1} 
    这是一个关于\textbf{“点滴汇”}与\textbf{“量变引起质变”}的动词。它涵盖了从财富积累、知识增长到自然界物质堆积的所有领域。其核心逻辑在于强调\textbf{“持续不断的增加与沉淀”}。
    \begin{enumerate} 
        \item \textbf{核心内涵:过程的“增量叠加”}

        $Accumulate$ 的逻辑源自拉丁语 $accumulare$(堆成堆),可以概括为:
        \begin{itemize}
            \item \textbf{Incremental Growth (增量增长):} 并非一次性爆发,而是通过细小单位的不断加入而变大。
            \item \textbf{Long-term Process (长期性):} 强调时间的跨度,通常需要一段时间的持续运作。
            \item \textbf{Collection and Storage (收集与储存):} 结果通常是形成了一个可观的整体或存量。
        \end{itemize}

        \item \textbf{多维语境下的语义表达}
        \begin{itemize}
            \item \textbf{经济/财富(积累):}
            
            描述资金、资本或债务的逐渐增加。
            
            \es{He managed to \textbf{accumulate} a fortune through savvy investments.} (他通过精明的投资积累了一笔财富。)
            
            \item \textbf{自然/物理(堆积):}
            
            描述灰尘、积雪或化学物质在某处的物理留存。
                
            \es{Dust had \textbf{accumulated} on the old books over the decades.} (几十年来,旧书上堆满了灰尘。)
            
            \item \textbf{抽象/知识(增长):}
            
            描述经验、证据或信息的逐步收集。
            
            \es{As we \textbf{accumulate} more data, the pattern becomes clearer.} (随着我们积累了更多数据,模式变得更加清晰。)
        \end{itemize}

        \item \textbf{程度与性质修饰}
        \begin{itemize}
            \item \textbf{速度修饰 (Speed)}
            \begin{itemize}
                \item \textbf{Rapidly accumulate}:迅速积累。
                \item \textbf{Gradually / Slowly accumulate}:逐渐/缓慢积累。
                \item \textbf{Steadily accumulate}:稳定地积累。
            \end{itemize}
            \item \textbf{结果状态 (Result)}
            \textbf{Inevitably accumulate}:不可避免地堆积(常指垃圾、债务)。
        \end{itemize}

        \item \textbf{近义词辨析}
        \begin{itemize}
            \item \textbf{Amass}:侧重于“大量收集”。通常指非常有意识地汇聚财富、权力或证据,规模比 \textit{Accumulate} 更宏大。
            \item \textbf{Collect}:最通用的词。侧重于“收集”这一动作,不一定强调“增长”或“堆积”。
            \item \textbf{Accrue}:财务术语。侧重于“自然增长”,如利息或假期随时间自动产生。
            \item \textbf{Gather}:侧重于从分散到集中的过程(如聚拢人群)。
        \end{itemize}

        \item \textbf{反义/消散动作}
        \begin{itemize}
            \item \textbf{Dissipate}:驱散、浪费。指累积的东西消失散尽。
            \item \textbf{Disperse}:分散。
            \item \textbf{Squander}:挥霍(常指金钱或机会)。
            \item \textbf{Deplete}:耗尽、枯竭。
        \end{itemize}

        \item \textbf{常见固定搭配}
        \begin{itemize}
            \item \textbf{Accumulated experience}:累积的经验。
            \item \textbf{Accumulate wealth/capital}:积累财富/资本。
            \item \textbf{Accumulate evidence}:收集证据。
            \item \textbf{Accumulate over time}:随时间推移而积累。
        \end{itemize}
    \end{enumerate}
\end{multicols}

\wsitem{Cling vs. Stick}
\begin{multicols}{1}
    这两个词虽然都和“粘、贴、附着”有关,但在物理状态、主被动关系以及语感上有非常显著的区别。

    \begin{enumerate}
        \item \textbf{物理本质:吸引 vs. 粘连}
        \begin{itemize}
            \item \textbf{Cling (依附、紧贴)}
            \begin{itemize}
                \item \textbf{动作特征:} 通常指\textbf{物理上的抱紧、缠绕或因为静电/潮湿而贴在一起}。它不涉及胶水等化学粘合剂。
                \item \textbf{画面感:} 湿衣服贴在身上、孩子紧紧抱住母亲、藤蔓缠绕墙壁。
                \item \textbf{常用介词:} cling \textbf{to}
            \end{itemize}

            \item \textbf{Stick (粘住、粘贴)}
            \begin{itemize}
                \item \textbf{动作特征:} 通常指通过\textbf{胶水、背胶或物质本身的粘性}固定在一个表面上。
                \item \textbf{画面感:} 邮票贴在信封上、胶带粘在桌子上、口香糖粘在鞋底。
                \item \textbf{常用介词:} stick \textbf{to / on}
            \end{itemize}
        \end{itemize}
        \item \textbf{抽象语义:情感依恋 vs. 坚持原则}
        这两个词在比喻用法中也非常活跃:
        \begin{itemize}
            \item \textbf{Cling (带有“不愿放手”的情绪):}
            
            Cling to hope (紧紧抓住希望)
            
            Cling to the past (沉溺于过去,不肯走出来)

            语感: 往往带有一种无助感或强烈的依赖感。


            \item \textbf{Stick (带有“保持一致”的毅力):}
            
            Stick to the plan (坚持计划)

            Stick to the rules (遵守规则)

            语感: 强调忠诚、持久和毫不动摇。
        \end{itemize}
        \item \textbf{常见生活案例对比}
        \begin{itemize}
            \item \textbf{保鲜膜}:Cling film,它是靠静电和物理吸附“包”在碗上的。
            \item \textbf{便利贴}:Sticky note,它的背面有胶水,是“粘”上去的。
            \item \textbf{湿衬衫贴背}:Shirt clings to back,因为汗水的张力,不是因为衬衫上有胶水。
            \item \textbf{海报上墙}:Stick a poster,需要用图钉或胶带。
        \end{itemize}
    \end{enumerate}
\end{multicols}

\wsitem{Discover vs. Find vs. Spot}
\begin{multicols}{1}
    \begin{enumerate}
        \item Discover (发现,揭示)
        \begin{itemize}
            \item 含义: 强调首次发现、揭示了原本未知的事物、规律或地方。
            \item 侧重: 开拓性、科学、地理发现等。
            \item 例子:
            
            \es{They discovered a new planet. (他们发现了一颗新行星)。}
            
            \es{Scientists discovered a cure for the disease. (科学家们发现了一种治愈该疾病的方法)。 }
        \end{itemize}
        \item Find (找到,发现)
        \begin{itemize}
            \item 含义: 词义最广,可指偶然发现、通过努力找到、找到答案,甚至感觉。
            \item 侧重: 结果、找到了目标或信息。
            \item 例子:
            
            \es{I found my lost keys. (我找到了丢失的钥匙)。}
            
            \es{We found out the truth. (我们查明了真相)。}
            
            \es{I find it difficult to learn Chinese. (我觉得学中文很困难)。 }
        \end{itemize}
        \item Spot (注意到,认出)
        \begin{itemize}
            \item 含义: 强调通过仔细看或敏锐的眼光在人群、物体中辨认、认出或注意到某人某物。
            \item 侧重: 视觉的、瞬间的、认出的能力。
            \item 例子:
            
            \es{I spotted him in the crowd. (我在人群中认出了他)。}
            
            \es{Can you spot the difference between these two pictures? (你能看出这两张图片之间的区别吗?)。 }
        \end{itemize}
    \end{enumerate}
\end{multicols}

\wsitem{feel obliged to ...}

\begin{multicols}{1}

这是一个非常实用的表达,通常用于描述一种\textbf{“感到有义务”或“不得不”}的心态。它既可以来自法律、规章的约束,也可以来自道德、人情或社交礼仪的压力。

\begin{enumerate}
    \item 语法结构分析
    
    这是一个典型的 “主语 + 系动词 + 形容词短语” 结构。
    \begin{itemize}
        \item feel: 连系动词(Linking Verb),意为“感到”。
        \item obliged: 形容词,源自动词 oblige(强迫、使负有义务)。
        \item to do something: 不定式短语,作形容词 obliged 的补足语,说明具体在什么方面感到有义务。
    \end{itemize}

    \item 核心含义与语境
    
    这个短语的精妙之处在于它涵盖了两种不同的“压力”:
    \begin{itemize}
        \item 道德或社交上的义务 (Social/Moral Duty)
        
        你觉得如果不做某事,会显得无礼、不真诚或背弃道德。

        Example: \textit{I felt obliged to invite them to the party because they invited me to theirs. (我感到有必要邀请他们参加聚会,因为他们之前也邀请过我。)}

        \item 法律或规定上的义务 (Legal/Formal Duty)
        
        通常指因为规则或合同,你必须这样做。

        Example: \textit{The company felt obliged to offer a refund after the error was discovered. (在发现错误后,公司感到有义务提供退款。)}
    \end{itemize}
    \item 同义词辨析:Obliged vs. Forced
        虽然都表示“不得不”,但语气上有微妙差别:
        \begin{itemize}
            \item Feel obliged to: 侧重于内在的责任感或社交压力。通常你还会保持基本的体面。
            \item Be forced to: 侧重于外在的绝对强迫,通常没有选择余地,且带有违背意愿的感觉。
        \end{itemize}
    
\end{enumerate}
\end{multicols}

\wsitem{... proved difficult}

\begin{multicols}{1}
    这是一个非常地道的表达,属于英语中的 “主语 + 系动词 + 形容词” 结构。它比简单的 "was difficult" 听起来更正式,且带有一种\textbf{“经过尝试或实践后发现”}的意味。

    \begin{enumerate}
        \item 语法结构解析
        \begin{itemize}
            \item Subject (主语): 通常是一个任务、计划、过程或某种尝试(如 The task, Finding a solution)。
            \item Proved (谓语): 这里是 连系动词 (Linking Verb),相当于 turned out to be(结果证明是……)。
            \item Difficult (表语): 形容词,描述主语的性质。
        \end{itemize}
        \item 核心语义:Why not "was"?
        
        虽然两者都可以翻译为“很困难”,但 "proved" 传达了更多的信息:
        \begin{itemize}
            \item Was difficult: 只是陈述一个客观事实(它很难)。
            \item Proved difficult: 强调一个过程。暗示起初可能觉得不难,或者在实际操作过程中遇到了预料之外的阻碍,最终得出了“困难”这个结论。
            \item Example: The climb proved difficult for the amateur hikers. (事实证明,这次攀登对业余登山者来说非常困难。)
        \end{itemize}
        \item 常见搭配
        
        这个结构非常灵活,可以替换不同的形容词:

        \begin{itemize}
            \item Proved successful: 证明是成功的。
            \item Proved fatal: 证明是致命的。
            \item Proved useful: 证明是有用的。
        \end{itemize}

    \end{enumerate}
\end{multicols}

\wsitem{It is + adj. + to do sth.}
\begin{multicols}{1}
    \begin{enumerate}
        \item 语法结构解析:形式主语
        
        这是一个典型的 "It is + adj. + to do sth." 结构。
        \begin{itemize}
            \item It: 形式主语 (Formal Subject)。它本身没有实际意义,只是为了平衡句子结构,避免“头重脚轻”。
            \item is: 系动词。
            \item disturbing: 形容词作表语,意为“令人不安的、扰人的”。
            \item to think that...: 真正的主语 (Real Subject)。这是一个动词不定式引导的短语,其中 that 引导的是一个宾语从句,作为 think 的内容。
        \end{itemize}
        \item 核心词汇与地道表达
        \begin{itemize}
            \item At large
            
            这是本课最核心的短语,有两个主要含义:
            \begin{itemize}
                \item 指罪犯或危险动物: 逍遥法外、尚未捕获(本句含义)。
                
                The murderer is still at large. (凶手仍逍遥法外。)

                \item 指群体: 整个地、大多数。
                
                The public at large. (整个公众。)
            \end{itemize}
            \item Disturbing
            
            源自动词 disturb。
            
            \begin{itemize}
                \item Sound/Noise 回顾: 如果深夜的 noise 干扰了你,这种感觉就是 disturbing。它比 worrying 程度更深,带有“心神不宁”的感觉。
            \end{itemize}
            \item Quiet countryside
            
            这里的 quiet 与前文提到的 cat-like noises 形成了鲜明的对比(Contrast),增加了文字的张力。
        \end{itemize}
    \end{enumerate}
\end{multicols}

\newpage