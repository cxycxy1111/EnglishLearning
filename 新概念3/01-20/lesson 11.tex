\section{Lesson 11 No guilty}

\begin{paracol}{2}


Customs Officers are quite \nw{tolerant} these days, but they can still stop you when you are going through the Green Channel and have nothing to \nw{declare}. 

\switchcolumn

\nwe{guilty}{ˈɡɪlti}{adj. 内疚的;有罪的;}
\nwe{tolerant}{ˈtɑːlərənt}{adj. 宽容的;容忍的,忍受的;(植物、动物或机器)能在困难条件下生存(或操作)的;能耐…的;}
\nwe{declare}{dɪˈkler}{v. 宣称;表明;断言;公布;申报;(在击球员还未全部出局时)宣布结束赛局;}

\switchcolumn*

Even really honest people are often made to feel guilty. 

\switchcolumn

\switchcolumn*

The \nw{hardened} professional \nw{smuggler}, \newsentence{on the other hand}, \ns{is never troubled by} such feelings, \newsentence{even if he has five hundred gold watches hidden in his suitcase. }

\switchcolumn

\nwe{hardened}{'hɑdənd}{adj. 变硬的,坚毅的;v. (使)变硬( harden的过去式和过去分词 );(使)坚固;(使)硬化;(使)变得坚强;}
\nse{be troubled by ...}{}{为某事烦恼}

\switchcolumn*

When I returned form abroad recently, a particularly \nw{officious} young Customs Officer clearly regarded me as a smuggler.

\switchcolumn

\nwe{officious}{əˈfɪʃəs}{adj. 过分殷勤的,爱管闲事的,爱显示权力的;}

\switchcolumn*

'Have you anything to declare?' he asked, \newsentence{\ns{looking me in the eye}}.

\switchcolumn

\nse{Look sb. in the eye}{}{直视某人的眼睛}

\switchcolumn*

'No', I answered \nw{confidently}.

\switchcolumn

\nwe{confidently}{ˈkɑnfədəntlɪ}{adv. 确信地,肯定地;自信地;}

\switchcolumn*

'\newsentence{Would you mind unlocking this suitcase please?}'

\switchcolumn

\switchcolumn*

'\ns{Not at all},' I answered.

\switchcolumn

\nse{not at all}{}{别客气;没什么,哪儿的话;并不;不谢;}

\switchcolumn*

The Officer \ns{went through} the case with great care. 

\switchcolumn

\nse{go through}{}{经历;度过;通读;}

\switchcolumn*

All the thing I had packed so carefully were soon in a \nw{dreadful} mess. 

\switchcolumn

\nwe{dreadful}{ˈdredfl}{adj. 讨厌的,令人不快的;极坏的;可怕的;}

\switchcolumn*

I felt sure I would never be able to close the case again. 

\switchcolumn

\switchcolumn*

Suddenly, I saw the Officer's face \ns{light up}. 

\switchcolumn

\nse{light up}{}{照亮;点烟;开街灯或车灯;(使)变得喜悦;}

\switchcolumn*

He had spotted a tiny bottle at the bottom of my case and he \nw{pounced} on it with delight.

\switchcolumn

\nwe{pounce}{paʊns}{vi. 突袭,猛扑;掌握,抓住;vt. 扑过去抓住;}

\switchcolumn*

'\nw{Perfume}, eh?' he asked \nw{sarcastically}. '

\switchcolumn

\nwe{perfume}{pərˈfjuːm}{n. 香水;香精;芳香;v. 使香气弥漫;为…增加香味;抹香水;}
\nwe{sarcastically}{sɑrˈkæstɪkəlɪ}{adv. 讽刺地,挖苦地;}

\switchcolumn*

\newsentence{You should have declared that. }

\switchcolumn

\switchcolumn*

Perfume \ns{is not \nw{exempt} from} \ns{import \nw{duty}}.'

\switchcolumn

\nwe{exempt}{ɪɡˈzempt}{v. 免除,豁免;adj. 被免除的,被豁免的;}
\nwe{duty}{ˈduːti}{n. 责任;职务,职责;上班;税;}
\nse{be exempt from}{}{被免除于;}
\nse{import duty}{}{进口税;}

\switchcolumn*

'But it isn't perfume,' I said. 'It's hair \nw{gel}.' 

\switchcolumn

\nwe{gel}{dʒɛl}{n. 凝胶,冻胶;定型发胶;vi. 形成胶体;胶化;}

\switchcolumn*

Then I added with a smile, 'It's a strange \nw{mixture} I make myself.'

\switchcolumn

\nwe{mixture}{ˈmɪkstʃər}{n. 混合;混合物;}

\switchcolumn*

As I expected, he did not believe me.

\switchcolumn

\switchcolumn*

'Try it!' I said \nw{encouragingly}.

\switchcolumn

\nwe{encouragingly}{ɪn'kʌrɪdʒɪŋlɪ}{adv. 鼓励地,勉励人地;}

\switchcolumn*

The officer \nw{unscrewed} the \nw{cap} and put the bottle to his \nw{nostrils}. 

\switchcolumn

\nwe{unscrew}{ʌnˈskru}{vt.& vi. 从…旋出螺丝,旋开,松开;}
\nwe{cap}{kæp}{n. 帽子;国家队队员资格;帽,盖;v. 覆盖;胜过;选入国家队;限定(预算);箍牙;}
\nwe{nostril}{ˈnɑstrəl}{n. 鼻孔;鼻孔内壁;}

\switchcolumn*

He was greeted by an unpleasant smell which convinced him that \newsentence{I was telling the truth}. 

\switchcolumn

\switchcolumn*

A few minutes later, I was able to \ns{hurry away} with precious \nw{chalk} marks on my \nw{baggage}.

\switchcolumn
\nwe{chalk}{tʃɑːk}{n. 粉笔,白垩;v. 用粉笔画(或写);}
\nwe{baggage}{ˈbæɡɪdʒ}{n. 行李;(思想)包袱;偏见;}
\nse{hurry away}{}{(使)迅速离开;匆匆而去;}
\switchcolumn*


\end{paracol}

\worddifference
\wsitem{Disturb vs. Trouble vs. Bother}
\begin{multicols}{1}
    这三个词都含有“打扰、麻烦”的意思,但在干扰程度、侧重点以及语境上有着微妙且关键的区别。我们可以通过一个“由外向内”的逻辑来记忆:

    \begin{enumerate}
        \item Disturb (打扰 / 扰乱)
        \begin{itemize}
            \item 核心逻辑:强调打破了某种平衡、平静或专注的状态。
            \item 侧重点:通常指外界的干扰(如声音、动作)。
            \item 常见语境:
            \begin{itemize}
                \item 比如你在写作业或睡觉。
                \item 比如扰乱公共秩序(disturb the peace)。
            \end{itemize}
            \item 例句:Please do not disturb me while I’m recording. (我录音时请不要打扰我。)
            \item 记忆点:酒店门口挂的牌子是 "Do Not Disturb"(请勿打扰)。
        \end{itemize}
        \item Trouble (麻烦 / 使忧虑)
        \begin{itemize}
            \item 核心逻辑:强调增加了负担或引起内心的不安。
            \item 侧重点:既可以是物理上的“费事”,也可以是心理上的“困扰”。
            \item 常见语境:
            \begin{itemize}
                \item 寻求帮助时:比 bother 更正式、更有礼貌。
                \item 内心焦虑时: be troubled by (被……所困扰)。
            \end{itemize}
            \item 例句:I'm sorry to trouble you, but could you move your car? (很抱歉给您添麻烦,但能请您挪下车吗?)
            \item 记忆点:它带有一种“产生问题(problem)”的沉重感。
        \end{itemize}
        \item Bother (烦扰 / 纠缠)
        \begin{itemize}
            \item 核心逻辑:强调反复的、琐碎的、令人厌烦的骚扰。
            \item 侧重点:侧重于一个人的烦躁感和不快。
            \item 常见语境:
            \begin{itemize}
                \item 琐事缠身时:有人一直问你无聊的问题。
                \item 否定句中:表示“不用费心去做某事”。
            \end{itemize}
            \item 例句:Stop bothering me with those silly questions! (别用那些愚蠢的问题来烦我了!)
            \item 记忆点:它常表达一种“讨厌、烦人”的情绪。
        \end{itemize}
    \end{enumerate}
\end{multicols}

\grammarpoints

\wsitem{even if ...}
\begin{multicols}{1}
    "Even if" 在这里是一个非常关键的逻辑工具,它引导了一个让步状语从句(Concession Clause)。

    \begin{enumerate}
        \item 核心逻辑:极端条件的“让步”
        
        Even if 的意思是“即使”、“即便”。

        \begin{itemize}
            \item 逻辑功能:它用于引出一种极端或假设的情况,并表示即便在这种情况下,主句的结果依然成立,不会发生改变。
            \item 句中体现:
            \begin{itemize}
                \item 极端情况:行李箱里藏了 500 块金表(这显然是严重的走私行为)。
                \item 结果不变:职业走私犯依然“从未被那种(愧疚)感所困扰”。
            \end{itemize}
            \item 对比力量:它通过这种极端的对比,突出了“职业走私犯”心理素质的强大(或道德的沦丧),与前文提到的“诚实的人”形成鲜明反差。
        \end{itemize}
        \item Even if vs. Even though
        
        这两个词长得很像,但在语感和逻辑上有一线之差:

        \begin{itemize}
            \item Even if:假设性/虚拟性。即使某事发生(可能还没发生)。
            
            \es{I will go even if it rains. (即使下雨我也去。—— 还没下)}
            \item Even though:事实性。虽然某事已经发生或确实存在。
            
            \es{I went even though it was raining. (虽然当时在下雨,但我还是去了。)}
        \end{itemize}

        在本句中:作者用 even if 带有一种“假设”的意味——“哪怕他真的藏了 500 块表(这种极端假设下),他也不会心慌”。

        \item 语法结构与位置
        \begin{itemize}
            \item 位置:可以放在句中(如本句),也可以放在句首。
            
            \es{Even if he has 500 watches, he is never troubled. (放在句首时,语气更重。)}
            \item 时态:在引导假设性让步时,通常遵循“主将从现”或类似的一般时态规则。
        \end{itemize}
    \end{enumerate}
\end{multicols}

\wsitem{looking me in the eye}
\begin{multicols}{1}
    这是本句的灵魂,它运用了 现在分词短语作伴随状语,描述了海关官员说话时的神态。
    \begin{enumerate}
        \item 语法功能:looking 动作与主句动作 asked 同时发生,主语都是 he。
        \item 固定搭配:Look someone in the eye(直视某人的眼睛)。
        \begin{itemize}
            \item 语感解析:这表达了一种极具压迫感的审视,旨在通过心理压力让“不诚实的人”露出马脚。
            \item 对比:Look at me 只是看着我;Look me in the eye 则是带着目的、严肃且专注地对视。
        \end{itemize}
    \end{enumerate}
\end{multicols}

\wsitem{Would you mind ...}
\begin{multicols}{1}
    这是一个在英语中极具迷惑性的委婉请求句式。它的字面意思和逻辑反馈往往会让初学者感到困扰。
    \begin{enumerate}
        \item 核心逻辑:你“介意”吗?
        
        理解这个句式的关键在于 mind 这个词的本义:介意、反对。

        \begin{itemize}
            \item 如果你愿意帮忙,你要说:"No, not at all."(不,我不介意 $\rightarrow$ 好的)。
            \item 如果你不想帮忙,你要说:"Yes, I do mind."(是的,我介意 $\rightarrow$ 不行)。
            \item 切记:千万不要直接用 "Yes" 来表达“好的”,那意味着“我介意/我反对”。
        \end{itemize}
        \item 两种常见的语法结构
        
        根据你想请求的对象不同,后面接的内容也不同:

        \begin{itemize}
            \item Would you mind + doing...(请你做某事)
            
            请求对方亲自去做某事。

            \begin{itemize}
                \item 例句:Would you mind opening your suitcase? (你介意打开你的手提箱吗?—— 即:请打开手提箱。)
                \item 语感:这比 "Open your suitcase" 要礼貌得多,海关官员在维持礼貌但坚定的立场时常这么说。
            \end{itemize}

            \item Would you mind + if I did...(你介意我做某事吗?)
            
            请求对方允许“我”去做某事。注意:这里通常用虚拟语气(过去时)。

            \begin{itemize}
                \item 例句:Would you mind if I looked in your bag? (你介意我翻看一下你的包吗?)
                \item 用法提示:虽然口语中也有人用 if I do,但 if I did 是最标准的礼貌表达。
            \end{itemize}
        \end{itemize}

        \item 常见的回答方式(避坑指南)
        
        由于 "No" 表示同意,"Yes" 表示拒绝,为了避免尴尬,人们常使用更清晰的词:

        \begin{itemize}
            \item 同意/不介意:Not at all. / Certainly not. / Go ahead.
            \item 拒绝/介意:'m sorry, but... / I'd rather you didn't.
        \end{itemize}
    \end{enumerate}
\end{multicols}

\wsitem{sb. should have done sth.}
\begin{multicols}{1}
    这是一个非常经典且实用的虚拟语气结构。它不仅是一个语法点,更是一种表达情感(后悔、责备、遗憾)的重要手段。

    \begin{enumerate}
        \item 核心定义与逻辑反差
        
        should have done 的核心在于**“理想与现实的背离”**:

        \begin{itemize}
            \item 字面意思:某人本应该做某事。
            \item 隐含事实:但实际上当时没做。
            \item 逻辑关系:这是一种对过去行为的评判或反思。
        \end{itemize}
        \item 常见使用语境
        
        这个句式通常带有三种不同的情绪色彩:

        \begin{itemize}
            \item 责备他人 (Criticism)
            
            当你觉得别人做错了事或漏掉了某项任务时。

            \es{You should have declared that gold watch at the Customs. (你本该在海关申报那块金表的。—— 暗示你没报,现在麻烦了。)}

            \item 自我遗憾 (Regret)
            
            当你反思自己的过去,觉得本可以做得更好时。

            \es{I should have studied harder for the exam. (我本该为考试更努力学习的。—— 暗示我没努力,结果考砸了。)}

            \item 逻辑推测 (Expectation)
            
            虽然不常见,但在某些语境下表示“按理说应该已经发生了”。

            \es{The bus should have arrived ten minutes ago. (公共汽车按理说十分钟前就该到了。)}
        \end{itemize}
        \item 语法对比:肯否之间的逻辑
        \begin{itemize}
            \item Should have done:本该做,事实上没做
            \item Shouldn't have done,本不该做,事实上做了。
            \es{例子:Jane shouldn't have put the wet money in the oven. (简本不该把湿钱放进烤箱的。—— 事实:她放了,钱糊了。)}
        \end{itemize}
        \item 易混淆句式辨析
        
        很多人会将 should have done 与其他情态动词混淆,它们的“确定感”和“情感”完全不同:

        \begin{itemize}
            \item Must have done: 肯定做了(对过去的笃定推测)。
            \item Could have done: 本可以做(但没做,强调可能性)。
            \item Should have done: 本应该做(强调义务或责任)。
        \end{itemize}
    \end{enumerate}
\end{multicols}

\wsitem{I was telling the truth}
\begin{multicols}{1}
    在这一句中,使用 "was telling"(过去进行时) 而不是简单的 "told"(一般过去时),是为了捕捉动作的持续性和即时背景。

    \begin{enumerate}
        \item 强调“正在进行”的背景 (The ongoing background)
        
        在故事发生的那个时刻,作者(我)正在和海关官员对话。

        \begin{itemize}
            \item was telling: 表达的是“我正在陈述事实”这个持续的过程。
            \item greeted / convinced: 这两个动作是突然发生的点(闻到味道、瞬间相信)。
        \end{itemize}

        逻辑关系: 就在我“喋喋不休地解释”(长动作)的过程中,那股难闻的味道突然冒了出来(短动作),并瞬间让他信服了。这种时态对比增强了故事的画面感。

        \item 对应“当下”的真实感 (Immediacy)
        
        如果用 told(一般过去时),听起来更像是一个已经结束的事实汇报;而用 was telling the truth,侧重于表达:“我刚才说的那些话,在此刻被证实是真的了。”

        \item 固定搭配与语感
        
        在英语中,描述某人是否诚实时,常用进行时来表示一种状态的延续:

        \begin{itemize}
            \item I am telling the truth. (我现在说的是实话。)
            \item He was telling the truth. (他当时说的是实话。)
        \end{itemize}

        如果改为 I told the truth,通常指代一个具体的、完成的动作(比如:我在法庭上说了真话)。而在这种边解释边检查的场景下,进行时更能体现出那种**“正在解释说明”**的氛围。

        \item 逻辑串联:缝合本课语境
        
        我们可以把这句话放入海关查验的完整逻辑中:
        
        \begin{itemize}
            \item Action: The officer pounced on a tiny bottle in my luggage.
            \item Dialogue: I tried to explain what it was (I was telling the truth).
            \item Turning Point: Suddenly, he was greeted by an unpleasant smell.
            \item Result: The smell convinced him. He realized he shouldn't have doubted me!
        \end{itemize}
    \end{enumerate}
\end{multicols}

\wsitem{on the other hand ...}
\begin{multicols}{1}
    这是一个非常经典且实用的连接短语(Transitional Phrase),主要用于对比两种不同的观点、情况或事物的两个方面。它就像一个天平,平衡地展示事物的正反两面。

    \begin{enumerate}
        \item 核心逻辑:平衡与对比
        \item 常见语法结构与标点
        
        这个短语通常作为插入语或连接状语。

        \begin{itemize}
            \item 句首用法:
            
            \es{On one hand, I want to travel. On the other hand, I need to save money.}

            \item 句中用法(更地道,常作为插入语):
            
            \es{The job pays well. On the other hand, the hours are very long.}

            \es{The professional smuggler, on the other hand, is never troubled by such feelings.}
        \end{itemize}

        \item 与 On the contrary 的区别
        
        这是很多学习者最容易混淆的地方:

        \begin{itemize}
            \item On the other hand,逻辑功能是对比/权衡:两个方面都可能成立,只是重点不同。语感是“虽然……但是另一方面……”。
            \item On the contrary,逻辑功能是反驳/否定:前一个观点是错的,后一个才是对的。语感是“恰恰相反”、“完全不是这样”。
        \end{itemize}

    \end{enumerate}
\end{multicols}

\newpage