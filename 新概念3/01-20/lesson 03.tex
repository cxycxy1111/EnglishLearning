\section{Lesson 3 An unknown goddess}

\begin{paracol}{2}

\englishtext{\ns{Some time ago}, an interesting discovery was made by archaeologists on \ns{the Aegean island of Kea}.}

\switchcolumn

\nwe{archaeologist}{ˌɑːrkiˈɑːlədʒɪst}{n. 考古学家;}
\nse{some time ago}{}{不久前;}
\nse{the Aegean island of Kea}{}{爱琴海的基亚岛}

\switchcolumn*

\englishtext{An American team explored a temple which stands in an \ns{ancient city} on the \nw{promontory} of Ayia Irini.}

\switchcolumn

\nwe{promontory}{ˈprɑməntɔri}{n. 岬,隆起,海角;}
\nse{ancient city}{}{古城;}

\switchcolumn*

\englishtext{The city \ns{at one time} must have been \nw{prosperous}, for it enjoyed \ns{a high level of civilization}.}

\switchcolumn

\nwe{prosperous}{ˈprɑːspərəs}{adj. 繁荣的,兴旺的;富裕的;幸福的,运气好的;良好的;}
\nse{at one time}{}{一度, 从前;}
\nse{a high level of civilization}{}{高度文明}


\switchcolumn*

\englishtext{Houses -- often three \nw{storeys} high -- \ns{were built of stone}.}

\switchcolumn

\nwe{storey}{ˈstɔri, ˈstori}{n. <英>楼层;叠架的一层;“story”的变体;}
\nse{be built of stone}{}{石砌的}

\switchcolumn*

\englishtext{They had large rooms with beautifully decorated walls.}

\switchcolumn

\switchcolumn*

\englishtext{The city was even equipped with a \ns{\nw{drainage} system}, for \ns{a great many} \ns{\nw{clay} pipes} were found beneath the \ns{narrow streets}.}

\switchcolumn

\nwe{drainage}{ˈdreɪnɪdʒ}{n. 排水,放水;排水系统;}
\nwe{clay}{kleɪ}{n. 黏土,陶土;}
\nse{drainage system}{}{排水系统}
\nse{a great many}{}{很多, 许多;指不胜屈;浩;}
\nse{clay pipe}{}{瓦管,陶(土)管;}
\nse{narrow street}{}{狭窄的街道}

\switchcolumn*

\englishtext{The temple which the archaeologists explored \ns{was used as a place of} \nw{worship} from the fifteenth century B.C. until \ns{Roman times}.}

\switchcolumn
\nwe{worship}{}{v. 敬奉(神);做礼拜;爱慕;n. 崇拜;礼拜;爱慕;阁下;}
\nse{be used as a place of}{}{被用作......的地方}
\nse{Roman times}{}{罗马时代}

\switchcolumn*

\englishtext{In the most \nw{sacred} room of the temple, \ns{clay \nw{fragments}} of fifteen statues were found.}

\switchcolumn

\nwe{sacred}{ˈseɪkrɪd}{adj. 宗教的;神圣的;受崇敬的,值得崇敬的;有宗教性质的;}
\nwe{fragment}{ˈfræɡmənt , fræɡˈment}{n. 碎片;片段,未完成的部分;(将文件内容)分段;vt. (使)碎裂,破裂,分裂;vi. 破碎,碎裂;}
\nse{clay fragment}{}{陶土碎片}

\switchcolumn*

\ns{Each of these} represented a goddess and had, \ns{at one time}, been painted.

\switchcolumn

\nse{each of these}{}{这些中的每一个}
\nse{at one time}{}{一度, 从前;}



\switchcolumn*

\englishtext{The body of one statue was found among remains \ns{dating from} the fifteenth century B.C.}

\switchcolumn

\nse{dating from}{}{追溯到,从什么时候开始;}

\switchcolumn*

\englishtext{Its missing head \ns{happened to be} among remains of the fifth century B.C.}

\switchcolumn

\nse{happen to be}{}{碰巧是,恰巧是;}

\switchcolumn*

\newsentence{This head must have been found in \ns{Classical times} and carefully preserved}.

\switchcolumn

\nse{Classical times}{}{古典时期}

\switchcolumn*

\englishtext{It was very old and precious \ns{even then}.}

\switchcolumn

\nse{even then}{}{尽管那样;}

\switchcolumn*

\englishtext{When the archaeologists \nw{reconstructed} the fragments, they \ns{were amazed to} find that the goddess \nw{turned out to be} a very modern-looking woman.}

\switchcolumn

\nwe{reconstruct}{ˌriːkənˈstrʌkt}{vt. 重建;重现;改造;复兴;}
\nse{be amazed to}{}{惊讶地}
\nse{turn out to be}{}{结果是,原来是,证明是;}

\switchcolumn*

\englishtext{She \ns{stood three feet high} and \ns{her hands rested on her hips}.}

\switchcolumn

\nse{stand ... foot/feet high}{}{...英尺高}
\nse{one's hands rest on one's hips}{}{双手叉腰}

\switchcolumn*

\englishtext{She was wearing a \ns{full-length skirt} which swept the ground.}

\switchcolumn

\nse{full-length skirt}{}{长裙}

\switchcolumn*

\englishtext{Despite her great age, she was very graceful indeed, but, \ns{so far}, the archaeologists have been unable to discover her identity.}

\switchcolumn
\nse{so far}{}{到目前为止,迄今为止;}

\end{paracol}


\grammarpoints
\wsitem{preserve save keep}

\begin{multicols}{1}
    \begin{enumerate}
    \item preserve (保护/维护/保鲜)
    \begin{enumerate}
        \item 侧重于保持原状,防止事物被破坏、腐烂或消失。
        \item 语境: 自然环境、古迹、食物、传统、现状。
        \item 用法:
        \begin{itemize}
            \item 保护环境/古迹: Preserve the environment / historical sites.
            \item 保鲜/腌制食物: Preserve fruit with sugar.
            \item 维持现状: Preserve the status quo.
        \end{itemize}
        \item 核心意图: 为了不让它变质或消失而采取行动。
    \end{enumerate}

    \item save (节省/挽救/储存)
    \begin{enumerate}
        \item 这是一个多义词,核心在于避免损失或浪费,或保留以备后用。
        \item 语境: 金钱、时间、生命、电子文件、救援。
        \item 用法:
        \begin{enumerate}
            \item 挽救生命: Save a life.
            \item 省钱/攒钱: Save money for a car.
            \item 电脑存档: Save a document. (防止进度丢失)
        \end{enumerate}
        \item 核心意图: 从危险、浪费或消耗中“救出”或“积攒”。
    \end{enumerate}
    \item keep (继续拥有/保持状态)
    \begin{enumerate}
        \item 侧重于持有权或维持某种动作/状态,是最基础、最常用的词。
        \item 语境: 拥有物品、保守秘密、保持习惯、保持联系。
        \item 用法:
        \begin{enumerate}
            \item 保留物品: You can keep the change. (你可以留着零钱).
            \item 保守秘密: Keep a secret.
            \item 保持联系: Keep in touch
            \item 存放: I keep my keys in the drawer
        \end{enumerate}
        \item 核心意图: 强调“留在身边”或“状态的延续”
    \end{enumerate}
\end{enumerate}
\end{multicols}

\wsitem{happen to be ...}

\begin{multicols}{1}
    用来描述一种“碰巧、偶然”的情况,增加叙事的戏剧性。

    \begin{enumerate}
        \item 语法结构解析
        
        Happen to be + [形容词/名词/地点]

        \begin{itemize}
            \item Happen: 动词,这里不是“发生”,而是“碰巧”。
            \item to be: 不定式,连接状态。
        \end{itemize}

        用法变体:

        \begin{itemize}
            \item Happen to do sth: 碰巧做了某事。
            
            $\rightarrow$ \textit{I happened to see him on the street. (我碰巧在街上看到他。)}
            \item It so happened that...: 碰巧……(句式更正式)。
            
            $\rightarrow$ \textit{It so happened that I had my camera with me. (碰巧我带了相机。)}
        \end{itemize}

        \item 课文语境
        
        这个词组描述了考古学家在爱琴海的凯阿岛(Kea)上的发现:

        $\rightarrow$ \textit{One of the most important significant discoveries... happened to be the head of a goddess.} (其中一项最重要的发现……碰巧是一个女神的头部。)

        这种表达方式比直接用 "was" 要生动得多,因为它强调了发现过程中的偶然性和惊喜感。
    \end{enumerate}
\end{multicols}

\wsitem{be amazed to ...}
\begin{multicols}{1}
    这个短语是描述“惊喜”或“诧异”的高级表达。当考古学家们在古庙里发现那尊精美绝伦的女神像时,他们的心情正是 amazed。

    \begin{enumerate}
        \item 语法结构解析
        
        Be amazed + to do something / that从句

        \begin{itemize}
            \item amazed: 形容词(由动词 amaze 演变),意为“大为惊奇、惊愕”。
            \item to do: 不定式引导原因,解释为什么感到惊讶。
            \item 程度: 它比 surprised(惊讶)程度要深得多,通常带有“难以置信”或“欣赏”的意味。
        \end{itemize}

        \item 结合课文语境
        
        在第三课中,考古学家在凯阿岛(Kea)的发现令人震撼:

        Archaeologists were amazed to find that the goddess turned out to be a very modern-looking woman. (考古学家们惊讶地发现,这位女神竟然是一个看起来非常现代的女性。)
        
        这里用 be amazed to find 准确地捕捉到了考古人员在面对数千年前精湛艺术时的那种震惊。

        \item 同类表达辨析 (Feeling vs. Causing)
        
        英语中这种表示情感的形容词通常成对出现:
        \begin{itemize}
            \item Amazed: 感到惊讶的(修饰人)。
            \item Amazing: 令人惊讶的(修饰物/事)。
            
            $\rightarrow$ Example: I was amazed by the amazing statue. (我被那尊令人惊叹的雕像震撼了。)
        \end{itemize}  
    \end{enumerate}
\end{multicols}

\wsitem{turned out to be ...}

\begin{multicols}{1}

    用来描述一个\textbf{“出乎意料的结果”},非常适合用在有反转情节的叙事中。

    \begin{enumerate}
        \item 语法结构解析
        
        Turn out to be + [名词/形容词]

        \begin{itemize}
            \item Turn out: 这是一个动词短语,意为“结果是、证明是”。
            \item to be: 后面常接描述性质的词。
            \item 含义: 强调经过一段时间、一系列动作或调查后,真相终于大白,且这个真相往往与最初的预想不同。
            \item 等价句式:It turned out that... (后面接一个完整的从句)
            $\rightarrow$ Example: It turned out that the statue was very old.
        \end{itemize}

        \item 结合课文语境
        
        在第三课中,考古学家的发掘过程充满惊喜:
        
        The goddess turned out to be a very modern-looking woman. (那位女神最终证明是一个长相非常现代的女性。)
        
        考古学家起初可能以为会发现一个传统的、古板的雕像,但随着碎片的拼凑,结果却让他们大吃一惊。

        \item 常见用法对比
        \begin{itemize}
            \item Was/Were,仅仅陈述一个事实。
            \item Proved to be,强调经过验证或测试。
            \item Turned out to be,强调结局或意想不到的发现。
        \end{itemize}
    \end{enumerate}
\end{multicols}

\wsitem{must have been done}
\begin{multicols}{1}
    它不仅是一个事实陈述,还包含了一个非常重要的推测性语气和被动结构

    \begin{enumerate}
        \item 语法深度解析
        
        这个结构是 “情态动词 + 过去完成时被动语态”:

        \begin{itemize}
            \item Must have been done:
            \begin{itemize}
                \item Must have done: 表示对过去发生的动作进行肯定的推测(“准是……”、“一定已经……”)。
                \item Been found: 被动语态,因为“头”(head)是被发现的。
            \end{itemize}
            \item 句意: “这个头像一定是(在某个时候)被发现的。”
        \end{itemize}

        \item 知识点对比:各种推测语气
        \begin{itemize}
            \item Must have been found: 一定是被发现的, 90\% - 100\% (非常确定)
            \item Might/Could have been found: 可能被发现了, 30\% - 50\% (不太确定)
            \item Can't have been found, 不可能被发现, 0\% (否定推测)
        \end{itemize}
        \item 课文语境与逻辑
        
        在课文中,考古工作是分阶段进行的。当身体部分被拼凑起来后,考古学家们意识到,为了完整性,这个头像在过去某个时间点一定已经被找到了。
        
        随后他们发现,这个头像竟然被放在一个箱子里存放了好几年!
    \end{enumerate}
\end{multicols}

\newpage