\section{Lesson 5 The fact}

\begin{paracol}{2}

Editors of newspapers and magazines often \ns{go to extremes} to provide their reader with unimportant facts and \ns{statistics}. 

\switchcolumn

\nwe{statistic}{stəˈtɪstɪk}{n. 统计数字;统计学;}
\nse{go to extremes}{}{走极端;偏激;过火;}

\switchcolumn*

Last year a journalist had been instructed by a well-known magazine to write an article on the president's palace in a new African republic. 

\switchcolumn

\switchcolumn*

When the article arrived, the editor read the first sentence and then refuse to publish it. 

\switchcolumn

\switchcolumn*

The article began: 'Hundreds of steps lead to the high wall which surrounds the president's palace'. 

\switchcolumn

\switchcolumn*

The editor \ns{at once} sent the journalist a fax instructing him \ns{find out} \ns{the exact number of} steps and \ns{the height of} the wall.

\switchcolumn

\nse{at once}{}{adv. 立即;一起;同时;立刻;}
\nse{find out}{}{发现;使发作;使受惩罚;通过探询[访问]获悉(某人)不在;}
\nse{the exact number of ...}{}{确切的...}
\nse{the height of ...}{}{...的高度}

\switchcolumn*

The journalist immediately \ns{set out} to obtain these important facts, \newsentence{but he took a long time to send them}.

\switchcolumn

\nse{set out}{}{出发;启程;(怀着目标)开始工作;}
\switchcolumn*

Meanwhile, the editor was \ns{getting impatient}, for the magazine would soon \ns{go to press}. 

\switchcolumn

\nse{get ...}{}{变得}
\nse{go to press}{}{付梓,付印;}


\switchcolumn*

He sent the journalist two more faxes, but \ns{received no reply}. 

\switchcolumn

\nse{receive no reply}{}{未收到答复}

\switchcolumn*

He sent yet another fax informing the journalist that if he did not reply soon he would be fired. 

\switchcolumn

\switchcolumn*

When the journalist again failed to reply, the editor \nw{reluctantly} published the article as it had originally been written. 

\switchcolumn

\nwe{reluctantly}{rɪˈlʌktəntlɪ}{adv. 不情愿地,勉强地;}

\switchcolumn*

A week later, the editor at last received a fax from the journalist. 

\switchcolumn

\switchcolumn*

\newsentence{Not only had the poor man been arrested, but he had been sent to prison as well}. 

\switchcolumn

\nse{not only ..., but ... as well}{}{不仅……而且……}

\switchcolumn*

However, he had at last been allowed to send a fax in which he informed the editor that the he had been arrested while counting the 1,084 steps leading to the fifteen-foot wall which surrounded the president's palace.

\switchcolumn

\switchcolumn*

\end{paracol}

\worddifference
\wsitem{obtain vs. get}
\begin{multicols}{1}
    虽然两者都表示“得到”,但在语境上有明显区别:
    \begin{itemize}
        \item Get: 最通用的词,口语化。
        \item Obtain: 更加正式,通常指经过努力、搜索或正式程序才得到的。
        \item 搭配: Obtain permission (获得许可), Obtain information (获取信息), Obtain the facts (查明事实)。
    \end{itemize}
\end{multicols}

\wsitem{tell, instruct, order, command, direct}
\begin{multicols}{1}
    \begin{enumerate}
    \item \textbf{Tell: 最普通的“吩咐”}
    \begin{itemize}
        \item \textbf{特点:} 语气最弱,最不正式,日常生活中最通用的表达。
        \item \textbf{场景:} 朋友、家人或同事之间简单的要求。
        
        \textbf{例句:} \es{The editor \textbf{told} the reporter to call him later.} (编辑告诉记者晚点给他打电话。)
    \end{itemize}

    \item \textbf{Instruct: 正式的“指示/指导” (Lesson 5 核心词)}
    \begin{itemize}
        \item \textbf{特点:} 带有权威性,且通常包含“如何操作”的详细指导或具体任务要求。
        \item \textbf{场景:} 上级对下级下达工作任务、律师对当事人、教官对学员。
        
        \textbf{例句:} \es{The editor \textbf{instructed} the reporter to obtain the factual background.} (编辑指示记者去获取事实背景。)
    \end{itemize}

    \item \textbf{Order: 强制性的“命令”}
    \begin{itemize}
        \item \textbf{特点:} 语气强硬,强调绝对服从,不带有指导色彩,不服从通常意味着违抗。
        \item \textbf{场景:} 警察、军队或极度紧急、带有强制权力的场合。
        
        \textbf{例句:} \es{The policeman \textbf{ordered} the journalist to stop taking photos.} (警察命令记者停止拍照。)
    \end{itemize}

    \item \textbf{Command: 威严的“统帅/指挥”}
    \begin{itemize}
        \item \textbf{特点:} 比 Order 更庄严,通常指作为领袖或统帅发布的一整套正式指令。
        \item \textbf{场景:} 军事行动、大型仪式或指挥系统中。
        
        \textbf{例句:} \es{The general \textbf{commanded} his men to attack.} (将军统领部下发起进攻。)
    \end{itemize}

    \item \textbf{Direct: 正式的“指引/指派”}
    \begin{itemize}
        \item \textbf{特点:} 侧重于在管理层面“指明方向”或“指派某人负责某事”,非常商务化。
        \item \textbf{场景:} 董事会决策、导演指挥、官方机构下达导向性任务。
        
        \textbf{例句:} \es{The chairman \textbf{directed} the committee to investigate the matter.} (主席指派委员会调查此事。)
    \end{itemize}
\end{enumerate}
\end{multicols}

\wsitem{tell, apprise, inform, notify, brief}


\begin{multicols}{1}
\begin{enumerate}
    \item \textbf{Tell: 最通用的“告诉”}
    \begin{itemize}
        \item \textbf{特点:} 非正式,频率最高。强调“说话”这个动作本身。
        \item \textbf{场景:} 朋友间聊天、讲故事、日常口语交流。
        \item \textbf{例句:} \es{Can you \textbf{tell} me the time?} (你能告诉我几点了吗?)
    \end{itemize}

    \item \textbf{Inform: 正式的“通报” (Lesson 5 核心词)}
    \begin{itemize}
        \item \textbf{特点:} 正式、中性词。强调“传递事实、数据或知识”,常用于商务或正式报道。
        \item \textbf{场景:} 报刊通知、官方函件、工作汇报。
        \item \textbf{例句:} \es{The reporter \textbf{informed} the editor of his arrest.} (记者通知了编辑他被捕的消息。)
    \end{itemize}

    \item \textbf{Notify: 程序化的“通知”}
    \begin{itemize}
        \item \textbf{特点:} 极其正式。通常指“履行法律或行政上的义务”,往往伴随着正式的书面通知。
        \item \textbf{场景:} 银行扣款通知、警察告知家属、法院下达传票。
        \item \textbf{例句:} \es{The tenants were \textbf{notified} that the rent would be increased.} (租户们接到通知,房租即将上涨。)
    \end{itemize}

    \item \textbf{Brief: 概要的“简报”}
    \begin{itemize}
        \item \textbf{特点:} 专业、高效。指在执行任务前,提供必要的背景信息和明确指示。
        \item \textbf{场景:} 军队下达任务、会议前向领导汇报要点、项目启动会。
        \item \textbf{例句:} \es{The captain \textbf{briefed} the soldiers on the mission.} (上尉向士兵们简要交代了任务内容。)
    \end{itemize}

    \item \textbf{Apprise: 高端的“使知晓”}
    \begin{itemize}
        \item \textbf{特点:} 文雅、罕见(属于文学/大词)。侧重于“让某人对整个局势、进展有所了解”。
        \item \textbf{场景:} 外交辞令、高级商务报告、文学作品叙述。
        \item \textbf{例句:} \es{Please keep me \textbf{apprised} of any new developments.} (如有任何新进展,请务必让我知晓。)
    \end{itemize}
\end{enumerate}
\end{multicols}


\grammarpoints

\wsitem{It takes time to do sth.}
\begin{multicols}{1}
    这是一个非常经典的时间花费表达的变体
    \begin{itemize}
        \item Take a long time: 花费很长时间。
        \item To send them: 这里的 them 指代的是前文提到的 the facts(事实真相)。
        \item 逻辑主语: 这里的 he 是 take 的主语,构成了 Somebody + take + time + to do sth. 的结构。
    \end{itemize}
    等价句式转换:
    \begin{itemize}
        \item It takes sb. time to do: \es{It took him a long time to send them.}
        \item Spend 结构: \es{He spent a long time sending them.} (注意 spend 后面接 doing)
    \end{itemize}
\end{multicols}

\wsitem{not only..., but... as well}

\begin{multicols}{1}

是一个非常实用的关联句型,意思等同于中文的 “不仅……而且……” 或 “不但……也……”。

它主要用于强调两个并列的信息,且通常第二个信息比第一个更重要、更令人惊讶或更进一步。


\begin{enumerate}
    \item 基本用法 (结构一致性):在使用这个句型时,前后连接的词性必须保持一致(例如:名词对名词,形容词对形容词)。
    \begin{itemize}
        \item 形容词: She is not only beautiful but intelligent as well. (她不仅漂亮,而且也很聪明。)
        \item 名词: He is not only a teacher but a writer as well. (他不仅是老师,也是一名作家。)
        \item 动词: They not only built the house but designed it as well. (他们不仅盖了这栋房子,还设计了它。)
    \end{itemize}
    \item 倒装用法 (更高级、正式):如果你想让表达更有气势或更正式,可以将 Not only 放在句首。此时,前半个分句需要使用“倒装句”(像变疑问句一样把助动词提前),而后半句保持正常语序。
    \begin{itemize}
        \item 普通句: He not only speaks English, but he speaks French as well.
        \item 倒装句: Not only does he speak English, but he speaks French as well. (他不仅会说英语,还会说法语。)
    \end{itemize}
    \item 三种常见的变体:这三种表达方式的意思几乎完全一样,只是结尾词不同:
    \begin{itemize}
        \item ...not only... but... as well. (非常地道)
        \item ...not only... but also... (最经典、教科书式的用法)
        \item ...not only... but... too. (口语中也很常用)
    \end{itemize}
    \item 小贴士
    \begin{itemize}
        \item 强调重心: 重点通常在 but 之后的内容。
        \item 标点符号: 如果连接的是两个完整的句子(如上面的倒装句例子),通常在 but 前面加逗号。
    \end{itemize}
\end{enumerate}
\end{multicols}

\newpage