\section{Lesson 9 Flying cats}

\begin{paracol}{2}

Cats never \ns{fail to} \nw{fascinate} human beings. 

\switchcolumn

\nwe{fascinate}{ˈfæsɪneɪt}{v. 深深吸引;迷住;使着迷;}
\nse{fail to do sth.}{}{}

\switchcolumn*

They can \newsentence{be friendly and \nw{affectionate} towards} humans, but they \ns{lead \nw{mysterious} lives} \ns{of their own} as well. 

\switchcolumn

\nwe{affectionate}{əˈfekʃənət}{adj. 表示关爱的;充满深情的;}
\nwe{mysterious}{mɪˈstɪriəs}{adj. 神秘的,诡秘的;难以解释的;故弄玄虚的;}
\nse{lead ... life}{}{过着...的生活}
\nse{of one's own}{}{...自己的}

\switchcolumn*

they never become \nw{submissive} like dogs and horses. \ns{As a result}, humans have learned to respect \nw{feline} independence. 

\switchcolumn

\nwe{submissive}{səbˈmɪsɪv}{adj. 顺从的,唯命是从的;柔顺;帖;贴;}
\nwe{feline}{ˈfiˌlaɪn}{adj. 猫(科)的;n. 猫科动物;}
\nse{as a result}{}{所以;结果(是);}

\switchcolumn*

Most cats \newsentence{remain \nw{suspicious} of humans} all their lives. 

\switchcolumn

\nwe{suspicious}{səˈspɪʃəs}{adj. 可疑的;猜疑的,怀疑的;多疑的,不信任的;多心;}

\switchcolumn*

\newsentence{\ns{One of the things that} fascinates us most about cats is the \ns{popular belief} that they have nine lives. }

\switchcolumn

\nse{one of the things that}{}{...事情之一}
\nse{popular belief}{}{民间信仰;}

\switchcolumn*

Apparently, there is \ns{a good deal of} truth in this idea. 

\switchcolumn

\nse{a good deal of}{e ɡʊd dil ʌv}{adj. 很多的;好些;}

\switchcolumn*

A cat’s ability to survive falls \ns{is based on} fact.

\switchcolumn

\nse{be based on}{bi beɪst ɑːn}{...以...为根据;}

\switchcolumn*

Recently the New York Animal medical Centre made a study of 132 cats over \ns{\ns{a period of} five months}. 

\switchcolumn

\nse{a period of...}{}{为期...}
\nse{a period of ... seconds/minutes/hours/days/...}{}{为期...秒/分/时/天/...}

\switchcolumn*

All these cats \ns{had one experience in common}: they had fallen off high buildings, yet only eight of them \ns{died from} shock or injuries. 

\switchcolumn

\nse{have ... in common}{}{在...方面有共同点}
\nse{die from}{}{死于(某种原因,不包括疾病、过度悲伤等);}

\switchcolumn*

Of course, New York is the \nw{ideal} place for such an interesting study, because there is \nw{no shortage of} tall buildings. 

\switchcolumn

\nwe{ideal}{aɪˈdiːəl}{adj. 理想的;不实际的;n. 理想;完美典型;}

\switchcolumn*

There are plenty of \nw{high-rise} \nw{windowsills} to fall from. 

\switchcolumn

\nwe{high-rise}{ˈhaɪ raɪz}{adj. 高层的;高耸的;}
\nwe{windowsill}{ˈwɪndoʊsɪl}{n. 窗沿,窗台;}

\switchcolumn*

One cat ,Sabrina, fell 32 storeys, yet only \ns{suffered from} a broken tooth. 

\switchcolumn

\nse{suffer from}{}{忍受,遭受,患...病,受...之苦;}

\switchcolumn*

‘Cats \ns{behave like} \nw{well-trained} \nw{paratroopers},’ a doctor said. 

\switchcolumn

\nwe{well-trained}{ˌwel'treɪnd}{adj. 训练有素的,受过良好训练的;}
\nwe{paratrooper}{ˈpærəˌtrupɚ}{n. 伞兵;}
\nse{... behave like ...}{}{...表现得像...}

\switchcolumn*

\newsentence{It seems that the further cats fall, the less they are likely to injure themselves}. 

\switchcolumn

\switchcolumn*

In a \nw{long drop}, they reach \ns{speeds of 60 miles an hour} and more. 

\switchcolumn

\nse{long drop}{}{长长的跌落过程}
\nse{speeds of ... metres/miles/... an seconds/minutes/...}{}{... 米/英里/...每秒/分钟/... 的速度}

\switchcolumn*

At high speeds, falling cats have time to relax. 

\switchcolumn

\switchcolumn*

They \ns{stretch out} their legs like flying \nw{squirrels}. 

\switchcolumn

\nwe{squirrel}{ˈskwɜːrəl}{n. 松鼠;}
\nse{stretch out}{}{(使)伸直身子躺下;伸出,伸开(身体的某一部分);}

\switchcolumn*

This \ns{increases their \nw{air-resistance}} and \ns{reduces the \ns{shock of impact}} when they hit the ground. 

\switchcolumn

\nwe{air-resistance}{'eərrɪz'ɪstəns}{空气阻力}
\nse{increases air-resistance}{}{增加空气阻力}
\nse{shock of impact}{}{冲击力}
\nse{reduces the shock of impact}{}{减少冲击力}

\switchcolumn*


\end{paracol}

\worddifference
\wsitem{high, tall}

\begin{multicols}{1}

在英语中,虽然两者都译为“高”,但逻辑出发点不同:\textbf{Tall 侧重于“个体的高度”,而 High 侧重于“所处的位置”。}

\begin{enumerate}

\item Tall (侧重:身材、纵向长度)}
用于描述从底部到顶部的距离,通常指那些\textbf{细长、垂直}且底部接触地面的物体。

\begin{itemize}[leftmargin=1.5cm]
    \item\textbf{人:} 描述身高时只能用 \textit{tall}。
    \begin{itemize}
        \item \textit{He is 180 cm \textbf{tall}.} (正) / \textit{He is high.} (误)
    \end{itemize}
    \item\textbf{细长物:} 如树木、塔楼、梯子等。
    \begin{itemize}
        \item \textit{A \textbf{tall} skyscraper.} (摩天大楼)
        \item \textit{A \textbf{tall} tree.} (一棵高树)
    \end{itemize}
\end{itemize}

\item High (侧重:海拔、位置、抽象高度)}

用于描述物体距离地面的高度(海拔),或者形容那些\textbf{宽大}的物体及\textbf{抽象}概念。

\begin{itemize}
    \item \textbf{位置:} 描述在空中的、远离地面的事物。
    \begin{itemize}
        \item \textit{The plane is flying \textbf{high} in the sky. (飞机飞得很高)}
    \end{itemize}
    \item \textbf{宽大物体:} 如山脉、围墙、天花板。
    \begin{itemize}
        \item \textit{A \textbf{high} mountain. (高大的山脉)}
        \item \textit{A \textbf{high} wall. (一堵高墙)}
    \end{itemize}
    \item \textbf{抽象概念:} 价格、速度、质量、地位等。
    \begin{itemize}
        \item \textit{\textbf{High} prices, \textbf{High} quality, \textbf{High} speed.}
    \end{itemize}
\end{itemize}

\end{enumerate}

\end{multicols}

\worddifference
\wsitem{Affectionate vs. Friendly vs. Loving}

\begin{multicols}{1}

    深度辨析:
    
    这三个词都表示“友好”或“爱”,但在程度和表达方式上有所不同:

    \begin{itemize}
        \item Be affectionate towards: 侧重于“亲昵的”。它通常指通过动作(如拥抱、抚摸)或言语公开表达出的慈爱和关怀。
        
        \es{She is very affectionate towards her cat. (她对她的猫非常亲昵。)}

        \item Be friendly towards: 侧重于“友好的”。指态度和蔼,好相处,但不一定有深厚的情感羁绊。
        
        \es{The villagers were friendly towards the strangers. (村民们对陌生人很友好。)}
        
        \item Be loving: 程度最深,指内心深处深沉的爱,但不一定总是表现得那么“亲昵”。
    \end{itemize}
\end{multicols}

\wsitem{Suspicious vs. Doubtful vs. Skeptical}
\begin{multicols}{1}
    这三个词在中文里都翻译为“怀疑”,但在英语逻辑中完全不同:

    \begin{itemize} 
    \item Be suspicious of: 侧重于“起疑心”。怀疑某人做了坏事,或者觉得某个情况有诈。带有强烈的警觉感。
    
    \es{The policeman was suspicious of Alf's smart clothes. (警察对阿尔弗雷德笔挺的衣服起了疑心。)}

    \item Be doubtful about: 侧重于“不确定”。对事实的真实性或结果的成功率表示犹豫。
    
    \es{I am doubtful about whether it will rain. (我拿不准会不会下雨。)}

    \item Be skeptical of: 侧重于“怀疑主义”。通常指对某种理论、说法持怀疑态度,不轻易相信。
    
    \es{Scientists are skeptical of his new discovery. (科学家们对他新发现持怀疑态度。)}

    \end{itemize}
\end{multicols}

\grammarpoints


\wsitem{never fail to ...}
\begin{multicols}{1}
    使用了双重否定(never + fail)来表达肯定的含义,语气比单纯的 always 要强烈且正式得多。该短语常用来赞扬某人的可靠性,或描述某种必然发生的自然/社会规律。

    \begin{enumerate}

        \item \textbf{语法成分拆解}
            \begin{itemize}
                \item \textbf{双重否定结构:} \textit{never} (否定词) + \textit{fail} (失败/未能)。
                \item \textbf{后续搭配:} \textit{fail} 后面固定接动词不定式 (\textit{to do})。
                \item \textbf{时态变化:} 常用一般现在时表达规律,或一般过去时描述一贯的表现。
            \end{itemize}

        \item \textbf{反向应用 (单重否定)}
            \begin{itemize}
                \item 如果去掉 \textit{never},变成 \textit{fail to do},则表示“未能做到某事”。
                \item \textit{Example:} \textit{He \textbf{failed to} arrive on time.} (他没能准时到达。)
            \end{itemize}

        \item 深度辨析:Never fail to vs. Always
        
        虽然两者在中文里都可以翻译为“总是”,但在语感上有显著差异:

        \begin{itemize}
            \item Always do sth: 描述一种常规的习惯或客观事实,语气平实。
            \item Never fail to do sth: 强调一种“绝无例外”的承诺、可靠性或必然性。它暗示了尽管环境复杂,但结果依然会发生。
            
            \es{He always helps me. (他总是帮我。)}

            \es{He never fails to help me when I'm in trouble. (每当我遇到困难,他准会向我伸出援手。—— 强调他的可靠。)}
        \end{itemize}
    \end{enumerate}
    
\end{multicols}

\wsitem{be affectionate towards ...}

\begin{multicols}{1}

\begin{enumerate}
    \item 语法结构拆解
    
    这个短语属于典型的 “连系动词 + 形容词 + 介词” 结构。

    \begin{itemize}
        \item Be (Linking Verb): 连系动词,根据主语的人称和时态变化(am, is, are, was, were 等)。
        \item Affectionate (Adjective): 形容词,意为“充满深情的”、“亲昵的”。
        \item Towards (Preposition): 介词,引出情感的对象。
    \end{itemize}
    \item 介词的选择:Towards vs. To
    
    在实际使用中,你会发现 towards 和 to 都可以接在后面,但存在细微差别:

    \begin{itemize}
        \item Towards (更常见): 强调情感的方向性,多用于描述一种持续的态度或倾向。在英国英语中非常普遍。
        \item To: 较直接,但在表示“对某人亲昵”时,使用频率略低于 towards。
    \end{itemize}

    \item 语感辨析:如何精准使用?
    
    使用这个结构时,要注意 affectionate 强调的是外露的情感(如肢体动作、温柔的语气)。

    \begin{itemize}
        \item 如果你说 "He is fond of his dog",这可能只是心理上的喜欢。
        \item 如果你说 "He is affectionate towards his dog",画面感就变成了他正在摸狗的头,或者在跟狗亲昵地说话。
    \end{itemize}

    \item 相似结构对比
    \begin{itemize}
        \item \textit{Be cruel to...} (对……残忍)
        \item \textit{Be indifferent towards...} (对……冷漠)
        \item \textit{Be enthusiastic about...} (对……热衷)
    \end{itemize}
\end{enumerate}

\end{multicols}

\wsitem{be suspicious of ...}
\begin{multicols}{1}
    "Be + 形容词 + of" 是英语中一个非常经典且高效的句式。这种结构通常用来描述主语对某事物的某种态度、认知、状态或情感。
    
    根据表达意思的不同,我为你整理了以下几类最常用的类似短语:

    \begin{enumerate}
        \item 表示“认知与觉察” (Awareness)
        
        这类短语描述你大脑中是否拥有某种信息或意识。

        \begin{itemize}
            \item Be aware of: 意识到、察觉到。
            
            \es{We must be aware of the potential risks. (我们必须意识到潜在的风险。)}
            
            \item Be conscious of: 意识到(通常指感官上或心理上的觉察)。
            
            \es{He was conscious of someone following him. (他察觉到有人在跟踪他。)}
            
            \item Be ignorant of: 对……一无所知(Aware的反义词)。
            
            \es{She was ignorant of the new rules. (她对新规则一无所知。)}
        \end{itemize}

        \item 表示“情感与态度” (Emotions \& Attitudes)
        
        描述你对某人或某事的主观感受。

        \begin{itemize}
            \item Be proud of: 为……感到自豪。
            
            \es{They are proud of their daughter’s achievements. (他们为女儿的成就感到骄傲。)}

            \item Be afraid of / Be terrified of: 害怕…… / 恐惧……。
            
            \es{Many people are afraid of spiders. (很多人怕蜘蛛。)}

            \item Be fond of: 喜欢……(语气比 like 稍微正式或温和一点)。
            
            \es{My grandmother is very fond of gardening. (我祖母非常喜欢园艺。)}

            \item Be envious of / Be jealous of: 羡慕…… / 嫉妒……。
            
            \es{He was a little envious of his friend's success. (他有点羡慕朋友的成功。)}
        \end{itemize}

        \item 表示“特征与属性” (Characteristics)
        
        描述事物的构成、性质或能力。

        \begin{itemize}
            \item Be capable of: 有能力做……。
            
            \es{The human brain is capable of amazing things. (人类大脑能做出了不起的事情。)}

            \item Be characteristic of: 是……的特征(典型的……)。
            
            \es{Quick decisions are characteristic of her style. (果断是她的风格特点。)}

            \item Be composed of: 由……组成(常用于科学或正式语境)。
            
            \es{Water is composed of hydrogen and oxygen. (水是由氢和氧组成的。)}

            \item Be short of: 缺乏……。
            
            \es{We are a bit short of time. (我们时间有点紧。)}
        \end{itemize}

        \item 表示“怀疑与确定” (Certainty & Doubt)
        
        描述你对某件事的把握程度。

        \begin{itemize}
            \item Be sure of / Be certain of: 对……有把握/确定。
            
            \es{Are you sure of your facts? (你确定你的事实无误吗?)}

            \item Be skeptical of: 对……持怀疑态度(逻辑上的怀疑)。
            
            \es{Scientists are skeptical of the new findings. (科学家对这些新发现持怀疑态度。)}
        \end{itemize}

        \item 核心规律总结
        
        通过观察这些短语,你可以发现一个记忆的小技巧:
        
        \begin{itemize}
            \item 结构:Subject + Be + Adjective + Of + Object
            \item 为什么用 Of? 在这些结构中,of 起到了连接作用,用来引出这种“态度”或“状态”所指向的对象。
        \end{itemize}

        \item 避坑指南:注意介词搭配
        
        并不是所有的 "Be + Adj" 后面都接 of。这是学习中最容易混淆的地方:

        \begin{itemize}
            \item Be interested in (不是 of)
            \item Be good at (不是 of)
            \item Be responsible for (不是 of)
        \end{itemize}
    \end{enumerate}
\end{multicols}


\wsitem{one of the things that...}
\begin{multicols}{1}

"One of the things that..." 是英语中极高频的句型,用于从多个因素中提取出一个重点来讨论。

\begin{enumerate}
    \item 语法结构
    
    这个句型后面通常接一个定语从句,用来修饰 "things"。

    结构: One of the things that + [动词/从句] + is + [重点内容]

    注意点: 尽管 "things" 是复数,但句子的核心主语是 "One",所以后面的系动词要用单数 is 或 was。

    \item 常见用法与例句
    \begin{itemize}
        \item 引出原因或特征
        \begin{itemize}
            \item "One of the things that makes this app great is its simplicity."(让这个 App 表现出色的原因之一是它的简洁性。)
            \item "One of the things that I love about 2025 is the advancement in AI."(关于 2025 年我喜欢的事情之一是 AI 的进步。)
        \end{itemize}
        \item 引出令人困扰的事
        
        "One of the things that bothers me is the loud noise outside."(让我的困扰之一是外面的噪音。)

    \end{itemize}
    \item 为什么使用这个句型?
    \begin{itemize}
        \item 起到缓冲作用: 直接说 "The reason is..." 显得生硬,用这个句型听起来更自然、更像在聊天。
        \item 强调筛选感: 暗示有很多相关的事,但我现在只挑其中一件最重要的来说。
    \end{itemize}
    \item 口语变体
    
        在非正式口语中,人们经常省略 "that":

        "One of the things I noticed about this year is..."

        (关于今年我注意到的一件事是……)

    \item 相似句型
    \begin{itemize}
        \item One of the reasons why... (原因之一是……)
        \item One of the best ways to... (最好的方法之一是……)
        \item The thing is... (口语常用:问题是/情况是……)
    \end{itemize}
\end{enumerate}

\end{multicols}

\wsitem{have ... in common}
\begin{multicols}{1}

    这个短语的核心在于 in common(共有/共同)。

    \begin{enumerate}
        \item 结构成分
        \begin{itemize}
            \item \textbf{谓语动词:} \textit{have} (根据时态变化:has, had, having)。
            \item \textbf{宾语:} 通常是表示“程度”的代词,如 \textit{something, much, little, a lot, nothing}。
            \item \textbf{状语:} \textit{in common} 作为固定短语修饰整个结构,表示“共同地”。
        \end{itemize}
        \item 常见结构
        \begin{itemize}
            \item Have something in common: 有一些共同点。
            \item Have a lot in common: 有很多共同点。
            \item Have nothing in common: 毫无共同之处。
            \item Have everything in common: 完全志同道合。
        \end{itemize}

        \item 语义重心
        
    这个短语强调的是“属性或特征的重合”。

    \es{Jane and I have a lot in common; we both love traveling. (简和我有很多共同点;我们都热爱旅游。)}
    \es{The two cases have nothing in common. (这两起案件毫无共同之处。)}

    \item \textbf{扩展句型:与某人的共同点}
    \begin{itemize}
        \item \textbf{公式:} \textit{have something in common \textbf{with} somebody}
        \item \textit{Example:} \textit{I discovered that I \textbf{had little in common with} my roommate.} (我发现我和室友几乎没有共同语言。)
    \end{itemize}

    \item \textbf{同义表达对比}
    \begin{itemize}
        \item \textit{be similar to} (侧重外观或性质相似)
        \item \textit{share the same...} (侧重分享同一个目标或观点)
    \end{itemize}

    \item 语感提示:位置的灵活性
    
    虽然通常宾语放在中间(have something in common),但在强调“共同点很多”时,也可以说:

    \es{They have in common a love of music. (他们共同的爱好是音乐。) —— 这种情况通常是因为宾语(a love of music)后面接了较长的修饰语。}
        
    \end{enumerate}
\end{multicols}


\newpage