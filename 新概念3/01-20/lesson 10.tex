

\section{Lesson 10 The loss of Titanic}

\begin{paracol}{2}
    The great ship, Titanic, sailed for New York from \nw{Southampton} on April 10th, 1912. 

\switchcolumn

\nwe{Southampton}{saʊθ'hæmptənˌ saʊ'θæmp-}{南安普敦(英国英格兰南部港市);}

\switchcolumn*

    She was carrying 1316 passengers and \newsentence{a crew of 89l}. 

\switchcolumn

\switchcolumn*

    Even by \ns{modern standards}, the 46,000 ton Titanic was a \nw{colossal} ship. 

\switchcolumn

\nwe{colossal}{kəˈlɑːsl}{adj. 巨大的;<口语>异常的;}
\nse{modern standard}{}{现代标准}

\switchcolumn*

    At that time, however, she was not only the largest ship that had ever been built, but was \ns{regarded as} unsinkable, for she had sixteen \nw{watertight} \nw{compartments}. 

\switchcolumn

\nwe{watertight}{ˈwɔːtərtaɪt}{adj. 防水的;}
\nwe{compartment}{kəmˈpɑːrtmənt}{n. 隔间;隔层;}
\nse{regard as}{rɪˈɡɑrd æz}{v. 把…认作;当做;看成;看做;}

\switchcolumn*

    Even if two of these were \nw{flooded}, she would still be able to \nw{float}. 

\switchcolumn

\nwe{flood}{flʌd}{n. 洪水;泛滥;溢流;探照灯;vi. 涌出;为水淹没;vt. 淹没;}
\nwe{float}{floʊt}{v. 使漂浮;飘荡;优雅地走动;提出;使上市;货币自由浮动;n. 浮体;鱼漂浮子;彩车;备用零钱;加冰激凌的饮料;}

\switchcolumn*

    The \nw{tragic} sinking of this great \nw{liner} will always be remembered, for she \ns{went down} on her first \nw{voyage} with \ns{heavy loss of life}. 

\switchcolumn

\nwe{tragic}{ˈtrædʒɪk}{adj. 悲剧的;悲惨的;}
\nwe{liner}{ˈlaɪnər}{n. 班轮,班机;衬垫,衬套;衬里;画线者;}
\nwe{voyage}{ˈvɔɪɪdʒ}{n. 航行,航海;航天;v. 旅行;(尤指)航海旅行;}
\nse{go down}{}{停止;被接受;沉下;被打败;}
\nse{heavy loss of life}{}{重大伤亡}

\switchcolumn*

    Four days after \ns{setting out}, while the Titanic was sailing across the \newsentence{icy waters} of the North Atlantic, a huge \nw{iceberg} was suddenly spotted by a \nw{lookout}. 

\switchcolumn

\nwe{iceberg}{ˈaɪsbɜːrɡ}{n. 冰山;(事物全貌的)一小部分;冷若冰霜的人;}
\nwe{lookout}{ˈlʊkˌaʊt}{n. 守望;远景,前途;瞭望台;警戒;}
\nse{set out}{}{出发;启程;(怀着目标)开始工作;}

\switchcolumn*

    After the alarm had been given, the great ship turned sharply to avoid a direct \nw{collision}. 

\switchcolumn

\nwe{collision}{kəˈlɪʒn}{n. 碰撞;冲突;}

\switchcolumn*

    The Titanic turned just \ns{in time}, narrowly missing the immense wall of ice which rose over 100 feet out of the water beside her. 

\switchcolumn

\nse{in time}{ɪn taɪm}{迟早; 最后;及时; 经过一段时间之后;}

\switchcolumn*

    Suddenly, there was a slight \nw{trembling} sound from below, and the captain went down to see what had happened. The noise had been so \nw{faint} that no one thought that the ship had been damaged. 

\switchcolumn

\nwe{tremble}{ˈtrembl}{v. 颤抖;颤动;使担心;n. 战栗;}
\nwe{faint}{feɪnt}{adj. 微弱的,暗淡的;敷衍的;晕眩的;v. 晕倒;n. 昏厥;}

\switchcolumn*

    Below, \newsentence{the captain realized to his horror that the Titanic was sinking rapidly}, for five of her sixteen watertight compartments had already been flooded ! 

\switchcolumn

\nse{realize to his horror}{}{}

\switchcolumn*

    The order to abandon ship was given and hundreds of people \nw{plunged} into the icy water. 

\switchcolumn

\nwe{plunge}{plʌndʒ}{vt. 用力插入;使陷入;vi. 跳入;全心投入;突降,俯冲;n. 投入,陷入;游泳,跳水;暴跌;}

\switchcolumn*

    As there were not enough lifeboats for everybody, 1500 lives were lost.

\switchcolumn

\nwe{lifeboat}{ˈlaɪfboʊt}{n. 救生艇,救生船;}

\switchcolumn*

\end{paracol}

\grammarpoints
\wsitem{sail, voyage}
\begin{multicols}{1}
    这两个词虽然都和“航行”有关,但在用法、侧重点以及航行的“规模”上有显著区别。简单来说:\textbf{Sail 是“怎么走”,而 Voyage 是“走多远”。}

\begin{enumerate}
    \item Voyage (名词 \& 动词)
    \textbf{侧重点:长距离、正式、探索性}

\begin{itemize}
    \item \textbf{作为名词 (最常用):} 指一段\textbf{漫长的旅程},通常是跨洋、去遥远的异国他乡,甚至是太空旅行。
    \begin{itemize}
        \item \textit{The Titanic sank on her maiden \textbf{voyage}.} (泰坦尼克号在首航中沉没了。)
        \item \textit{A manned \textbf{voyage} to Mars.} (载人火星航行。)
    \end{itemize}
    \item \textbf{作为动词 (较文学化):} 指进行长途航行。
    \begin{itemize}
        \item \textit{They \textbf{voyaged} to distant lands.} (他们远航到了遥远的土地。)
    \end{itemize}
\end{itemize}

\item Sail (动词 \& 名词)
\textbf{侧重点:动作、操控、休闲}

\begin{itemize}
    \item \textbf{作为动词 (最常用):} 指船在水上移动的\textbf{动作},或者人驾驶、操纵船只。
    \begin{itemize}
        \item \textit{The ship \textbf{sails} for London tomorrow.} (这艘船明天启程前往伦敦。)
        \item \textit{I want to learn to \textbf{sail}.} (我想学习驾驶帆船。)
    \end{itemize}
    \item \textbf{作为名词:} 1. 指船帆;2. 指一次简短的水上旅行。
    \begin{itemize}
        \item \textit{Let's go for a \textbf{sail}.} (我们出海转一圈吧。)
    \end{itemize}
\end{itemize}
\end{enumerate}

\end{multicols}

\wsitem{realize to one's horror}
\begin{multicols}{1}

    这是一个非常生动的英语习语,通常用来描述一种突然意识到某种可怕事实的瞬间。
    \begin{enumerate}
        \item 语法结构分析
        
        这个表达属于 "to + one's + 情感名词" 结构,在句中通常作状语(状语从句的缩略形式)。

        realize: 动词,意为“意识到”。

        to: 介词,在这里表示“结果”,引出某种情感反应。

        one's: 物主代词(如 my, his, her, their),指代意识到事实的那个人。

        horror: 抽象名词,意为“惊恐、恐惧”。

        \item 常见用法与句式
        
        这个结构通常有两种摆放方式:

        \begin{itemize}
            \item 放在句首(作评注性状语,强调情感)
            
            \textit{To his horror, he realized he had left the stove on. (令他惊恐的是,他意识到自己没关煤气灶。)}

            \item 紧跟动词(作方式状语)
            
            \textit{She realized to her horror that she was being followed. (她惊恐地意识到自己正被人跟踪。)}
        \end{itemize}
        \item 类似的“情感状语”结构
        \begin{itemize}
            \item To my surprise:令我吃惊的是
            \item To our relief:令我们欣慰的是
            \item To her disappointment:令她失望的是
            \item To their amazement:令他们惊叹的是
        \end{itemize}
        \item 深度解析:为什么不直接说 "realized horribly"?
        
        虽然 "realized horribly" 在语法上可能通顺,但含义不同:

        \begin{itemize}
            \item Realized horribly: 侧重于意识到的这个“动作”很糟糕(不常用)。
            \item Realize to one's horror: 侧重于意识到的内容引发了内心的恐惧。这是一种更地道、更具画面感的表达方式。
        \end{itemize}
    \end{enumerate}
\end{multicols}

\wsitem{a crew of ...}
\begin{multicols}{1}
    "A crew of..." 是英语中一个非常典型的集体名词(Collective Noun)表达方式。它不仅表示“一群人”,更强调这群人是为了同一个目标、在同一个交通工具或专业团队中协作的。

    以下是针对这个短语的深度解析:

    \begin{enumerate}
        \item 核心属性:专业性与协作
        
        Crew 最核心的含义是“全体工作人员”。

        \begin{itemize}
            \item 海事/航空:指船上、飞机上或航天器上的全体成员。
            
            \es{A crew of 20 sailors. (一个由20名水手组成的船员团队。)}

            \item 专业技术团队:指电影剧组、急救小组或施工队伍。
            \begin{itemize}
                \item A camera crew. (摄像小组。)
                \item An ambulance crew. (救护小组。)
            \end{itemize}
        \end{itemize}
        \item 语法结构:名词的数
        
        A crew of + 复数名词

        \begin{itemize}
            \item 谓语动词的单复数:
            \begin{itemize}
                \item 如果你把 crew 看作一个整体,谓语动词用单数:
                
                \es{A crew of specialists is needed for this mission.}

                \item 如果你强调团队里的每个成员,谓语动词用复数:
                
                \es{A crew of rescuers were searching for the missing dog.}
            \end{itemize}
        \end{itemize}
        \item 集体名词
        
        在英语中,集体名词 (Collective Nouns) 是一个非常有趣的类别。它们指代的是“一组人、动物或事物”,虽然形式上通常是单数,但在逻辑上包含多个个体。


        理解集体名词的关键在于:你是把它看作一个整体,还是看作其中的每一个成员?

        \begin{itemize}
            \item 集体名词的“变色龙”属性 (单复数一致性)

            \begin{itemize}
                \item 看作整体:My family is a happy one. (我的家庭是个幸福的整体。)
                \item 看作成员:My family are all sitting at the table. (我家人都坐在桌边。——强调每个人都在座。)
            \end{itemize}
        \end{itemize}

        \item 四大类常见的集体名词
        
        为了方便记忆,我们可以将它们按“物种”和“语境”分类:

        \begin{itemize}
            \item 人的职业与协作 (Professional Groups)
            \begin{itemize}
                \item Crew: 全体船员/机组人员(你之前问过的专业团队)。
                \item Staff: 全体职员。
                \item Committee: 委员会。
                \item Audience: 观众。
            \end{itemize}

            \item 亲属与社会属性 (Social Groups)
            \begin{itemize}
                \item Family: 家庭。
                \item Team: 队伍。
                \item Class: 班级。
                \item Crowd: 人群(通常指无序的)。
            \end{itemize}

            \item 动物界(最地道的表达)
            
            英语中对不同动物的“群”有非常具体的称呼,这能体现你的词汇水平:

            \begin{itemize}
                \item A pack of dogs/wolves (一群狗/狼)。
                \item A pride of lions (一群狮子)。
                \item A flock of birds/sheep (一群鸟/羊)。
                \item A school of fish (一群鱼)。
            \end{itemize}
            \item 物品 (Groups of Things)
            \begin{itemize}
                \item A set of keys/tools (一套钥匙/工具)。
                \item A bunch of flowers/grapes (一束花/一串葡萄)。
                \item A fleet of ships/cars (一个船队/车队)。
            \end{itemize}
        \end{itemize}

        \item 语法进阶:只有复数意义的集体名词
        
        有些集体名词“天生”就必须接复数动词,因为它们从不被视为单一实体:

        \begin{itemize}
            \item Police: 警察。 (不说 The police is... 而说 The police are...)
            \item Cattle: 牛群。
            \item People: 人们。
        \end{itemize}
    \end{enumerate}
\end{multicols}

\wsitem{waters}
\begin{multicols}{1}
    这是一个非常精妙的语法点。在英语中,waters(water 的复数形式)并不是指“很多杯水”,而是一个专门用来描述大面积海域、领海或具有特定特征的水域的术语。

    \begin{enumerate}
        \item \textbf{特定海域的地理概念 (Specific Areas of Sea)}
        \begin{itemize}
            \item \textbf{逻辑意义:} 当 \textit{water} 变成 \textbf{waters} 时,它不再指代物质本身,而是指代一个**特定的地理区域**(通常是海洋或大湖)。
            \item \textbf{语境解析:} \text{The North Atlantic}(北大西洋)是一个极其广阔的区域,使用 \textbf{waters} 能够体现出那种“波澜壮阔、无边无际”的海域感。
        \end{itemize}

        \item \textbf{强调水域的“性质”或“状态” (Character and Condition)}
        \begin{itemize}
            \item \textbf{核心逻辑:} 当我们想给某片海域贴上“标签”(如:冰冷的、危险的、外国的)时,习惯用复数。
            \item \textbf{常见搭配:}
                \begin{itemize}
                \item \textbf{Icy waters:} 冰冷的海域。
                \item \textbf{Deep waters:} 深水区(也常比喻深陷麻烦)。
                \item \textbf{International waters:} 公海。
                \item \textbf{Territorial waters:} 领海。
            \end{itemize}
        \end{itemize}

        \item \textbf{文学与叙事的庄重感 (Stylistic Effect)}
        \begin{itemize}
            \item 使用 \textbf{waters} 比单数更能营造出一种**史诗般的、危机四伏**的氛围。它让读者感觉到船只不仅仅是在水上漂,而是在一个充满力量和危险的环境中航行。
        \end{itemize}
    \end{itemize}
\end{multicols}

\newpage