\section{Lesson 6 Smash-and-grab}

\begin{paracol}{2}

The \ns{expensive shops} in a famous \nw{arcade} near Piccadilly were just opening. 

\switchcolumn

\nwe{arcade}{ɑrˈkeɪd}{n. 商场,游乐中心;拱廊,连拱廊;有拱廊的街道;}
\nse{expensive shop}{}{昂贵的商店}


\switchcolumn*

\ns{At this time of the morning}, the arcade was almost empty. 

\switchcolumn

\nse{at this time of the morning}{}{在早晨的这个时候}

\switchcolumn*

Mr. Taylor, the owner of a jewellery shop was admiring a new display. 

\switchcolumn

\switchcolumn*

Two of his assistants had been working busily since eight o'clock and had only just finished. 

\switchcolumn

\switchcolumn*

Diamond \nw{necklaces} and rings had been beautifully arranged on a background of black \nw{velvet}. 

\switchcolumn

\nwe{necklace}{ˈnekləs}{n. 项链;v. 给(某人)戴火项链(将燃烧的轮胎挂在脖子上将其杀死);}
\nwe{velvet}{ˈvelvɪt}{n. 丝绒;天鹅绒;立绒;赢得的钱;}

\switchcolumn*

After \ns{gazing at} the display for several minutes, Mr. Taylor went back into his shop.

\switchcolumn

\nse{gaze at}{}{凝视;瞄;注目;盯;}

\switchcolumn*

The silence was suddenly broken when a large car, with its \nw{headlights} on and its \nw{horn} \nw{blaring}, \nw{roared} down the arcade. 

\switchcolumn

\nwe{headlights}{ˈhɛdˌlaɪt}{n. (汽车等)的前灯;照明灯;}
\nwe{horn}{hɔːrn}{n. 角,角质物;号,喇叭;角状物;}
\nwe{blaring}{bler}{vi. 发出响而刺耳的声音;vt. 嘟嘟地发出;高声发出;高声宣布;n. 嘟嘟声,巨响;(颜色等的)光泽;}
\nwe{roar}{rɔːr}{vi. 咆哮;喧闹;吼叫;混乱或吵闹;vt. 大声喊出;使…轰鸣;吼,咆啸;n. 吼叫声,咆哮声,呼啸声;狂笑,大笑;}

\switchcolumn*

It \newsentence{came to a stop} outside \newsentence{the jeweller's}. 

\switchcolumn

\switchcolumn*

One man \ns{stayed at the wheel} while two others \ns{with black stocking over their faces} \nw{jumped out} and smashed the window of the shop with iron bars. 

\switchcolumn

\nse{stay at the wheel}{}{}
\nse{with black stocking over one's face(s)}{}{蒙着黑布脸}
\nse{jump out}{}{视觉(或精神)上影响强烈的;令人眼花缭乱;引人注目的;醒目的;}

\switchcolumn*

While this was going on, Mr. Taylor was upstairs. 

\switchcolumn

\switchcolumn*

He and his staff began throwing furniture out of the window. 

\switchcolumn

\switchcolumn*

Chairs and tables went flying into the arcade. 

\switchcolumn

\switchcolumn*

One of the thieves was struck by a heavy statue, but he was too busy \ns{helping himself to} diamonds to notice any pain. 

\switchcolumn

\nse{help oneself to}{}{擅自取用,侵占某物}

\switchcolumn*

The \nw{raid} was all over in three minutes, for the men \nw{scrambled} back into the car and it \ns{moved off} \ns{at a \nw{fantastic} speed}. 

\switchcolumn

\nwe{fantastic}{fænˈtæstɪk}{adj. 极好的;极大的;奇异的;不切实际的;}
\nwe{raid}{reɪd}{n. (骑兵队等的)急袭,突然袭击,突击,(军舰等的)游击;劫掠,劫夺,(盗贼,狐等的)侵入;突然查抄[搜捕],围捕;vi. 对…进行突然袭击;进行奇袭;vt. 突然袭击[抢劫];突然搜查;劫掠;打劫;}
\nwe{scramble}{ˈskræmbl}{vi. 攀登,爬;争夺,抢夺;(植物)蔓延;[航]紧急起飞;vt. 攀登,爬;把…搅乱;炒(蛋);n. 攀登;抢夺;混乱;[航]紧急起飞;}
\nse{move off}{}{v. 离开,死,畅销;}
\nse{at a fantastic speed}{}{以极快的速度}

\switchcolumn*

Just as it was leaving, Mr. Taylor \ns{rushed out} and \ns{ran after} it throwing \nw{ashtrays} and \nw{vases}, but it was impossible to stop the thieves. 

\switchcolumn

\nwe{ashtray}{ˈæʃtreɪ}{n. 烟灰缸;}
\nwe{vase}{veɪs}{n. 装饰瓶,花瓶;}
\nse{rush out}{}{仓促地跑出;赶着生产出;冲出;}
\nse{run after}{}{追赶;追求;伺候;奔逐;}


\switchcolumn*

They had \ns{got away} with \ns{thousands of pounds worth of} diamonds.

\switchcolumn

\nse{get away}{}{离开,脱身;逃掉;抽身;拔身;}
\nse{thousands of pounds worth of}{}{价值数千英镑的}

\switchcolumn*

\end{paracol}

\grammarpoints
\wsitem{come to a/an ...}
\begin{multicols}{1}
    \begin{enumerate}
    \item \textbf{语感辨析:} 与简单的动词 \textit{stop} 相比,\textit{come to a stop} 更具画面感。它描述的是一个“由动到静”的完整过程,强调车辆减速直至最终稳稳停住的状态。
        \item \textbf{结构拓展:} 这种“\textit{come to + a/an + 名词}”的结构在英语中非常地道,常用于替代对应的动词。
        \begin{itemize}
            \item \textit{come to an end} (结束 — 替代 finish/end)
            \item \textit{come to a conclusion} (得出结论 — 替代 conclude)
        \end{itemize}
        \item \textbf{例句:} \textit{The train slowed down and gradually \textbf{came to a stop} at the platform.} (火车减速并缓缓停靠在站台旁。)
\end{enumerate}
\end{multicols}

\wsitem{The jeweller's}
\begin{multicols}{1}
    \begin{enumerate}

    \begin{itemize}
        \item \textbf{语法点:} 所有格省略 (Elliptical Possessive)。在英语中,当所有格修饰的名词是指代商店、诊所或住宅时,该名词(如 \textit{shop, office, house})通常会被省略。
        \item \textbf{常见同类用法:}
        \begin{itemize}
            \item \textit{at the chemist's} (在药店)
            \item \textit{at the barber's} (在理发店)
            \item \textit{at my grandmother's} (在我奶奶家)
        \end{itemize}
        \item \textbf{课文语境:} 这里指的就是那家倒霉的、被劫匪撞破橱窗的珠宝店。
    \end{itemize}
\end{enumerate}
\end{multicols}

\wsitem{help oneself to...}
\begin{multicols}{1}
    \begin{enumerate}
    \item \textbf{社交语境:礼貌的“请自便”}
    \begin{itemize}
        \item \textbf{语义:} 这是该短语最常见的用法,用于主人邀请客人随意取用食物、饮料或设施。语气热情、友好。
        \item \textbf{例句:} \textit{``Please \textbf{help yourself to} some coffee and cakes,'' said the host.} (“请随便喝点咖啡,吃点蛋糕,”主人说。)
    \end{itemize}

    \item \textbf{负面语境:讽刺性的“擅自取用/洗劫”}
    \begin{itemize}
        \item \textbf{语义:} 当主语是非法闯入者(如劫匪、小偷)时,该短语带有一种强烈的讽刺意味。它并非指“帮忙”,而是形容对方像在自家一样肆无忌惮地拿走财物。
        \item \textbf{课文应用:} \textit{Two men scrambled out and \textbf{helped themselves to} the jewellery.} (两名男子爬了出来,肆无忌惮地洗劫了珠宝。)
        \item \textbf{效果:} 这种表达比单纯用 \textit{steal} 更具文学表现力,突出了劫匪的胆大妄为(audacity)。
    \end{itemize}

    \item \textbf{语法结构限制}
    \begin{itemize}
        \item \textbf{人称一致:} 反身代词 \textit{oneself} 必须随主语改变。例如:\textit{I helped myself... / They helped themselves...}
        \item \textbf{介词固定:} 必须搭配介词 \textit{to},其后接具体的物品名称,不可省略。
    \end{itemize}

    \item \textbf{近义词力度对比}
    \begin{itemize}
        \item \textbf{Take:} 中性,无感情色彩。
        \item \textbf{Grab:} 强调动作的粗鲁和匆忙(第6课中用来形容劫匪抓取珠宝的动作)。
        \item \textbf{Help oneself to:} 强调过程的“从容”与结果的“非法占有”,带有叙事上的嘲讽。
    \end{itemize}
\end{enumerate}
\end{multicols}

\wsitem{get away with}
\begin{multicols}{1}
    这个短语是本课的灵魂。它不仅仅是“逃跑”,更强调\textbf{“做坏事没被抓到/没受惩罚”}。
    \begin{enumerate}
        \item 核心语义层级
        
        第一层(字面): 带着某物逃走。

        \es{The thieves got away with the diamonds. (劫匪带着钻石逃跑了。)}

        第二层(抽象): 做了坏事却没受惩罚/侥幸逃脱。

        \es{You can't get away with lying to the teacher. (你撒谎骗老师是瞒不过去的。)}

        \item 词义对比
        \begin{itemize}
            \item Escape: 侧重于从封闭的地方(监狱、火灾)逃出来。
            \item Flee: 侧重于因为恐惧而迅速逃离。
            \item Get away with: 侧重于\textbf{“得逞”},不仅人跑了,而且还没付出代价。
        \end{itemize}
    \end{enumerate}
\end{multicols}
\newpage