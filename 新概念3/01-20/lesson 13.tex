\section{Lesson 13 It's only me}


\begin{paracol}{2}

After her husband had gone to work, Mrs. Richards sent her children to school and \nw{went upstairs} to her bedroom. 

\switchcolumn

\nse{go upstairs}{}{上楼}

\switchcolumn*

She was too excited to do any housework \ns{that morning}, for in the evening she would be going to a \ns{fancy-dress party} with her husband. 

\switchcolumn

\nse{that morning}{}{那个早上}
\nse{fancy-dress party}{}{化妆舞会}

\switchcolumn*

She \ns{intended to} \ns{dress up as} a ghost and as she had made her \nw{costume} \ns{the night before}, she \ns{was impatient to} \ns{try it on}. 

\switchcolumn

\nwe{costume}{ˈkɑstum}{n. (戏剧或电影的)戏装,演出服;服装,衣服;泳装,常用于英式英语;(某地或某历史时期的)服装;}
\nse{intend to}{}{打算(做)…,想要(做)…;}
\nse{dress up as ...}{}{化妆成...}
\nse{the night before}{}{昨天晚上}
\nse{try on ...}{}{试穿,试戴}

\switchcolumn*

Though the costume \ns{consisted only of} a \nw{sheet}, it was very \nw{effective}. 

\switchcolumn

\nwe{sheet}{ʃit}{n. 纸;被单;一张(通常指标准尺寸的纸);一大片(覆盖物);表格;}
\nwe{effective}{ɪˈfɛktɪv}{adj. 有效的;起作用的;实际的,实在的;给人深刻印象;}
\nse{consist of}{}{由…组成;由 ... 组成;包括;}

\switchcolumn*

After putting it on, Mrs. Richards \ns{went downstairs}. 

\switchcolumn

\nse{go downstairs}{}{下楼梯}

\switchcolumn*

She wanted to \ns{find out} whether it would be comfortable to wear.

\switchcolumn

\nse{find out}{}{发现;使发作;使受惩罚;通过探询[访问]获悉(某人)不在;}

\switchcolumn*

\ns{Just as} Mrs. Richards was entering the \ns{dinning room}, there was a knock on the front door. 

\switchcolumn

\nse{just as}{}{正当...的时候}
\nse{dinning room}{}{餐厅}

\switchcolumn*

She knew that it must be the \nw{baker}. 

\switchcolumn

\switchcolumn*

She had told him to come straight in \ns{if ever} she failed to open the door and to leave the bread on the kitchen table. 

\switchcolumn

\nse{if ever}{}{很少,难得;如果有过的话;}

\switchcolumn*

\newsentence{Not wanting to frighten the poor man}, Mrs. Richards quickly hid in the small \nw{storeroom} under the stairs. 

\switchcolumn

\nwe{storeroom}{ˈstɔrˌrum, -ˌrʊm, ˈstor-}{n. 贮藏室,商品陈列室;}

\switchcolumn*

She heard the front door open and \ns{heavy footsteps} in the \ns{hall}. 

\switchcolumn

\nwe{hall}{hɔl}{n. 过道,走廊;大厅,前厅;娱乐中心,会所;}
\nse{heavy footsteps}{}{沉重的脚步声}

\switchcolumn*

Suddenly the door of the storeroom was opened and a man entered. 

\switchcolumn

\switchcolumn*

Mrs. Richards realized that it must be the man from the \ns{Electricity Board} who had come to \ns{read the metre}. 

\switchcolumn

\nse{Electricity Board}{}{供电局}
\nse{read the metre}{}{查电表}

\switchcolumn*

She tried to explain the situation, \newsentence{saying 'It's only me'}, but it was too late. 

\switchcolumn

\switchcolumn*

The man \ns{let out a cry} and jumped back several \nw{paces}. 

\switchcolumn

\nwe{pace}{peɪs}{n. 一步;长度单位;步幅,步调;快步;vt. 踱步,走来走去;步测;调整步调;训练马溜蹄;vi. 踱;溜蹄;}
\nse{let out a cry}{}{大叫一声}

\switchcolumn*

When Mrs. Richards walked towards him, he \nw{fled}, \newsentence{\nw{slamming} the door behind him}.

\switchcolumn

\nwe{flee}{fli}{vi. 逃走,逃掉;消失;vt. 逃离,逃避;}
\nwe{slam}{slæm}{vt. 砰地关上(门或窗);猛烈抨击;使劲一推;猛劲一摔;vi. 砰然关上;猛力打击,碰撞;撞到了一辆卡车上;n. 猛然关闭的声音;猛击;}

\switchcolumn*

\end{paracol}

\grammarpoints

\wsitem{Just as ...}
\begin{multicols}{1}
    "Just as" 是一个非常实用的连词短语,它在句子中主要起到“同步”或“类比”的作用。在《新概念英语》中,它经常被用来营造一种\textbf{“无巧不成书”}的画面感。
    
    逻辑深度解析:Just as 深度语法解析
    \begin{enumerate}
        \item \textbf{核心功能一:时间同步 (Simultaneous Action)}
        \begin{itemize}
            \item \textbf{逻辑:} 强调两个动作在“那一瞬间”完全同时发生。
            \item \textbf{语感:} 正当……之时,就在……那一刻。语气比 \textit{as} 或 \textit{when} 更强烈、更具即时感。
            
            \es{例句:Just as} they were about to leave, the boat began to sink.(就在他们准备离开时,船开始下沉了。)
        \end{itemize}

        \item \textbf{核心功能二:程度或方式类比 (Comparison)}
        \begin{itemize}
            \item \textbf{逻辑:} 表示“正如……一样”。
            \item \textbf{常见结构:} \textbf{Just as ..., (so) ...} (正如……,……也同样)。
            
            \es{例句:Just as the sea was rough, the wind was strong. (正如大海波涛汹涌,风也同样猛烈。)}
        \end{itemize}

        \item \textbf{与 Just like 的逻辑区别}
        \begin{itemize}
            \item \textbf{Just as + 句子 (从句):} \textbf{Just as} I said... (正如我所说)。
            \item \textbf{Just like + 名词/代词:} \textbf{Just like} my father. (就像我父亲一样)。
        \end{itemize}
        
        \item \textbf{词义辨析:Just as vs. Just like}
        
        虽然都翻译为“正如”,但在语法上不能混用:

        \begin{itemize}
            \item \textbf{Just as:} 后接从句(主语+谓语)。
            
            \es{Just as I thought... (正如我所想的...)}

            \item Just like: 后接名词或代词。

            \es{Just like me. (就像我一样。)}
        \end{itemize}
    \end{enumerate}
\end{multicols}

\wsitem{Just as Mrs. Richards was entering the dinning room...}
\begin{multicols}{1}
    在 Just as 引导的时间状语从句中,时态的选择(过去进行时 vs. 一般过去时)直接决定了故事的\textbf{“镜头感”}。

    \begin{enumerate}
        \item \textbf{背景与突发的逻辑对比 (Background vs. Interruption)}
        \begin{itemize}
            \item \textbf{过去进行时 (was entering):} 充当“背景动作”。它表示一个在过去某一时刻正在进行的、持续性的动作。就像电影中的长镜头,交代故事发生的环境。
            \item \textbf{一般过去时 (there was a knock):} 充当“突发动作”。它表示一个突然发生、打破了原有连续状态的瞬间动作。
            \item \textbf{逻辑缝合:} 敲门声(瞬间)发生在 Richards 太太走进餐厅(持续过程中)的中间。
        \end{itemize}

        \item \textbf{Just as 的增强效应 (The Power of "Just as")}
        \begin{itemize}
            \item \textbf{Just as} 的意思是“正当……那一刻”。它要求从句展现一个“动态的过程”。
            \item 如果用一般过去时 (\textit{Just as she entered...}),语感上更强调两个动作先后发生;而用进行时 (\textit{was entering}),则强调敲门声响起来时,她人正在门口,脚可能还没落地,极大地增强了戏剧冲突感。
        \end{itemize}

        \item \textbf{时态对比模型 (Temporal Model)}
        \begin{itemize}
            \item \textbf{进行时 = 过程 (Process):} \textit{She was entering...} (走进去的动作正在发生)
            \item \textbf{一般时 = 结果 (Result):} \textit{There was a knock...} (响起了敲门声)
        \end{itemize}
    \end{enumerate}
\end{multicols}

\wsitem{Intend vs. Want vs. Plan}
\begin{multicols}{1}
    \begin{enumerate}
        \item \textbf{Want to: 原始欲望 (Desire)}
        \begin{itemize}
            \item \textbf{定义:} 最通用的词,表达内心的一种愿望或冲动,但不一定有实际行动。
            \item \textbf{语感:} 比较感性、口语化。可能只是“想一想”,没有具体的步骤。
            
            \es{例句: I want to visit the Caribbean someday. (我想去加勒比海——只是个愿望。)}
        \end{itemize}

        \item \textbf{Intend to: 正式意图 (Formal Intention)}
        \begin{itemize}
            \item \textbf{定义:} 比 want 正式,强调心中已经形成了“目标”或“打算”。
            \item \textbf{语感:} 比较理性、正式。它暗示你已经把这个想法当成一个认知的决定,但还没有进入细节安排。
            
            \es{例句: We intend to find out the truth. (我们打算查明真相——这是一个明确的目标。)}
        \end{itemize}

        \item \textbf{Plan to: 实际规划 (Detailed Arrangement)}
        \begin{itemize}
            \item \textbf{定义:} 确定性最高。它意味着你已经考虑了“如何做”,可能已经安排了时间、准备了工具。
            \item \textbf{语感:} 强调行动力和具体的步骤。
            
            \es{例句: They plan to row across the sea tomorrow. (他们计划明天划船横渡大海——时间工具都定好了。)}
        \end{itemize}
    \end{enumerate}
\end{multicols}

\wsitem{find out}

\begin{multicols}{1}
    \begin{enumerate}
    \item \textbf{核心语义:发现真相 (Discovery of Truth)}
    \begin{itemize}
        \item find  out 强调的是通过“观察、询问、研究或调查”而得知原本不知道的信息。
        \item 它后面经常接 whether,  if,  what,  when 等引导的从句。
        \item 在本句中,她不知道这件衣服是否舒服,所以需要通过“试穿”这个行为来“查明”这个信息。
    \end{itemize}

    \item \textbf{词义辨析:find  out  vs.  find}
    \begin{itemize}
        \item \textbf{find  (偶然发现/寻回):} 强调找到具体的东西或人,往往是无意中看到或找回丢失之物。
        \begin{itemize}
            \es{例:She  found  a  gold  coin  on  the  beach. (她在沙滩上发现了一枚金币。)} 
        \end{itemize}
        \item \textbf{find  out  (查明/得知):} 强调获取“抽象的事实”或“隐藏的秘密”。
        \begin{itemize}
            \es{例:I  need  to  find  out  when  the  train  leaves. (我需要查一下火车什么时候开。)} 
        \end{itemize}
    \end{itemize}

    \item \textbf{常见场景对比}
    \begin{itemize}
        \item find  the  keys  (找到钥匙 - 具体的物)
        \item find  out  the  reason  (找出原因 - 抽象的信息)
    \end{itemize}
    \end{enumerate}
\end{multicols}


\wsitem{Consist of ...}
\begin{multicols}{1}
    \begin{enumerate}
        \item \textbf{核心定义与逻辑方向 (Definition and Logic)}
        \begin{itemize}
            \item 意为“由...组成”或“由...构成”。
            \item \textbf{逻辑路径:} 整体 + consist of + 部分 (A consist of B, C, and D)。
            
            \es{例如:A week consists of seven days. (一周由七天组成。)} 
        \end{itemize}

        \item \textbf{三大关键语法限制 (Grammar Constraints)}
        \begin{itemize}
            \item \textbf{无被动语态:} 这是最容易犯错的地方。consist of 本身就含有被动意味,
            \item 所以绝对不能说 \textit{is consisted of}。
            \item \textbf{无进行时:} 通常用于描述事实状态,因此不使用 \textit{is consisting of}。
            \item \textbf{主谓一致:} 主语是单数时,记得加 -s (consists of)。
        \end{itemize}

        \item \textbf{近义词逻辑辨析 (Synonym Comparison)}
        \begin{itemize}
            \item \textbf{Consist of:} 强调列出所有组成部分,逻辑最直接。
            \item \textbf{Be made up of:} 含义相同,但可以使用被动语态。
            \item \textbf{Include:} 强调“包含”,通常只列出部分成员,而非全部。
            \item \textbf{Comprise:} 较为正式,既可以表示“整体包含部分”,也可以表示“部分构成整体”。
        \end{itemize}

        \item \textbf{易混短语扩展:Consist in}
        
        虽然只差一个介词,但逻辑完全不同:
        \begin{itemize}
            \item Consist of: 由...组成(事实/成分)。
            
            \es{The team consists of ten members.}

            \item Consist in: 在于...(本质/定义)。
            
            \es{Happiness consists in helping others. (幸福在于助人。)}
        \end{itemize}
    \end{enumerate}
\end{multicols}

\wsitem{Close vs. Slam}
\begin{multicols}{1}
    \begin{enumerate}
        \item \textbf{动作本质与力度 (Nature of Action)}
        \begin{itemize}
            \item \textbf{Close:} 是一个中性词,指代任何“关闭”的动作。它只强调结果(门关上了),不涉及过程的激烈程度。
            \item \textbf{Slam:} 强调“砰地关上”。这个动作通常伴随着巨大的力量和极快的速度。
        \end{itemize}

        \item \textbf{声音与物理效果 (Sound and Effect)}
        \begin{itemize}
            \item \textbf{Close:} 可能是无声的,也可能是轻微的咔哒声。
            \item \textbf{Slam:} 必然伴随着巨大的噪音(a loud bang)。由于用力过猛,甚至可能引起墙壁或地板的震动。
        \end{itemize}

        \item \textbf{情绪逻辑与隐含语境 (Emotional Context)}
        \begin{itemize}
            \item \textbf{Close:} 通常是日常、礼貌或平静的行为。
            \item \textbf{Slam:} 往往暗示强烈的负面情绪。
            \begin{itemize}
                \item \textbf{Anger:} 生气地甩门而出。
                \item \textbf{Impatience:} 不耐烦地摔上抽屉。
                \item \textbf{Emergency:} 紧急情况下用力关上舱门。
            \end{itemize}
        \end{itemize}

        \item \textbf{常见搭配对比 (Common Collocations)}
        \begin{itemize}
            \item \textbf{Close:} \textit{Close the door quietly} (轻轻关门), \textit{close an account} (销户), \textit{close a deal} (达成交易)。
            \item \textbf{Slam:} \textit{Slam the door in someone's face} (当面把门摔上——极度无礼), \textit{slam the brakes on} (急刹车)。
        \end{itemize}
    \end{enumerate}
\end{multicols}

\wsitem{be impatient to ...}
\begin{multicols}{1}
    \begin{enumerate}
        \item \textbf{核心词义与语感 (Core Meaning)}
        \begin{itemize}
            \item impatient 是由前缀 im- (不) + patient (有耐心的) 构成。
            \item 搭配 to do sth. 时,它表示“急于做某事”或“迫不及待要做某事”。
            \item 语感逻辑:这种“急”通常源于对现状的不满,或是对目标达成的强烈渴望,带有一种时间上的紧迫感。
        \end{itemize}

        \item \textbf{语法结构对比 (Structural Comparison)}
        \begin{itemize}
            \item \textbf{be impatient to do sth. (重点):} 后接不定式,强调动作的倾向。
            \begin{itemize}
                \item \textit{例句:} The survivors were impatient to leave the tiny island. 
                \item (幸存者们迫不及待地想离开那个小岛。)
            \end{itemize}
            \item \textbf{be impatient for sth.:} 后接名词,强调对某个结果或事物的渴望。
            \begin{itemize}
                \item \textit{例句:} They were impatient for the arrival of the rescue boat. 
                \item (他们焦急地等待救援船的到来。)
            \end{itemize}
            \item \textbf{be impatient with sb./sth.:} 强调对某人或某事感到“不耐烦”。
            \begin{itemize}
                \item \textit{例句:} Don't be impatient with the damaged boat; it still floats. 
                \item (别对这艘破船没耐心,它还能漂呢。)
            \end{itemize}
        \end{itemize}

        \item \textbf{同义词辨析:迫切程度的阶梯 (Intensity Scale)}
        \begin{itemize}
            \item want to do: 表达简单的意愿(我想做)。
            \item be eager to do: 表达热切的期望(我很想做)。
            \item \textbf{be impatient to do:} 表达一种“等不及了”的焦躁感(我一秒都不想等了)。
        \end{itemize}

        \item \textbf{逻辑缝合:实战应用 (Contextual Usage)}
        \begin{itemize}
            \item 结合课文背景:Spending five days on the coral island, the men grew \textbf{impatient to return to Miami.}
            \item 这里使用现在分词做状语 (spending),完美衔接了由于时间流逝导致的心理变化。
        \end{itemize}
    \end{enumerate}
\end{multicols}

\wsitem{Not wanting to frighten the poor man...}
\begin{multicols}{1}

    在英语语法中,Not 和 No 的分工非常明确。在这个分词短语中,必须使用 Not,因为这里涉及的是对动作(动词)的否定,而不是对名词的否定。

    \begin{enumerate}
        \item \textbf{修饰对象的区别 (Part of Speech)}
        \begin{itemize}
            \item \textbf{Not:} 是副词 (Adverb),专门用来否定\textbf{动词、形容词或副词}。
            \item \textbf{No:} 是限定词 (Determiner) 或形容词,专门用来修饰\textbf{名词}。
            \item 在 \textit{Not wanting...} 中,wanting 是动词 want 的现在分词形式,因此必须用副词 not 来否定它。
        \end{itemize}

        \item \textbf{非谓语动词的否定规则 (Negating Participles)}
        \begin{itemize}
            \item 英语中所有非谓语形式(不定式 to do、分词 doing/done、动名词 doing)的否定,统一要求在前面直接加 \textbf{not}。
            
            \es{Not wanting to...} (不想...)
            \es{Not knowing what to do...} (不知道该做什么...)
            \es{Not having repaired the boat...} (还没修好船...)
        \end{itemize}

        \item \textbf{逻辑意义对比 (Logic Comparison)}
        \begin{itemize}
            \item \textbf{Not wanting (动作否定):} 强调“不想、不愿”这个心理意愿。
            \item \textbf{No + 名词 (存在否定):} 强调“没有...”。
            
            \es{No food, no beer.} (没有食物,没有啤酒。)
        \end{itemize}

        \item \textbf{结构转换 (Sentence Transformation)}
        \begin{itemize}
            \item 这个分词短语可以还原为一个原因状语从句:
            
            \es{Because he did \textbf{not} want to frighten the poor man...}

            \item 当我们把从句简化成分词短语时,助动词 did 消失,但否定词 \textbf{not} 必须保留。
        \end{itemize}
    \end{enumerate}
\end{multicols}

\wsitem{if ever ...}
\begin{multicols}{1}
    \begin{enumerate}
        \item \textbf{核心含义与逻辑 (Core Meaning)}
        \begin{itemize}
            \item 意为“如果曾经有过”、“即便有的话”。
            \item 它通常暗示某种情况\textbf{极少发生},或者说话人对其发生的可能性持怀疑态度。
        \end{itemize}

        \item \textbf{常见用法结构 (Usage Structures)}
        \begin{itemize}
            \item \textbf{Seldom, if ever...} (极少,如果曾经有过的话): 这是最经典的固定搭配,用来强调频率极低。
            \begin{itemize}
                \item 例:He seldom, if ever, goes to the cinema. (他极少去电影院,即便去过,次数也屈指可数。)
            \end{itemize}
            \item \textbf{If ever there was...} (如果真的有过...): 用于强调某个特质达到了顶峰。
            \begin{itemize}
                \item 例:If ever there was a paradise, this coral island is it. (如果这世上真的有天堂,这座珊瑚岛就是。)
            \end{itemize}
        \end{itemize}

        \item \textbf{逻辑辨析:If ever vs. If any}
        \begin{itemize}
            \item \textbf{If ever:} 针对 \textbf{时间或频率} (Time/Frequency)。
            \begin{itemize}
                \item 例:He rarely speaks, if ever. (他很少说话。)
            \end{itemize}
            \item \textbf{If any:} 针对 \textbf{数量或存在} (Quantity/Existence)。
            \begin{itemize}
                \item 例:There is little hope, if any. (希望渺茫,如果有的话。)
            \end{itemize}
        \end{itemize}
    \end{enumerate}
\end{multicols}

\wsitem{现在分词作状语}
\begin{multicols}{1}
    在英语语法中,现在分词($V\text{-}ing$)作状语是一个非常高效的表达方式,它能把两个简单的句子合并为一个结构紧凑的高级句子。
    
    它的核心逻辑是:分词动作与主句动作同时发生,或者分词动作是主句背景的一部分。
    \begin{enumerate}
        \item 现在分词作状语的四个主要功能
        
        \begin{itemize}
            \item \textbf{时间状语 (Time):} 表示动作发生的背景时间。
            
            \es{例句: Arriving at the tiny island, they let out a cry of joy.当到达小岛时,他们发出了欢呼。}
            
            \item \textbf{原因状语 (Reason):} 解释主句动作发生的原因。
            
            \es{例句: Being impatient to leave, they rowed across the sea in a hurry.(因为急于离开,他们匆忙划船横渡大海。}
            
            \item \textbf{伴随状语 (Accompanying):} 描述与主语动作同时进行的次要动作。
            
            \es{例句:They spent five days on the island, starving to death.(他们在岛上度过了五天,忍受着极度的饥饿。)}
            
            \item \textbf{结果状语 (Result):} 表示自然而然产生的结果。
            
            \es{例句:The boat struck a coral reef, causing a dreadful mess.(船撞上了珊瑚礁,导致了一场可怕的混乱。)}
        \end{itemize}
    \end{enumerate}
\end{multicols}

\wsitem{let out a cry}

\begin{multicols}{1}
    "Let out a cry" 是一个非常生动的情境描写短语。相比于简单的 "cry" 或 "shout",它更强调声音从体内“迸发”或“释放”出来的那一瞬间,通常带有突发性和强烈的情感冲击。

    \begin{enumerate}
        \item 核心定义与语感
        \begin{itemize}
            \item Let out: 原意为“释放、发出”。它暗示这种声音之前可能被憋着,或者是因为突然的刺激(惊吓、疼痛、狂喜)而瞬间爆裂。
            \item Cry: 在这里不是指“哭泣”,而是指“叫喊、大喊”。
        \end{itemize}

        语感对比:

        \es{He cried. (他喊了一声 —— 比较平淡)}

        \es{He let out a cry. (他爆发出一声大喊 —— 更有画面感,仿佛你能听到那一声尖叫或惊呼。)}
        
        \item 常见的“声音释放”组合
        
        这个结构非常灵活,你可以通过更换名词来精准描述各种情绪:

    
        \begin{itemize}
            \item \textbf{Let out a cry of pain (痛呼)}
            \begin{itemize}
                \item \textbf{对应背景:} 身体受损、剧烈疼痛。
                \item \textbf{逻辑释义:} 侧重于生理上的本能反应,如被珊瑚礁割伤脚时。
            \end{itemize}
            
            \item \textbf{Let out a cry of surprise (惊呼)}
            \begin{itemize}
                \item \textbf{对应背景:} 意外发现、面对突发状况。
                \item \textbf{逻辑释义:} 侧重于心理上的“措手不及”,不一定带恐惧感。
            \end{itemize}
            
            \item \textbf{Let out a cry of joy (欢呼)}
            \begin{itemize}
                \item \textbf{对应背景:} 极度兴奋、成功、获救。
                \item \textbf{逻辑释义:} 情感的正面爆发。
                
                \es{例:Arriving at the shore, they let out a cry of joy.}
            \end{itemize}
            
            \item \textbf{Let out a cry of horror (惊恐)}
            \begin{itemize}
                \item \textbf{对应背景:} 极度恐惧、目睹可怕的场面。
                \item \textbf{逻辑释义:} 伴随生理上的战栗。
                
                \es{例:看到钞票烧成灰时,Richards 太太的表现。}
            \end{itemize}
            
            \item \textbf{Let out a sigh of relief (松了一口气)}
            \begin{itemize}
                \item \textbf{对应背景:} 虚惊一场、困难解决。
                \item \textbf{逻辑释义:} 情绪由紧绷转为平复,动作是“长舒一口气”。
            \end{itemize}
        \end{itemize}

        \item 句法总结

        \begin{itemize}
            \item \textbf{基本句式结构 (Basic Structure)}
            \begin{itemize}
                \item \textbf{S + let out + a/an + [Noun for Sound]}
                
                \es{例句:The boat hit the reef, and he let out a roar.}
            \end{itemize}
            
            \item \textbf{核心逻辑意义 (Logic Meaning)}
            \begin{itemize}
                \item \textbf{To suddenly produce a sound (often emotional)}
                \item \textbf{释义:} 强调声音是“突然”且“不由自主”地发出的,通常与强烈的情绪(如惊恐、狂喜、愤怒)挂钩。
            \end{itemize}
            
            \item \textbf{常见声音变体 (Common Variations)}
            \begin{itemize}
                \item \textbf{let out a roar / scream:} 发出咆哮 / 尖叫。
                \item \textbf{let out a groan / laugh:} 发出呻吟 / 爆发大笑。
                \item \textbf{let out a gasp:} 发出倒吸气声(通常由于惊讶或恐惧)。
            \end{itemize}
            
            \item \textbf{时态注意事项 (Tense Note)}
            \begin{itemize}
                \item \textbf{Let 的不规则变形:Let - Let - Let}
                \item \textbf{提醒:} 即使在描述过去发生的故事中,\textbf{let} 也不需要加 -ed。但在主语是单数第三人称且为一般现在时时,需加 -s (\textbf{lets out})。
            \end{itemize}
        \end{itemize}
    \end{enumerate}
\end{multicols}



\newpage