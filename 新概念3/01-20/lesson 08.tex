\section{Lesson 8 A famous monastery}

\begin{paracol}{2}
    
The Great St Bernard \nw{Pass} \ns{connects Switzerland to} Italy.

\switchcolumn

\nwe{pass}{pæs}{v. 使经过;通过;超过;转交;转变;消逝,度过;结束;陈述,宣布;被当作;排泄;n. 合格;通行证;传球;关口;飞跃;阶段;}
\nse{connect ... to ...}{}{将……连接到……}

\switchcolumn*


At 2,473 metres, it is the highest \ns{mountain pass} in Europe. 

\switchcolumn

\nse{mountain pass}{}{山口;山道;}

\switchcolumn*

The famous \nw{monastery} of St. Bernard, which was founded in the eleventh century, \ns{lies about a mile away}. 

\switchcolumn

\nwe{monastery}{ˈmɑnəsteri}{n. 修道院,寺院;[复数]全体僧侣;}
\nse{lie ... miles/meters/kilometers/... away}{}{坐落在...英里/米/公里之外}

\switchcolumn*

\ns{For hundreds of years}, St. Bernard dogs have saved the lives of travellers \ns{crossing the dangerous Pass}. 

\switchcolumn

\nse{for hundreds/thousands/millions of years}{}{几百/千/百万年来}
\nse{cross the pass}{}{横穿山口}

\switchcolumn*

These friendly dogs, which were first brought from Asia, were used as \nw{watchdogs} even in Roman times. 

\switchcolumn

\nwe{watchdog}{ˈwɑːtʃdɔːɡ}{n. 看门狗,看家狗;监视者;}

\switchcolumn*

\newsentence{Now that a tunnel has been built through the mountains}, the Pass is less dangerous, but each year, \newsentence{the dogs are still sent out into the snow whenever a traveller is \ns{in difficulty}}. 

\switchcolumn

\nse{in difficulty}{}{处于困境,在困难中;}

\switchcolumn*

Despite the new tunnel, there are still a few people who \nw{rashly} attempt to cross the Pass \nw{on foot}.

\switchcolumn

\nwe{rashly}{ˈræʃli}{adv. 鲁莽地,轻率地,仓促地;冒;愣;贸然;}
\nse{on foot}{}{步行,进行起来,在筹划中;徒步;徒;}

\switchcolumn*

During the summer months, the monastery is very busy, for it is visited by thousands of people who cross the Pass \nw{in cars}. 

\switchcolumn

\switchcolumn*

As there are so many people \nw{about}, the dogs have to be kept in a special \nw{enclosure}. 

\switchcolumn

\nwe{about}{əˈbaʊt}{prep. 关于,对于;为了;忙于;在…地方,在各处;围绕;adv. 大约,左右;几乎;到处;闲着;附近;掉头;adj. 现成的;即将发生的;}
\nwe{enclosure}{ɪnˈkloʊʒər}{n. 围场;圈地;(信)附件;}

\switchcolumn*

In winter, however, life at the monastery is quite different. 

\switchcolumn

\switchcolumn*

The temperature drops to -30 $^\circ C$ and \newsentence{very few} people attempt to cross the Pass. 

\switchcolumn

\switchcolumn*

\newsentence{The \nw{monks} \ns{prefer winter to summer} for they have more \nw{privacy}. }

\switchcolumn

\nwe{monk}{mʌŋk}{n. 修道士,僧侣;}
\nwe{privacy}{ˈpraɪvəsi}{n. 隐私,秘密;隐居;私事;不受公众干扰的状态;}
\nse{prefer ... to ...}{}{比起(后者),更喜欢(前者)}

\switchcolumn*

The dogs have \nw{greater freedom}, too, for they are allowed to wander outside their enclosure. 

\switchcolumn

\nse{greater freedom}{}{更大自由}

\switchcolumn*

The only \nw{regular visitors} to the monastery in winter are \nw{parties of} skiers who go there at Christmas and Easter. 

\switchcolumn

\nse{regular visitor}{}{常客;固定访问者;}
\nse{parties of ...}{}{一批批...}

\switchcolumn*

These young people, who love the peace of the mountains, always \ns{receive a \ns{warm welcome}} at St Bernard's monastery.

\switchcolumn

\nse{receive welcome}{}{受到欢迎}
\nse{warm welcome}{}{热烈的欢迎}

\switchcolumn*


\end{paracol}

\worddifference
\wsitem{found vs. build vs. erect vs. establish}

\begin{multicols}{1}
\begin{enumerate}
    \item Establish (最正式/最广泛)
    \begin{itemize}
        \item 含义:指使某物在稳固的基础上开始存在。它不仅指建筑,更常指制度、法律、关系、名声或外交关系。
        \item 侧重:强调“确立”和“长期存在”。
        
        \es{例子:Establish a diplomatic relationship (建立外交关系)。}
    \end{itemize}
    \item Found (强调“基石”与“初创”)
    \begin{itemize}
        \item 含义:指为组织或建筑奠定基础。它带有明显的历史起源色彩,常用于城市、学校、修道院等。
        \item 侧重:强调“创始人”和“起始点”。
        
        \es{例子:The university was founded in 1890 (这所大学创立于1890年)。}
    \end{itemize}
    \item Build (最通用/物理性)
    \begin{itemize}
        \item 含义:指通过组合材料(木材、砖块、代码等)来创造某物。
        \item 侧重:强调施工的过程。
        
        \es{例子:Build a house / Build a website (盖房子 / 建网站)。}
    \end{itemize}
    \item Erect (强调“垂直”与“挺拔”)
    \begin{itemize}
        \item 含义:字面意思是“使竖立”。指建造高耸、垂直的物体。
        \item 侧重:强调物理上的高度或正式的落成。
        
        \es{例子:Erect a flagpole / Erect a monument (竖起旗杆 / 树立纪念碑)。}
    \end{itemize}
\end{enumerate}
\end{multicols}

\grammarpoints

\wsitem{cross the pass}
\begin{multicols}{1}
    在描述“过山”时,除了 cross,英语中还有其他地道的表达方式:

    \begin{enumerate}
        \item Go over the Pass: 最通俗的表达,指翻越动作。
        \item Traverse the Pass: 比较正式或文学化的词汇,指“横越”。
        \item Navigate the Pass: 强调在复杂地形中“寻找路径”或“导航”。
        \item Make one's way across: 强调翻越过程的艰难与缓慢。
    \end{enumerate}
\end{multicols}

\wsitem{prefer ... to ...}
\begin{multicols}{1}

    "Prefer ... to ..." 是表达“比起(后者),更喜欢(前者)”最标准的句型,它是表达个人偏好(Preferences)的核心词汇。 

    \begin{enumerate}
        \item 语法结构
        
        这个短语的重点在于:中间必须用介词 "to",而不是 "than"。

        \begin{itemize}
            \item \textbf{接名词: Prefer A to B}
            \item "I prefer coffee to tea." (比起茶,我更喜欢咖啡。)
            \item \textbf{接动名词 (-ing): Prefer doing A to doing B}
            \item "In 2025, many people prefer working from home to commuting."(在 2025 年,许多人比起通勤,更喜欢居家办公。)
        \end{itemize}
        \item 三种同义句型的对比
        
        如果你想表达相同的意思,有三种常用的变化方式,注意它们连接词的不同:

        \begin{itemize}
            \item \textbf{Prefer ... to ...}
            \item I prefer swimming to running.
            \item \textbf{Would rather ... than ...}
            \item I would rather swim than run
            \item \textbf{Like ... better than ...}
            \item I like swimming better than running.
        \end{itemize}
        \item 高频使用场景
        \begin{itemize}
            \item 数字偏好:"Do you prefer digital notes to paper ones?" (你喜欢电子笔记还是纸质笔记?)
            \item 内容偏好:"I prefer short videos to long articles." (比起长文章,我更喜欢短视频。)
            \item AI 交互:"Most users prefer talking to AI to searching manually." (大多数用户比起手动搜索,更喜欢与 AI 对话。) 
        \end{itemize}
        \item 总结: 只要记住 "Prefer A to B" 且 A 和 B 的词性要一致(都是名词或都是 -ing),你就能准确表达你的偏好。
    \end{enumerate}
\end{multicols}

\wsitem{now that ...}
\begin{multicols}{1}
    在语法连接中,“now that” 是一个非常高效的从属连词(Subordinating Conjunction),它将“已知的事实”与“随之而来的结果或情况”连接起来。

    它的中文意思通常翻译为:“既然”、“由于” 或 “既然现在已经……”。

    “Now that” 引导的是一个原因状语从句。它的特殊之处在于,它所连接的原因必须是此时此刻已经发生或变成现实的情况。

    \begin{enumerate}

        \item 逻辑解析:因果的“连接”
        \begin{itemize}
            \item 逻辑结构: Now that + 事实 (原因) , + 结果/决定
            \item 连接点: 它把“时间上的现在”和“逻辑上的原因”无缝对接。
        \end{itemize}

        \item 结合你之前的句子
        
        \es{"Now that we are near the monastery, we should go and visit it." (既然我们已经到了修道院附近,我们应该去参观一下。)}

        \item 语法小贴士
        \begin{itemize}
            \item that 可省略: 在非正式口语中,常简写为 "Now..."。
            
            \es{Example: Now (that) you mention it, I do remember him. (既然你提到了,我确实想起他了。)}

            \item 时态习惯: 后面常接现在时(Present Tense)或现在完成时(Present Perfect),因为连接的是“现在的状态”。
        \end{itemize}

        \item 总结对比
        \begin{itemize}
            \item Because:侧重单纯的因果(不一定强调时间)。
            \item Now that:侧重“既然情况已经变了”这一前提下的新行动。
        \end{itemize}
    \end{enumerate}
    
\end{multicols}

\wsitem{... whenever ...}

\begin{multicols}{1}

 whenever 扮演着从属连词的角色,引导一个时间状语从句。

它的核心作用是表达“无条件的频率”,可以拆解为以下三个层面来解析:

\begin{enumerate}
    \item 逻辑含义:连接“时间”与“条件”
    
    whenever 的意思是 “每当……” 或 “无论什么时候”。 在这个句子里,它不仅交代了救援犬出发的时间,还隐含了一个触发条件:

    \begin{itemize}
        \item 动作: 派出救援犬 (the dogs are sent out)
        \item 触发点: 旅客遇到困难 (a traveller is in difficulty)
        \item 解析: 只要“旅客遇险”这个情况发生(无论发生多少次,无论在白昼还是黑夜),“派狗救援”这个动作就会随之发生。
    \end{itemize}

    \item 语法功能:引导时间状语从句
    
    在长难句分析中,whenever 引导的部分是对主句动作发生频率的补充:

    \begin{itemize}
        \item 主句分句: ...the dogs are still sent out into the snow...
        \item whenever 从句: ...whenever a traveller is in difficulty.
    \end{itemize}

    它与 when 的区别:

    \begin{itemize}
        \item 如果用 when:通常指某一次具体的、特定的时间点。
        \item 使用 whenever:强调“反复性”和“必然性”。它在告诉读者,这种救援行动是一个持续多年的惯例,只要条件满足,结果必然发生。
    \end{itemize}
\end{enumerate}

\end{multicols}

\wsitem{For, As, Since, Because}

\begin{multicols}{1}
    \begin{enumerate}
    \item \textbf{$Because$ —— 最强因果与核心重点}
    \begin{itemize}
        \item \textbf{语义逻辑}:语气最强,指明直接的因果必然联系。用于回答 Why 引导的提问,提供对方未知的新信息。
        \item \textbf{语法特征}:从属连词,可置于句首或句中。
        
        \es{示例:He failed the exam because he did not study hard.}
    \end{itemize}

    \item \textbf{$Since$ —— “既然”之意的已知事实}
    \begin{itemize}
        \item \textbf{语义逻辑}:语气较弱,侧重引导显而易见或双方已知的事实,逻辑重点在于主句的结果。
        \item \textbf{语法特征}:从属连词,常置于句首。
        
        \es{示例:Since we have no money, we cannot buy that car.}
    \end{itemize}

    \item \textbf{$As$ —— 随意的背景铺垫}
    \begin{itemize}
        \item \textbf{语义逻辑}:语气最弱,将原因视为背景信息或附带说明,常用于口语。
        \item \textbf{语法特征}:从属连词,功能与 Since 类似但更口语化。
        
        \es{示例:As it was getting late, I decided to leave.}
    \end{itemize}

    \item \textbf{$For$ —— 文学性的补充推论}
    \begin{itemize}
        \item \textbf{语义逻辑}:并列连词,不提供直接原因,而是为前文陈述提供逻辑依据或证据,具文学色彩。
        \item \textbf{语法特征}:\textbf{禁止置于句首};前面必须使用逗号隔开。
        
        \es{示例:I believed her, for she had never lied to me before.}
    \end{itemize}
\end{enumerate}

\end{multicols}

\wsitem{Very few vs. A few}
\begin{multicols}{1}
    在英语语法中,Few 和 A few 的区别不仅仅是多了一个单词,而是代表了完全相反的逻辑态度(Affirmative vs. Negative)。

    \begin{enumerate}
        \item 核心差异:肯定 vs. 否定
        
        这是理解这两个词最关键的切入点:

        \begin{itemize}
            \item A few (肯定含义): 意为“有一些”、“几个”。它强调“虽然不多,但还是有的”。其逻辑重心在于“存在”。
            \item Few / Very few (否定含义): 意为“几乎没有”、“极少”。它强调“太少了,几乎可以忽略不计”。其逻辑重心在于“缺失”。
        \end{itemize}

        \item 修饰语的“站队”
        
        为了加强语气,它们通常会配合不同的修饰语:

        \begin{itemize}
            \item Only a few: 依然是肯定含义,但语气变得委婉,表示“仅仅只有几个”。
            \item Very few / Extremely few: 进一步强化否定含义,表示“少之又少”,几乎为零。
        \end{itemize}

        \item 延伸:Little vs. A little
        
        这个逻辑同样适用于不可数名词:

        \begin{itemize}
            \item A little (肯定): I have a little water. (我还有点水,够喝。)
            \item Little (否定): I have little water. (我几乎没水了,很危险。)
        \end{itemize}
    \end{enumerate}
\end{multicols}

\wsitem{Drop vs. Fall vs. Go down vs. Plummet / Plunge}
\begin{multicols}{1}
    在描述气温、价格或水平的“下降”时,drop 是一个非常地道且高频的词汇,但根据想表达的“下降方式”(是缓慢下降还是突然暴跌),还有几个其他的动词可以选择。

    \begin{enumerate}
        \item Drop (最常用)
        
        Drop 通常暗示一种明显或突然的下降。在描述天气突变(比如寒潮来袭)时非常合适。

        \begin{itemize}
            \item 特点: 既可以作动词,也可以作名词(a drop in temperature)。
            
            \es{例句: The temperature dropped to -30°C overnight. (气温在夜间降到了零下30度。)}
            
        \end{itemize}

        \item Fall (通用)
        
        Fall 是 drop 的近义词,语气稍微平缓一点,是一个中性词。

        \es{例句: Temperatures are expected to fall below freezing tonight. (预计今晚气温将降至冰点以下。)}

        \item Go down (口语化)
        
        在日常对话中,人们最常用 go down。

        \es{例句: The temperature is going down fast. (气温正在快速下降。)}

        \item Plummet / Plunge (剧烈下降)
        
        如果想表达气温“断崖式下跌”或“暴降”,这两个词更有画面感。

        \es{例句: As the cold front arrived, the temperature plummeted. (随着冷锋到达,气温骤降。)}
    \end{enumerate}

\end{multicols}

\newpage