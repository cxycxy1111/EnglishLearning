\section{Lesson 20 Pioneer pilots}
\begin{paracol}{2}

In 1908 \nw{Lord} Northcliffe offered a prize of \$1,000 to the first man who would fly across the English Channel. 

\switchcolumn

\chinesetext{1908年,诺斯克利夫勋爵拿出1,000英镑,作为对第一个飞越英吉利海峡的人的奖励。}
\nwe{pioneer}{ˌpaɪəˈnɪr}{n. 先驱;开拓者;v. 开创;}
\nwe{lord}{lɔːrd}{n. 主,上帝;勋爵(英国贵族的称号);阁下,大人;(中世纪欧洲的)领主;v. 举止霸道(或逞威风);}

\switchcolumn*

\newsentence{Over a year passed before the first attempt was made. }

\switchcolumn

\chinesetext{然而一年多过去了才有人出来尝试。}

\switchcolumn*

On July 19th, 1909, in the early morning, Hubert Latham took off from the French coast in his plane the 'Antoinette IV.' 

\switchcolumn

\chinesetext{1909年7月19日凌晨,休伯特.莱瑟姆驾驶“安特瓦特4号”飞机从法国海岸起飞。}

\switchcolumn*

He had travelled only seven miles across the Channel when his engine failed and he was forced to land on sea. 

\switchcolumn

\chinesetext{但他只在海峡上空飞行7英里,引擎就发生了故障,他只好降落在海面上。}

\switchcolumn*

The 'Antoinette' floated on the water until Latham was picked up by a ship.

\switchcolumn

\chinesetext{“安特瓦特”号飞机在海上漂浮,后来有船经过,莱瑟姆方才获救。}

\switchcolumn*

Two days later, Louis Bleriot arrived near Calais with a plane called 'No. XI'. 

\switchcolumn

\chinesetext{两天之后,路易斯.布莱里奥驾驶一名为“11号”的飞机来到加来附近。}

\switchcolumn*

Bleriot had been making planes since 1905 and this was his latest model. 

\switchcolumn

\chinesetext{布莱里奥从1905年起便开始研制飞机,“11号”飞机是他制作的最新型号。}

\switchcolumn*

A week before, he had completed a successful \nw{overland} flight during which he covered twenty-six miles. 

\switchcolumn

\chinesetext{一周以前,他曾成功地进行了一次26英里的陆上飞行。}
\nwe{overland}{ˈoʊvərlænd}{adj. 陆上的;经由陆路的;横跨大陆的;}

\switchcolumn*

Latham, however, did \ns{not give up easily}. 

\switchcolumn

\chinesetext{但是莱瑟姆不肯轻易罢休。}
\nse{not give up easily}{}{不轻易作罢}

\switchcolumn*

He, too, arrived near Calais on the same day with a new  'Antoinette'. 

\switchcolumn

\chinesetext{同一天,他驾驶一架新的“安特瓦特”号飞机来到了加来附近。}

\switchcolumn*

\newsentence{It looked as if there would be an exciting race across the Channel.} 

\switchcolumn

\chinesetext{看来会有一场激烈的飞越英吉利海峡的竞争。}

\switchcolumn*

Both planes were going to take off on July 25th, but Latham failed to get up early enough.

\switchcolumn

\chinesetext{两天飞机都打算在7月25日起飞,但莱瑟姆那天起床晚了。}

\switchcolumn*

\newsentence{After making a short test flight at 4,15 a.m., Bleriot set off half an hour later. }

\switchcolumn

\chinesetext{布莱里奥凌晨4点15分作了一次短距离试飞,半小时后便正式出发了。}

\switchcolumn*

His great flight lasted thirty-seven minutes. 

\switchcolumn

\chinesetext{他这次伟大的飞行持续37分钟。}

\switchcolumn*

When he landed near Dover, the first person to greet him was a local policeman. 

\switchcolumn

\chinesetext{当他在多佛着陆后,第一个迎接他的是当地一名警察。}

\switchcolumn*

Latham made another attempt a week later and got within half a mile of Dover, but he was unlucky again. 

\switchcolumn

\chinesetext{莱瑟姆一周以后也作了一次尝试,飞到离多佛不到半英里的地方,但这次他又遭厄运。}

\switchcolumn*

His engine failed and he landed on the sea for the second time.

\switchcolumn

\chinesetext{他因引擎故障第二次降落在海面上。}

\switchcolumn*

\end{paracol}

\retellingpoints
\begin{multicols}{1}
    \begin{enumerate} 
        \item \textbf{The Challenge (1908 -- 1909):} 
        \begin{itemize} 
            \item Lord Northcliffe / prize of \$1,000. 
            \item Fly across the English Channel. 
            \item Over a year passed before the first attempt. 
        \end{itemize}

        \item \textbf{Latham's First Attempt (July 19th):}
        \begin{itemize}
            \item Hubert Latham / 'Antoinette IV'.
            \item Seven miles across / engine failed.
            \item Forced to land on sea / picked up by a ship.
        \end{itemize}

        \item \textbf{The Rivalry (Two days later):}
        \begin{itemize}
            \item Louis Bleriot / 'No. XI' (latest model).
            \item Bleriot's record: 26-mile overland flight.
            \item Latham's persistence: arrived with a new 'Antoinette'.
            \item An exciting race expected on July 25th.
        \end{itemize}

        \item \textbf{The Historic Flight (July 25th):}
        \begin{itemize}
            \item Latham failed to get up early.
            \item Bleriot: test flight at 4:15 a.m. $\rightarrow$ set off at 4:45 a.m.
            \item Duration: 37 minutes / landed near Dover.
            \item Greeted by a local policeman.
        \end{itemize}

        \item \textbf{Latham's Final Failure:}
        \begin{itemize}
            \item Another attempt a week later.
            \item Within half a mile of Dover.
            \item Engine failed / landed on sea for the second time.
        \end{itemize}
    \end{enumerate}
\end{multicols}

\grammarpoints
\wsitem{Attempt vs. Try}
\begin{multicols}{1}
    这两个词虽然都翻译为“尝试”,但在正式程度、成功的可能性以及词法习惯上有着非常明显的区别。

    \begin{enumerate}
        \item \textbf{基本属性对比 (Fundamental Comparison)}
        \begin{itemize}
            \item \textbf{Try (最常用、非正式)}
            \begin{itemize}
                \item \textbf{语感:} 生活化、口语化。强调“试一试”这个动作本身。
                \item \textbf{含义:} 既可以指尽力去做一件难事,也可以指为了看看结果而做的轻微尝试(如“试吃”、“试穿”)。
            \end{itemize}
            
            \item \textbf{Attempt (正式、书面语)}
            \begin{itemize}
                \item \textbf{语感:} 庄重、严肃。常用于报道、法律、科学或文学语境。
                \item \textbf{含义:} 强调“试图去完成一项具有挑战性或困难的任务”。它往往暗示**结果可能并不成功**。
            \end{itemize}
        \end{itemize}
        \item \textbf{核心逻辑点解析 (Core Logical Points)}
        \begin{itemize}
            \item \textbf{难度的差异 (Degree of Difficulty)}
            \begin{itemize}
                \item Try 可以是小事:"Try this soup." (尝尝这汤。)
                \item Attempt 通常是大事:"He \textbf{attempted to swim across the Channel."} (他试图横渡英吉利海峡。) —— 这里用 attempt 体现了任务的艰巨性。
            \end{itemize}
            
            \item \textbf{结果的暗示 (Implication of Outcome)}
            \begin{itemize}
                \item Attempt 经常与“失败”联系在一起。如果说 "The \textbf{first attempt was made,"} 读者通常预感到后面会有第二次、第三次,或者这次失败了。
                \item Try 则更中性,侧重于“付出努力”的过程。
            \end{itemize}
            
            \item \textbf{词性变化的丰富度}
            \begin{itemize}
                \item Attempt 作为名词非常常用,如 "make an attempt"。
                \item Try 作为名词通常指“一次尝试”,如 "Give it a try." (口语中极其常见)。
            \end{itemize}
        \end{itemize}
        \item \textbf{语法搭配与惯用法 (Grammar and Collocations)}
        \begin{itemize}
            \item \textbf{Try doing vs. Try to do}
            \begin{itemize}
                \item Try to do: 努力尝试做某事(可能很难)。
                \item Try doing: 试着做某事(看看有没有效果,比如试着换个电池)。
                \item \textbf{注意:} Attempt 几乎只接 to do,不接 doing。
            \end{itemize}
            
            \item \textbf{典型短语搭配:}
            \begin{itemize}
                \item \textbf{Attempt:} In an attempt to... (为了试图……), Criminal attempt (犯罪未遂)。
                \item \textbf{Try:} Try your best (尽你最大努力), Trial and error (尝试与错误/试错法)。
            \end{itemize}
        \end{itemize}
    \end{enumerate}    
\end{multicols}

\wsitem{It looked as if there would be an exciting race across the Channel. }
\begin{multicols}{1}
    这句话使用了 "It looked as if..." 结构,配合虚拟语气或推测语气,生动的描绘了一场呼之欲出的竞争。

    \begin{enumerate}
        \item \textbf{句法结构总览 (Overall Syntactic Structure)}
        
        \begin{itemize}
            \item \textbf{It looked as if... (看起来似乎...)}
            \begin{itemize}
                \item It 是无人称代词,作形式主语。
                \item Looked 设定了过去的时间背景。
                \item As if 引导表语从句,表示一种基于当时观察得出的推断。
            \end{itemize}
            
            \item \textbf{There would be (将会有)}
            \begin{itemize}
                \item 这里的 would 是过去将来时。
                \item 因为主句是过去时 (looked),所以从句中描述“在那时看来即将发生的事”要用过去将来时。
            \end{itemize}
        \end{itemize}

        \item \textbf{核心逻辑点解析 (Core Logical Points)}
        \begin{itemize}
            \item \textbf{悬念的构建 (Building Suspense)}
            \begin{itemize}
                \item 这个句子出现在故事背景铺垫阶段。它暗示了当时不仅有一名飞行员,而是有多名竞争者(如 Bleriot 和 Latham)都在等待时机飞越海峡。
                \item “Exciting race” (激动人心的竞赛) 这个词赋予了这次技术尝试以竞技体育的紧张感。
            \end{itemize}
            
            \item \textbf{"As if" 的真实性判断}
            \begin{itemize}
                \item 在英语中,as if 后接 would 有时带有“未必成真”或“尚未发生”的色彩。在课文中,这种“竞赛”确实开始了,但过程却充满了意外(比如一方引擎故障坠海)。
            \end{itemize}
        \end{itemize}

        \item \textbf{高级词汇与表达进阶 (Advanced Vocabulary)}
        
        \begin{itemize}
            \item \textbf{Race across the Channel (横跨海峡的竞赛)}
            \begin{itemize}
                \item Across 强调从一岸到另一岸的横跨。
                \item 在这里不仅指物理上的飞越,也指谁能第一个获得“首飞成功”头衔的竞争。
            \end{itemize}
            
            \item \textbf{类似推测句型:}
            \begin{itemize}
                \item \textbf{It appeared that} (看上去似乎...)
                \item \textbf{The prospect of} an exciting race was looming. (一场激动人心竞赛的前景正在隐现。)
            \end{itemize}
        \end{itemize}
    \end{enumerate}
\end{multicols}

\grammarquestions
\wsitem{Over a year passed before the first attempt was made.为什么这么写,不是the first attempt was made over a year later?}
\begin{multicols}{1}
    这是一个非常棒的语法对比。虽然这两句话传达的意思基本相同,但在\textbf{叙事重心、逻辑连词的使用以及“地道感”}上,原句 "Over a year passed before..." 要高级得多。

    \begin{enumerate}
        \item \textbf{句法结构对比 (Syntactic Comparison)}
        \begin{itemize}
            \item \textbf{原句结构:Time + passed + before + Event}
            \begin{itemize}
                \item \textbf{主语:} Over a year (一年多时间)。
                \item \textbf{谓语:} passed (流逝了)。
                \item \textbf{状语从句:} before the first attempt was made (在第一次尝试做出之前)。
                \item \textbf{逻辑:} 这种结构把“时间”作为主角,强调了这段漫长的\textbf{等待过程}。
            \end{itemize}
            
            \item \textbf{你写的结构:Event + happened + Time later}
            \begin{itemize}
                \item \textbf{主语:} The first attempt (第一次尝试)。
                \item \textbf{谓语:} was made (被做出)。
                \item \textbf{状语:} over a year later (一年多以后)。
                \item \textbf{逻辑:} 这种结构是典型的“结果导向”,重心在“尝试”这个动作上,时间只是一个附属的标签。
            \end{itemize}
        \end{itemize}
        \item \textbf{为什么原句更高级? (Why the Original is Better?)}
        \begin{itemize}
            \item \textbf{强调“漫长感” (Emphasizing the Duration)}
            \begin{itemize}
                \item 使用 passed before 结构,读者的第一印象是“时间走得真慢”。它在心理上拉长了跨度,暗示在这期间可能发生了许多困难、准备或犹豫。
                \item 在文学叙事中,这通常用于引出一段\textbf{难产的}或\textbf{重大的}事件。
            \end{itemize}
            
            \item \textbf{戏剧性的铺垫 (Narrative Suspense)}
            \begin{itemize}
                \item “一年过去了……”(读者会问:然后呢?)“……才进行了第一次尝试。”
                \item 这种先交代背景、后抛出事件的写法,比直接说“一年后发生了某事”更具节奏感。
            \end{itemize}
            
            \item \textbf{"Before" 的地道用法}
            \begin{itemize}
                \item Before 在这里不仅是“在……之前”,它暗含了“\textbf{直到……才……}”的意思。
                \item \textbf{固定模式:} It was + [Time] + before... 或 [Time] + passed + before... 都是英语中表达“过了多久才发生某事”的标准地道句式。
            \end{itemize}
        \end{itemize}

        \item \textbf{类似用法的进阶模仿 (Advanced Imitation)}
        
        如果你想让你的描述更有质感,可以尝试模仿这个 "Time + before" 结构:

        \begin{itemize}
            \item \textbf{描述等待:}
            
            \es{\textbf{Many years passed before} his \textbf{life work} was finally recognized by the public.(许多年过去了,他的毕生心血才最终获得公众认可。)}
            
            \item \textbf{描述变化:}
            
            \es{\textbf{A few minutes passed before} he \textbf{made it clear} that he would \textbf{go to the police}.(过了几分钟,他才明确表示他会去报案。)}
        \end{itemize}
    \end{enumerate}
\end{multicols}

\wsitem{He had travelled only seven miles across the Channel when his engine failed and he was forced to land on sea.为什么这么写?}
\begin{multicols}{1}
    在这个句子里面,only 修饰数量,搭配 when 引导的一个特殊时间状语从句。在英语中,这种结构通常被称为 "when" 的突发性用法,常与 "had (only/just) done" 连用。
    \begin{enumerate}
        \item \textbf{句法结构拆解 (Syntactic Breakdown)}
        \begin{itemize}
            \item \textbf{Only 的作用范围}
            \begin{itemize}
                \item 这里的 only 是副词,直接修饰数量词 seven miles。
                \item \textbf{含义:} “仅仅、只有”。它强调航行的距离非常短,还没飞多远就出事了。
            \end{itemize}
            
            \item \textbf{When 的特殊逻辑}
            \begin{itemize}
                \item 这里的 when 并不表示“当……的时候”,而是相当于 \textbf{"and at that moment"}(就在那时,突然……)。
                \item \textbf{时态搭配:} 前句用过去完成时(had travelled),后句用一般过去时(failed/was forced)。
            \end{itemize}
        \end{itemize}
        \item \textbf{核心表达解析 }
        \begin{itemize}
            \item \textbf{Across the Channel (横跨英吉利海峡)}
            \begin{itemize}
                \item Channel 特指英国和法国之间的英吉利海峡。
            \end{itemize}
            
            \item \textbf{Be forced to land on sea (被迫降落在海上)}
            \begin{itemize}
                \item Forced 体现了身不由己的紧急情况。
                \item 注意 land on sea 这种表达,虽然 land 本意是陆地,但在这里作为动词表示“降落/着陆”。
            \end{itemize}
        \end{itemize}
        \item \textbf{实战模仿:如何描述“突然的转折”}
        
        这种 "had only... when..." 的结构能让你的叙事非常有画面感,特别适合描述意外:

        \begin{itemize}
            \item \textbf{描述事故:}
            
            \es{The driver \textbf{had travelled only} a few meters \textbf{when} a tire burst.(司机才开了几米,就在那时轮胎爆了。)}
            
            \item \textbf{描述发现:}
            
            \es{I \textbf{had only} just entered the room \textbf{when} I noticed the \textbf{oddly shaped forms} hanging from the ceiling.(我刚进房间,就在那时我注意到了天花板上挂着的形状奇特的物体。)}
        \end{itemize}
    \end{enumerate}
\end{multicols}

\wsitem{A week before, he had completed a successful overland flight during which he covered twenty-six miles. 为什么during which会连用?}
\begin{multicols}{1}
    这是一个非常经典的\textbf{“介词 + 关系代词”}引导定语从句的结构。
    
    在英语中,当先行词(这里的 $twenty-six miles$)或者引导从句的时间/空间范围(这里的 $flight$)需要与一个介词搭配时,这个介词就会被提到引导词 $which$ 的前面。

    \begin{enumerate}
        \item \textbf{句法拆解与还原 (Syntactic Breakdown and Restoration)}
        
        为了理解为什么连用,我们可以把这个长句拆解成两个独立的简单句:

        \begin{itemize}
            \item \textbf{Sentence A:} He had completed a successful overland \textbf{flight}.
            \item \textbf{Sentence B:} He covered twenty-six miles \textbf{during} the \textbf{flight}.
            \item \textbf{Combination Logic:}
            \begin{itemize}
                \item 为了避免重复使用 flight,我们用关系代词 which 代替它。
                \item 介词 during (在……期间) 必须跟着它的宾语走。在正式英语中,介词通常置于 which 之前,形成 \textbf{during which}。
            \end{itemize}
        \end{itemize}

        \item \textbf{为什么不用 $when$ 或单独用 $which$? (Why this specific structure?)}
        
        \begin{itemize}
            \item \textbf{与 $when$ 的区别:}
            \begin{itemize}
                \item When 引导的是时间点或时间段。但这里的先行词是 flight(飞行过程),它既是一个时间段,也是一个**事件**。
                \item During which 比 when 更精确地强调了“**在这次飞行的整个航程中**”。
            \end{itemize}
            
            \item \textbf{与单独使用 $which$ 的区别:}
            \begin{itemize}
                \item 如果只说 "...flight which he covered twenty-six miles",语法是错误的。
                \item 因为 cover miles(飞行里程)不是在“飞行”这个动作本身上发生的,而是在这个“飞行过程**之中**”发生的。
            \end{itemize}
        \end{itemize}

        \textbf{“介词 + Which” 的常见模式 (Common Patterns)}

        这种结构在《新概念英语》和正式写作中极其常见,它能让逻辑非常严密:

        \begin{itemize}
            \item \textbf{Period + during which:}
            
            \es{The years \textbf{during which} he worked on his \textbf{life work} were difficult.(他致力于毕生心血的那几年非常艰辛。)}
            
            \item \textbf{Point + at which:}
            
            \item He reached a point \textbf{at which his \textbf{engine failed}.(他到达了一个引擎故障的点。)}
            
            \item \textbf{Tool/Means + by which:}
            
            \es{This is the method \textbf{by which} he \textbf{made it clear}.}(这就是他阐明观点的方法。)
        \end{itemize}
    \end{enumerate}
\end{multicols}

\wsitem{Latham made another attempt a week later and got within half a mile of Dover, but he was unlucky again. 这个句子中got within的作用}
\begin{multicols}{1}
    "Got within..." 是一个在描述距离、目标或进度时非常简洁且高频的短语,意为\textbf{“到达距离……不到……的地方”或“接近到……范围之内”}。

    \begin{enumerate}
        \item \textbf{句法结构总览 (Overall Syntactic Structure)}
        
        \begin{itemize}
            \item \textbf{Syntactic Role (句法角色):} Verb Phrase (动词短语)
            \begin{itemize}
                \item \textbf{Core Verb:} Get (过去式 got) —— 在这里表示“移动并到达(某个位置)”。
                \item \textbf{Preposition:} Within —— 核心介词,意为“在……之内”。它界定了一个范围。
                \item \textbf{Measure of Distance:} Half a mile —— 具体的距离数值。
                \item \textbf{Destination:} Of Dover —— 参照点。
            \end{itemize}
        \end{itemize}

        \item \textbf{核心逻辑点解析 (Core Logical Points)}
        
        \begin{itemize}
            \item \textbf{差之毫厘的遗憾 (The "Almost" Logic)}
            \begin{itemize}
                \item Within 强调的是“最后一段微小的差距”。
                \item 在文中,Latham 距离多佛(Dover)仅剩半英里,这与全长 20 多英里的航程相比微不足道。这个短语生动地衬托出后半句的 "unlucky again" —— 这种失败比一开始就坠海更令人惋惜。
            \end{itemize}
            
            \item \textbf{动态的终点}
            \begin{itemize}
                \item Get within 常用于描述**充满挑战的接近**。它暗示了主体(飞行员)付出了极大的努力才推进到这个范围内。
            \end{itemize}
        \end{itemize}

        \item \textbf{高级用法与同类替换 (Advanced Substitutions)}
        \begin{itemize}
            \item \textbf{Close in on... (逼近/接近)}
            \begin{itemize}
                \item \textbf{语感:} 强调正在缩短距离的动态过程。
            \end{itemize}
            
            \item \textbf{Within striking distance of... (在可以触及/攻击的距离内)}
            \begin{itemize}
                \item \textbf{语感:} 这是一个非常地道的习语,指距离成功只有一步之遥。
                
                \es{例句:He was \textbf{within striking distance of} his \textbf{life work}'s completion.}
            \end{itemize}
            
            \item \textbf{Be within reach (在够得着的范围内)}
            \begin{itemize}
                \item \textbf{语感:} 强调目标的触手可及。
            \end{itemize}
        \end{itemize}

    \end{enumerate}
\end{multicols}

\wsitem{After making a short test flight at 4,15 a.m., Bleriot set off half an hour later. 为什么不是after having made ...?}
\begin{multicols}{1}
    其实,"After having made..." 在语法上是完全正确的,但作者选择 "After making..." 是出于对简洁性和语言节奏的考虑。

    \begin{enumerate}
        \item \textbf{语法功能对比 (Grammatical Comparison)}
        \begin{itemize}
            \item \textbf{After making... (动名词/分词结构)}
            \begin{itemize}
                \item \textbf{特点:} 这里的 after 作为介词,直接接动名词。
                \item \textbf{逻辑:} 介词 after 本身已经具备了明确的“先后顺序”含义。在英语中,如果先后关系已经由 after 或 before 明确交代,动词通常不需要再使用完成式来重复强调。
            \end{itemize}
            
            \item \textbf{After having made... (完成时动名词)}
            \begin{itemize}
                \item \textbf{特点:} 这种结构被称为“动名词的完成式”。
                \item \textbf{逻辑:} 它专门用于强调一个动作在另一个动作开始之前已“完全结束”。
                \item \textbf{现状:} 虽然在语法上非常严谨,但在现代英语(尤其是叙事性文章)中,它常被认为过于冗长 (redundant)。
            \end{itemize}
        \end{itemize}
        \item \textbf{为什么作者选择 "After making"? (Stylistic Choice)}
        \begin{itemize}
            \item \textbf{避免“语义重叠” (Avoiding Redundancy)}
            \begin{itemize}
                \item After 已经告诉我们动作是先发生的,having made 再次告诉我们动作是先完成的。
                \item 这种“双重强调”除非是为了对比极其细微的时间差,否则在文学叙事中会显得拖沓。
            \end{itemize}
            
            \item \textbf{叙事的流畅性 (Narrative Flow)}
            \begin{itemize}
                \item 课文描述的是 Bleriot 紧凑的行动:4:15 试飞 $\rightarrow$ 紧接着半小时后出发。
                \item 使用 After making 能体现出一种**利落、连贯**的动作感,符合当时飞行员争分夺秒的气氛。
            \end{itemize}
        \end{itemize}
        \item \textbf{什么时候“必须”用 Having made?}
        
        虽然在本句中可以通用,但在以下情况,having done 会更有优势:
        
        \begin{itemize}
            \item \textbf{当不使用 After/Before 时:}
            
            如果去掉介词,只用分词作状语,则必须用完成式来体现先后。
                
            \es{\textbf{Having made} a test flight, he felt confident.} (做完试飞后,他感到有信心了。) —— 此时不能只用 Making,否则会变成“正在试飞时感到有信心”。
            
            \item \textbf{强调“因果”而非仅仅是“时间”:}
            
            \es{\textbf{Having made} several mistakes, he decided to \textbf{go to the police}. (正因为犯了好几个错误,他才决定去报案。) —— 这里强调的是“完成这些动作后导致的心理变化”。}
        \end{itemize}
        \item \textbf{总结与实践 (Summary and Practice)}
        \begin{itemize}
            \item \textbf{Rule of Thumb (经验准则):} 
            \begin{itemize}
                \item 有介词 (After/Before) 时,优先用 **doing** (简洁);
                \item 没介词且强调动作先后时,必须用 **having done** (精准)。
            \end{itemize}
        \end{itemize}
    \end{enumerate}
\end{multicols}



\newpage