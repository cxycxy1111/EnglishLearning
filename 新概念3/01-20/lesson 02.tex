\section{Lesson 2 Thirteen equals one}

\begin{paracol}{2}

\englishtext{Our vicar is always \ns{raising money} for one cause or another, but he has never \ns{managed to} get enough money to \newsentence{have the church clock repaired}. }

\switchcolumn

\nse{raise money}{}{筹款,募捐}
\nse{manage to}{}{达成,设法;}

\switchcolumn*

\englishtext{The big clock which \ns{used to} \ns{strike the hours} \ns{day and night} was damaged many years ago and has been silent \ns{ever since}.}

\switchcolumn

\nse{used to}{}{过去经常,曾经;}
\nse{strike the hours}{}{报时}
\nse{day and night}{}{日日夜夜;夜以继日;}
\nse{ever since}{}{adv. 从那时到现在; 自从;自…以后;从…起;}

\switchcolumn*

\englishtext{One night, however, our vicar \ns{woke up with a start}: the clock was striking the hours! }

\switchcolumn

\nse{wake up with a start}{}{惊醒}

\switchcolumn*

\englishtext{Looking at his watch, he saw that it was one o'clock, but the bell struck thirteen times before it stopped. }

\switchcolumn

\switchcolumn*

\englishtext{Armed with a torch, the vicar went up into the clock tower to see what was going on. }

\switchcolumn

\switchcolumn*

\englishtext{In the torchlight, he \ns{caught sight of} a \nw{figure} \newsentence{whom} he immediately \ns{recognized as} Bill Wilkins, our local grocer.}

\switchcolumn

\nwe{figure}{ˈfɪɡjər}{n. 数字;位数;人影;雕像;体形;知名人士;形象;图,表;几何图形;金额;冰上花样;v. 认为;计算;出现;}
\nse{catch sight of}{}{一下子看到;看见;瞥见;见;}
\nse{recognize ... as ...}{}{认出某人是……}

\switchcolumn*

\newsentence{'Whatever are you doing up here Bill?'} asked the vicar \ns{in surprise}.

\switchcolumn

\nse{in surprise}{}{惊奇地;}

\switchcolumn*

\englishtext{'I'm trying to repair the bell,' answered Bill. }

\switchcolumn

\switchcolumn*

\englishtext{'\newsentence{I've been coming up here \ns{night after night} for weeks now.} You see, \newsentence{I was hoping to give you a surprise.}'}

\switchcolumn

\nse{night after night}{}{adv. 一夜又一夜;}

\switchcolumn*

\englishtext{'\newsentence{You certainly did give me a surprise!}' said the vicar. }

\switchcolumn

\switchcolumn*

\englishtext{'You've probably woken up everyone in the village as well. Still, I'm glad the bell is working again.'}

\switchcolumn

\switchcolumn*

\englishtext{'That's the trouble, vicar,' answered Bill. }

\switchcolumn

\switchcolumn*

\englishtext{'It's working all right, but I'm afraid that at one o'clock it will strike thirteen times and there's nothing I can do about it.'}

\switchcolumn

\switchcolumn*

\englishtext{'We'll \ns{get used to} that, Bill,' said the vicar. }

\switchcolumn

\nse{get used to}{}{习惯于;}

\switchcolumn*

\englishtext{"\newsentence{Thirteen is not as good as one, but it's better than nothing.} Now let's go downstairs and \newsentence{have a cup of tea.}"}

\switchcolumn

\switchcolumn*

\end{paracol}

\worddifference
\wsitem{raise money, collect money}
\begin{multicols}{1}
    \begin{enumerate}
        \item Collect money: 侧重于“收集”这个动作(比如去每个人桌子前收钱),或者收集硬币、纸币。
        \item Raise money: 侧重于为了特定目的(慈善、创业、维修)而发起的一系列筹款活动。它带有一种“从无到有、努力筹措”的意味。
        \begin{itemize}
            \item 慈善与公益 (Charity): The school held a bake sale to raise money for the new library. (学校举办义卖活动为新图书馆筹款。)
            \item 商业投资 (Business): The startup is trying to raise money from venture capitalists. (这家创业公司正尝试从风险投资人那里融资。)
        \end{itemize}
    \end{enumerate}
\end{multicols}

\grammarpoints
\wsitem{Have sth. done}
\begin{multicols}{1}
    这个结构通常用于以下两种情况:
    \begin{enumerate}
        \item 请别人为自己做某事(最常用): 你不亲自动手,而是付钱或派人去做。
        \begin{itemize}
            \item I need to have my hair cut. (我需要理发了。——是理发师理,不是自己理。)
            \item I'm going to have my car washed. (我要去洗车。——洗车店洗。)
        \end{itemize}
        \item 遭遇某种不幸:
        \begin{itemize}
            \item He had his wallet stolen. (他的钱包被偷了。)
        \end{itemize}
    \end{enumerate}
\end{multicols}

\wsitem{Get sth. done}
\begin{multicols}{1}
    这个结构是英语中非常核心的使役用法,与 have sth. done 在很多语境下可以互换,但 get 的语气更具主动性和说服力。

    \begin{enumerate}
        \item 核心公式与逻辑:Get + something (宾语) + done (过去分词)
        \begin{itemize}
            \item 逻辑: 宾语(something)和后面的动词(done)之间是被动关系。
            \item 含义: 设法使某事被完成,或者请/叫别人做某事。
        \end{itemize}
        \item 主要用法场景
        \begin{itemize}
            \item 设法完成某项困难的任务 (Emphasis on Task)
            
            强调经过一番努力,最终完成了某事。
            
            $\rightarrow$ Example: \textit{I finally got the report finished at midnight. (我终于在半夜把报告写完了。)}

            \item 请/雇人做某事 (Emphasis on Service)
            
            类似于 have sth. done,但 get 往往暗示需要一点“说服”或“安排”。

            $\rightarrow$ Example: \textit{I need to get my car fixed. (我得找人把车修了。)}

            \item 遭遇不幸 (Emphasis on Accident)
            
            $\rightarrow$ Example: \textit{He got his leg broken during the football match. (他在足球赛中腿部骨折了。)}

            \item Get sth. done vs. Have sth. done
            
            虽然两者常通用,但有细微差别:

            \begin{itemize}
                \item Have sth. done: 语气更正式,强调结果或授权(比如牧师想 have the church clock repaired,这是一种官方/正式的安排)。
                \item Get sth. done: 语气更口语化,强调动作的过程或克服阻力去完成。
            \end{itemize}
        \end{itemize}
    \end{enumerate}
\end{multicols}


\wsitem{get used to ..., be used to ...}
\begin{multicols}{1}
    

\begin{enumerate}
    \item 核心含义
    
    Get used to something / doing something 表示“变得习惯于……”。 它强调的是一个过程,即从“不习惯”到“习惯”的转变。
    \begin{enumerate}
        \item get: 体现了动作的改变(become)。
        \item used: 这里的 used 是形容词,意为“习惯的”。
        \item to: 这是一个介词(Preposition)。这是最重要的语法点! 因为它是介词,所以后面必须接名词或 动名词 (V-ing)。
    \end{enumerate}
    \item 对比
    \begin{itemize}
        \item be used to (doing):习惯于(已经是习惯),侧重于当前的状态。
        \item get used to (doing):变得习惯于(适应中),侧重于转变的过程
    \end{itemize}
    \item 例句
    \begin{itemize}
        \item 适应环境: The puma had to \textbf{get used to} hunting in the quiet countryside. (这只美洲狮不得不习惯在宁静的乡村打猎。)
        \item 心理状态: Residents will never \textbf{get used to} the fact that a wild animal is \textbf{at large}. (居民们永远无法习惯一只野生动物在逃这一事实。)
    \end{itemize}
\end{enumerate}

\end{multicols}

\wsitem{he caught sight of a figure whom he immediately recognized as Bill Wilkins}
\begin{multicols}{1}
    \begin{enumerate}
        \item 这是一个带有定语从句的复杂句:
        \begin{itemize}
            \item 状语 (Adverbial): In the torchlight(在手电筒光照下)。
            \item 主句 (Main Clause): He (主) + caught sight of (谓) + a figure (宾)。
            \item 定语从句 (Relative Clause): whom he immediately recognized as Bill Wilkins...
            
            这个从句修饰先行词 a figure。

            \item whom: 在从句中作 recognized 的宾语。

        \end{itemize}
        \item 重点难点:Whom 的使用
        
        在正式英语(如《新概念三》)中,当关系代词在从句中充当宾语时,使用 whom 而不是 who。
        
        He recognized the figure (宾语).

        $\rightarrow$ The figure whom he recognized.
    \end{enumerate}
\end{multicols}

\wsitem{Whatever are you doing up here Bill}
\begin{multicols}{1}
    \begin{enumerate}
        \item 语法核心:Whatever 的特殊用法
        
        在这里,Whatever 并不是引导一个让步状语从句(比如“无论什么”),而是用来加强语气。

        \begin{itemize}
            \item 功能: 在疑问词(what, where, who, how)后加上 -ever,常用于非正式口语中,表示极度的惊讶、困惑、愤怒或怀疑。
            \item 同义替换: 相当于 "What on earth are you doing" 或 "What in the world are you doing"。
            \item 例句:
            \begin{itemize}
                \item Whatever do you mean? (你到底是什么意思?!)
                \item Wherever did you find that? (你究竟是在哪儿找到那东西的?!)
            \end{itemize}
        \end{itemize}
        \item Up here 的空间感
        \begin{itemize}
            \item Up here: 在这里指“在这上面”。
            \item 语境: 结合课文,当时牧师是爬到了钟楼(Church tower)的顶端。所以他用 "up here" 来强调这个位置极不寻常——一个杂货店老板大半夜不在家睡觉,跑到这么高的钟楼上干什么?
        \end{itemize}
    \end{enumerate}
\end{multicols}

\wsitem{I’ve (I have) + been + coming (V-ing)}
\begin{multicols}{1}
    \begin{enumerate}
        \item 语法深度解析
        
        I’ve (I have) + been + coming (V-ing)

        \begin{itemize}
            \item 时态:现在完成进行时 (Present Perfect Continuous)
            \begin{itemize}
                \item 构成: \textit{have/has + been + doing}
                \item 含义: 强调一个动作从过去开始,一直持续到现在,而且极有可能还会继续下去。
            \end{itemize}
            \item Up here: 再次出现了这个短语,指代钟楼顶部。
            \item 语境含义: 比尔想表达的是:“我这段时间以来,一直都在往这儿(钟楼)跑。” 这暗示了他为了修钟付出了长期的努力,而不是只来了一次。
        \end{itemize}

        \item 为什么要用这个时态?
        
        如果比尔说 "I come up here",那是陈述一个经常性的事实(习惯)。 但他说 "I've been coming...",带有强烈的解释色彩和情感诉求:
        \begin{itemize}
            \item 解释原因: 解释为什么牧师此时此刻会在这里撞见他。
            \item 强调辛苦: 暗示他为了这口破钟,已经往返折腾了很多次了。
        \end{itemize}
    \end{enumerate}
\end{multicols}

\wsitem{I was hoping...}
\begin{multicols}{1}
    比尔(Bill Wilkins)对牧师解释他为什么深夜偷偷修钟的原因。这句话充满了温情和幽默,同时也包含了一个非常微妙的时态用法。

    \begin{enumerate}
        \item 语法深度解析:过去进行时的委婉语气
        
        I was hoping...

        \begin{itemize}
            \item 时态:过去进行时 (Past Continuous)
            \item 深层含义: 虽然形式上是过去进行时,但在口语中,这是一种极具委婉、礼貌色彩的表达方式。
            \begin{itemize}
                \item 如果你说 "I hope to give you a surprise" (一般现在时),听起来比较直接。
                \item 如果你说 "I was hoping...",它把这种想法稍微推向了“过去”和“过程”,表达出一种“我之前一直这么想着,但现在被你发现了”的羞涩感或解释语气。
            \end{itemize}
        \end{itemize}
    \end{enumerate}
\end{multicols}

\wsitem{give sb. a surprise}
\begin{multicols}{1}
    \begin{itemize}
        \item 含义: 给某人一个惊喜。
        \item 搭配拓展:
        \begin{itemize}
            \item take someone by surprise: 趁某人不备(强调突然性)。
            \item To someone's surprise: 令某人惊讶的是(常作状语)。
        \end{itemize}
    \end{itemize}
\end{multicols}

\wsitem{... is not as good as ..., but it is better than ...}

\begin{multicols}{1}
    这句话是一个经典的三方对比句型,用于表达某事物处于“中等水平”或“比上不足,比下有余”。
    \begin{enumerate}
    \item 句型结构解析
    
    \textbf{A is not as good as B}:A 不如 B 好(B 是最高标准)。
    
    \textbf{But it is better than C}:但 A 比 C 好(C 是更差的标准)。
    
    逻辑关系: B > A > C

    \item 常用口语替代(更地道的说法)
    
    虽然上面的句子语法正确,但在日常生活中,人们常会根据语境使用以下表达:

    \begin{itemize}
        \item 强调“中规中矩”:"It’s somewhere in the middle; better than C, but not quite as good as B."
        \item 强调“折中方案”:"It’s a decent compromise. It doesn't beat B, but it’s a step up from C."
        \item 强调“还可以,但不是最好的”:"It’s okay, but it’s no [B]." (例如:It’s okay, but it’s no iPhone. 表示它比一般的强,但比不上 iPhone。)
    \end{itemize}
    \item 实际应用举例
    \begin{itemize}
        \item 谈论电子产品:"The budget model is not as good as the Pro version, but it is better than the entry-level one."(这款经济型不如 Pro 版好,但比入门级要强。)
        \item 谈论餐厅:"This cafe is not as good as the one downtown, but it is better than the one near my house."(这家咖啡馆不如市中心那家,但比我家附近那家好。)
        \item 谈论个人表现:"My English is not as good as a native speaker's, but it is better than it was last year."(我的英语不如母语者,但比去年进步了。)
    \end{itemize}
    \item 语法进阶:Not so... as...
    \begin{itemize}
        \item 在否定句中,你也可以把第一个 as 换成 so,这在文学或正式语境中更常见:
        \item "The movie is not so good as the book, but it is better than the sequel."(电影不如原著好,但比续集强。)
    \end{itemize}
    \item 总结: 这个句型非常实用,可以帮你精准地描述一个事物在同类产品或情况中的相对位置。

\end{enumerate}
\end{multicols}


\wsitem{have a ...}

\begin{multicols}{1}
    在英语中,have 是一个极具“包容性”的动词,可以代替许多具体的动作动词(如 eat, drink, enjoy)。

    \begin{itemize}
        \item Have a tea/coffee:喝茶/咖啡 (Drink)
        \item Have a look:看一眼 (Look)
        \item Have a try:试一下 (Try)
        \item Have a rest:休息一下 (Rest)
        \item Have a cigarette	抽根烟 (Smoke)
    \end{itemize}
\end{multicols}






\newpage
