\section{Lesson 19 A very dear cat}
\begin{paracol}{2}

\nw{Kidnappers} are rarely interested in animals, but they recently \ns{took considerable interest in} Mrs. Eleanor Ramsay's cat. 

\switchcolumn

\chinesetext{绑架者很少对动物感兴趣。最近,绑架者却盯上了埃莉诺·拉姆齐太太的猫。}
\nwe{kidnapper}{'kɪdnæpə(r)}{n. 拐子,绑匪;}
\nse{take considerable interest in ...}{}{}

\switchcolumn*

Mrs. Eleanor Ramsay, a very wealthy old lady, has shared a flat with her cat, Rastus, for a great many years. 

\switchcolumn

\chinesetext{埃莉诺·拉姆齐太太是一个非常富有的老妇人,多年来,一直同她养的猫拉斯一起住在一所公寓里。}

\switchcolumn*

Rastus \ns{leads an orderly life}. 

\switchcolumn

\chinesetext{拉斯特斯生活很有规律。}
\nse{lead an orderly life}{}{是一个非常优雅且地道的短语,用于描述一种有规律、有条理、严谨的生活方式}

\switchcolumn*

He usually \ns{takes a short walk} in the evenings and is always home by seven o'clock. 

\switchcolumn

\chinesetext{他傍晚常常出去溜达一会儿,并且总是在7点钟以前回来。}
\nse{take a short walk}{}{不仅表达“有兴趣”,还通过 considerable(相当大的/值得注意的)强调了这种兴趣的深度和专注度。}

\switchcolumn*

One evening, however, he failed to arrive. 

\switchcolumn

\chinesetext{可是,有一天晚上,它出去后再也没回来。}

\switchcolumn*

Mrs. Ramsay got very worried. She looked everywhere for him but could not find him.

\switchcolumn

\chinesetext{拉姆齐太太急坏了,四处寻找,但没有找着。}

\switchcolumn*

There days after Rastus' disappearance, Mrs. Ramsay received an \nw{anonymous} letter. 

\switchcolumn

\chinesetext{拉斯特斯失踪3天后,拉姆齐太太收到一封匿名信。}
\nwe{anonymous}{əˈnɑːnɪməs}{adj. 匿名的,不知名的;不记名的;没有特色的;}

\switchcolumn*

The writer stated that Rastus was \ns{in safe hands} and would be returned immediately if Mrs. Ramsay paid a \nw{ransom} of \$1,000. 

\switchcolumn

\chinesetext{写信人声称拉斯特斯安然无恙,只要拉姆齐太太愿意支付1,000 英镑赎金,可以立即将猫送还。}
\nwe{ransom}{ˈrænsəm}{n. 付赎金救人,赎金;[神]赎罪;vt. 赎回;[神]赎救;}
\nse{in safe hands}{}{在可靠的人手里;}

\switchcolumn*

Mrs. Ramsay was instructed to place the money in a \nw{cardboard} box and to leave it outside her door. 

\switchcolumn

\chinesetext{他让拉姆齐太太把钱放在一个纸盒里,然后将纸盒放在门口。}
\nwe{cardboard}{ˈkɑrdbɔrd}{n. 硬纸板;卡纸板;adj. 不真实的,虚假的;}

\switchcolumn*

\ns{At first} she decided to \ns{go to the police}, but fearing that she would never see Rastus again -- \newsentence{the letter had made that quite clear} -- she \ns{changed her mind}. 

\switchcolumn

\chinesetext{一开始拉姆齐太太打算报告警察,但又害怕再也见不到拉斯特斯——这点,信上说得十分明白——于是便改变了主意。}
\nse{at first}{æt fɚst}{起初,当初;}
\nse{go to the police}{}{报案,向警方求助;}
\nse{change one's mind}{tʃeɪndʒ wʌnz maɪnd}{改变主意;变卦;}

\switchcolumn*

She withdrew \$1000 from her bank and followed the kidnapper's instructions. 

\switchcolumn

\chinesetext{她从银行取出1,000 英镑,并照绑架者的要求做了。}

\switchcolumn*

The next morning, the box had disappeared but Mrs. Ramsay was sure that the kidnapper would \ns{keep his word}. 

\switchcolumn

\chinesetext{次日早晨,放钱的盒子不见了。但拉姆齐太太确信绑架者是会履行诺言的。}
\nse{keep one's word}{kiːp wʌnz wɜːrd}{遵守诺言,守信用;说话算数;信守诺言;}

\switchcolumn*

\ns{Sure enough}, Rastus arrived \nw{punctually} at seven o'clock that evening. 

\switchcolumn

\chinesetext{果然,当天晚上7点正,拉斯特斯准时回来了。}
\nwe{punctually}{ˈpʌŋktʃʊəlɪ}{adv. 如期地,准时地;正点;}
\nse{sure enough}{ʃʊr ɪˈnʌf}{果然;的的确确;}

\switchcolumn*

He looked very well, though he was \newsentence{rather} \nw{thirsty}, for he drank half a bottle of milk. 

\switchcolumn

\chinesetext{它看上去一切正常,只是口渴得很,喝了半瓶牛奶。}
\nwe{thirsty}{ˈθɜːrsti}{adj. 口渴的;渴望(求)…的;干旱的,缺水的;耗油的;}

\switchcolumn*

The police were \nw{astounded} when Mrs. Ramsay told them what she had done. 

\switchcolumn

\chinesetext{拉姆齐太太把她所做的事告诉了警察,警察听后大为吃惊。}
\nwe{astounded}{əˈstaʊndɪd}{v. 使震惊(astound的过去式和过去分词);愕然;愕;惊讶;adj. 感到震惊的;大吃一惊的;}

\switchcolumn*

She explained that Rastus \ns{was very dear to} her. 

\switchcolumn

\chinesetext{拉姆齐太太解释说她心疼她的猫拉斯特斯。}
\nse{be very dear to sb.}{}{对某人来说非常珍贵;深受某人喜爱}

\switchcolumn*

Considering the amount she paid, he was dear in \ns{more ways than one}!

\switchcolumn

\chinesetext{想到她所花的那笔钱,她的心疼就具有双重意义了。}
\nse{more ways than one}{}{在不止一个方面}

\switchcolumn*

\end{paracol}


\grammarpoints

\wsitem{... though he was rather thirsty, ...}
\begin{multicols}{1}
    在这个句式中,rather 是一个程度副词 (Adverb of Degree)。它的作用非常微妙,主要用于修饰形容词或副词,表示“相当、颇、在某种程度上”。

    \begin{enumerate}
        \item \textbf{逻辑功能:委婉的强调 (Subtle Emphasis)}
        \begin{itemize}
            \item \textbf{程度的中置化 (Moderate Degree)}
            \begin{itemize}
                \item rather 表达的程度通常高于 fairly (尚可) 或 quite (很),但低于 very (非常)。
                \item 它传达出一种“比预想的要更甚一些”的感觉,但语气又比 very 更加含蓄、克制。
            \end{itemize}
            
            \item \textbf{语气的评价性 (Evaluative Tone)}
            \begin{itemize}
                \item 在英语习惯中,rather 常用于修饰**负面或不理想**的形容词(如 \textit{thirsty, cold, difficult, annoyed})。
                \item 当它修饰这类词时,带有一种“**令人遗憾地、不凑巧地**”这种潜在意味。
                \item \textbf{例句:} The soup is \textbf{rather cold.} (汤相当冷了——暗示这不太好。)
            \end{itemize}
        \end{itemize}
        \item \textbf{语境中的对比作用 (Contrast in Context)}
        
        在这个特定的句子中("He looked very well, though he was rather thirsty..."),rather 起到了平衡叙事节奏的作用:

        \begin{itemize}
            \item \textbf{对比效应 (Contrast Effect):}
            \begin{itemize}
                \item \textbf{前半句:} looked \textbf{very well} (看起来状态非常好)。
                \item \textbf{后半句:} was \textbf{rather thirsty} (只是相当口渴)。
            \end{itemize}
            \item \textbf{逻辑作用:} 使用 rather 使得“口渴”这个负面状态看起来不像是一种严重的病态,而更像是一种可以理解的小瑕疵。这解释了为什么他虽然喝了很多奶,但“看起来气色依然很好”。
        \end{itemize}
        \item \textbf{"Rather" 的特殊用法进阶 (Advanced Usage)}
        \begin{itemize}
            \item \textbf{与比较级连用 (With Comparatives)}
            \begin{itemize}
                \item 它常用来修饰比较级,表示“稍微……一点”。
                \item \textbf{例句:} The project is \textbf{rather more difficult than we thought.} (项目比我们预想的要难一些。)
            \end{itemize}
            
            \item \textbf{修饰正面词汇 (With Positive Words)}
            \begin{itemize}
                \item 当 rather 修饰褒义词(如 \textit{good, pleasant})时,往往带有“**出乎意料地好**”这种赞赏感。
                \item \textbf{例句:} It was a \textbf{rather pleasant surprise.} (这是一个相当令人惊喜的意外。)
            \end{itemize}
        \end{itemize}
    \end{enumerate}
\end{multicols}

\wsitem{be dear to sb.}
\begin{multicols}{1}
    "Be dear to sb." 是一个情感深度很高的表达,通常用来形容某人或某物在某人心中占据着极其重要、珍贵且不可替代的地位。

    \begin{enumerate}
        \item \textbf{词义本质 (Core Meaning)}
        \begin{itemize}
            \item \textbf{情感价值 (Emotional Value)}
            \begin{itemize}
                \item Dear 在这里不表示“亲爱的”这种称呼,也不仅仅是“喜欢”,它强调一种“**珍视**” (precious) 的情感。
                \item 如果某物对你很 dear,意味着失去它会让你感到难过。
            \end{itemize}
            
            \item \textbf{语义范围 (Semantic Scope)}
            \begin{itemize}
                \item \textbf{人:} 形容至亲好友(如 a dear friend)。
                \item \textbf{物:} 形容具有纪念意义的物品(如旧照片、礼物)。
                \item \textbf{抽象概念:} 形容理想、家乡或一段珍藏的回忆(如 memories dear to my heart)。
            \end{itemize}
        \end{itemize}
        \item \textbf{句法结构 (Syntactic Structure)}
        \begin{itemize}
            \item \textbf{结构式:[Subject] + be + (degree) + dear + to + [Somebody]}
            \begin{itemize}
                \item \textbf{Degree Modifiers:} 常搭配 \text{very, so, deeply, especially}。
                \item \textbf{Position:} 常用作表语。
            \end{itemize}
            
            \item \textbf{常见变体:Hold sb./sth. dear}
            \begin{itemize}
                \item \textbf{用法:} \text{He holds his freedom dear.} (他珍视自由。)
                \item \textbf{逻辑:} 这个动词词组比 \text{be dear to} 更有“主动保护、捍卫”的意味。
            \end{itemize}
        \end{itemize}
        \item \textbf{地道用法与语境对比 (Usage & Contrast)}
        \begin{itemize}
            \item \textbf{与 Important 的区别}
            \begin{itemize}
                \item \text{Important} 侧重于**功能、地位或影响**(如:这份文件很重要)。
                \item \text{Dear} 侧重于**个人情感和爱**(如:这张破旧的毯子对我来说很珍贵)。
            \end{itemize}
            
            \item \textbf{文学色彩}
            \begin{itemize}
                \item 这是一个带有温情、甚至略带忧伤或怀旧色彩的词。在《新概念英语》中,当描述艺术家的作品或某人的毕生心血(\text{life work})时,用这个词能体现出创造者对作品近乎痴迷的热爱。
            \end{itemize}
        \end{itemize}
        \item \textbf{实战模仿 (Practical Application)}
        \begin{itemize}
            \item \textbf{描述个人物品:}
            
            \es{ This old watch \textbf{is very dear to me, for it was a gift from my grandfather.}(这块旧表对我来说非常珍贵,因为它是我祖父送给我的礼物。)}
            
            \item \textbf{描述理想或事业:}
            
            \es{ The project was \textbf{dear to his heart; he had spent ten years on it.}(这个项目是他心之所系,他为此倾注了十年的心血。)}
        \end{itemize}
    \end{enumerate}
\end{multicols}

\wsitem{more ways than one}
\begin{multicols}{1}
    这是一个非常地道的英语习语,通常以 "in more ways than one" 的形式出现。它用来形容某种情况、影响或意义是多方面的、不止一种的,或者具有双关/深层含义。

    \begin{enumerate}
        \item \textbf{语义逻辑解析 (Semantic Logic)}
        \begin{itemize}
            \item \textbf{字面与引申含义 (Literal vs. Figurative)}
            \begin{itemize}
                \item \textbf{字面:} 指通过多种途径、方法或手段。
                \item \textbf{引申:} 指某件事在多个层面(如物质与精神、表面与深层)都具有意义。
            \end{itemize}
            
            \item \textbf{语义重心 (Emphasis)}
            \begin{itemize}
                \item 这个短语强调“多样性”和“程度深”。它告诉读者:不要只看表面,这件事的影响比你想象的要丰富得多。
            \end{itemize}
        \end{itemize}
        \item \textbf{常见使用语境 (Common Contexts)}
        \begin{itemize}
            \item \textbf{形容成功的维度 (Dimensions of Success)}
            \begin{itemize}
                \item \textbf{例句:} The project was a success \textbf{in more ways than one.}
                \item \textbf{逻辑:} 这意味着项目不仅赚了钱(经济层面),还赢得了声誉(社会层面),甚至锻炼了团队(管理层面)。
            \end{itemize}
            
            \item \textbf{形容双关语或多重身份 (Puns and Multiple Identities)}
            \begin{itemize}
                \item 如果一个人既是你的上司又是你的好友,你可以说:"He is my leader \textbf{in more ways than one."} 
                \item 这暗示了他不仅在工作上领导你,在生活品味或精神导向上也是你的榜样。
            \end{itemize}
            
            \item \textbf{形容改变或影响 (Impact and Change)}
            \begin{itemize}
                \item This experience changed my life \textbf{in more ways than one.}
                \item 指这次经历改变了我的性格、职业规划以及对他人的看法。
            \end{itemize}
        \end{itemize}
        \item \textbf{句法结构与替换 (Syntax and Substitutions)}
        \begin{itemize}
            \item \textbf{句中位置 (Position)}
            \begin{itemize}
                \item 通常置于句末,作为状语修饰整个句子或动词短语。
            \end{itemize}
            
            \item \textbf{近义词替换 (Synonyms)}
            \begin{itemize}
                \item On many levels (在许多层面上)
                \item Multifaceted (多层面的/多才多艺的)
                \item In various respects (在各个方面)
            \end{itemize}
        \end{itemize}
    \end{enumerate}
\end{multicols}

\grammarquestions
\wsitem{the letter had made that quite clear...为什么是the letter,不是the kidnapper?}
\begin{multicols}{1}
    这个问题切中了英语叙事艺术的核心。虽然这封信确实是绑匪(kidnapper)写的,但作者选择用 the letter 作为主语,是为了在读者的脑海中构建一种特定的心理氛围。

    以下是从文学叙事、心理逻辑和语言习惯三个维度的深度解析:

    \begin{enumerate}
        \item \textbf{创造“冰冷的物证感” (The Power of the Object)}
        \begin{itemize}
            \item \textbf{消除人的温情 (Dehumanization)}
            \begin{itemize}
                \item 绑匪是一个人,人可能会犹豫、会说谎、会改变主意。
                \item 但一封信(The letter)是白纸黑字。将“信”作为主语,赋予了威胁一种**不可更改、冷酷无情**的属性。这种“物”的冰冷感比“人”的威胁更让人感到绝望。
            \end{itemize}
            
            \item \textbf{模拟女主人的视角 (Immediate Perception)}
            \begin{itemize}
                \item 当时绑匪并不在现场,女主人手里紧握的是那张纸。
                \item 在那个惊恐的瞬间,对她产生直接心理冲击的不是远方的绑匪,而是眼前的这封信。这种写法让读者能身临其境地感受到那份威胁的\textbf{即时性}。
            \end{itemize}
        \end{itemize}
        \item \textbf{英语中的“物称主语”习惯 (Impersonal Subjects)}
        \begin{itemize}
            \item \textbf{逻辑重心的转换 (Focus Shift)}
            \begin{itemize}
                \item \textbf{If using "the kidnapper":} 重点在于人的行为(绑匪威胁了她)。
                \item \textbf{Using "the letter":} 重点在于证据的清晰度(信里写得清清楚楚)。
            \end{itemize}
            
            \item \textbf{常见的地道表达:}
            \begin{itemize}
                \item The map tells you the way. (地图会告诉你路——而不是说地图绘图员告诉你路。)
                \item The facts speak for themselves. (事实胜于雄辩——事实自己会说话。)
            \end{itemize}
        \end{itemize}
        \item \textbf{强化“不可辩驳性” (Indisputability)}
        \begin{itemize}
            \item \textbf{法律般的效力}
            \begin{itemize}
                \item The letter had made that quite clear. 
                \item 这句话听起来就像法官在宣读判决。信件的内容成为了一个**既定事实 (Established Fact)**。这种表达方式加强了女主人“不得不屈服”的逻辑合理性。
            \end{itemize}
            
            \item \textbf{避免冗余}
            \begin{itemize}
                \item 读者已经知道信是绑匪写的。如果再说 the kidnapper,叙事会显得有些啰嗦;而用 the letter,则精准地捕捉到了威胁的传播媒介。
            \end{itemize}
        \end{itemize}
    \end{enumerate}
\end{multicols}

\wsitem{She withdrew \$1000 from her bank and followed the kidnapper's instructions. 为什么时her bank, 但是bank不属于她的?}
\begin{multicols}{1}
    在英语中,当我们要表达某人经常使用的、存钱的或与其有业务往来的机构时,习惯性地使用形容词性物主代词(如 my, your, her)。这并不意味着她拥有这家银行的所有权,而是强调一种\textbf{“所属关系”或“业务关联”}。
    \begin{enumerate}
        \item \textbf{“关联性”而非“所有权” (Connection vs. Ownership)}
        \begin{itemize}
            \item \textbf{1. 业务归属感 (Service Relationship)}
            \begin{itemize}
                \item Her bank 指的是“她开户的那家银行”或“她平时使用的银行”。
                \item 这类词在英语中被称为**习惯性领属**。类似于我们会说 "my doctor"(我的医生)或 "my school"(我的学校),显然我们并不拥有那位医生或那所学校,但我们与他们有固定的联系。
            \end{itemize}
            
            \item \textbf{2. 唯一性与确定性}
            \begin{itemize}
                \item 如果只说 "from \textbf{a bank"},听起来像是她随便找了一家银行(可能是抢劫或随机选择);而 "from \textbf{her bank"} 则明确了这是她自己的账户所在地,逻辑上更符合取钱(withdraw)的行为。
            \end{itemize}
        \end{itemize}
        \item \textbf{英语中的类似表达习惯 (Common Parallel Expressions)}
        
        这种用法在日常英语中随处可见,我们可以通过下表对比:

        \begin{itemize}
            \item \textbf{My seat}
            \begin{itemize}
                \item \textbf{含义:} 我正坐着的那个位置。
                \item \textbf{逻辑:} 并不拥有该椅子,但拥有此时此刻的使用权。
            \end{itemize}
            
            \item \textbf{His office}
            \begin{itemize}
                \item \textbf{含义:} 他办公的地方。
                \item \textbf{逻辑:} 通常是公司资产,但他与其有固定的职业关联。
            \end{itemize}
            
            \item \textbf{Her bus}
            \begin{itemize}
                \item \textbf{含义:} 她通常搭乘的那班公交车。
                \item \textbf{逻辑:} 指代特定的路线或习惯,而非她买下了这辆车。
            \end{itemize}
            
            \item \textbf{My country}
            \begin{itemize}
                \item \textbf{含义:} 我的祖国。
                \item \textbf{逻辑:} 强调我是其中的公民,而非这片领土的所有者。
            \end{itemize}
        \end{itemize}
        \item \textbf{动词搭配的逻辑强制性 (Lexical Logic)}
        \begin{itemize}
            \item \textbf{Withdraw (取款) 的语境要求:}
            \begin{itemize}
                \item 取钱的前提是你必须在银行有账户。
                \item She withdrew \$1000 from \textbf{her bank.} 这一短语组合在英语母语者看来是非常连贯的,因为它暗示了:取钱 $\rightarrow$ 账户 $\rightarrow$ 她的银行。
            \end{itemize}
        \end{itemize}
    \end{enumerate}
\end{multicols}

\newpage