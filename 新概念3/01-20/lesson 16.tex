\section{Lesson 16 Mary had a little lamb}

\begin{paracol}{2}

    Mary and her husband Dimitri lived in the tiny village of Perachora in southern Greece. 

    \switchcolumn

    \chinesetext{玛丽与丈夫迪米特里住在希腊南部一个叫波拉考拉的小村庄里。}

    \switchcolumn*

    One of Mary's \nw{prize} possessions was a little white lamb which her husband had given her. 

    \switchcolumn

    \chinesetext{玛丽最珍贵的财产之一就是丈夫送给她的一只白色小羔羊。}

    \nwe{prize}{praɪz}{n. 奖品;非常珍贵的人/物;adj. 可以获奖的;优秀的;v. 重视;撬动;竭力打探;}

    \switchcolumn*

    \newsentence{She kept it \nw{tied} to a tree in a field \ns{during the day}} and went to \nw{fetch} it every evening. 

    \switchcolumn

    \chinesetext{白天,玛丽把羔羊拴在地里的一颗树上,每天晚上把它牵回家。}

    \nwe{tie}{taɪ}{v. 系,扣,捆,打结;(使)关系密切;束缚,限制;打成平局;n. 领带;绳子,金属丝;联系;束缚;轨枕;淘汰赛;}
    \nwe{fetch}{fetʃ}{v. 去拿;卖得;售得;去取;}
    \nse{during the day}{}{在白天;在白天期间;}

    \switchcolumn*

    \newsentence{One evening, however, the lamb was missing. }

    \switchcolumn

    \chinesetext{可是,一天晚上,那只小羔羊失踪了。}

    \switchcolumn*

    The rope had been cut, so \newsentence{it was obvious that the lamb had been stolen}.

    \switchcolumn

    \chinesetext{绳子被人割断,很明显小羔羊是被人偷走了。}

    \switchcolumn*

    \newsentence{When Dimitri came in from the fields}, His wife told him what had happened.

    \switchcolumn

    \chinesetext{迪米特里从地里回来,妻子把情况跟他一说。}

    \switchcolumn*

    Dimitri at once set out to find the thief.

    \switchcolumn

    \chinesetext{他马上出去找偷羔羊的人。}

    \switchcolumn*

    He knew \newsentence{it would not prove difficult in such a small village}. 

    \switchcolumn

    \chinesetext{他知道在这样一个小村庄里抓住小偷并不困难。}

    \switchcolumn*

    After telling several of his friends about the \nw{theft}, Dimitri found out that his neighbour, Aleko, had suddenly \newsentence{acquired} a new lamb.

    \switchcolumn

    \chinesetext{把失窃的事告诉几个朋友后,迪米特里发现他的邻居阿列科家突然多了一只小羔羊。}

    \nwe{theft}{θeft}{n. 偷窃,盗窃(罪);}

    \switchcolumn*

    Dimitri immediately went to Aleko's house and angrily \ns{accused him of} stealing the lamb. 

    \switchcolumn

    \chinesetext{迪米特里立刻去了阿列科家,气呼呼地指责他偷了羔羊。}

    \nse{accused sb. of sth./doing sth.}{}{指责某人某事}

    \switchcolumn*

    He told him he \ns{had better} return it or he would \ns{call the police}. 

    \switchcolumn

    \chinesetext{他告诉他最好把羊交还,否则就去叫警察。}
    \nse{have better do sth.}{}{最好...}
    \nse{call the police}{}{报警}

    \switchcolumn*

    Aleko \nw{denied} taking it and led Dimitri into his \nw{backyard}. 

    \switchcolumn

    \chinesetext{阿列科不承认,并把迪米特里领进院子。}

    \nwe{deny}{dɪˈnaɪ}{v. 否认;拒绝承认;戒绝(尤因道德或宗教原因);拒绝给予;}
    \nwe{backyard}{ˌbækˈjɑrd}{n. 后院;}

    \switchcolumn*

    \newsentence{It was true that he had just bought a lamb}, he explained, but his lamb was black. 

    \switchcolumn

    \chinesetext{不错,他的确刚买了一只羔羊,阿列科解释说,但他的羔羊是黑色的。}

    \switchcolumn*

    \newsentence{\nw{Ashamed} of having acted so rashly, Dimitri \nw{apologized} to Aleko for having accused him. }

    \switchcolumn

    \chinesetext{迪米特里为自己的鲁莽而感到不好意思,向阿列科道了歉,说是错怪了他。}
    \nwe{ashamed}{əˈʃeɪmd}{[电影]真丢人;}
    \nwe{apologize}{əˈpɑːlədʒaɪz}{v. 道歉,认错;}

    \switchcolumn*

    While they were talking it began to rain and Dimitri stayed in Aleko's house until the rain stopped.

    \switchcolumn
    \chinesetext{就在他俩说话的时候,天下起了雨,迪米特里便呆在阿列科家里避雨,一直等到雨停为止。}

    \switchcolumn*

    When he went outside half an hour later, he \newsentence{was astonished to find} that the little black lamb was almost white. 

    \switchcolumn

    \chinesetext{半小时后,当他从屋里出来时,他惊奇地发现小黑羔羊全身几乎都变成白色。}

    \switchcolumn*

    \newsentence{Its wool, which had been \nw{dyed} black, had been washed clean by the rain!}

    \switchcolumn

    \chinesetext{原来羊毛上染的黑色被雨水冲掉了!}

    \nwe{dye}{daɪ}{v. 给…染色;n. 染料,染液;}
    \switchcolumn*

\end{paracol}

\grammarpoints
\wsitem{Keep + Object + Participle}
\begin{multicols}{1}
    
    在英语逻辑中,使用 "kept it tied to a tree" 而不是简单的 tied it to a tree,是为了区分“动作的发生”与“状态的持续”。

    \begin{enumerate}
        \item \textbf{动作 vs. 状态 (Action vs. State)}
        \begin{itemize}
            \item \textbf{Tied it to a tree:} 侧重于“拴”这个瞬间动作。如果你说 "She tied it to a tree during the day",听起来像是她一整天都在重复“拴”这个动作,或者这只是一个动作描述,没强调结果。
            \item \textbf{Kept it tied:} 侧重于“保持这种被拴住的状态”。它强调在这一整段时间(during the day)里,这种状态一直没有改变。
        \end{itemize}

        \item \textbf{维持某种受控环境 (Maintenance)}
        \begin{itemize}
            \item \textbf{Keep} 含有“看管、保留、维持”的逻辑。在叙事中,它常用于描述一种例行公事(routine)或一种长期的安排。
            \item \textbf{逻辑链条:} 动作发生(拴住) $\rightarrow$ 维持状态(Keep) $\rightarrow$ 持续时间(During the day)。
        \end{itemize}

        \item \textbf{语法结构:Keep + 宾语 + 宾语补足语}
        \begin{itemize}
        \item 这里 \textbf{tied} 是过去分词作补足语,表示宾语 \textbf{it}(那个存钱罐或动物)“被拴着”的被动状态。
        \item \textbf{类似用法:}
            \begin{itemize}
                \item \textit{Keep the door locked.} (让门一直锁着。)
                \item \textit{Keep your eyes closed.} (闭上眼睛别睁开。)
            \end{itemize}
        \end{itemize}
    \end{enumerate}
\end{multicols}

\wsitem{It was obvious that ...}
\begin{multicols}{1}
    "It was obvious that..." 是一个非常经典且高效的句式,用于引出一个显而易见、无需多言的事实。

    \begin{enumerate}
        \item \textbf{语法结构:形式主语 (Formal Subject)}
        \begin{itemize}
            \item \textbf{It (形式主语):} 指代后面整个 \textit{that} 从句。
            \item \textbf{Obvious (形容词):} 核心判断词,意为“显而易见的”。
            \item \textbf{That 从句 (真正的主语):} 描述那个具体的事实。
            \item \textbf{为什么这么用?} 英语倾向于把“重”的信息放在后面,让句子听起来更稳重。
        \end{itemize}

        \item \textbf{逻辑功能:建立共识 (Establishing Consensus)}
        \begin{itemize}
            \item 当你使用这个句式时,你是在暗示读者:“这个结论是基于证据的,任何智商正常的人都能看出来。”
            \item 它比直接说 \textit{"Clearly, he was lying"} 语气更正式、更有分量。
        \end{itemize}

        \item \textbf{语境差异:Obvious vs. Clear vs. Apparent}
        \begin{itemize}
            \item \textbf{Clear:} 最普通,指清楚、不含糊。
            \item \textbf{Obvious:} 更强烈,暗示“就在眼皮底下,想看不见都难”。
            \item \textbf{Apparent:} 侧重于“表面上看是这样”(但事实未必如此,带有一丝怀疑)。
        \end{itemize}
    \end{itemize}
\end{multicols}

\wsitem{At once vs. Immediately}
\begin{multicols}{1}
    这两个词都表示“立即、马上”,在很多语境下可以互换,但它们在语气强度、逻辑侧重和语体色彩上存在细微而关键的区别。
    \begin{enumerate}
        \item \textbf{语气的“急迫感” (Sense of Urgency)}
        \begin{itemize}
            \item \textbf{At once:} 语气更强,通常带有一种命令式或不容置疑的压力。它暗示“现在,一秒钟都不能耽误”。
            \item \textbf{Immediately:} 虽然也很快,但语气相对客观、职业,强调的是两个动作之间没有时间间隔。
        \end{itemize}

        \item \textbf{逻辑侧重:时间点 vs. 动作链}
        \begin{itemize}
            \item \textbf{At once:} 侧重于时间点。强调“当下这个瞬间”。
            \item \textbf{Immediately:} 侧重于因果衔接。常用于描述“一……就……”的连锁反应。
            \item \textbf{例子对比:}
            
            \es{Come here \textbf{at once}! (命令:立刻给我过来!)}

            \es{He answered the phone \textbf{immediately} it rang. (描述:电话一响他立刻就接了。)}
        \end{itemize}

        \item \textbf{语体色彩 (Register)}
        \begin{itemize}
            \item \textbf{At once:} 较多出现在口语命令或富有戏剧性的文学叙事中。
            \item \textbf{Immediately:} 更正式,常见于商务邮件、法律条款、科学实验描述或说明书。
        \end{itemize}
    \end{enumerate}
\end{multicols}

\wsitem{it would not prove difficult in ...}
\begin{multicols}{1}
    在英语排版和叙事逻辑中,"It would not prove difficult" 是一种非常委婉、正式且富有预测性的表达方式。
    
    这里使用 prove 而非简单的 be,以及使用 would 而非 will,都有其深层含义。

    \begin{enumerate}
        \item \textbf{Prove 的逻辑:事实的验证 (Verification of Truth)}
        \begin{itemize}
            \item \textbf{核心逻辑:} \textbf{Prove} 在这里作为系动词(Linking Verb),意思是“证明是……/结果是……”。
            \item \textbf{深层含义:} 它暗示了这不仅是一个判断,而是一个经过实践检验后会得出的结论。
            \item \textbf{对比:} 
            \begin{itemize}
                \item \textit{It will not be difficult.} (主观预测:不会难。)
                \item \textbf{It will not prove difficult.} (客观推断:事实将证明这并不难。)
            \end{itemize}
        \end{itemize}

        \item \textbf{Would 的语体色彩:假设与委婉 (Hypothesis and Softening)}
        \begin{itemize}
            \item \textbf{逻辑意义:} 这里的 \textbf{would} 表达的是一种对可能发生的情况的推测。在叙事中,它比 \textit{will} 听起来更客观、更有“智者”的姿态。
            \item \textbf{语境解析:} 作者在分析案情:在这样一个抬头不见低头见的小村子里,找一只羊“按理说”不应该难。
        \end{itemize}

        \item \textbf{句式结构:形式主语 (Formal Subject)}
        \begin{itemize}
            \item \textbf{Structure:} \textbf{It (形式主语) + would + prove + adj. + (to do sth.)}
            \item 这种结构让句子的重心落在 difficult 这个评价词上。
        \end{itemize}
    \end{enumerate}
\end{multicols}

\wsitem{had better do sth.}
\begin{multicols}{1}
    "Had better" 是一个非常独特的结构,虽然它看起来像过去时(had),但它表达的是针对现在或未来的建议、告诫甚至轻微的威胁。

    \begin{enumerate}
        \item \textbf{语法结构:情态动词逻辑}
        \begin{itemize}
            \item \textbf{公式:} \textbf{had better + do} (动词原形)。
            \item \textbf{否定式:} \textbf{had better NOT do} (注意 \textit{not} 的位置)。
            \item \textbf{缩略形式:} 口语中常缩略为 \textbf{'d better}。
            \item \textbf{例句:} You'd better \textbf{not be late.} (你最好别迟到。)
        \end{itemize}

        \item \textbf{逻辑张力:带有后果的建议 (Advice with Consequences)}
        \begin{itemize}
            \item \textbf{核心差异:} 
            \begin{itemize}
                \item \textbf{Should:} 只是建议,不听也没事。
                \item \textbf{Had better:} 强烈建议,并暗示如果不这么做,会有不好的事情发生。
            \end{itemize}
            \item \textbf{语感:} 它带有一种“警告”的意味。
        \end{itemize}

        \item \textbf{时间逻辑:伪过去式}
        \begin{itemize}
            \item 虽然用了 \textbf{had},但它永远不指过去。如果想说过去“最好做了某事”,要改用 \textit{It would have been better if...}。
        \end{itemize}
    \end{enumerate}
\end{multicols}

\wsitem{It was true that ..., but ....}
\begin{multicols}{1}
    "It was true that..." 是一个极具辩论色彩的句式。它通常用于先承认一个事实(让步),然后再引出更重要的反驳或转折。

    在逻辑学中,这被称为“承认与对比”策略。

    \begin{enumerate}
        \item \textbf{语法功能:让步性主语从句 (Concessive Subject Clause)}
        \begin{itemize}
            \item \textbf{逻辑意义:} 相当于“虽然……是真的” (\textit{Although it was true that...})。
            \item \textbf{结构:} \textbf{It (形式主语) + was + true + that-从句 (真实内容)}。
            \item \textbf{语用功能:} 作者通常用它来展示自己是公平客观的。我承认 A 是事实,但我要讨论的是 B。
        \end{itemize}

        \item \textbf{逻辑期待:等待 "But" 或 "However"}
        \begin{itemize}
            \item 当读者看到这一句开头,心理上会自动预期一个**转折**。
            
            \item \es{{It was true that} the lamb was missing, \textbf{but} it hadn't been stolen. (小羊失踪了是事实,但它并没有被偷。)}
        \end{itemize}

        \item \textbf{与 "It was obvious that" 的区别}
        \begin{itemize}
            \item \textbf{Obvious:} 强调一眼就能看出的结果,通常直接引出结论。
            \item \textbf{True:} 强调对某个观点的确认,通常是为了后续的“反转”做铺垫。
        \end{itemize}
        \item \textbf{高级表达进阶:承认事实的其他方式}
        \begin{itemize}
            \item \textbf{It was undeniable that...}
            
            \textbf{语感:} \textit{不可否认的是……} (语气极强,表示事实确凿,没有任何反驳余地)。
                
            \es{例句: \textbf{It was undeniable that} the gangsters had total control over the small town.(不可否认的是,黑帮曾经完全控制了那个小镇。)} 
            
            \item \textbf{It was granted that...}
            
            \textbf{语感:} \textit{诚然…… / 假定……是事实} (通常用于正式辩论,承认对方的一个前提点,以便引出自己的观点)。
                
            \es{例句:\textbf{It was granted that} he acted rashly, but his intention was to protect his family.(诚然,他的行为确实鲁莽,但他的意图是保护家人。)} 
            
            \item \textbf{Admittedly, ...}
            
            \textbf{语感:} \textit{必须承认的是……} (非常地道且简洁,常用于句首作为插入语)。
                
                \es{例句:\textbf{Admittedly}, the lamb looked like a thief in the darkness of the night.(必须承认,在黑夜中,那只小羊看起来确实像个小偷。)} 
            
            \item \textbf{There was no doubt that...}
            
            \textbf{语感:} \textit{毫无疑问的是……} (强调结论的确定性,通常用于总结一个明显的发现)。
                
            \es{例句:\textbf{There was no doubt that} the rain had done what Dimitri could not do: wash the wool clean.(毫无疑问,雨水做到了 Dimitri 做不到的事:把羊毛洗干净。)} 
        \end{itemize}
    \end{enumerate}
\end{multicols}

\wsitem{Ashamed of having acted so rashly, Dimitri apologized to Aleko for having accused him.}
\begin{multicols}{1}
    这里连续使用了两次 "having acted" 和 "having accused"(这些是 现在分词的完成式,本质上承载了过去完成时的逻辑),其核心原因在于:严格区分动作的先后顺序。

    \begin{enumerate}

        \item 逻辑顺序:先“做错”,后“道歉”
        \begin{itemize}
            \item \textbf{动作 A (先发生):} Dimitri 鲁莽行事 (\textit{acted rashly}),指责了 Aleko (\textit{accused him})。
            \item \textbf{动作 B (后发生):} Dimitri 感到羞愧 (\textit{ashamed}) 并道歉 (\textit{apologized})。
        \end{itemize}

        \item 语法解析:现在分词的完成式 (Having done)
        \begin{itemize}
            \item \textbf{为什么不用 having acting?}
            \begin{itemize}
                \item \textbf{Having acting} 语法错误。
                \item \textbf{Acting} (现在分词一般式) 表示动作与主句谓语(apologized)同时发生。如果用 \textit{Acting so rashly, Dimitri apologized},听起来像是他一边鲁莽行事,一边在道歉,逻辑不通。
            \end{itemize}
            \item \textbf{为什么用 having acted?}
            \begin{itemize}
                \item \textbf{Having + done} 专门用来强调该动作在主句动作(apologized)之前已经完成。
                \item \textbf{逻辑链条:} 因为他在“更早的过去”指责了朋友,所以他“后来”才道歉。
            \end{itemize}
        \end{itemize}

        \item 语境深度:因果与反思 (Cause and Reflection)

        使用完成式不仅是时间上的先后,还带有一种反思性。

        \begin{itemize}
            \item \textbf{Ashamed of having acted...} (为已经做过的鲁莽行为感到羞愧)
            \item \textbf{Apologized for having accused...} (为已经发生过的指责而道歉)
        \end{itemize}
        
        这种表达方式让读者的注意力集中在 Dimitri 的心理转变上——他回顾过去(完成式),对比现状(一般过去时),从而产生了悔意。

    \end{enumerate}
\end{multicols}

\wsitem{be astonished to find ...}
\begin{multicols}{1}
    "Be astonished to find..." 是一个极具戏剧色彩的结构,常用于描述那种出乎意料、令人瞠目结舌的发现。

    \begin{enumerate}
        \item \textbf{情感强度:Beyond Surprised}
        \begin{itemize}
            \item \textbf{Astonished 的逻辑:} 它的强度远高于 \textit{surprised}。如果 \textit{surprised} 是“没想到”,那么 \textbf{astonished} 就是“万万没想到”或“大吃一惊”。
            \item \textbf{语感:} 它常伴随着一种冲击感,仿佛大脑瞬间空白。
        \end{itemize}

        \item \textbf{语法逻辑:不定式作原因状语 (To-infinitive for Cause)}
        \begin{itemize}
            \item \textbf{结构:} \textbf{Subject + be + emotional adj. + to do sth.}
            \item \textbf{逻辑:} 后面的 \textit{to find} 是导致前面这种惊讶情感的原因。
            \item \textbf{翻译:} “发现……之后感到十分震惊”。
        \end{itemize}

        \item \textbf{连带动作:Find 的瞬间性}
        \begin{itemize}
            \item \textbf{Find} 在这里通常指“撞见”或“猛然意识到”某个事实。这种瞬间的发现与 astonished 的剧烈反应在节奏上非常契合。
        \end{itemize}
    \end{enumerate}
\end{multicols}

\wsitem{Get vs. Obtain vs. Acquire}
\begin{multicols}{1}
    \begin{enumerate}
        \item \textbf{Get (最通用、最口语)}
        \begin{itemize}
            \item \textbf{逻辑内核:} 一个中性词,强调“动作的结果”,不强调过程。无论你是买的、捡的、借的还是别人给的,都可以用 \textit{get}。
            \item \textbf{语境:} 日常生活。
            
            \es{I need to \textbf{get} some fifty pence pieces for the parking meter.}
        \end{itemize}

        \item \textbf{Obtain (更正式、强调努力)}
        \begin{itemize}
            \item \textbf{逻辑内核:} 强调通过特定的努力、申请、流程或途径才拿到的东西。它带有一种“好不容易弄到手”的感觉。
            \item \textbf{语境:} 官方文档、工作机会、许可证明。
            
            \es{He managed to \textbf{obtain} a permit to open his shop in the village.}
        \end{itemize}

        \item \textbf{Acquire (最正式、强调积累或收购)}
        \begin{itemize}
            \item \textbf{逻辑内核:} 强调一个缓慢、持续的获取过程,或者指通过购买/接管获得大宗财产。常用于描述“技能、名声、公司、知识”。
            \item \textbf{语境:} 商业收购、知识习得、长期声誉。
            
            \es{Over the years, the gangsters \textbf{acquired} a reputation for being ruthless.}
        \end{itemize}
    \end{itemize}
\end{multicols}

\grammarquestions

\wsitem{When Dimitri came in from the fields这句中,为什么用come in from the field?}
\begin{multicols}{1}
    在英语的逻辑中,come in(或者 go in)不仅仅是“进入”的意思,它通常带有一种**“由外而内”的空间转换感**,特别是指从开阔的户外进入私密的、有遮蔽的室内。

    \begin{enumerate}
        \item \textbf{空间的内外对比 (Outside vs. Inside)}
        \begin{itemize}
            \item \textbf{核心逻辑:} "Fields"(田野/野外)是开阔、无边界的外部空间;而 Dimitri 最终回到的地方(通常是家或农舍)是封闭的内部空间。
            \item \textbf{画面感:} \textbf{Come in} 捕捉到了他跨过门槛、进入屋内的那一刻。
        \end{itemize}

        \item \textbf{“归巢”的语义惯例 (Returning Home)}
        \begin{itemize}
            \item 在农耕语境下,人们习惯用 \textit{come in} 来表达“收工回家”。
            \item \textbf{例句对比:} 
            
            \es{He is working in the fields. (他在田里工作——强调位置)}

            \es{He has just \textbf{come in} from the fields. (他刚从田里回来——强调“进屋”这个动作)}
        \end{itemize}

        \item \textbf{介词的连用:In 与 From 的配合}
        \begin{itemize}
            \item \textbf{In:} 描述终点(室内)。
            \item \textbf{From:} 描述起点(田野)。
            \item 这个组合精准地勾勒出了一条移动轨迹:田野 \rightarrow 室内。
        \end{itemize}
    \end{enumerate}
\end{multicols}

\wsitem{Its wool, which had been dyed black, had been washed clean by the rain!这句中,动词为什么可以直接连接形容词?}
\begin{multicols}{1}
    在英语中,这种结构被称为结果补足语 (Resultative Complement)。
    
    简单来说,当一个动作直接导致了宾语状态的改变时,我们可以在动词后面直接加一个形容词,来描述那个“变出来的状态”。

    \begin{enumerate}
        \item \textbf{逻辑解析:动作 + 结果状态}
        \begin{itemize}
            \item \textbf{Dye it black (把它染成黑色)}
            \begin{itemize}
                \item \textbf{逻辑:} 动作是 \textit{dye}(染),其直接结果是羊毛变成了 \textit{black}(黑色)。
                \item \textbf{为什么不加介词:} 你不需要说 \textit{dye into black},因为 \textit{black} 在这里就是描述 \textit{wool} 在动作完成后的最终属性。
            \end{itemize}

            \item \textbf{Wash it clean (把它洗干净)}
            \begin{itemize}
                \item \textbf{逻辑:} 动作是 \textit{wash}(洗),其直接结果是羊毛变成了 \textit{clean}(干净的)。
                \item \textbf{对比:} 如果只说 \textit{The rain washed the wool},我们只知道雨淋了羊毛;加上 \textbf{clean},我们就知道了雨淋之后的程度和结果。
            \end{itemize}
        \end{itemize}

        \item \textbf{为什么不用副词 (Cleanly/Blackly)?}
        
        这是很多学习者的误区。请看逻辑差异:

        \begin{itemize}
            \item \textbf{副词修饰动作}
            
            \es{He washed the wool \textbf{carefully}. (他洗得很仔细。—— 描述的是“洗”的方式,没说洗没洗干净。)}
            
            \item \textbf{形容词描述结果}
            
            \es{He washed the wool \textbf{clean}. (他把羊毛洗干净了。—— 描述的是“羊毛”的状态,它是干净的。)}
        \end{itemize}

    \end{enumerate}
\end{multicols}

\wsitem{... is missing,为什么用现在进行时?}
\begin{multicols}{1}
    这是一个非常经典的语法陷阱。在英语逻辑中,missing 在这里并不是动词 miss 的进行时,而是一个形容词,表示“失踪了”或“不在原处”。
    \begin{enumerate}
        \item \textbf{词性本质:形容词 (Missing) vs. 动词 (Missed)}
        \begin{itemize}
            \item \textbf{Was missing:} 这里的 \textbf{missing} 是形容词,描述的是一种状态。
            \item \textbf{逻辑意义:} 当主人去找小羊时,发现它“处于失踪的状态”。这强调的是一个客观事实:羊不在那儿。
            \item \textbf{Had missed:} 这是动词 \textit{miss} 的过去完成时,表示“想念”或“错过”。
            \item \textbf{逻辑冲突:} 如果说 \textit{The lamb had missed},听起来像是“这只小羊曾经想念过某人”或者“这只小羊错过了一班公交车”。这完全不符合语境。
        \end{itemize}

        \item \textbf{状态的发现 (The Discovery of a State)}
        \begin{itemize}
            \item \textbf{Was missing} 配合 \textbf{One evening},营造了一种“在那一刻发现它不见了”的画面感。
            \item 在英语中,描述某人/某物不见了,标准表达就是 \textbf{be missing}。
            
            \es{The child \textbf{is missing}. (孩子失踪了。)}

            \es{A page \textbf{is missing} from the book. (书里缺了一页。)}
        \end{itemize}

        \item \textbf{语态的逻辑:主动感 vs. 被动感}
        \begin{itemize}
            \item \textbf{Missing} 虽然长得像进行时,但它描述的是主体本身的缺失状态。
            \item 如果你想用动词表达,通常会用 \textbf{was lost} (丢了) 或 \textbf{had disappeared} (消失了)。
        \end{itemize}
    \end{enumerate}
\end{multicols}

\newpage