\section{Lesson 18 Electric currents in modern}

\begin{paracol}{2}

\newsentence{Modern sculpture rarely surprises us any more. }

\switchcolumn

\chinesetext{现代雕塑不再使我们感到惊讶了。}

\switchcolumn*

\newsentence{The idea that modern art can only be seen in museums is mistaken. }

\switchcolumn

\chinesetext{那种认为现代艺术只能在博物馆里才能看到的观点是错误的。}

\switchcolumn*

Even people who \ns{take no interest in} art \newsentence{cannot have failed to} notice examples of modern sculpture on display in public places. 

\switchcolumn

\chinesetext{即使是对艺术不感兴趣的人也不会注意到在公共场所展示的现代艺术品。}
\nse{take no interest in}{}{对…不[不太]感兴趣;}

\switchcolumn*

Strange forms stand in gardens, and outside buildings and shops. 

\switchcolumn

\chinesetext{公园里、大楼和商店外竖立着的奇形怪状的雕塑。}

\switchcolumn*

\newsentence{We have \ns{got quite used to} them. }

\switchcolumn

\chinesetext{对这些,我们已经司空见惯了。}
\nse{get used to}{}{习惯于;}

\switchcolumn*

Some \nw{so-called} 'modern' pieces have been on display for nearly eighty years.

\switchcolumn

\chinesetext{有些所谓的“现代”艺术品在那里已经陈列了近80年了。}
\nwe{so-called}{ˌsoʊ ˈkɔːld}{adj. 所谓的,号称的;}

\switchcolumn*

\ns{In spite of} this, some people -- including myself -- were surprise by a recent exhibition of modern sculpture. 

\switchcolumn

\chinesetext{尽管如此,最近举办的一次现代雕塑展览还是使一些人(包括我在内)大吃了一惊。}
\nse{in spite of}{}{不管;尽管;}

\switchcolumn*

\newsentence{The first thing I saw when I entered the \ns{art gallery} was a notice which said: 'Do not touch the exhibits. Some of them are dangerous!' }

\switchcolumn

\chinesetext{走进展厅首先看到的是一张告示,上面写着“切勿触摸展品,某些展品有危险!”}
\nse{art gallery}{ɑrt ˈɡæləri}{n. 艺廊;画廊;}

\switchcolumn*

The objects on display were pieces of moving sculpture. 

\switchcolumn

\chinesetext{展品都是些活动的雕像。}

\switchcolumn*

Oddly shaped forms that are suspended from the \nw{ceiling} and move \ns{in response to} \ns{a gust of wind} \ns{are quite familiar to} everybody. 

\switchcolumn

\chinesetext{人们所熟悉的是悬挂在天花板上、造型奇特、随风飘荡的雕塑品。}
\nwe{ceiling}{ˈsiːlɪŋ}{n. 天花板,顶棚;上限;最高飞行限度;}
\nse{in response to}{}{对…做出反应;}
\nse{a gust of wind}{e ɡʌst ʌv wɪnd}{一阵狂风;}
\nse{be familiar to ...}{bi fəˈmɪljɚ tu}{被某人熟悉;}

\switchcolumn*

These objects, however, were different. 

\switchcolumn

\chinesetext{这些展品却使人大开眼界。}

\switchcolumn*

\newsentence{Lined up against the wall}, there were long thin wires attached to metal \nw{spheres}. 

\switchcolumn

\chinesetext{靠墙排列着许多细长的电线,而电线又连着金属球。}
\nwe{sphere}{sfɪr}{n. 球(体);(兴趣或活动的)范围;势力范围;天体,如行星或恒星;v. 将…封入球体中;包围;}

\switchcolumn*

The spheres had been \nw{magnetized} and \newsentence{attracted or \nw{repelled} each other \ns{all the time}}. 

\switchcolumn

\chinesetext{金属球经过磁化,互相之间不停地相互吸引或相互排斥。}
\nwe{magnetize}{ˈmæɡnɪˌtaɪz}{vt. 使有磁性;使磁化;紧紧吸引;迷住;}
\nwe{repel}{rɪˈpel}{vt. 击退;使厌恶;抵制;使不愉快;排斥,(磁极间) 相斥;}
\nse{all the time}{}{(在该段时间内)一直;向来, 一向;时时刻刻;每时每刻;}

\switchcolumn*

In the centre of the hall, there were a number of tall structures which contained coloured lights. 

\switchcolumn

\chinesetext{展厅中央是装有彩色灯泡的许多高高的构件。}

\switchcolumn*

These lights \nw{flickered} continuously like \ns{traffic lights} which have gone mad. 

\switchcolumn

\chinesetext{灯泡一刻不停地闪烁着,就像失去了控制的红绿灯。}
\nwe{flicker}{ˈflɪkər}{v. 闪烁;飘扬,摆动;昏倒;n. 闪烁;闪光;电影;假装昏倒的乞丐;}
\nse{traffic light}{ˈtræfɪk laɪt}{n. 红绿灯;}

\switchcolumn*

Sparks were \nw{emitted} from small black boxes and red \nw{lamps} flashed on and off angrily. 

\switchcolumn

\chinesetext{小黑盒子里迸出火花,红色灯泡发怒似地忽明忽暗。}
\nwe{emit}{iˈmɪt}{v. 发出;散发(光、热等);}
\nwe{lamp}{læmp}{n. 灯;发热灯,照射灯;}

\switchcolumn*

It was rather like an exhibition of \nw{prehistoric} electronic equipment. 

\switchcolumn

\chinesetext{这儿倒像是在展览古老的电子设备。}
\nwe{prehistoric}{ˌpriːhɪˈstɔːrɪk}{adj. 史前的;陈旧的;}

\switchcolumn*

These \nw{peculiar} forms not only seemed designed to shock people emotionally, but to give them electric shocks as well!

\switchcolumn

\chinesetext{好像设计这些奇形怪状的展品不仅是为了给人感情上的强烈刺激,而且还想给人以电击似的!}
\nwe{peculiar}{pɪˈkjuːliər}{adj. 奇怪的,古怪的;异常的;特有的,特殊的;}

\switchcolumn*

\end{paracol}

\grammarpoints

\wsitem{... rarely ... any more}
\begin{multicols}{1}
    \begin{enumerate}
        \item \textbf{句法结构拆解 (Syntactic Breakdown)}
        \begin{itemize}
            \item \textbf{Subject (主语):} \textit{Modern sculpture} (现代雕塑) —— 这是一个抽象名词短语,作为动作的发起者。
            \item \textbf{Adverb (程度状语):} \textit{rarely} (很少/罕见) —— 这是一个**准否定词**(Semi-negative),奠定了全句的否定基调。
            \item \textbf{Verb (谓语):} \textit{surprises} (使……惊讶) —— 使用一般现在时,描述一种普遍存在的现状或规律。
            \item \textbf{Object (宾语):} \textit{us} (我们) —— 动作的承受者,即普通大众。
            \item \textbf{Adverbial (时间状语):} \textit{any more} (再,更) —— 与 \textit{rarely} 呼应,强调这种状态的持续和改变。
        \end{itemize}
        \item \textbf{核心逻辑点解析}
        \begin{itemize}
            \item \textbf{"Rarely" 与 "Any more" 的“双重否定”效果}
            
            在英语中,虽然 rarely 本身不是 not,但它的逻辑含义是负面的。

            \begin{itemize}
                \item 通常我们说 not... any more 表示“不再……”。
                \item 这里用 rarely... any more 表达的是一种更微妙的语气:并不是说“完全不让我们惊讶了”,而是说“那种让我们感到惊讶的情况几乎不再发生了”。
                \item 这种表达比单纯的否定(doesn't surprise us)更具文采,也更符合事实的灰度。
            \end{itemize}
            \item \textbf{心理预期与现状的对比}
            这个句子暗示了一个前提:过去的雕塑经常让我们感到惊讶(也许是因为激进的形式或材料)。而现在,我们的审美阈值变高了,或者现代艺术变得套路化了,因此“惊喜”变得稀缺。
        \end{itemize}
        \item \textbf{高级表达进阶:频率副词的逻辑梯队}
        
        在写作中,选择不同的副词可以精准控制你“否定”的程度:

        \begin{itemize}
            \item \textbf{Seldom / Rarely... any more}
            \begin{itemize}
                \item \textbf{语义:} 几乎不再…… (强调频率极低)。
                \item \textbf{例句:} People \textbf{rarely write long letters to each other \textbf{any more}.}
            \end{itemize}
            
            \item \textbf{Hardly ever... any more}
            \begin{itemize}
                \item \textbf{语义:} 简直从不…… (语气比 rarely 更重,接近于零)。
                \item \textbf{例句:} He \textbf{hardly ever visits his hometown \textbf{any more}.}
            \end{itemize}
            
            \item \textbf{Scarcely... any more}
            \begin{itemize}
                \item \textbf{语义:} 几乎不…… (常带有“勉强、不足”的意味)。
                \item \textbf{例句:} There is \textbf{scarcely any hope of finding the missing lamb \textbf{any more}.}
            \end{itemize}
        \end{itemize}
    \end{enumerate}
\end{multicols}

\wsitem{The idea that ... is mistaken.}
\begin{multicols}{1}
    这个句子是一个非常经典的\textbf{名词从句(同位语从句)}结构,常用于批驳某种普遍存在的错误观点。它的结构清晰、逻辑严密,是议论文写作的模范句式。

    \begin{enumerate}
        \item \textbf{句法结构拆解 (Syntactic Breakdown)}
        \begin{itemize}
            \item \textbf{Subject (主语):} The idea (观点/想法) —— 句子的核心主语。
            \item \textbf{Appositive Clause (同位语从句):} that modern art can only be seen in museums
            \begin{itemize}
                \item \textbf{功能:} 紧跟在 \textit{idea} 后面,具体解释这个“观点”的内容是什么。
                \item \textbf{连接词:} that 在这里不充当成分,只起连接作用,且不能省略。
            \end{itemize}
            \item \textbf{Predicate (谓语):} is (是) —— 系动词。
            \item \textbf{Predictive (表语):} mistaken (错误的) —— 形容词,对主语 \textit{The idea} 进行定性。
        \end{itemize}
        \item \textbf{核心逻辑点解析}
        \begin{itemize}
            \item \textbf{批驳式开篇 (The Refutation Strategy)}
            
            这个句子采用了“\textbf{先树靶子,再打靶子}”的策略。

            \begin{itemize}
                \item 它先提出了一个普遍的社会认知(艺术只能在博物馆看)。
                \item 然后迅速用一个极其简洁的词 \textbf{mistaken} 予以否定。这种“头重脚轻”的结构(复杂的想法 + 简单的结论)能产生强烈的视觉和心理反差,增强说服力。
            \end{itemize}

            \item "Only" 的限定作用
            
            从句中的 only 是作者批驳的关键。作者并不是否定博物馆有艺术,而是否定“\textbf{只能 (only)}”在博物馆看。这为下文引出街道、广场或大桥上的现代雕塑做了逻辑铺垫。
        \end{itemize}
        \item \textbf{高级表达进阶:同类批驳句式}
        
        如果你想表达“某种观点是错误的”,可以根据语气的轻重选择以下句式:
        \begin{enumerate}
            \item \textbf{The notion that... is a common misconception.}
            \begin{itemize}
                \item \textbf{语感:} \textit{认为……的观点是一个普遍的误区} (比 \textit{mistaken} 更正式,带有科普或纠正意味)。
                \item \textbf{例句:} The notion \textbf{that modern sculpture is easy to create is a common \textbf{misconception}.}
            \end{itemize}
            
            \item \textbf{The belief that... holds no water.}
            \begin{itemize}
                \item \textbf{语感:} \textit{认为……的信念是站不住脚的} (使用比喻,语气较强,暗示逻辑有漏洞)。
                \item \textbf{例句:} The belief \textbf{that art must be beautiful \textbf{holds no water} in the 21st century.}
            \end{itemize}
            
            \item \textbf{It is a fallacy to assume that...}
            \begin{itemize}
                \item \textbf{语感:} \textit{假定……是一种谬论} (语气最重,通常用于严谨的辩论)。
                \item \textbf{例句:} It is a \textbf{fallacy to assume \textbf{that} all modern art is meaningless.}
            \end{itemize}
        \end{itemize}
    \end{enumerate}
\end{multicols}

\wsitem{We have got quite used to them.}
\begin{multicols}{1}
    这个句子看似简单,实际上包含了口语中非常地道的语气助词和固定搭配。它描述了一种**“从陌生到习以为常”**的心理转变。

    \begin{enumerate}
        \item \textbf{句法结构拆解 (Syntactic Breakdown)}
        \begin{itemize}
            \item \textbf{Subject (主语):} We (我们) —— 泛指大众或说话者群体。
            \item \textbf{Predicate (谓语):} have got used to
            \begin{itemize}
                \item \textbf{have got:} 这里是现在完成时,强调动作的结果,即“现在已经处于某种状态”。
                \item \textbf{get used to:} 动作短语,表示“变得习惯于……”。相比 \textit{be used to}(强调习惯的状态),\textit{get} 强调了**从不习惯到习惯的转变过程**。
            \end{itemize}
            \item \textbf{Adverb (程度状语):} quite (相当地) —— 用来修饰形容词化的 \textit{used},加强语气。
            \item \textbf{Object of Preposition (介词宾语):} them (它们) —— 代词,在文中指代前文提到的“现代雕塑”(modern sculptures)。
        \end{itemize}
        \item \textbf{核心逻辑与语法点 (Core Logical Points)}
        \begin{itemize}
            \item \textbf{"Get used to" 的语法要求}
            \begin{itemize}
                \item \textbf{结构:} get used to + n./pron./doing sth.
                \item \textbf{注意:} 这里的 to 是**介词**,所以后面必须接名词、代词或动名词。
                \item \textbf{例句:} I \textbf{got used to the cold weather.} (我变得习惯这种寒冷的天气了。)
            \end{itemize}
            
            \item \textbf{"Quite" 的修饰作用}
            \begin{itemize}
                \item \textbf{位置:} 放在 \textit{used} 之前。
                \item \textbf{功能:} 表达一种“颇为充分”的程度。它暗示这种习惯过程已经完成得相当彻底,不再感到突兀或不适。
            \end{itemize}
            
            \item \textbf{语境中的“心理防御”}
            \begin{itemize}
                \item 结合前文“现代雕塑不再让我们惊讶”,这句话解释了原因:因为我们已经**习以为常**了。这种心理过程被称为“脱敏”(Desensitization)。
            \end{itemize}
        \end{itemize}
        \item \textbf{高级表达进阶:描述“习惯”的不同层次}
        \begin{itemize}
            \item \textbf{Take ... for granted (想当然地认为/视……为理所当然)}
            \begin{itemize}
                \item \textbf{语感:} 习惯到了完全忽略其存在的程度,甚至不再感激。
                \item \textbf{例句:} We have \textbf{taken} these sculptures \textbf{for granted}.
            \end{itemize}
            
            \item \textbf{Become accustomed to (习惯于)}
            \begin{itemize}
                \item \textbf{语感:} 比 \textit{get used to} 更加正式,常用于文学或学术语境。
                \item \textbf{例句:} The public has slowly \textbf{become accustomed to} these strange shapes.
            \end{itemize}
            
            \item \textbf{Acclimatize oneself to (使自己适应/服水土)}
            \begin{itemize}
                \item \textbf{语感:} 通常指适应新的环境、气候或极端的条件。
                \item \textbf{例句:} Sailors must \textbf{acclimatize themselves to} the \textbf{icy waters} of the North Atlantic.
            \end{itemize}
        \end{itemize}
    \end{enumerate}
\end{multicols}

\wsitem{The first thing sb. saw when ... is ...}
\begin{multicols}{1}
    这是一个非常经典的引导式叙述句型。它通过限定时间(when)和观察顺序(the first thing),迅速将读者的注意力锁定在某个特定的视觉焦点上。

    \begin{enumerate}
        \item \textbf{句法结构总览 (Overall Syntactic Structure)}
        \begin{itemize}
            \item \textbf{Subject (主语部分):} The first thing (that) sb. saw [when...]
            \begin{itemize}
                \item \textbf{The first thing:} 核心主语。
                \item \textbf{(That) sb. saw:} 定语从句,修饰 thing。在口语和非正式书面语中 that 常省略。
                \item \textbf{When-clause:} 时间状语从句,交代观察发生的背景或瞬间。
            \end{itemize}
            
            \item \textbf{Predicate (谓语):} is / was
            \begin{itemize}
                \item 注意时态一致:如果前面是 saw,后面通常用 was。
            \end{itemize}
            
            \item \textbf{Predictive (表语):} A noun / A clause
            \begin{itemize}
                \item 交代具体看到的那个事物。
            \end{itemize}
        \end{itemize}
        \item \textbf{核心逻辑点解析 (Core Logical Points)}
        \begin{itemize}
            \item \textbf{视觉第一印象 (First Impression)}
            \begin{itemize}
                \item 这个句式强调了“视觉冲击力”。它暗示该事物非常巨大、鲜艳或突兀,以至于观察者根本无法忽略。
            \end{itemize}
            
            \item \textbf{叙事节奏的铺垫}
            \begin{itemize}
                \item 在文学描写中,这通常是一个“由面到点”的过程:先交代大环境(when I entered...),再抛出视觉重心(the first thing... was...)。
            \end{itemize}
            
            \item \textbf{强调效果}
            \begin{itemize}
                \item 相比于直接说 \textit{"I saw a dog when I opened the door,"} 使用 \textit{"The first thing I saw... was a dog"} 更加强调“发现”的瞬间感。
            \end{itemize}
        \end{itemize}
        \item \textbf{高级表达进阶:变换“感官”与“重点”}
        \begin{itemize}
            \item \textbf{What struck sb. first was... (首先击中某人的是……)}
            \begin{itemize}
                \item \textbf{语感:} 语气更强烈,常用于描述震撼的景观或深刻的印象。
                \item \textbf{例句:} \textbf{What struck me first when I saw the bridge \textbf{was} its immense size.}
            \end{itemize}
            
            \item \textbf{The first sight to greet sb. was... (迎接某人的第一道景观是……)}
            \begin{itemize}
                \item \textbf{语感:} 具有拟人化色彩,常用于旅游文学。
                \item \textbf{例句:} \textbf{The first sight to greet the sailors \textbf{was} the towering statue.}
            \end{itemize}
            
            \item \textbf{Sb. was immediately confronted by... (某人立刻面对着……)}
            \begin{itemize}
                \item \textbf{语感:} 带有一定的被动感,暗示事物是不请自来地出现在眼前。
                \item \textbf{例句:} Upon entering the exhibition, we \textbf{were immediately confronted by oddly shaped forms.}
            \end{itemize}
        \end{itemize}
    \end{enumerate}
\end{multicols}

\wsitem{Attracted or repelled each other}
\begin{multicols}{1}
    \begin{enumerate}
        \item \textbf{Exert a force on each other (相互施加压力/作用力)}
        
        \es{The two objects \textbf{exerted a gravitational force on each other}.}
        
        \item \textbf{Be in a state of constant motion (处于持续运动状态)}
        
        \es{Because of the magnets, the sculpture was \textbf{in a state of constant motion}.}
        
        \item \textbf{Pull and push (拉与推)}
        
        \es{It was as if invisible hands were \textbf{pulling and pushing} the spheres \textbf{all the time}.}
    \end{itemize}
\end{multicols}

\wsitem{It was rather like ...}
\begin{multicols}{1}
    "It was rather like..." 是一个非常具有文学色彩且生动形象的句式。它用于将一个复杂、庞大或难以理解的事物,类比成一个大家熟悉的事物,从而增加文字的画面感。
    
    这里的 rather 起到了“相当地、某种程度上”的修饰作用,使语气显得委婉且地道。

    \begin{enumerate}
        \item \textbf{逻辑功能:建立类比 (Creating an Analogy)}
        \begin{itemize}
            \item \textbf{核心逻辑:} 当作者描述维拉扎诺大桥这种宏伟建筑时,直接给数据(如 4,260 英尺)可能不够直观。通过 \textit{"It was rather like..."},作者可以把大桥比作一个巨大的秋千或者其他日常事物。
            \item \textbf{语用:} 它能拉近读者与枯燥事实之间的距离。
        \end{itemize}
        
        \item \textbf{词汇拆解}
        \begin{itemize}
            \item \textbf{Rather:} 副词,在这里表示“颇为、相当”,用来缓和语气。
            \item \textbf{Like:} 介词,意为“像……一样”。
            \item \textbf{对比:} 
            \begin{itemize}
                \item \textit{It was like a dream.} (普通:像一场梦。)
                \item \textit{It was \textbf{rather like} a dream.} (更地道:颇有点像一场梦。)
            \end{itemize}
        \end{itemize}
    \end{enumerate}

    \textbf{高级表达进阶:描述“相似性”的同类句式}
    在写作中,如果你想换种方式打比方,可以参考以下句式:
    \begin{enumerate}
        \item \textbf{It bore a resemblance to...}
        \begin{itemize}
            \item \textbf{语感:} \textit{它与……有相似之处} (非常正式的表达,常用于描述外观)。
            \item \textbf{例句:} The shape of the towers \textbf{bore a resemblance to giant gateways.}
        \end{itemize}
        
        \item \textbf{It might be compared to...}
        \begin{itemize}
            \item \textbf{语感:} \textit{它可以被比作……} (常用于逻辑推导或功能类比)。
            \item \textbf{例句:} The complex cable system \textbf{might be compared to} a giant spider web.
        \end{itemize}
        
        \item \textbf{It was not unlike...}
        \begin{itemize}
            \item \textbf{语感:} \textit{它并非不像…… (即:它很像)} (双重否定,语气更加委婉、考究)。
            \item \textbf{例句:} Standing on top of the tower \textbf{was not unlike} being on a mountain peak.
        \end{itemize}
    \end{enumerate}
\end{multicols}

\wsitem{Lined up against ...}
\begin{multicols}{1}
    这是一个非常具有画面感的过去分词短语,常用于文学描写中,用来设定一个静态的背景。它的核心逻辑是描述一系列物体被“整齐地靠墙排列”的状态。

    \begin{enumerate}
        \item \textbf{句法结构总览 (Overall Syntactic Structure)}
        \begin{itemize}
            \item \textbf{Syntactic Role (句法角色):} Past Participle Phrase (过去分词短语)
            \begin{itemize}
                \item 这是一个典型的被动含义短语。它通常作为**状语**(描述主语的状态)或**后置定语**。
            \end{itemize}
            
            \item \textbf{Core Verb (核心动词):} Line up
            \begin{itemize}
                \item \textbf{Line up:} 排队、排成行。
                \item \textbf{Lined up:} 被排列成行。
            \end{itemize}
            
            \item \textbf{Prepositional Phrase (介词短语):} Against the wall
            \begin{itemize}
                \item \textbf{Against:} 这里的含义是“背对着”或“紧靠着”。它强调了物体与墙壁之间的支撑或位置关系。
            \end{itemize}
        \end{itemize}
        \item \textbf{高级表达进阶:描述“排列与位置”}
        \begin{itemize}
            \item \textbf{Arranged in a row (成排布置)}
            \begin{itemize}
                \item \textbf{语感:} 更加正式,强调有目的的组织。
            \end{itemize}
            
            \item \textbf{Flanking the entrance (侧立于入口两旁)}
            \begin{itemize}
                \item \textbf{语感:} 描述左右对称的位置关系,常用于建筑描写。
            \end{itemize}
            
            \item \textbf{Positioned at regular intervals (等距离安置)}
            \begin{itemize}
                \item \textbf{语感:} 极具工程感或科学精确性。
            \end{itemize}
        \end{itemize}
    \end{enumerate}
\end{multicols}

\wsitem{Cannot have failed to ...}
\begin{multicols}{1}
    "Cannot have failed to..." 是英语中语气最强、最地道的双重否定表达之一。它通过“推测”和“否定”的结合,传达出一种“绝对肯定”的必然性。

    \begin{enumerate}
        \item \textbf{句法结构总览 (Overall Syntactic Structure)}
        \begin{itemize}
            \item \textbf{Syntactic Breakdown:} Modal (cannot) + Perfect (have done) + Negative Verb (fail)
            \begin{itemize}
                \item \textbf{Cannot have done:} 对过去或现状的强力推测,意为“不可能已经……”。
                \item \textbf{Fail to do:} 未能做成某事/漏掉某事。
                \item \textbf{Double Negative Logic:} Not (cannot) + Not (fail) = \textbf{Must} (必定)。
            \end{itemize}
            
            \item \textbf{Meaning:} 相当于 \textit{"must have noticed/seen/done"},但语气更加肯定且带有一种“不言自明”的逻辑压力。
        \end{itemize}
        \item \textbf{核心逻辑点解析 (Core Logical Points)}
        \begin{itemize}
            \item \textbf{强调“显而易见” (Highlighting Obviousness)}
            \begin{itemize}
                \item 该句式通常用于描述那些体积巨大、色彩鲜艳或影响深远的事物。
                \item \textbf{隐含前提:} 除非观察者完全心不在焉或视而不见,否则必然会注意到。
            \end{itemize}
            
            \item \textbf{语气的优越感 (Rhetorical Tone)}
            \begin{itemize}
                \item 在议论文中,使用此句式能迫使读者承认某个事实。它暗示:“如果你没注意到,那说明你太不留心了。”
            \end{itemize}
            
            \item \textbf{时态的固定性}
            \begin{itemize}
                \item 习惯上使用 \textit{have failed},因为它描述的是直到说话那一刻为止,对方“不可能还没有”经历过某事。
            \end{itemize}
        \end{itemize}
        \item \textbf{高级表达进阶:双重否定的不同变体 }
        \begin{itemize}
            \item \textbf{It is impossible not to... (不……是不可能的)}
            
            \es{It is \textbf{impossible not to} admire the courage of the bridge builders.}
            
            \item \textbf{There is no one but... (没有人不……)}
            
            \es{There is \textbf{no one} who has seen the bridge \textbf{but} admires it.}
            
            \item \textbf{Never ... without ... (每逢……必……)}
            
            \es{I \textbf{never} cross this bridge \textbf{without} thinking of the work involved.}
            
        \end{itemize}
    \end{enumerate}
\end{multicols}

\wsitem{Even people who take no interest in art cannot have failed to notice examples of modern sculpture on display in public places.}
\begin{multicols}{1}
    这个句子是《新概念英语》中非常经典的长难句,采用了“双重否定”和“复杂定语从句”来表达一个肯定的事实:现代雕塑在公共场所随处可见,谁都无法忽视。

    \begin{enumerate}
        \item \textbf{句法结构总览 (Overall Syntactic Structure)}
        \begin{itemize}
            \item \textbf{Subject (主语):}Even people [who take no interest in art] 
            \begin{itemize}
                \item \textbf{Even:} 副词,表示甚至,用于加强语气,强调连“最不可能的人”也不例外。
                \item \textbf{People:} 句子的核心主语。
                \item \textbf{Who-从句:} 限定性定语从句,修饰 people,界定了这群人的特征(对艺术没兴趣)。
            \end{itemize}
            
            \item \textbf{Predicate (谓语):} cannot have failed to notice
            \begin{itemize}
                \item \textbf{Double Negative:} cannot 与 failed 构成双重否定。
                \item \textbf{Modal Verb:} cannot have done 表示对过去或现状的强力推测(不可能已经……)。
            \end{itemize}
            
            \item \textbf{Object (宾语):} examples of modern sculpture
            \begin{itemize}
                \item \textbf{Core Object:} examples (例子/实例)。
                \item \textbf{Modifier:} of modern sculpture (现代雕塑的)。
            \end{itemize}
            
            \item \textbf{Adverbial (状语):} on display in public places
            \begin{itemize}
                \item \textbf{On display:} 介词短语作后置定语,修饰雕塑(正在展出的)。
                \item \textbf{In public places:} 地点状语(在公共场所)。
            \end{itemize}
        \end{itemize}
        \item \textbf{核心逻辑点深度解析 (Core Logical Points)}
        \begin{itemize}
            \item \textbf{双重否定的威慑力 (The Power of Double Negative)}
            \begin{itemize}
                \item \textbf{公式:} Cannot + fail to do = Must have done。
                \item \textbf{逻辑逻辑:} “不可能没注意到” $\Rightarrow$ “肯定已经注意到了”。
                \item \textbf{修辞效果:} 这种表达比直接说 Everyone must have noticed 更有冲击力,它暗示这些雕塑非常显眼,除非你是瞎子,否则不可能看不见。
            \end{itemize}
            
            \item \textbf{动词短语的习惯搭配 (Idiomatic Usage)}
            \begin{itemize}
                \item \textbf{Take no interest in:} 对……不感兴趣。这是 take an interest in 的否定形式。
                \item \textbf{On display:} 固定搭配,意为“展出中”。
            \end{itemize}
            
            \item \textbf{语气推测:Cannot have done}
            \begin{itemize}
                \item 这里的 cannot have done 不是指“没能力做”,而是指根据逻辑常识,某种情况“绝无可能发生”。
            \end{itemize}
        \end{itemize}
        \item \textbf{高级句式进阶与替换 (Advanced Variations)}
        \begin{itemize}
            \item \textbf{替换“对……不感兴趣”}
            \begin{itemize}
                \item Even those who are \textbf{indifferent to art...} (甚至那些对艺术漠不关心的人……)
            \end{itemize}
            
            \item \textbf{替换“双重否定”}
            \begin{itemize}
                \item It is \textbf{impossible} for them \textbf{not to} notice... (对他们来说,不注意到……是不可能的。)
            \end{itemize}
            
            \item \textbf{替换“随处可见”}
            \begin{itemize}
                \item ...cannot have failed to notice \textbf{ubiquitous} examples of modern sculpture. (\dots 不可能没注意到那些无处不在的现代雕塑。)
            \end{itemize}
        \end{itemize}
    \end{enumerate}
\end{multicols}

\wsitem{Oddly shaped forms that are suspended from the ceiling and move in response to a gust of wind are quite familiar to everybody. }
\begin{multicols}{1}
    这个句子是典型的“长主语+短谓语”结构,通过复杂的修饰成分描述了现代艺术中一种常见的形式——动态雕塑(Mobile)。
    \begin{enumerate}
        \item \textbf{句法结构总览 (Overall Syntactic Structure)}
        \begin{itemize}
            \item \textbf{Subject (主语):} Oddly shaped forms [that are suspended... and move...]
            \begin{itemize}
                \item \textbf{Oddly shaped forms:} 核心主语,意为“形状奇特的物体”。
                \item \textbf{That-从句:} 限定性定语从句,修饰 forms。
            \end{itemize}
            
            \item \textbf{Predicate (谓语):} are
            \begin{itemize}
                \item 使用一般现在时,陈述一个普遍的事实。
            \end{itemize}
            
            \item \textbf{Predictive (表语):} \textit{quite familiar to everybody}
            \begin{itemize}
                \item \textbf{Quite:} 程度副词,意为“相当地”。
                \item \textbf{Familiar to:} 固定搭配,意为“为某人所熟悉”。
            \end{itemize}
        \end{itemize}
        \item \textbf{定语从句内部逻辑解析 (Logic inside the Relative Clause)}
        \begin{itemize}
            \item \textbf{Are suspended from the ceiling}
            \begin{itemize}
                \item \textbf{Suspend from:} 悬挂于……。这与我们之前讨论大桥缆索的 \textbf{from which it is suspended} 逻辑一致。
                \item \textbf{Ceiling:} 天花板。这交代了物体的空间位置。
            \end{itemize}
            
            \item \textbf{Move in response to a gust of wind}
            \begin{itemize}
                \item \textbf{In response to:} 响应…… / 对……做出反应。
                \item \textbf{A gust of wind:} 一阵风。这是一个非常地道的量词表达,形容风突如其来的动作。
                \item \textbf{物理逻辑:} 强调这些物体不是靠电力驱动,而是随自然的空气流动而摆动。
            \end{itemize}
        \end{itemize}
        \item \textbf{词汇与表达进阶 (Advanced Vocabulary)}
        \begin{itemize}
            \item \textbf{Oddly shaped (形状奇特的)}
            \begin{itemize}
                \item \textbf{Oddly} 是副词,修饰过去分词 \textbf{shaped}。
                \item \textbf{同义替换:} \textbf{Bizarrely shaped} 或 \textbf{Unconventionally formed}。
            \end{itemize}
            
            \item \textbf{Familiar to vs. Familiar with}
            \begin{itemize}
                \item \textbf{Something is familiar to sb.:} 某事对某人来说是熟悉的(物作主语)。
                \item \textbf{Sb. is familiar with sth.:} 某人熟悉某事(人作主语)。
            \end{itemize}
        \end{itemize}
    \end{enumerate}
\end{multicols}

\newpage