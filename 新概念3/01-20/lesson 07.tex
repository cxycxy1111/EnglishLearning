\section{Lesson 7 Mutilate Ladies}

\begin{paracol}{2}

Has it ever happened to you?

\switchcolumn

\switchcolumn*

Have you ever put your \nw{trousers} in the \ns{washing machine} and then remembered there was a \ns{large bank note} in your \ns{back pocket}?

\switchcolumn
\nwe{trousers}{ˈtraʊzərz}{n. 裤子;}
\nse{washing machine}{}{洗衣机}
\nse{large bank note}{}{大额银行券}
\nse{back pocket}{}{后兜;}

\switchcolumn*

When you recued your trousers, did you find the note was \ns{whiter than white}?

\switchcolumn

\nse{whiter than white}{}{特白,雪白,煞白;}

\switchcolumn*

People who live in Britain needn’t \nw{despair} when they \ns{made mistakes} \ns{like this}(and a lot of people do)!

\switchcolumn

\nse{make mistake}{}{v. 犯错;}
\nse{like this}{}{这样地,如此;这般;这么;这么样;}

\switchcolumn*

Fortunately for them, the Bank of England has a team called \nw{Mutilated} Ladies.

\switchcolumn

\nwe{mutilated}{m'jutɪleɪtɪd}{v. 严重残害…的身体,使残缺不全,肢解( mutilate的过去式和过去分词 );}

\switchcolumn*

Which deals with claims from people who fed their money to a machine or to their dog.

\switchcolumn

\switchcolumn*

Dogs, it seems, \newsentence{love to \ns{chew up} money!}

\switchcolumn

\nse{chew up}{}{充分咀嚼;弄坏;罚; 申斥;}

\switchcolumn*

A recent case concerns Jane Butlin whose \nw{fianc\'e}, John, \ns{runs a successful furniture business}.

\switchcolumn

\nwe{fianc\'e}{ˌfiən'seɪ}{n. <法>未婚夫;}
\nwe{fianc\'ee}{'faɪənsi}{未婚妻;}
\nse{run business}{}{[贸易] 经营商业;}

\switchcolumn*

John had a very good day and put his wallet containing \$3,000 into the \ns{microwave oven} for \nw{safekeeping}.

\switchcolumn

\nwe{safekeeping}{ˈsefˈkipɪŋ}{n. (对贵重物品的)保护,保管;}
\nse{microwave oven}{}{n. 微波炉;}

\switchcolumn*

Then he and Jane \ns{went horse-riding}.

\switchcolumn

\nse{go doing sth}{}{去做某事}

\switchcolumn*

When they \ns{got home}, Jane \ns{cooked their dinner} in the microwave oven and without realizing it, cooked her fianc\'e’s wallet as well.

\switchcolumn

\nse{get home}{}{v. 回家;}
\nse{cook dinner}{}{煮饭}

\switchcolumn*

\newsentence{Imagine their \nw{dismay} when they found a beautifully-cooked wallet and notes \ns{turned to} ash!}

\switchcolumn

\nwe{dismay}{dɪsˈmeɪ}{v. 使惊愕;使焦虑;使失望;n. 惊恐;焦虑;哀伤;}
\nse{turn to}{}{向…求助;}

\switchcolumn*

John went to see his \ns{bank manager} who sent \ns{the remains of} wallet and the money to the special department of the Bank of England in Newcastle: the Mutilate Ladies!

\switchcolumn

\nse{bank manager}{}{n. 银行经理;}
\nse{the remain of ...}{}{...的残骸}

\switchcolumn*

They examined the remains and John got all his money back.

\switchcolumn

\switchcolumn*

‘\ns{So long as} there’s something to identify, we will give people their money back,’ said a spokeswoman for the Bank.

\switchcolumn

\nse{so long as}{}{只要;}

\switchcolumn*

‘Last year, we paid \$1.5m on 21,000 claims.’

\switchcolumn

\switchcolumn*



\end{paracol}


\grammarpoints


\wsitem{note,cash,money}
\begin{multicols}{1}
    在英语中,这三个词都与“钱”有关,但它们的形式和使用范围有很大区别。在 2025 年这个移动支付高度发达的时代,区分它们尤为重要:
    \begin{enumerate}
        \item Note (纸币/钞票)
        
        主要指纸质的货币实体。在美式英语中常用 Bill。

        \begin{itemize}
            \item 侧重点: 强调“一张一张”的纸币。
            \item 语境: 当你提到具体的面额时。
            \item 用法:
            \begin{itemize}
                \item A £20 note (一张 20 英镑的纸币)
                \item He took a bundle of notes out of his pocket. (他从兜里掏出一叠钞票。)

            \end{itemize}
            \item 注意: 它是可数名词。
        \end{itemize}

        \item Cash (现金)
        
        指现钱,包括纸币(notes)和硬币(coins)。

        \begin{itemize}
            \item 侧重点: 强调支付方式,即相对于信用卡、扫码支付或支票的“实物货币”。
            \item 语境: 购物结账、提款机取钱。
            \item 用法:
            \begin{itemize}
                \item Do you accept cash? (你们收现金吗?)
                \item I'm short of cash at the moment. (我手头现金不够。)
                \item Pay in cash (现金支付)
            \end{itemize}
            \item 注意: 它是不可数名词,不能说 "a cash"。
        \end{itemize}

        \item Money (钱/货币)
        
        这是一个最通用、最宏观的词,涵盖了所有形式的财富。

        \begin{itemize}
            \item 侧重点: 强调价值或财富总称,包括银行存款、现金、数字货币等。
            \item 语境: 谈论收入、价格、经济、存钱。
            \item 用法:
            \begin{itemize}
                \item Time is money. (时间就是金钱。)
                \item How much money do you have in your bank account? (你银行账户里有多少钱?)
                \item Digital money (数字货币)
            \end{itemize}
            \item 注意: 它是不可数名词。
        \end{itemize}
    \end{enumerate}
\end{multicols}

\wsitem{run business}
\begin{multicols}{1}
    在商业语境下,"run a business" 是一个非常地道且核心的表达。虽然 run 在初级英语中是“跑”的意思,但在管理语境下,它意为“经营、管理、运作”。

    \begin{enumerate}
        \item 核心含义解析
        
        Run 强调的是过程(Process)和持续的操纵(Operation)。

        \es{定义:负责一个组织的日常运作、决策和管理。}

        \item Run vs. Operate vs. Manage
        
        这三个词都与“经营”有关,但侧重点不同:

        \begin{itemize}
            \item Run,侧重点在于整体掌控(最常用)
            
            \es{She runs a family hotel. (她经营着一家家族旅馆。)}

            \item Operate,侧重点在于业务运作/技术执行
            
            \es{The company operates in 20 countries. (公司在20个国家开展业务。)}

            \item Manage,侧重点在于具体管理/调控
            
            \es{He manages a team of ten. (他管理着一个十人的团队。)}

        \end{itemize}
        \item 语法搭配与扩展
        
        "Run" 是一个非常活跃的动词,可以连接多种商业对象:

        \begin{itemize}
            \item Run a shop/factory: 经营商店/工厂。
            \item Run a campaign: 发起/运作一场活动(如广告或选举)。
            \item Run a risk: 冒风险(在商业决策中常用)。
            \item Run smoothly: 运作顺利。
        \end{itemize}
    \end{enumerate}
\end{multicols}

\wsitem{get home}
\begin{multicols}{1}
    "Get home" 是英语中极其高频且地道的短语,但它的语法结构中隐藏着一个初学者很容易掉进去的“小陷阱”。
    
    以下是针对这个短语的深度语法解析:
    \begin{enumerate}
        \item 核心结构:省略介词的秘密
        
        在英语中,我们说 go to school 或 get to the office,通常 get 表示“到达”时需要接介词 to。但在 get home 中,to 被省略了。

        \begin{itemize}
            \item 语法原因:这里的 home 不是名词,而是副词(Adverb),意为“往家地/在家地”。
            \item 规则:当 get 表示“到达”且后接地点副词时,不能加介词。
            \item 同类词:get here (到这), get there (到那), get abroad (出国), get upstairs (上楼)。
        \end{itemize}

        \item Get 的多义性
        
        在 "Get home" 这个短语中,$Get$ 扮演的是趋向性动词的角色:

        \begin{itemize}
            \item 到达,对应词Arrive at/Reach。
            
            \es{I didn't get home until midnight. (我直到半夜才到家。)}

            \item 回到,对应词Return
            
            \es{What time do you usually get home? (你通常几点回家?)}
        \end{itemize}
    \end{enumerate}
\end{multicols}

\wsitem{Love to do vs. Love doing}
\begin{multicols}{1}
    这里使用 love to chew up:
    \begin{itemize}
        \item Love to do:往往强调某一次具体的行为,或者由于某种习惯、目的而去做某事。
        \item Love doing:更强调长期的爱好或享受这个过程。在本文语境下,两者区别不大,但 to chew up 听起来更具动感,仿佛能看到狗正在下嘴的那一刻。
    \end{itemize}
\end{multicols}

\wsitem{Imagine their ... turned to ash!}
\begin{multicols}{1}
    \begin{enumerate}


        \item 意群断句建议
    
    你可以按照以下节奏来朗读或理解:

    Imagine their dismay / when they found / a beautifully-cooked wallet / and notes turned to ash!

    \begin{itemize}
        \item 第一层:主句(祈使句) Imagine their dismay —— 想象一下他们的沮丧。
        \item 第二层:时间状语从句 when they found —— 当他们发现……时。
        \item 第三层:并列的宾语
        \begin{itemize}
            \item a beautifully-cooked wallet —— 一个“煮得精美的”钱包。
            \item and notes turned to ash —— 以及化为灰烬的钞票。
        \end{itemize}
    \end{itemize}

    \item 深度语法解析
    
    \begin{itemize}
        \item "Imagine their dismay" —— 祈使句开头
        
        作者用 Imagine 开头,直接把读者拉入场景,增强了代入感。Dismay 是一个程度很深的词,指“幻灭、沮丧、惊愕”。

        \item "A beautifully-cooked wallet" —— 幽默的讽刺
        
        这里使用了 过去分词作定语。通常 beautifully-cooked 是形容美味的大餐,这里用来形容被烤焦的钱包,是一种典型的英式幽默(Irony)。

        \item "Notes turned to ash" —— 省略结构
        
        这里的 turned to ash 是 过去分词短语作后置定语,修饰 notes。

        \begin{itemize}
            \item 完整形式是:notes (which had been) turned to ash。
            \item Turn to ash: 化为灰烬。
        \end{itemize}
    \end{itemize}
    \end{enumerate}
\end{multicols}

\wsitem{Turn to vs. Become}
\begin{multicols}{1}

    为什么这里用 turned to 而不用 became?

    \begin{itemize}
        \item Turn to: 强调形态或性质的彻底改变(从纸张变成灰烬)。
        \item Become: 只是简单的状态变化。
        \item 在文学表达中,turn to ash 或 turn to dust 是非常有画面感的固定搭配。
    \end{itemize}
\end{multicols}

\wsitem{So long as ...}
\begin{multicols}{1}

    \begin{enumerate}
    
    \item 核心定义与逻辑
    
    "So long as" (或 "As long as") 意为“只要……就……”。

    \begin{itemize}
        \item 逻辑性质:它引导一个充分条件状语从句。
        \item 深层含义:它不仅表达“如果”,还带有一种 “在……期间”或“以此为前提” 的持续性暗示。
        \item 语气倾向:通常用于表达一种承诺、协议或某种必然的结果。
    \end{itemize}

    \item 语法解析:时态的“陷阱”
    
    在使用 so long as 引导的从句时,必须遵守 “主将从现” 原则(这与 if 引导的条件状语从句一致):

    \begin{itemize}
        \item 从句:用一般现在时(表达将来)。
        \item 主句:用一般将来时(或情态动词)。
    \end{itemize}

    \item So long as vs. As long as
    
    虽然两者在大多数情况下可以互换,但微小的区别如下:

    \begin{itemize}
        \item As long as:最通用,既可用于肯定句也可用于否定句。
        \item So long as:略显正式,或者在否定句和强调条件时更为常见。
        \item Only if:语气更重,强调“只有在……的情况下”。
    \end{itemize}

    \end{enumerate}

    \item 深度延伸:Long 的其他用法
    
    在这个短语中,long 已经失去了“长度”的字面意思。但在其他语境下:

    \begin{itemize}
        \item Before long: 不久之后。
        
        \es{He realized that before long, he would have to find a new job. (他意识到不久以后,他将不得不找一份新工作。)}
        \item For long: 长时间地。
        
        \es{Although they were lost, they didn't believe the search would last for long. (尽管他们迷路了,但他们不相信搜救会持续很久。)}
        \item No longer: 不再(强调状态的终结)。
        
        \es{Jane was sad because the bank notes no longer looked like money. (简很伤心,因为那些钞票不再看起来像钱了。)}
    \end{itemize}
\end{multicols}

\newpage