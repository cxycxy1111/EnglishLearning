\section{Lesson 12 Life on a desert island}

\begin{paracol}{2}

Most of us have formed an \nw{unrealistic} picture of life on a \ns{desert island}. 

\switchcolumn

\nwe{unrealistic}{ˌʌnriəˈlɪstɪk}{adj. 不切实际的;不现实的;空想的;不实在的;}
\nse{desert island}{}{荒岛}

\switchcolumn*

\newsentence{We sometimes imagine a desert island to be a sort of paradise} where the sun always shines. 

\switchcolumn

\switchcolumn*

Life there is simple and good. 

\switchcolumn

\switchcolumn*

\nw{Ripe} fruit falls from the trees and you never have to work. 

\switchcolumn

\nwe{ripe}{raɪp}{adj. 成熟的;醇美可口的;老练的;时机成熟的;}

\switchcolumn*

\ns{The other side of} the picture is \ns{quite the opposite.} 

\switchcolumn

\nse{the other side of}{}{...的另一面}
\nse{quite the opposite}{}{完全相反,截然不同}

\switchcolumn*

Life on a desert island is \nw{wretched}. 

\switchcolumn

\nwe{wretched}{ˈrɛtʃɪd}{adj. 不幸的,悲惨的,可怜的;卑鄙的;恶劣的;(用于表示烦恼)讨厌的;}

\switchcolumn*

\newsentence{You either \ns{starve to death} or live like Robinson Crusoe, waiting for a boat which never comes.}

\switchcolumn

\nse{starve to death}{}{饿得要死}

\switchcolumn*

Perhaps there is an \nw{element} of truth in both these pictures, \newsentence{but few of us have had the opportunity to find out}.

\switchcolumn

\nwe{element}{ˈɛləmənt}{n. [化]元素;要素;原理;[电]电阻丝;}
\nse{an element of truth}{}{}

\switchcolumn*

Two men who recently spent five days on a \ns{\nw{coral} island} wished \newsentence{they had stayed there longer}. 

\switchcolumn

\nwe{coral}{ˈkɔrəl}{n. 珊瑚;珊瑚虫;珊瑚色;龙虾卵;adj. 珊瑚的;珊瑚色的;}
\nse{coral island}{}{珊瑚岛}

\switchcolumn*

They were taking a badly damaged boat from the Virgin Islands to Miami to have it repaired. 

\switchcolumn

\switchcolumn*

During the journey, their boat began to sink. 

\switchcolumn

\switchcolumn*

They quickly loaded a small \ns{\nw{rubber} \nw{dinghy}} with food, matches, and cans of beer and rowed for a few miles across the \nw{Caribbean} until they arrived at a tiny \ns{coral island}. 

\switchcolumn

\nwe{rubber}{ˈrʌbɚ}{n. 橡胶;橡皮;决胜局;避孕套;}
\nwe{dinghy}{ˈdɪŋi}{n. 无篷小船,小艇;救生艇;}
\nwe{Caribbean}{ˌkærəˈbiən, kəˈrɪbiən}{n. 加勒比海;}
\nse{rubber dinghy}{}{救生筏,小船}


\switchcolumn*

\newsentence{There were hardly any trees on the island} and there was no water, \newsentence{but this did not prove to be a problem}. 

\switchcolumn

\switchcolumn*

The men collected \nw{rainwater} in the rubber dinghy. 

\switchcolumn

\nwe{rainwater}{ˈrenˌwɔtɚ, -ˌwɑtɚ}{n. 雨水,软水;}

\switchcolumn*

As they had brought a \nw{spear gun} with them, they had plenty to eat. 

\switchcolumn

\nwe{spear gun}{spɪr-ɡʌn}{鱼枪}

\switchcolumn*

They caught \nw{lobster} and fish every day,and, \newsentence{as one of them put it 'ate like kings'}. 

\switchcolumn

\nwe{lobster}{ˈlɑbstə(r)}{n. 龙虾;龙虾肉;}

\switchcolumn*

When a passing tanker rescued them five days later, both men were \nw{genuinely} sorry that they had to leave.

\switchcolumn

\nwe{genuinely}{ˈdʒɛnjʊɪnlɪ}{adv. 真地;真正地;真诚地;诚实地;}

\switchcolumn*

\end{paracol}

\grammarpoints

\wsitem{Tiny vs. Little}
\begin{multicols}{1}
    这两个词都表示“小”,但在程度、感情色彩以及用法逻辑上有着明显的区别。简单来说,Tiny 强调的是“极小(尺寸)”,而 Little 更多地带有“可爱、怜悯”等主观情感。

    \item 核心定义与程度 (Degree)
    \begin{itemize}
        \item Tiny: 意为“极小的、微小的”。它的程度比 small 要深得多。通常指肉眼看上去非常细微,或者在同类事物中尺寸极小。
        \item Little: 意为“小的”。它的程度通常等同于 small,但它不仅仅描述尺寸,还描述一种心理感受。
    \end{itemize}

    \item 情感色彩 (Emotional Color)
    
    这是两者最本质的区别:

    \begin{itemize}
        \item Little (带有主观情感):当你使用 little 时,你通常在表达一种喜爱、轻蔑或同情。
        
        \es{A little dog (一只可爱的小狗 —— 带有喜爱感)。}

        \es{Poor little thing (可怜的小东西 —— 带有同情感)。}

        \item Tiny (客观描述尺寸):它是一个相对中性的词,纯粹描述体积、重量或比例非常之小。
        
        \es{A tiny insect (一只微小的昆虫 —— 只是描述它真的很小)。}
    \end{itemize}

    \item 特殊用法:不可数名词的陷阱
    
    这一点非常重要,涉及到“数量”概念:

    \begin{itemize}
        \item Little: 可以作为限定词修饰不可数名词,表示“几乎没有”(半否定)。
        
        \es{There is little water. (几乎没水了。)}

        \item Tiny: 永远不能这样用。它只能作为形容词修饰具体的事物。
        
        \es{There is a tiny drop of water. (有一粒微小的水珠。)}
    \end{itemize}
\end{multicols}

\wsitem{Cross vs. Through}
\begin{multicols}{1}
    虽然 across 和 through 都可以翻译为“穿过”,但在英语的“空间逻辑”中,它们描述的是完全不同的物理路径。
    \begin{enumerate}
        \item 空间逻辑图解:平面 vs. 立体
        \begin{itemize}
            \item Across (面状穿过):指从一个表面的这一边到另一边。强调的是在二维平面上移动。
            \begin{itemize}
                \item 场景:横渡海洋、穿过马路、游过河流。
                \item 逻辑:就像在纸上划过一条横线。
            \end{itemize}
            \item Through (立体穿过):指从一个三维空间的内部穿过。强调的是四周被包围的状态。
            \begin{itemize}
                \item 场景:穿过森林、钻过隧道、走过人群、透过浓雾。
                \item 逻辑:就像子弹穿过苹果,或者人走在茂密的丛林里。
            \end{itemize}
        \end{itemize} 

        \item 为什么课文里必须用 Across?
        
        在这个句子中,动作是 "rowed... across the Caribbean"(划船横渡加勒比海):

        \begin{itemize}
            \item 水面是平面:划船是在海的“表面”移动,而不是潜入水底从水里钻过去。
            \item 视野开阔:除非当时海面上全是大雾(那可能会用 through the mist),否则在开阔的海域航行,逻辑上就是横跨一个平面。
        \end{itemize}

        \item 对比示例
        
        \es{They rowed across the lake. (他们划船横渡湖泊 —— 正常航行)}

        \es{The submarine traveled through the water. (潜水艇在水里穿行 —— 三维空间穿透)}
    \end{enumerate}
\end{multicols}


\wsitem{an element of truth}
\begin{multicols}{1}

    这是一个非常有文学色彩的短语,常用来表达“并非全然虚构,其中包含了一定的真实成分”。它比单纯说 "it is true" 要委婉且精准得多。

    \begin{enumerate}
        \item 核心词义拆解
        \begin{itemize}
            \item Element: 原意为“元素”或“基本组成部分”。在这里引申为“少量、一点点、一丝”。
            \item Of truth: 修饰这个少量成分的性质。
            \item 整体含义:指在一段话、一个故事或一个借口中,虽然大部分可能是夸张或虚假的,但其中确实有“一部分事实”。
        \end{itemize}
        \item 常见使用场景
        \begin{itemize}
            \item 评价一个借口或谎言
            
            当某人撒了谎,但谎言是基于部分事实编造时:

            \es{There was an element of truth in his excuse, but most of it was made up. (他的借口里有几分属实,但大部分都是编的。)}
            \item 评价一个谣言或猜测
            
            当一个传闻听起来很荒唐,但核心逻辑是对的时:

            \es{The story sounds incredible, but it contains an element of truth. (这个故事听起来不可思议,但它包含了一定的真实成分。)}
        \end{itemize}
        \item 语义强度对比
        \begin{itemize}
            \item Completely true,真实程度100\%,事实确凿。
            \item Largely true,真实程度80\% - 90\%,基本上是真的。
            \item An element of truth,真实程度10\% - 20\%,只有一点点是真的(带有怀疑态度)。
            \item Entirely false,真实程度0\%,全然虚假。
        \end{itemize}

        \item Tiny little... 连用
        
        在口语中,你经常会听到有人说 "a tiny little thing"。这是一种双重强调,既表达了物体极小(tiny),又表达了说话者觉得它很萌/很精巧(little)。
    \end{enumerate}
    
\end{multicols}

\wsitem{starve to death}
\begin{multicols}{1}
    "Starve to death" 是一个含义非常强烈的短语。在字面上,它描述的是由于极度缺乏食物而导致的悲剧性结果;但在日常口语中,它也常被夸张地用来形容“饿坏了”。

    \begin{enumerate}
        \item 语法结构与核心含义
        \begin{itemize}
            \item Starve: 动词,意为“(使)挨饿”或“饿得要命”。
            \item To death: 介词短语作为结果补语,表示动作程度达到了“死亡”的地步。
        \end{itemize}

        \item 基本用法
        \begin{itemize}
            \item 字面义(客观事实):指真正死于饥饿。
            
            \es{Many animals starved to death during the unusually cold winter. (在那个异常寒冷的冬天,许多动物饿死了。)}

            \item 夸张义(非正式口语):表示非常饥饿。
            
            \es{I haven't had lunch yet; I'm starving to death! (我还没吃午饭,快饿死了!)}
        \end{itemize}

        \item 逻辑内核:生存极限的对比
        
        $$\begin{array}{|l|l|l|}
            \hline
            \textbf{程度等级} & \textbf{对应表达} & \textbf{逻辑状态} \\ \hline
            \text{Level 1} & \text{Hungry} & \text{需要进食 (Normal)} \\ \hline
            \text{Level 2} & \text{Starving} & \text{极度饥饿 (Urgent)} \\ \hline
            \text{Level 3} & \textbf{Starve to death} & \text{生命威胁 (Critical)} \\ \hline
        \end{array}$$
        
        英语中有很多 "Verb + to death" 的结构,用来表达某种行为或情感达到了极限:

        \begin{itemize}
            \item Freeze to death: 冻死。
            \item Bore to death: 烦得要死。 (The long speech bored me to death.)
            \item Scare to death: 吓死。
            \item Work to death: 累死/过劳死。
        \end{itemize}

    \end{enumerate}
\end{multicols}

\wsitem{否定/半否定词}
\begin{multicols}{1}
    在英语中,否定词(如 Never, Not)和半否定词(如 Hardly, Seldom, Few, Little)构成了语言逻辑中的“冷色调”。它们不仅改变句子的意思,还会引发一系列复杂的语法变形,如倒装、反义疑问句以及词汇搭配的改变。

    \begin{enumerate}
        \item 否定程度的光谱 (The Spectrum of Negativity)
        
        我们可以把这些词放在一个从“100\% 肯定”到“0\% 绝对否定”的数轴上:

        \begin{itemize}
            \item 绝对否定 (100\% Negative): Never (从不), No/None (没有), Neither (两者都不), Nobody/Nothing (没人/没物).
            \item 半否定 (Semi-negative, 几乎没有): Hardly, Scarcely, Barely (几乎不); Seldom, Rarely (很少/频率极低); Few, Little (数量极少).
        \end{itemize}
        \item 半否定词的分类与陷阱
        
        半否定词最让学习者头疼,因为它们表面上没有 "Not",但在逻辑上等于 "Not"。

        \begin{itemize}
            \item 频率与程度:Hardly, Seldom, Rarely
            
            这些词通常修饰动词。
            \begin{itemize}
                \item Hardly/Scarcely/Barely: 强调“几乎达不到某个程度”。
                
                \es{I could hardly hear him. (我几乎听不到他的声音。)}

                \item Seldom/Rarely: 强调“频率极低”。
                
                \es{Great people seldom complain. (伟人很少抱怨。)}
            \end{itemize}

            \item 数量:Few vs. Little
            
            这是最经典的考试陷阱,关键在于可数性和心理倾向。
        \end{itemize}
        \item 三大核心语法规则
        \begin{itemize}
            \item 规则一:句首倒装 (Negative Inversion)
            
            \es{正常句: He seldom eats out.}

            \es{倒装句: Seldom does he eat out.}

            \es{公式: $\text{Negative Word} + \text{Auxiliary/Modal Verb} + \text{Subject} + \text{Verb}$}
            \item 规则二:反义疑问句 (Tag Questions)
            
            如果陈述句部分包含半否定词,它被视为否定句,因此后面的简短问句必须用肯定式。

            \es{He has few friends, has he / does he? (千万不能用 hasn't he)}
            \es{ou hardly know her, do you? (千万不能用 don't you)}
            \item 规则三:搭配 Any 而非 Some
            
            半否定词后的语境具有“否定倾向”,因此通常搭配 any/anything/anybody。

            \es{There is hardly any water left. (几乎没剩水了。)}
        \end{itemize}
    \end{itemize}
\end{multicols}

\wsitem{image ... to be ...}
\begin{multicols}{1}
    这是一个非常地道的表达方式,主要用于**“想象某人/某事是……”或“把……想象成……”**。在语法结构上,它属于“及物动词 + 宾语 + 宾语补足语”。

    \begin{enumerate}
        \item 结构拆解
        
        Imagine + 宾语 (sb./sth.) + to be + 补足语 (a/an...)

        \begin{itemize}
            \item Imagine: 动词,意为“想象、设想”。
            \item 宾语: 你想象的对象。
            \item To be: 不定式符号,用来连接对象及其身份/状态(在口语中,to be 有时可以省略,但加上它会显得更正式、更具文学感)。
            \item 补足语: 你赋予该对象的虚构身份。
        \end{itemize}

        \item 语境应用
        \begin{itemize}
            \item 纯粹的想象/比喻 (Creative Metaphor)
            
            当你把一个事物看作另一个完全不同的东西时:

            \es{As a child, I used to imagine the clouds to be giant sheep in the sky. (小时候,我常把云朵想象成天空中的巨型绵羊。)}

            \item 对某人身份的预设 (Assuming a role)
            
            \es{I had imagined him to be a serious person, but he turned out to be quite funny. (我原本以为他是个严肃的人,结果他挺幽默的。)}

            \item 文学描述(如课文语境)
            
            这种用法可以增强画面感:

            \es{The boy imagined himself to be a hero after saving the cat from the tree. (男孩在从树上救下猫后,想象自己是个英雄。)}
        \end{itemize}
    \end{enumerate}
\end{multicols}

\wsitem{Either ... or ...}
\begin{multicols}{1}
    "Either ... or ..." 是英语中非常重要的并列连词 (Correlative Conjunctions)。它的核心功能是在两个选项中进行“二选一”。
    
    以下是针对这个结构的深度解析,涵盖了你在写作和语法考试中必须掌握的三个关键点:

    \begin{enumerate}
        \item 核心逻辑:排他性的选择
        
        这个结构连接两个并列的成分(可以是名词、代词、动词、形容词或句子),表示“要么……要么……”。

        \begin{itemize}
            \item 连接名词:You can have either tea or coffee. (你要么喝茶,要么喝咖啡。)
            \item 连接动作:He will either call me or send an email. (他要么给我打电话,要么发邮件。)
        \end{itemize}
        \item 语法重难点:就近原则 (The Principle of Proximity)
        
        当 Either ... or ... 连接两个不同的人称或数的主语时,谓语动词的形式由离它最近的主语决定。
        \begin{itemize}
            \item 示例 A:Either you or he is responsible for the mess. (要么是你,要么是他对这烂摊子负责。) —— 动词随 he 用 is。
            \item 示例 B:Either he or you are going to talk to the customs officer. (要么是他,要么是你去跟海关官员谈。) —— 动词随 you 用 are。
        \end{itemize}
    \end{enumerate}

    其他并列连词包括:both ... and 和neither...nor...

    这两个结构与either... or... 构成了英语中最核心的关联连词 (Correlative Conjunctions) 体系。它们分别代表了逻辑上的“全选”与“全否”。

    \begin{enumerate}
        \item Both ... and ... (两者都)
        
        这是典型的“$1 + 1 = 2$”逻辑,强调两个选项同时成立。

        \begin{itemize}
            \item 核心语法:谓语动词永远用复数。 由于这个结构在逻辑上等同于 and,它连接的两个主语被视为一个复数整体。
            \item 示例:
            \begin{itemize}
                \item Both the pilot and the crew were exhausted. (飞行员和机组人员都精疲力竭了。)
                \item I like both tea and coffee. (茶和咖啡我都喜欢。)
            \end{itemize}
        \end{itemize}

        \item Neither ... nor ... (两者都不)
        
        这是 either... or... 的否定版,表示“既不……也不……”。

        \begin{itemize}
            \item 核心语法:就近原则 (The Principle of Proximity)。 和 either... or... 一样,谓语动词的形式由离它最近的那个主语决定。
            \item 示例:
            \begin{itemize}
                \item Neither Jane nor her parents know the truth. (简和她的父母都不知道真相。) —— 动词随复数 parents 用 know。
                \item Neither the parents nor Jane knows the truth. (父母和简都不知道真相。) —— 动词随单数 Jane 用 knows。
            \end{itemize}
            \item 易错陷阱:双重否定
            
            在使用 neither... nor... 时,句中不能再出现否定词。因为 neither 本身已经包含了 not。

            \begin{itemize}
                \item 错误:I don't like neither tea nor coffee. (重复否定)
                \item 正确:I like neither tea nor coffee.
                \item 或者:I don't like either tea or coffee. (这是最地道的否定表达方式,用 not... either... or... 代替 neither... nor...)
            \end{itemize}
        \end{itemize}
    \end{enumerate}

\end{multicols}

\wsitem{Prove to be}
\begin{multicols}{1}
    "Prove to be" 是一个非常高级且地道的连系动词结构,意为“证明是……”或“结果显示为……”。它在描述一个事物经过时间的检验或特定事件后,展露出真实面貌时非常有用。
    
    以下是关于这个短语的深度解析:

    \begin{enumerate}
        \item 核心定义与用法
        
        Prove to be + 形容词/名词

        它的逻辑不是“某人去证明某事”,而是“某事本身在后期显现出了某种特质”。

        \begin{itemize}
            \item 搭配形容词:The task proved to be difficult. (这项任务证明是很困难的。)
            \item 搭配名词:He proved to be a great friend. (事实证明他是个伟大的朋友。)
            \item 语法小贴士:在很多情况下,to be 是可以省略的,即 The task proved difficult。但在正式写作或强调状态时,保留 to be 听起来更加完整。
        \end{itemize}

        \item 与 "Turn out to be" 的区别
        \begin{itemize}
            \item Prove to be强调经过考验后,某种特质得到了证实。
            \item Turn out to be强调意想不到的结果(原来是……)。
        \end{itemize}

        \item 被动语态陷阱
        
        注意,prove to be 本身就含有“被证明是”的意思,所以通常用主动形式表示被动含义。
        \begin{itemize}
            \item 错误:The rumor was proved to be false. (虽然语法通,但太累赘)
            \item 正确:The rumor proved to be false. (简洁、地道)
        \end{itemize}
    \end{enumerate}
\end{multicols}

\wsitem{they had stayed there longer}



\begin{multicols}{1}
    这是一个非常典型的虚拟语气(Subjunctive Mood)用法。在英语中,动词 wish 后面的宾语从句需要使用虚拟语气来表达一种“与事实相反”或“难以实现的愿望”。
    
    使用 $had\ stayed$ 而不是 $would\ stay$,其根本原因在于时间维度的对立。

    \begin{enumerate}
        \item 时间轴的倒退 (The Backshift of Tense)
        
        虚拟语气的核心规律是:向过去倒退一个时态。

        \begin{itemize}
            \item 事实(Fact):他们已经在岛上待了五天,并且已经离开了(这是发生在过去的事实)
            \item 愿望(Wish):他们希望当初能待得更久一点。
            \item 语法逻辑:既然是对“过去已经发生的事实”表示遗憾,我们必须在“过去时(spent)”的基础上再往后退一格,即使用过去完成时($had\ done$)。
        \end{itemize}
        \item $Had\ stayed$ vs. $Would\ stay$ 的本质区别
        
        为什么不用 $would\ stay$?$would\ stay$ 通常表达对未来的愿望,或者对现状不满(希望某人改变行为)。

        \es{例:I wish it would stop raining. (我希望雨能停——针对现在或未来)}

        但在这句课文中,他们已经离开了小岛,这是一个改变不了的过去,所以必须用 $had\ stayed$。

        \item 如果是 $wished$ 怎么办?
        
        注意主句动词是 $wished$(过去时)。但这并不会改变从句用过去完成时的逻辑,因为“待在岛上”这个动作发生在“希望”这个动作之前。

        \es{They wished (过去某时希望) they had stayed (比希望更早的过去待在岛上).}

        \item 公式
        
        $\text{Wish (Past Regret)} = \text{Subject} + \text{wish} + \text{Subject} + \text{had} + \text{v-ed (Past Participle)}$
        
        $\text{Logic: } \text{Past Fact (They left)} \iff \text{Past Wish (They had stayed)}$

    \end{enumerate}
\end{multicols}

\wsitem{but few of us have had the opportunity to find out}
\begin{multicols}{1}

    在这一句中,使用 "have had" 而不是简单的 "have",涉及到英语中“拥有”的状态与“经历”的过程之间的微妙区别。

    \begin{enumerate}
        \item 时态的逻辑:现在完成时 vs. 一般现在时
        
        这里的 "have had" 是现在完成时($Present\ Perfect$):

        \begin{itemize}
            \item 第一个 "have": 助动词,构成完成时态。
            \item 第二个 "had": 实义动词 $have$ 的过去分词,意思是“获得/得到/经历”。
        \end{itemize}

        为什么不能只用 "have"?

        如果说 "...few of us have the opportunity"(一般现在时),它描述的是一个静态的现状。听起来像是“我们现在手头没有这个机会”。

        如果说 "...few of us have had the opportunity"(现在完成时),它描述的是从过去到现在的整个时间跨度。其逻辑内核是在截止到目前为止的人生经历中,很少有人“曾经获得过”去岛上长期生活的机会。它强调的是一种经验的缺乏,而不仅仅是现在的状态。

        \item 动词词义的偏向:Possess vs. Experience
        
        在英语中,当 $have$ 表示“经历”或“获得”时,更倾向于用完成时来表达“过往的资历”。

        \begin{itemize}
            \item Have (一般现在时): 倾向于“拥有某物”(Possession)。
            
            \es{I have a car. (我有一辆车。)}


            \item Have had (现在完成时): 倾向于“经历过某种机会/时间”(Experience)。
            
            \es{I have had a busy day. (我今天过得很忙——强调这一天的经历。)}

            \es{Few people have had the opportunity... (很少有人经历过这种机会。)}
        \end{itemize}

        \item 逻辑对比
        \begin{itemize}
            \item few us have...,其语法逻辑是General Fact,含义是我们现在没有机会 (静态现状)
            \item few us have had...,其语法逻辑是Experience,含义是截止到目前,很少人曾得到过机会 (动态经历)
        \end{itemize}
    \end{enumerate}
\end{multicols}

\wsitem{as one of them put it ...}
\begin{multicols}{1}

    这是一个非常地道的引语表达方式,常用于引用某人的原话或核心观点。在文章中,它能起到“证据支持”的作用,让你的叙述更具真实感。

    \begin{enumerate}
        \item 结构拆解
        \begin{itemize}
            \item As: 连词,意为“正如”。
            \item One of them: 特指前文提到的那群人(比如那两个在荒岛上的男人)中的一个。
            \item Put it: 这里的 put 不是“放置”,而是“表达、叙述、说”。
        \end{itemize}

        整体含义:正如他们其中一人所说…… / 正如他们其中一人的说法……

        \item 为什么用 "Put it"?
        
        在英语中,put 经常用来指代“用语言表达某种想法”的方式。

        \begin{itemize}
            \item To put it simply: 简单来说。
            \item To put it another way: 换句话说。
            \item As he put it: 正如他所表达的那样。
        \end{itemize}

        这种用法比 as one of them said 显得更加书面且专业,它侧重于表达的方式和措辞。
    \end{enumerate}
\end{multicols}

\newpage
