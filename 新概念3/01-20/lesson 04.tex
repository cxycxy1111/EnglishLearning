\section{Lesson 4 The double life of Alfred Bloggs}

\begin{paracol}{2}

These days, people who do \ns{manual work} often receive \newsentence{far more money than} people who work in offices. 

\switchcolumn

\nse{manual work}{}{人工,手工作业;苦工;}

\switchcolumn*

People who work in offices are frequently \ns{referred to as} ‘\ns{white-collar workers}' for the simple reason that they usually wear a \nw{collar} and tie to \ns{go to work}. 

\switchcolumn

\nwe{collar}{ˈkɑːlər}{n. 衣领;(动物)颈圈;箍;v. 抓住,逮住;拦住(某人与其)谈话;}
\nse{refer to as}{}{被称为,被认为;}
\nse{white-collar workers}{}{脑力劳动者;}
\nse{go to work}{}{开始行动;上班;上工;出工;}

\switchcolumn*

\newsentence{Such is human nature, that \ns{a great many} people \newsentence{are often willing to} \ns{sacrifice higher pay} for \ns{the privilege of} becoming white-collar workers. }

\switchcolumn

\nse{a great many}{}{很多, 许多;指不胜屈;浩;}
\nse{sacrifice higher pay}{}{牺牲高薪}
\nse{the privilege of}{}{特权}

\switchcolumn*

This can give rise to curious situations, as it did \ns{in the case of} Alfred Bloggs who \ns{worked as a} \nw{dustman} for the Ellesmere Corporation.

\switchcolumn

\nwe{dustman}{'dʌstmən}{n. 清洁工;}
\nse{in the case of}{}{至于…, 就…来说;}
\nse{work as a}{}{任…职,当…;}

\switchcolumn*

When he \ns{got married}, \newsentence{Alf was too embarrassed to say anything to his wife about his job}. 

\switchcolumn

\nse{get married}{}{结婚;结亲;成家立室;}

\switchcolumn*

He simply told her that he \ns{worked for} the Corporation. 

\switchcolumn

\nse{work for}{}{受雇于;为…而工作;效劳;致力;}

\switchcolumn*

Every morning, he left home \ns{dressed in} a smart black suit. 

\switchcolumn

\nse{dress in}{}{穿着…衣服;}

\switchcolumn*

He then changed into \nw{overalls} and \newsentence{spent the next eight hours as a dustman}. 

\switchcolumn

\nwe{overalls}{'oʊvərɔlz}{n. 工装裤;长罩衣( overall的名词复数 );工装裤,工作裤;套裤;}

\switchcolumn*

Before returning home at night, he \ns{took a shower} and changed back into his suit. 

\switchcolumn

\nse{take a shower}{}{沐浴;}

\switchcolumn*

Alf did this for over two years and his fellow dustmen \ns{kept his secret}. 

\switchcolumn

\nse{keep secret}{}{保守秘密}

\switchcolumn*

Alf's wife has never discovered that she married a dustman and she never will, for Alf has just found another job. 

\switchcolumn

\switchcolumn*

He will soon be working in an office. 

\switchcolumn

\switchcolumn*

\newsentence{He will be earning only half \ns{as much as} he used to}, but he feels that his \ns{rise in status} \ns{is well worth} \ns{the loss of money}.

\switchcolumn

\nse{as much as}{}{adv. 差不多;足;}
\nse{rise in status}{}{地位上的提升}
\nse{be well worth sth/doing}{}{非常值得}
\nse{the loss of money}{}{资金损失}

\switchcolumn*

\ns{From now on}, he will wear a suit all day and others will call him 'Mr. Bloggs', not 'Alf'.

\switchcolumn

\nse{from now on}{}{adv. 从现在开始;从此;从今以后;往后;}

\switchcolumn*


\end{paracol}

\worddifference
\wsitem{Embarrassed vs. Embarrassing}
\begin{multicols}{1}
    这两个词经常被混淆,在描述人的情感时要格外注意:
    \begin{enumerate}
        \item Embarrassed
        
        感到尴尬的,修饰人。

        \es{Alfred felt embarrassed.}
        \item Embarrassing
        
        令人尴尬的,修饰事。

        \es{His job was embarrassing.}
    \end{enumerate}
\end{multicols}

\grammarpoints

\wsitem{far more ... than ...}

\begin{multicols}{1}
    \begin{enumerate}
        \item 语法结构解析
        
        far + more + [不可数名词/形容词比较级] + than

        \begin{itemize}
            \item Far: 程度副词,意为“……得多”,用来加强比较级的语气。
            \item More money: 比较级结构。注意 money 是不可数名词,所以用 more。
            \item Than: 引导比较的对象。
        \end{itemize}

        常见的程度加强词 (Modifier + Comparative):
        
        如果你想表达“多得多”或“好得多”,除了 far,还可以使用:
        \begin{itemize}
            \item Much more money
            \item A lot more money
            \item Even more money (甚至更多)
        \end{itemize}

        \item 比较级转换练习 (A Little vs. Far)
        
        通过对比,你可以感受到 far 的力度:

        \begin{itemize}
            \item 略多: \es{We need a little more money than we have. (我们需要的钱比现有的多一点。)}
            \item 多得多: \es{We need far more money than we have. (我们需要的钱比现有的多得多。)}
        \end{itemize}
    \end{enumerate}
\end{multicols}

\wsitem{refer to ... as ...}
\begin{multicols}{1}
    它是一个非常实用的被动语态/命名表达。
    \begin{enumerate}
    \item 核心语法解析
    
    Refer to A as B 含义: 把 A 称作 B;将 A 看作 B。
    
    语法结构: 
    \begin{itemize}
        \item 主动: \es{We refer to the statue as a goddess.(我们把这座雕像称为女神。)} 
        \item 被动: \es{The statue is referred to as a goddess.(这座雕像被称作女神。)}
    \end{itemize}
    \item 使用 referred to as 而不是简单的 called,显得更加正式,带有一种学术界定或公认称呼的色彩。
\end{enumerate}
\end{multicols}

\wsitem{be willing to ...}
\begin{multicols}{1}
    它描述的是一种“主观意愿”
        
    深度辨析:Willing vs. Ready vs. Prepared
        
        这三个词在中文里都可能翻译为“准备好”或“愿意”,但它们在心理状态上有很大区别:

        \begin{itemize}
            \item Be willing to: 侧重于主观上的乐意或不反对。即使某事很麻烦或不愉快,我也“愿意”去做。
            
            \es{The bank is willing to help you if you have mutilated banknotes. (即使钞票碎了,银行也乐意帮忙。)}

            \item Be ready to: 侧重于时间或行动上的准备就绪。
            
            \es{I am ready to go. (我已经收拾好,随时可以出发。)}

            \item Be prepared to: 侧重于心理上的准备或防范,通常针对困难或复杂的情况。
            
            \es{You must be prepared to wait for a long time. (你必须做好长期等待的心理准备。)}
        \end{itemize}
\end{multicols}

\wsitem{spend time as ...}
\begin{multicols}{1}
    \begin{enumerate}
        \item 核心语法解析
        
        Spend + 时间 + as + 身份/职业

        \begin{itemize}
            \item Spend: 动词。注意它的搭配:$spend + time + (in) + doing$ 或 $spend + time + on + sth$。但当后面接 as 时,表示“以……的身份度过时间”。
            \item The next eight hours: 一段时间。这里的 the next 增加了叙事的连续性,表示“接下来的”八小时。
            \item As a dustman: 以一个清洁工(垃圾工)的身份。As 在这里是介词,表示“作为”。
        \end{itemize}
    \end{enumerate}
\end{multicols}

\wsitem{keep secret}
\begin{multicols}{1}
    \begin{enumerate}
        \item Keep a secret / Keep something secret
        
        这是最通用的表达。注意其细微的结构区别:

        \begin{itemize}
            \item Keep a secret: 守口如瓶(指一种行为习惯)。
            
            $\rightarrow$ \es{Can you keep a secret?}

            \item Keep something secret: 把某事保密(指针对特定的事)。
            
            $\rightarrow$ \es{Alfred wanted to keep his costume secret.}
        \end{itemize}
        \item Keep something to oneself
        
        指不把某事告诉别人,独自保留这个信息。

        $\rightarrow$ \es{He kept the news to himself until the party.}
        \item Keep someone in the dark
        
        这是一个非常地道的习语,指“瞒着某人”,让某人对某事一无所知。

        $\rightarrow$ \es{They kept the vicar in the dark about the repairs.}
    \end{enumerate}
\end{multicols}

\wsitem{such is ..., that ...}

\begin{multicols}{1}
    "Such is ..., that ..." 是一个非常正式且具有文学色彩的倒装结构,主要用于强调程度之深,以至于产生了某种结果。
    
    它相当于 "The degree of ... is so great that ..."(某事的程度如此之大,以至于……)。

    \begin{enumerate}
    \item 语法结构
    Such + is/was + 名词(主语) + that + 结果状语从句
    
    Such 在这里充当表语,被提前到句首以示强调。
    
    is/was 的单复数形式取决于后面的名词(主语)。

    \item 经典范例
    \begin{itemize}
        \item 强调力量/影响:
        
        \es{Such is the power of love that it can overcome any obstacle.(爱如此伟大,以至于它能战胜任何障碍。)}

        \item 强调程度/状态:
        
        \es{Such was her excitement that she couldn't sleep all night.(她当时如此兴奋,以至于整晚没睡着。)}

        \item 强调名声/地位:
        
        \es{Such is his fame that he is recognized everywhere he goes.(他的名气如此之大,以至于无论走到哪都会被认出来。)}
    \end{itemize}
    \item 为什么使用这个句型?
    \begin{itemize}
        \item 语气更强烈: 比起普通的 "His fame is so great that...",使用 "Such is..." 更有震撼力和戏剧效果。
        \item 书面感强: 常见于新闻报道、小说、演讲或正式的学术写作中。
    \end{itemize}
    \item 易混淆表达:Such as
    
    请注意,"Such is ..., that" 与 "Such as" 完全不同:

    \begin{itemize}
        \item Such as: 用于举例(例如……)。
        \item Such is ..., that: 用于强调程度(如此……以至于……)。
    \end{itemize}
    \item 变体练习
    
    如果你想把这个句型运用在描述 2025 年的科技上,可以这样写:
    
    \es{Such is the speed of AI development in 2025 that new breakthroughs are happening every single week.(2025年AI发展的速度如此之快,以至于每周都有新的突破。)}

    总结: 当你想要感叹某件事情的程度“简直到了极点”时,用这个句型会显得你的英语非常地道且高级。

\end{enumerate}
\end{multicols}

\wsitem{too... to...}
\begin{multicols}{1}
    \begin{enumerate}
        \item 核心语法解析:Too... to...
        
        Too + 形容词/副词 + to do something

        \begin{itemize}
            \item 含义: “太……而不能……”(表否定含义,虽然句中没有 not)。
            \item 句法结构:
            \begin{itemize}
                \item Too embarrassed: 太尴尬了。
                \item To say anything: 以至于没法说出任何话。
            \end{itemize}
            \item 等价转换: 我们可以用 so... that... 结构改写:
            
            \es{Alf was so embarrassed that he couldn't say anything to his wife.}
        \end{itemize}
    \end{enumerate}
\end{multicols}

\wsitem{倍数 + as + 形容词/副词原级 + as...}
\begin{multicols}{1}
    \begin{enumerate}
        \item 语法结构解析:倍数表达法
        
        这个句子的核心公式是:倍数 + as + 形容词/副词原级 + as...

        \begin{itemize}
            \item Half: 倍数词(一半)。
            \item As much as: 这里的 much 指代钱数(不可数)。
            \item He used to: 这里的 used to 后面省略了 earn,表示“他过去常常(挣的钱)”。
        \end{itemize}

        常见的倍数词:

        \begin{itemize}
            \item Twice / Double: 两倍
            \item Three times: 三倍
            \item Half: 一半
        \end{itemize}

        \item 深度对比:两种比较方式
        
        如果你想表达“他现在的工资只有过去的一半”,有两种地道的写法:

        \begin{enumerate}
            \item As...as 结构 (课文用法):
            
            $\rightarrow$ He earns half as much as he used to.

            \item 比较级结构:
            
            $\rightarrow$ He earns half less than he used to. (这种用法相对较少,第一种更常用)
        \end{enumerate}
    \end{enumerate}
\end{multicols}


\newpage