\section{Lesson 14 A noble gangster}

\begin{paracol}{2}
    \newsentence{There was a tine when the owners of shops and businesses in \nw{Chicago} had to pay \ns{large sums of money} to \nw{gangsters} \ns{in return for} 'protection.' }

    \switchcolumn

    \chinesetext{曾经有一个时期,芝加哥的店主和商行的老板们不得不拿出大笔的钱给歹徒以换取"保护"。}

    \nwe{Chicago}{ʃəˈkɑˌgoʊ}{n. 芝加哥(美国城市名);}
    \nwe{gangster}{ˈɡæŋstər}{n. 匪徒,歹徒;流氓;恶棍;}
    \nse{large sums of money}{}{大笔资金}
    \nse{in return for}{}{作为对……的回报;作为……的交换条件}

    \switchcolumn*

    If the money was not paid \nw{promptly}, the gangsters would quickly \ns{put a man out of business} by destroying his shop. 

    \switchcolumn

    \chinesetext{如果交款不及时,歹徒们就会很快捣毁他的商店,让他破产。}

    \nwe{promptly}{ˈprɑmptli}{adv. 敏捷地;迅速地;立即地;毫不迟疑;}
    \nse{put sb. out of business}{}{使某人破产、倒闭,或者迫使某人无法再经营下去}

    \switchcolumn*

    \newsentence{Obtaining 'protection money' is not a modern crime. }

    \switchcolumn

    \chinesetext{榨取"保护金"并不是一种现代的罪恶行径。}

    \switchcolumn*

    \ns{As long ago as} the fourteenth century, an Englishman, Sir John Hawkwood, \newsentence{made the \nw{remarkable} discovery that} \newsentence{people would rather pay large sums of money than have their \ns{life work} destroyed by gangsters.}

    \switchcolumn

    \chinesetext{早在14世纪,英国人约翰·霍克伍德就有过非凡的发现:人们情愿拿出大笔的钱,也不愿毕生的心血毁于歹徒之手。}

    \nwe{remarkable}{rɪˈmɑːrkəbl}{adj. 异常的,引人注目的,;卓越的;显著的;非凡的,非常(好)的;}
    \nse{as long ago as ...}{}{追溯到...,早在...}
    \nse{life work}{}{一个人倾注一生心血所成就的事业}

    \switchcolumn*

    Six hundred years ago, Sir Johan Hawkwood arrived in Italy with a \nw{band} of soldiers and settled near \nw{Florence}. 

    \switchcolumn

    \chinesetext{600年前,约翰·霍克伍德爵士带着一队士兵来到意大利,在佛罗伦萨附近驻扎下来。}

    \nwe{band}{bænd}{n. 乐队;一群人;带子,条纹;(数值)范围;波段;v. 把…划分为;加彩条;}
    \nwe{Florence}{'flɔrənsˌ 'flɑr-}{n. 弗洛伦斯(女子名);佛罗伦萨(意大利都市);}

    \switchcolumn*

    He soon \ns{made a name for himself} and \ns{came to be known to} the Italians as Giovanni Acuto. 

    \switchcolumn

    \chinesetext{他很快就出了名,意大利人叫他乔凡尼·阿库托。}

    \nse{make a name for oneself}{}{通过努力或成就而出名}
    \nse{come to be known to ...}{}{逐渐被……所知晓}

    \switchcolumn*

    Whenever the Italian \nw{city-states} were at war with each other, Hawkwood used to \nw{hire} his soldiers to \nw{princes} who were willing to pay the high price he demanded. 

    \switchcolumn

    \chinesetext{每次意大利各城邦之间打仗,霍克伍德把他的士兵雇佣给愿给他出高价的君主。}

    \nwe{city-states}{'sɪtɪ'steɪ}{n. 城邦}
    \nwe{hire}{ˈhaɪər}{v. 雇用,临时雇佣;租用;n. 租赁;新雇员;}
    \nwe{prince}{prɪns}{n. 王子;巨头;(某些欧洲国家的)贵族;小国的君主;王孙,亲王;}

    \switchcolumn*

    In times of peace, when business was bad, Hawkwood and his men would \nw{march} into a city-state and, after burning down a few farms, \newsentence{would offer to \nw{go away} if protection money was paid to them}. 

    \switchcolumn

    \chinesetext{和平时期,当生意萧条时,霍克伍德便带领士兵进入某个城邦,纵火烧毁一两个农场,然后提出,如向他们缴纳保护金,他们便主动撤离。}

    \nwe{march}{mɑːrtʃ}{v. 行军;前进;示威游行;行走;迫使…同行;n. 游行示威;行进;进步;进行曲;}
    \nse{go away}{}{vi. (症状)消失;离开;私奔;}

    \switchcolumn*

    Hawkwood made large sums of money \nw{in this way}. 

    \switchcolumn

    \chinesetext{霍克伍德用这种方法挣了大笔钱。}
    \nse{in this way}{}{adv. 这样;}

    \switchcolumn*

    \ns{In spite of} this, the Italians regarded him as a sort of hero. 

    \switchcolumn

    \chinesetext{尽管如此,意大利人还是把他视作某种英雄。}
    \nse{In spite of}{}{不管;尽管;}

    \switchcolumn*

    When he died at the age of eighty, the \nw{Florentines} gave him a \ns{state funeral} and had a picture painted which was \nw{dedicated} to the memory of 'the most \nw{valiant} soldier and most \nw{notable} leader, Signor Giovanni Haukodue.'

    \switchcolumn

    \chinesetext{他80岁那年死去时,佛罗伦萨人为他举行了国葬,并为他画像以纪念这位“骁勇无比的战士、杰出的领袖乔凡尼·阿库托先生。”}

    \nwe{Florentines}{}{佛罗伦萨人}
    \nwe{dedicate}{ˈdedɪkeɪt}{v. 献(身),致力;题献词;为…举行落成典礼;}
    \nwe{valiant}{ˈvæliənt}{adj. 勇敢的,英勇的;坚定的;}
    \nwe{notable}{ˈnoʊtəbl}{adj. 值得注意的;显著的;著名的;n. 名人;显要人物;}

    \nse{state funeral}{}{国葬}

    \switchcolumn*

\end{paracol}

\grammarpoints

\wsitem{Make money vs. Earn money}
\begin{multicols}{1}
    \begin{enumerate}
        \item \textbf{Earn money (挣钱/赢得报酬)}
        \begin{itemize}
            \item \textbf{核心逻辑:} 强调“通过劳动、汗水或技能”来获得应得的报酬。它与 \textbf{deserve}(应得)相关。
            \item \textbf{语感:} 比较正式,侧重于辛苦工作换取的工资(salary/wages)。
            
            \es{例句: She works two jobs to \textbf{earn} enough money for her tuition.}
        \end{itemize}

        \item \textbf{Make money (赚钱/获利)}
        \begin{itemize}
            \item \textbf{核心逻辑:} 强调“创造利润”或“获取收益”。它关注的是结果和数额,不一定意味着辛苦劳动。
            \item \textbf{语感:} 范围更广,通常与投资、经商、甚至是手运气(如彩票)有关。
            
            \es{例句: He \textbf{made} a lot of money by selling his old car.}
        \end{itemize}

        \item \textbf{关键对比模型 (Comparison Model)}
        \begin{itemize}
            \item \textbf{Earn:} 劳动 $\rightarrow$ 报酬(逻辑:交换)。
            \item \textbf{Make:} 投入/机会 $\rightarrow$ 收益(逻辑:生产/创造)。
        \end{itemize}
    \end{enumerate}
\end{multicols}

\wsitem{Obtaining 'protection money' is not a modern crime. }
\begin{multicols}{1}
    使用 obtain 而非普通的 get 或 take,不仅体现了语体(Style)的正式性,更深层地揭示了黑帮犯罪的逻辑本质。

    \begin{enumerate}
        \item \textbf{强调“获取的过程” (Effort or Process)}
        \begin{itemize}
            \item \textbf{逻辑:} \textit{Get} 过于简单随性,而 \textbf{Obtain} 暗示了一定的过程、手段或努力。
            \item \textbf{语境:} 获取“保护费”并非捡钱,黑帮需要通过威胁、恐吓或建立复杂的组织网络来确保钱款到账。这个词体现了犯罪行为背后的“运作”感。
        \end{itemize}

        \item \textbf{法律与正式用语的严谨性 (Formal/Legal Tone)}
        \begin{itemize}
            \item \textbf{语感:} \textbf{Obtain} 是法律文件、新闻报道中描述获取非法财物的标准用词。
            \item \textbf{对比:} \textit{Steal} 是偷,\textit{Rob} 是抢,而 \textbf{Obtain} 更像是一个中性的总结词,涵盖了通过敲诈勒索(extortion)获得利益的整套行为。
        \end{itemize}

        \item \textbf{逻辑进阶:Obtain vs. Gain vs. Earn}
        \begin{itemize}
            \item \textbf{Earn:} 辛勤劳动换取的报酬(正当)。
            \item \textbf{Gain:} 获得利益(侧重于增加、积累)。
            \item \textbf{Obtain:} 获得所需之物(侧重于“达到目的”)。黑帮的目标是钱,他们通过犯罪手段 \textbf{obtained} 了这个目标。
        \end{itemize}
    \end{enumerate}
\end{multicols}

\wsitem{made the remarkable discovery}
\begin{multicols}{1}
    在英语排版和叙事中,"made the remarkable discovery" 这种结构被称为“轻动词结构”(Light Verb Construction)或“去词汇动词结构”。

    \begin{enumerate}
        \item \textbf{语法平衡:为形容词提供“落脚点”}
        \begin{itemize}
            \item \textbf{核心逻辑:} 如果只用动词 \textbf{discovered},你想修饰这个动作时只能用副词,比如 \textit{remarkably discovered}。但这听起来像是在说“发现的过程很奇特”,而不是“发现的内容很了不起”。
            \item \textbf{优势:} 使用名词 \textbf{discovery} 后,\textbf{remarkable}(非凡的)可以直接修饰这个成果。这种“动词 + 形容词 + 名词”的结构比“副词 + 动词”更符合英语正式书面语的节奏感。
        \end{itemize}

        \item \textbf{强调“成果”而非“动作” (Result vs. Action)}
        \begin{itemize}
            \item \textbf{核心逻辑:} \textit{Discover} 强调的是“发现”这个瞬间动作;而 \textbf{make a discovery} 强调的是通过一系列观察后得出的\textbf{结论或知识资产}。
            \item \textbf{语境:} 霍克伍德爵士并不是在某个瞬间“看到”了什么,而是经过长期的勒索实践,总结出了一套商业逻辑。这种逻辑更像是一个“产品”,因此用名词表达更厚重。
        \end{itemize}

        \item \textbf{语体色彩:提升叙事的正式性 (Formal Tone)}
        \begin{itemize}
            \item \textbf{核心逻辑:} 在历史叙事和正式文体中,使用“动名结构”能延缓句子的节奏,给读者一种庄重感。
            \item \textbf{对比:} 
            
            \es{He discovered... (听起来像在讲故事,节奏快)}

            \es{He made the discovery that... (听起来像在评价历史人物,有分量)}
        \end{itemize}
    \end{itemize}
\end{multicols}

\wsitem{Would offer to go away ...}
\begin{multicols}{1}
    这里使用 "would offer to go away" 而非 would go away,精准地捕捉到了黑帮敲诈过程中的“交易本质”和“心理博弈”。

    \begin{enumerate}
        \item \textbf{建立“交换”逻辑 (Creating a Bargain)}
        \begin{itemize}
            \item \textbf{核心差异:} \textit{Go away} 只是一个单纯的物理位移;而 \textbf{offer to go away} 则是一个\textbf{谈判条件}。
            \item \textbf{逻辑链条:} 烧掉农场(展示武力)$\rightarrow$ \textbf{提出条件}(offer)$\rightarrow$ 支付保护费(成交)$\rightarrow$ 离开。
            \item \textbf{意义:} 它强调了霍克伍德并不是真的想走,他把“离开”变成了一种需要对方花钱购买的“商品”。
        \end{itemize}

        \item \textbf{体现“威胁”的虚伪性 (The Hypocrisy of Threats)}
        \begin{itemize}
            \item \textbf{语感辨析:} \textbf{Offer} 通常用于善意的帮助(如 \textit{offer help})。但在这种语境下,作者用这个词来描述敲诈勒索,产生了一种强烈的\textbf{讽刺感}。
            \item \textbf{逻辑:} 他明明是在抢劫,却表现得像是在提供一项“服务”——“只要你们付钱,我就‘优惠’地离开”。
        \end{itemize}

        \item \textbf{延续“Would”的习惯性倾向 (Habitual Action)}
        \begin{itemize}
            \item \textbf{语法逻辑:} 这里的 \textbf{would} 表示过去反复发生的习惯。
            \item \textbf{逻辑:} 霍克伍德的套路是:先破坏,再\textbf{开价}。如果不加 \textit{offer},句子就变成了“烧完农场就走了”,逻辑上无法解释为什么要给他们钱。
        \end{itemize}
    \end{itemize}
\end{multicols}

\wsitem{Promptly vs. In time}
\begin{multicols}{1}
    这两个表达都与“及时”有关,但它们的核心逻辑和侧重点完全不同:promptly 侧重于“反应的速度”,而 in time 侧重于“截止日期前的期限”。

    \begin{enumerate}
        \item \textbf{Promptly (迅速地 / 敏捷地)}
        \begin{itemize}
            \item \textbf{核心逻辑:} 强调动作发生的**即时性**,通常指在一个触发点之后“立刻、不拖延”地行动。
            \item \textbf{语感:} 类似于 \textit{quickly} 或 \textit{without delay}。它描述的是一种高效的、雷厉风行的态度。
            \es{例句: She arrived \textbf{promptly} at 8:00. (她准时在8点到了——一秒都没耽误。)}
        \end{itemize}

        \item \textbf{In time (及时 / 赶得上)}
        \begin{itemize}
            \item \textbf{核心逻辑:} 强调在**截止时间或某个结果发生之前**到达或完成。
            \item \textbf{语感:} 带有“幸亏、没迟到”的逻辑,通常与 \textit{for something} 或 \textit{to do something} 连用。
            \es{例句: They arrived \textbf{in time} for the last boat. (他们赶上了最后一班船——如果再晚一点就没了。)}
        \end{itemize}

        \item \textbf{关键对比模型 (Temporal Model)}
        \begin{itemize}
            \item \textbf{Promptly:} 反应速度快 (Reaction Speed)。
            \item \textbf{In time:} 满足期限要求 (Meeting a Deadline)。
        \end{itemize}
    \end{enumerate}
\end{multicols}

\wsitem{}
\begin{multicols}{1}
    \begin{itemize}
        \item \textbf{核心语义 (Core Meaning)}
        \begin{itemize}
            \item \textbf{字面意思:} 把某人置于商业活动之外。
            \item \textbf{引申含义:} 使某人破产、倒闭,或者迫使某人无法再经营下去。
            \item \textbf{逻辑特征:} 这通常不是一种自然的竞争结果,而是暗示了某种外力干预、强力压制甚至是不正当手段。
        \end{itemize}

        \item \textbf{词组结构分析 (Structural Breakdown)}
        \begin{itemize}
            \item \textbf{Put + A + out of B:} 这是一个常见的动力学结构,表示“将A移出B的状态”。
            \item \textbf{类似结构:}
            \begin{itemize}
                \item \textit{Put out a fire} (灭火 —— 使火熄灭)
                \item \textit{Put someone out of action} (使某人失去行动能力/丧失战斗力)
            \end{itemize}
        \end{itemize}

        \item \textbf{语境逻辑:Quickly 的作用}
        \begin{itemize}
            \item 这里的 \textbf{quickly} 呼应了 \textbf{promptly}。它强调了黑帮手段的雷厉风行,不给受害者任何喘息和反抗的机会。
        \end{itemize}
    \end{itemize}
\end{multicols}

\wsitem{Make a name for oneself}
\begin{multicols}{1}
    \begin{enumerate}
        \item \textbf{核心语义 (Core Meaning)}
        \begin{itemize}
            \item \textbf{定义:} 指某人变得为人所知、受人尊敬或在特定领域获得声望。
            \item \textbf{褒义色彩:} 这个短语通常带有正面评价,暗示这种名声是“挣来的”(earned)而非偶然获得的。
        \end{itemize}

        \item \textbf{语法逻辑 (Grammatical Logic)}
        \begin{itemize}
            \item \textbf{反身代词:} \textbf{oneself} 必须随主语变化(myself, himself, herself 等)。
            \item \textbf{介词搭配:} 常接 \textbf{in}(在某个领域)或 \textbf{as}(作为某种身份)。
            
            \es{例句: He \textbf{made a name for himself} as a brilliant lawyer in New York.}
        \end{itemize}

        \item \textbf{近义词对比 (Synonym Comparison)}
        \begin{itemize}
            \item \textbf{Become famous:} 最通用的表达,强调知名度。
            \item \textbf{Make a mark:} 强调“留下痕迹”或“产生重大影响”。
            \item \textbf{Distinguish oneself:} 强调“使自己出类拔萃”(语气更正式)。
        \end{itemize}
    \end{enumerate}
\end{multicols}

\wsitem{Come to be known to ...}
\begin{multicols}{1}
    \begin{itemize}
        \item \textbf{核心语法拆解 (Structural Breakdown)}
        \begin{itemize}
            \item \textbf{Come to + do:} 表示“逐渐变得……”。
            \item \textbf{Be known to (someone):} 被(某人)所熟知。这里介词用 \textbf{to} 而不是 \textbf{by},强调的是一种认知的状态。
            \item \textbf{逻辑:} 某物/某人的名声或身份,经过一段时间的传播,最终传到了特定人群的耳中。
        \end{itemize}

        \item \textbf{语感辨析:Known as vs. Known to}
        \begin{itemize}
            \item \textbf{Known as:} 以……身份/名称而闻名(强调称号)。
            
            \es{例:\textit{He is known as a hero.}}

            \item \textbf{Known to:} 被……所知(强调知晓的人群范围)。
            
            \es{例:\textit{The secret came to be known to everyone.}}
        \end{itemize}

        \item \textbf{叙事中的逻辑作用 (Narrative Function)}
        \begin{itemize}
            \item 这个短语常用于交代背景,比如一个偏僻的地方、一个隐秘的发现或一个无名小卒,如何最终引起了外界的注意。
        \end{itemize}
    \end{itemize}
\end{multicols}

\wsitem{There was a time when...}
\begin{multicols}{1}
    "There was a time when..." 是一个非常经典的叙事性句式,用于引出对过去某段特定时期的回忆或背景描写。它能瞬间建立一种“怀旧”或“对比”的语境。

    \begin{enumerate}
        \item \textbf{核心语法结构 (Sentence Structure)}
        \begin{itemize}
            \item \textbf{There was a time:} 主句,表示“曾经有一个时期”。
            \item \textbf{when ...:} 引导定语从句,修饰前面的名词 \textbf{time},在从句中充当时间状语。
            \item \textbf{逻辑特征:} 这种结构通常暗示“现在情况已经不同了”,隐含着一种**过去与现在的对比**。
        \end{itemize}

        \item \textbf{语感与修辞效果 (Stylistic Effect)}
        \begin{itemize}
            \item \textbf{怀旧感:} 类似于 \textit{Once upon a time},但比它更正式,常用于自传或评论文。
            \item \textbf{引出冲突:} 后面常接一个转折(如 \textit{But now...}),强调时代的变迁。
        \end{itemize}

        \item \textbf{常见变体与扩展 (Variations)}
        \begin{itemize}
            \item \textbf{There will be a time when...:} 将来总有一天……(表示预言)。
            \item \textbf{There are times when...:} 有时候……(表示频率,相当于 \textit{sometimes})。
            \item \textbf{There have been times when...:} 曾经多次出现过……(强调经历)。
        \end{itemize}
    \end{enumerate}
\end{multicols}

\wsitem{Would rather ... than ...}
\begin{multicols}{1}

    "Would rather ... than ..." 是一个极其经典的选择性句式,用于表达“宁愿……也不愿……”的偏好逻辑。它在语义上相当于 "prefer to ... rather than ...",但在语法结构上有着独特的“双原形”要求。

    \begin{enumerate}
        \item \textbf{核心语法结构 (The "Double Base" Rule)}
        \begin{itemize}
            \item \textbf{公式:} Subject + would rather + \textbf{动词原形} + than + \textbf{动词原形}。
            \item \textbf{逻辑特征:} \textbf{than} 前后的动词形式必须严格对称。即使 \textbf{would} 是过去式形式,动词也绝不能加 -ed。
            
            \es{例句: I \textbf{would rather} stay home \textbf{than} go to the party. (我宁愿待在家里也不愿去参加聚会。)}
        \end{itemize}

        \item \textbf{否定形式的陷阱 (Negative Form)}
        \begin{itemize}
            \item \textbf{位置:} \textbf{not} 必须紧跟在 \textbf{rather} 后面,放在动词原形之前。
            \item \textbf{结构:} \textbf{would rather not} + 动词原形。
            \item \textit{注:此结构通常不与 than 连用,仅表示“宁愿不……”。}
        \end{itemize}

        \item \textbf{逻辑进阶:主语不一致的情况 (Subjunctive Mood)}
        \begin{itemize}
            \item \textbf{逻辑:} 当你想表达“我宁愿【别人】做某事”时,句子结构会发生质变。
            \item \textbf{公式:} I would rather + \textbf{从句(过去时)}。
            
            \es{例句: I \textbf{would rather} you \textbf{stayed} here. (我宁愿你留下来。—— 这里的 stayed 表达的是虚拟语气,而非过去时间。)}
        \end{itemize}
    \end{enumerate}
\end{multicols}

\newpage