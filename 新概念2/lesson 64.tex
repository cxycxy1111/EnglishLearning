\section{Lesson 64 The Channel Tunnel}

\begin{paracol}{2}

In 1858, a French engineer, Aime Thome de Gamond, arrived in England with a plan for a twenty-one-mile tunnel under the English Channel.

\switchcolumn

1858年,一位名叫埃梅·托梅·德·干蒙的法国工程师带着建造一条长21英里、穿越英吉利海峡的隧道计划到了英国。

\switchcolumn*

He said that it would be possible to build a platform in the centre of the Channel.

\switchcolumn

他说,可以在隧道中央建造一座平台。

\switchcolumn*

This platform would \ns{serve as} a port and a railway station.

\switchcolumn

这座平台将用作码头和火车站。
\nse{serve as}{}{用作...,充当...}

\switchcolumn*

The tunneI would be \nw{well-ventilated} if tall \nw{chimneys} were built above \ns{sea level}.

\switchcolumn

如果再建些伸出海面的高大的烟囱状通风管,隧道就具备了良好的通风条件。
\nwe{well-ventilated}{}{通风良好的}
\nwe{chimney}{}{烟囱}
\nse{sea level}{}{海平面}

\switchcolumn*

In 1860, a better plan was \ns{put forward} by an Englishman, William Low.

\switchcolumn

1860年,一位名叫威廉·洛的英国人提出了一项更好的计划。
\nse{put forward}{}{提出}

\switchcolumn*

He suggested that a double railway-tunnel should be built.

\switchcolumn

他提议建一条双轨隧道,这样就解决了通风问题。

\switchcolumn*

This would solve the problem of \nw{ventilation}, for if a train entered this tunnel, it would draw in fresh air behind it.

\switchcolumn

因为如果有一列火车开进隧道,它就把新鲜空气随之抽进了隧道。
\nwe{ventilation}{}{通风}

\switchcolumn*

Forty-two years later a tunnel was actually begun.

\switchcolumn

42年以后,隧道实际已经开始建了。

\switchcolumn*

If, at the time, the British had not feared invasion, it would have been completed. 

\switchcolumn
如果不是因为那时英国人害怕入侵,隧道早已建成了。

\switchcolumn*

The world had to wait almost another 100 years for the ChanneI Tunnel.

\switchcolumn

世界不得不再等将近100年才看到海峡隧道竣工。

\switchcolumn*

It was officially opened on March 7, 1994, finally connecting Britain to the European continent.

\switchcolumn

它于1994年3月7日正式开通,将英国与欧洲大陆连到了一起。

\switchcolumn*

\end{paracol}

\newpage